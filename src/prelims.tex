

\subsection{KZG Commitment Scheme}
\label{sec:KZG}

Here, we first define the KZG polynomial commitment scheme \cite{KZG}, denoted by the tuple \\
$(\KZGsetup,\KZGcommit,\KZGopen, \KZGverify)$, to be used for our construction. We then establish the isomorphism of the KZG polynomial commitment scheme as a vector commitment scheme.

\begin{definition}[KZG Polynomial Commitment Scheme]
	Let $(p,\Gone,\Gtwo,\GT, e, \eltone{1},\elttwo{1})$ be a bilinear group. The KZG polynomial commitment scheme is defined as follows: 
	\begin{itemize}
		\item $\KZGsetup$ on input $(1^\lambda,d)$, where $\lambda$ is the security parameter and $d$ is the degree bound, outputs $\srs = \{ \eltone{\tau},\ldots,\eltone{\tau^d},\elttwo{\tau},\ldots,\elttwo{\tau^d}\}$
		\item $\KZGcommit$ on input $(\srs,p(X))$, where $p(X) \in \F_{\leq d}[X]$, outputs $C=\eltone{p(\tau)}$
		\item $\KZGopen$ on input $(\srs,p(X),\alpha)$, where $p(X) \in \F_{\leq d}[X]$ and $\alpha\in \F$, outputs $(\val, \pi)$ such that $\val=p(\alpha)$ and $\pi=\eltone{q(\tau)}$, for
		\[ q(X)=\frac{p(X)-p(\alpha)}{X-\alpha} \]
		\item $\KZGverify$ on input $(\srs,C,\alpha,\val,\pi)$, outputs $1$ if the following equation holds, and $0$ otherwise.
		\[ e(C-\val, \elttwo{\tau}) = e(\pi, \elttwo{\tau-\alpha}) \]
	\end{itemize}
\end{definition}


\begin{definition}[KZG Polynomial Commitment Scheme \textcolor{red}{degree check included}]
    Let $(p,\Gone,\Gtwo,\GT, e, \eltone{1},\elttwo{1})$ be a bilinear group. The KZG polynomial commitment scheme is defined as follows: 
    \begin{itemize}
	        \item $\KZGsetup$ on input $(1^\lambda,d)$, where $\lambda$ is the security parameter and $d$ is the degree bound, outputs $\srs = \{ \eltone{\tau},\ldots,\eltone{\tau^d},\elttwo{\tau},\ldots,\elttwo{\tau^d}\}$
	        \item $\KZGcommit$ on input $(\srs,p(X))$, where $p(X) \in \F_{\leq d}[X]$, outputs $C=\eltone{p(X)}:=\eltone{p(\tau)}$
	        \item $\KZGopen$ on input $(\srs,p(X),\alpha)$, where $p(X) \in \F_{\leq d}[X]$ and $\alpha\in \F$, outputs $(\val, \pi)$ such that $\val=p(\alpha)$ and $\pi=\Comtwo{q(X)}:=\elttwo{q(\tau)}$, for
	        \[ q(X)=\frac{p(X)-p(\alpha)}{X-\alpha} \]
	        \item $\KZGverify$ on input $(\srs,C,\deg,\alpha,\val,\pi)$, outputs $1$ if the following equation holds, and $0$ otherwise.
	        \[ e(C-\val, \elttwo{\tau}) = e(\pi, \elttwo{\tau-\alpha}) \]
	        { \color{teal} \moumita{var 3.}
		        Degree check has not been added. With degree check, $\pi$ is modified with
		        \[ q(X)=X^{d-\deg(p(X))+2}\frac{p(X)-p(\alpha)}{X-\alpha} \]
		        and the verification equation if modified as 
		        \[ e(C-\val, \elttwo{\tau}^{d-\deg+2}) = e(\pi, \elttwo{\tau-\alpha}) \]
		        } %stays unchanged from var 1
	    \end{itemize}
\end{definition}

\paragraph{Vector Commitment Scheme.} We note that there is a natural isomorphism between the set of vectors of length $n$ and polynomials of degree $n-1$, where any vector $\abf=(a_1,\ldots,a_n) \in \F^n$ has a natural equivalent representation as a polynomial $a(X)=a_1\lambda_1(X)+\ldots,a_n\lambda_n(X)$, where $\{\lambda_i(X)\}_{i\in [n]}$ is basis of $\F_{\leq n-1}[X]$. Hence, when we use the KZG polynomial commitment scheme for a vector $\abf=(a_1,\ldots,a_n) \in \F^n$, we can output the commitment as $\KZGcommit(\srs,a(X))$, where $a(X)=a_1\lambda_1(X)+\ldots,a_n\lambda_n(X)$.


Now, for vectors of length $n$, we consider the basis $\{\lambda_i(X)\}$ to be the langrange basis with respect to the set consisting of $n^{th}$ roots of unity $\doubleH=\{\omega,\ldots,\omega^n\}$. The lagrange basis helps us use the property that, for some $i\in [n]$ and a value $\alpha$, $a(\omega^i)=a_i=\alpha$ if and only if there exists a polynomial $W(X)$ such that $W(X)=\frac{a(X)-\alpha}{X-\omega^i}$. 

Throughout the draft, for $\abf\in \F^n$, we use $\Comone{.}$ to denote $\KZGcommit(\srs,\abf)$ for vectors of length $n$, using the langrange polynomial basis $\{\lambda_i(X)\}$, with respect to the $n^{th}$ root of unity $\omega$, and for $\bbf \in \F^m$, we use $\Comtwo{.}$ to denote $\KZGcommit(\srs,\bbf)$ for vectors of length $m$, using the langrange polynomial basis $\{\mu_i(X)\}$, with respect to the $m^{th}$ root of unity $\nu$.