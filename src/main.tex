%! Author = nitin
%! Date = 03/11/23

% Preamble
\documentclass[11pt]{article}

% Packages
\usepackage{llncsconf}
\usepackage{times}
\usepackage{amsmath,amsthm,amsfonts,enumitem,tikz, subcaption}
\usepackage{amssymb}
\usetikzlibrary{arrows,arrows.meta,shapes.arrows, calc, positioning,matrix}

\newtheorem{theorem}{Theorem}[section]
\newtheorem{lemma}{Lemma}[section]
\newtheorem{definition}{Definition}[section]


%! Author = nitin
%! Date = 03/11/23

\newcommand{\npol}{\mathsf{NP}}
\newcommand{\enc}[2]{\mathsf{Enc}_{{}_{#2}}(\vec{#1})}
\newcommand{\vecA}{\vec{a}}
\newcommand{\vecB}{\vec{b}}
\newcommand{\vecC}{\vec{c}}
\newcommand{\vecD}{\vec{d}}
\newcommand{\vecP}{\vec{p}}
\newcommand{\vecQ}{\vec{q}}
\newcommand{\vecS}{\vec{s}}
\newcommand{\vecT}{\vec{T}}
\newcommand{\vecX}{\vec{x}}
\newcommand{\vecY}{\vec{y}}
\newcommand{\vecZ}{\vec{z}}
\newcommand{\vecW}{\vec{w}}
\newcommand{\vecI}{\vec{I}}
\newcommand{\vecind}{\overrightarrow{\ind}}
\newcommand{\vecop}{\overrightarrow{\op}}
\newcommand{\vecval}{\overrightarrow{\val}}
\newcommand{\setind}{I}
\newcommand{\ins}{\mathsf{ins}}
\newcommand{\ind}{\mathsf{ind}}
\newcommand{\op}{\mathsf{op}}
\newcommand{\vecins}{\overrightarrow{\ins}}

%%% Base table and cache table
\newcommand{\vecTbase}{\vec{T}_{\mathsf{b}}}
\newcommand{\vecTcache}{\vec{T}_{\mathsf{ch}}}
\newcommand{\Tbasepoly}[1]{T_{\mathsf{b}}({#1})}
\newcommand{\Tcachepoly}[1]{T_{\mathsf{ch}}({#1})}
\newcommand{\Qbasepoly}[1]{Q_{\mathsf{b}}({#1})}
\newcommand{\Qcachepoly}[1]{Q_{\mathsf{ch}}({#1})}
\newcommand{\Qcachepolyone}[1]{Q'_{\mathsf{ch}}({#1})}
\newcommand{\Qcachepolytwo}[1]{Q''_{\mathsf{ch}}({#1})}
\newcommand{\Zinv}[1]{(\mathbb{Z}_#1)^{\times}}
\newcommand{\Zn}{\mathbb{Z}_N}
\newcommand{\setH}{\mathbb{K}}
\newcommand{\setV}{\mathbb{V}}
\newcommand{\setN}{\mathbb{H}}
\newcommand{\vpolyN}{Z_\mathbb{H}}
\newcommand{\mroots}{\mathbb{V}}
\newcommand{\nroots}{\mathbb{H}}
\newcommand{\iroots}{\mathbb{H}_I}
\newcommand{\BG}{\mathsf{BG}}
\newcommand{\F}{\mathbb{F}}
\newcommand{\Fp}{{\mathbb{F}_p}}
\newcommand{\Zp}{{\mathbb{Z}_p}}
\newcommand{\Zq}{{\mathbb{Z}_q}}
\newcommand{\G}{\mathbb{G}}
\newcommand{\N}{\mathbb{N}}
\newcommand{\V}{\mathsf{V}}
\newcommand{\Z}{\mathsf{Z}}

\newcommand{\Gone}{\mathbb{G}_1}
\newcommand{\Gtwo}{\mathbb{G}_2}
\newcommand{\GT}{\mathbb{G}_T}
\newcommand{\srs}{\mathsf{srs}}
\newcommand{\cm}{\mathsf{cm}}
\newcommand{\wt}[1]{\widetilde{#1}}
\newcommand{\outp}{\rightarrow}
\newcommand{\pp}{\mathsf{pp}}
\newcommand{\pk}{\ensuremath{\mathsf{pk}}}
\newcommand{\bal}{\ensuremath{\mathsf{bal}}}
\newcommand{\txno}{\ensuremath{\mathsf{txno}}}
\newcommand{\load}{\ensuremath{\mathsf{LD}}}
\newcommand{\store}{\ensuremath{\mathsf{STR}}}
\newcommand{\RAM}[2]{\ensuremath{\mathsf{RAM}}_{\mathcal{#1}, {#2}}}
\newcommand{\RAMOp}[1]{\ensuremath{\mathcal{O}_{{}_\mathcal{#1}}}}
\newcommand{\Rupd}[1]{\ensuremath{\mathsf{Upd}_{\mathcal{#1}}}}
\newcommand{\LRAM}[3]{\ensuremath{\mathsf{LRAM}_{\mathcal{#1},#2,#3}}}
\newcommand{\LTr}{\ensuremath{\mathcal{L}_{\mathsf{Tr}}}}
\newcommand{\tr}{\ensuremath{\mathsf{tr}}}
\newcommand{\TOT}{\ensuremath{\mathsf{TimeTr}}}
\newcommand{\AOT}{\ensuremath{\mathsf{AddrTr}}}
\newcommand{\evalH}[1]{\ensuremath{\vec{#1}_{|_\setH}}}
\newcommand{\evalV}[1]{\ensuremath{\vec{#1}_{|_\setV}}}
\newcommand{\LLconcat}{\ensuremath{\tilde{\mathcal{L}}_{\mathsf{concat}}}}
\newcommand{\Lperm}{\ensuremath{\mathcal{L}_{\mathsf{perm}}}}
\newcommand{\RRAM}[1]{\ensuremath{R}_{\mathsf{ram},#1}}
\newcommand{\RLOOK}{\ensuremath{\mathcal{R}^{\mathsf{lookup}}_{\srs,N,m}}}
\newcommand{\CRAM}{\ensuremath{\mathcal{R}^{\mathsf{ram}}_{\srs,N,m}}}
\newcommand{\CLRAM}{\ensuremath{\mathcal{R}^{\mathsf{LRAM}}_{\srs,m}}}
\newcommand{\val}[2]{\ensuremath{\mathsf{V}_{{#2},{#1}}}}
\newcommand{\uniq}[1]{\ensuremath{\mathsf{uniq}(\vec{{#1}})}}
% group encodings
\newcommand{\gone}[1]{\ensuremath{\left[{#1}\right]_1}}
\newcommand{\gtwo}[1]{\ensuremath{\left[{#1}\right]_2}}
\newcommand{\gany}[1]{\ensuremath{\left[{#1}\right]_g}}

% provers, verifiers, adversaries
\newcommand{\prover}{\ensuremath{\mathcal{P}}}
\newcommand{\verifier}{\ensuremath{\mathcal{V}}}
\newcommand{\Adv}{\ensuremath{\mathcal{A}}}
\newcommand{\kzg}{\ensuremath{\mathsf{KZG}}}
\newcommand{\kzgprove}{\ensuremath{\mathsf{KZG.Prove}}}
\newcommand{\kzgverify}{\ensuremath{\mathsf{KZG.Verify}}}
\newcommand{\kzgcommit}{\ensuremath{\mathsf{KZG.Commit}}}

%%%Author Comments--------------------------------------------
\newcommand{\moumita}[1]{\textcolor{orange}{M: #1}}
\newcommand{\chaya}[1]{\textcolor{cyan}{Chaya: #1}}
\newcommand{\sikhar}[1]{\textcolor{red}{Sikhar: #1}}
\newcommand{\nitin}[1]{\textcolor{blue}{Nitin: #1}}
\newcommand{\shubh}[1]{\textcolor{green}{Shubh: #1}}

%group elements and commitments
\newcommand{\eltone}[1]{[#1]_1}
\newcommand{\elttwo}[1]{[#1]_2}
\newcommand{\elany}[1]{[#1]_g}
\newcommand{\elt}[1]{[\,{#1}\,]}
\newcommand{\Comone}[1]{\mathsf{Com}_1(#1)}
\newcommand{\Comtwo}[1]{\mathsf{Com}_2(#1)}
\newcommand{\ComT}[1]{\mathsf{Com}_T(#1)}

\newtheorem{fact}{Fact}[section]

%%%KZG 
\newcommand{\Com}[1]{\mathsf{Com}(#1)}
\newcommand{\KZGsetup}{\mathsf{KZG.Setup}}
\newcommand{\KZGcommit}{\mathsf{KZG.Commit}}
\newcommand{\KZGopen}{\mathsf{KZG.Prove}}
\newcommand{\KZGverify}{\mathsf{KZG.Verify}}
\newcommand{\out}{\mathsf{out}}
\newcommand{\ppt}{\mathrm{PPT}}
% \newcommand{\in}{\mathsf{in}}
\newcommand{\rangeproof}{\mathsf{Range}}

\newcommand{\setup}{\mathsf{Setup}}
\newcommand{\commit}{\mathsf{Commit}}
\newcommand{\open}{\mathsf{Open}}
\newcommand{\verify}{\mathsf{Verify}}

%%%Elements
\newcommand{\abf}{\textbf{a}}
\newcommand{\bbf}{\textbf{b}}
\newcommand{\cbf}{\textbf{c}}
\newcommand{\ibf}{\textbf{i}}
\newcommand{\jbf}{\textbf{j}}
\newcommand{\kbf}{\textbf{k}}
\newcommand{\pbf}{\textbf{p}}
\newcommand{\qbf}{\textbf{q}}
\newcommand{\rbf}{\textbf{r}}
\newcommand{\sbf}{\textbf{s}}
\newcommand{\vbf}{\textbf{v}}
\newcommand{\xbf}{\textbf{x}}
\newcommand{\ybf}{\textbf{y}}
\newcommand{\zbf}{\textbf{z}}
\newcommand{\Tbf}{\textbf{T}}
\newcommand{\Sbf}{\textbf{S}}
\newcommand{\Mbf}{\textbf{M}}

%%%Sets
\newcommand{\doubleH}{\mathbb{H}}
\newcommand{\Hi}{{\mathbb{H}_I}}
\newcommand{\doubleV}{\mathbb{V}}
\newcommand{\Vm}{{\mathbb{V}_m}}
\newcommand{\zeroset}{\{0\}}


%% Bilinear group generator
\newcommand{\bg}{\ensuremath{\mathsf{BG}}}
%% Security parameter
\newcommand{\secp}{\ensuremath{\lambda}}
\newcommand{\negl}{\ensuremath{\mathsf{negl}}}

\title{Faster Rollups from Improved Persistent RAM}

% Document
\begin{document}
    \maketitle
    \begin{abstract}
        Persistent RAM (random access memory) is a useful primitive in verifiable computation.
        The persistent RAM primitive allows one to efficiently verify that execution of a program $\mathcal{P}$
        with read/write access to memory state $\mathsf{st}_0$ results in the final state $\mathsf{st}_1$.
        The efficiency requires that verification is cheaper than naive execution of $\mathcal{P}$
        (typically poly-logarithmic) and only requires access to {\em succinct} digests (e.g. Merkle hash) of the state(s).
        Persistence refers to the fact that state digests allow proving updates to the state across several computations
        in an incremental fashion.

        Current approaches realizing persistent RAM in verifiable computation generally use Merkle-trees
        or cryptographic accumulators based on unknown-order groups (RSA, class-groups) to compress the state. The
        Merkle-tree based approaches suffer from high concrete costs due to poor batching properties, leading to
        verification of several hash function evaluations inside VC circuits. While the accumulator based approaches
        offer significant improvement for bigger batch sizes, they still incur computations linear in the size of the
        accumulated set to compute witnesses for inclusion proofs.

        In this paper, we present a polynomial protocol that realizes an improved persistent RAM primitive, building
        on the techniques used in recent lookup arguments with table size independent prover complexity.
        The RAM state and updates are suitably represented as polynomials, and proving correctness of the state update
        reduces to proving identities over related polynomials.
        These can be compiled into succinct non-interactive arguments
        of knowledge (SNARK) using a polynomial commitment scheme like \textsf{KZG} under the Algebraic Group Model (AGM) and
        the Random Oracle model. Amortized cost of proving correctness of a batch of $m$ updates for RAM of size $N$ is
        $O(m^2)$ field operations and a multi-exponentiation of size $O(\sqrt{mN})$, thus achieving sublinear $(\sqrt{N})$
        dependence on the RAM size. We also show that our approach outperforms existing approaches in practice and
        experimentally validate superior performance of our primitive when applied to blockchain rollups.

    \end{abstract}

    \section{Introduction}\label{sec:introduction}
    General purpose {\em Succinct Non-interactive Arguments of Knowledge} (SNARKs) enable one to generate succinct
    proofs of membership of a statement in an $\npol$ relation expressed as an arithmetic circuit. These proofs are
    extremely cheap to verify, which makes them useful for {\em Verifiable Computation} (VC), where a low-powered
    client (e.g. mobile phone), can outsource an expensive computation to an untrusted server, and later
    verify the correctness of the results at a minimal cost.
    However,
    arithmetic circuit based representations are inefficient in expressing relations involving the result of
    a program execution on memory/state, which frequently arise in the context of verifiable computation.
    These include scenarios such as proving the correctness of a query execution against a database,
    proving the correctness of inference
    from a decision tree, or proving correct update of table of account balances when a batch of transactions (e.g. transfers)
    are applied to it.
    In aforementioned examples, database tables, decision trees and accounts table are naturally
    modelled as addressable memory and one needs to prove that it is accessed/updated in accordance with the correct execution
    of the computation. Accordingly, there is a rich and expanding body of work to efficiently model the abstraction of
    addressable memory in verifiable computations. While complete acknowledgement of this vast body of work is beyond
    the scope of this paper (a fairly recent survey in \cite{WB15} is a good starting point) we summarise key approaches towards modelling RAM, and more pertinently persistent RAM
    in verifiable computations in the next section.


    The final example above is especially relevant in the context of recent efforts to scale blockchains
    by moving expensive computation off the blockchain to the so-called {\em layer two} or L2 chains. The blockchain
    only needs to verify the succinct proofs attesting to the correctness of the off-chain computation. This approach
    is popularly called {\em rollup}, as it allows verifying the result of several transactions modifying the L2 state (
    rolled up transactions), as part of one transaction verified on the main chain. This simultaneously helps the scalability
    and lowers the cost (e.g. gas fees) per transaction (due to succinct verification). The specific rollup scenario
    is discussed later in the paper, to illustrate the efficacy of our approach to achieving much more efficient rollups.

    \subsection{Our work}\label{subsec:ourwork}
    We present a persistent RAM construction, which advances the efforts
    towards achieving {\em verifiable outsourcing of state update} such as in ~\cite{EPRINT:BFRSBW13}
    and more recently in ~\cite{USENIX:OWWB20, CCS:CFHKKO22}. The most popular approaches to succinctly represent
    state involve accumulators based on Merkle-trees ~\cite{C:Merkle87}, or ones based on groups of unknown order
    (e.g. RSA, class-groups) ~\cite{C:CamLys02,C:BonBunFis19,USENIX:OWWB20, CCS:CFHKKO22}.
    The updates to the state are effected by insertions or deletions in the  accumulated set.
    By contrast, in this work we
    model the state as an addressable memory (RAM) described by vector $\vecT$, which stores a values $v_i$ at addresses $i$.
    We denote this as $\vecT[\,i\,]=v_i$. The RAM supports two operations (i) {\em indexed lookups} ($v_i := \vecT[\,i\,]$),
    denoted by the tuple $(\load, i, v_i)$ and (ii) {\em indexed update} ($\vecT[\,i\,] := v_i$), denoted
    by the tuple $(\store, i, v_i)$. We think of addresses $i\in [0,N]$ for $N\in \mathbb{Z}$ while the
    values $v_i\in \F$ for some finite field $\F$. We encode both the RAM and operations as polynomials, and
    use appropriate polynomial commitment schemes to obtain succinct commitments (digests) to them.
    In this paper, we do not require commitments to be {\em hiding}, as our focus is on succinctness.
    We consider privacy as an orthogonal goal, one we believe is easily achievable via small adaptations to our construction.

    \noindent{\bf Application to Blockchain Rollups}: We consider the application of persistent RAM primitive to a common rollup scenario.
    Our application is a simplified adaptation of rollup protocols such as ZkSync ~\cite{ZkSync}.
    We assume a layer two (L2) chain \textsf{DemoChain}, which issues its clients \textsf{DemoCoins}. The clients have accounts on
    L2, and they transfer \textsf{DemoCoins} to each other using L2 transactions. Such a system will also have mechanism to transfer
    funds to and from main-chain (e.g. Ethereum) accounts, but we
    ignore those details here.
    Each client is assigned a unique identifier $i$, and associated account information  is maintained as
    a tuple $(\pk_i,\bal_i,\txno_i)$ which refer respectively to the public key for verifying client's signature on transactions,
    account balance (number of coins owned by the client) and total number of transactions made from the account (to prevent replay attacks).
    The L2 state thus consists of the above information for all accounts. We model an L2 chain supporting upto $N$ accounts as a RAM $\vecT$
    of size $N$, where $\vecT[\,i\,]=(\pk_i,\bal_i,\txno_i)$ denotes client $i$'s account information. A transfer transaction of amount $v$
    from client $i$ to client $j$ involves two load operations, and two store operations on the RAM (for reading and updating the referenced
    accounts). These transactions are submitted by clients directly to a designated L2 chain operator \textsf{DemoOperator}, who batches $m$
    such transactions and updates the RAM state accordingly. The operator maintains the entire state off-chain and locally computes the updates for
    each batch of transactions. Only the succinct digest of the state (polynomial commitments) is stored on the main chain, and the proof of
    validity of the state update is verified by a main-chain transaction. In later sections, we provide more details on the application and the
    performance achieved when instantiated using our primitive.

    \subsection{Techniques}\label{subsec:techniques}
    \noident{\bf Improved lookup from updatable tables}: Starting point for our work are the recent lookup arguments which prove that a vector of size $m$ appears as
    a sub-vector in a large fixed vector (table) of size $N$ with succinct proof sizes and verification, but most notably
    ensuring that prover runs in time sub-linear in the size of the table ($N$). The pioneering work ~\cite{CCS:ZBKMNS22}
    obtained prover complexity of $O(m^2+m\log N)$, which was improved in subsequent works to $O(m^2)$ ~\cite{EPRINT:PosKat22},
    $O(m\log^2 m)$ ~\cite{EPRINT:ZGKMR22}, and $O(m\log m)$ ~\cite{EPRINT:EagFioGab22}. However, the sub-linear prover
    complexity requires table-dependent $O(N\log N)$ pre-processing and $O(N)$ storage. This table-dependent
    pre-processing implies that while
    the aforementioned lookup arguments can be used to obtain efficient ROM (read only memory) semantics
    \footnote{The protocols for sub-vector relation in ~\cite{CCS:ZBKMNS22, EPRINT:ZGKMR22} also implicitly support indexed lookup semantics},
    they cannot be used as is for RAM (which supports update operations). Moreover, an update involving even a single
    index renders the entire $O(N)$ pre-processing unusable for further lookups, thus necessitating entire $O(N\log N)$
     re-computation. An important contribution of this work is modified pre-processing and prover for lookup scheme
    in ~\cite{EPRINT:PosKat22} which continues to ensure efficient lookups if updates to the table are small since the pre-processing.
    Informally, we achieve the following:

    \begin{theorem}[Informal]\label{thm:pre-process}
        There exists a deterministic $O(N\log N)$ time algorithm $\textsf{Params}(\vecT)\outp \pp_T$
        which on input $\vecT\in \F^N$, outputs parameters $\pp_T$ of size $O(N)$ such
        that: Given $\pp_T$, vectors $\vecT'\in \F^N$, $\vec{t}\in \F^m$ with $\vec{t}$ being a sub-vector of $\vecT'$
        an argument of knowledge for the same can be computed in time
        $O((m+\delta)\log^2 (m+\delta) + m^2)$ where $\delta=\Delta(\vecT, \vecT')$
    is the number of positions where $\vecT$ and $\vecT'$ differ.
    \end{theorem}
    For the construction in ~\cite{EPRINT:EagFioGab22}, a corresponding version of the above theorem holds with the prover complexity
    of $O((m+\delta)\log^2(m+\delta))$. We also remark that above theorem also holds for the indexed lookups that we briefly discussed before.\smallskip

    \noindent{\bf Polynomial Protocol for RAM}: Most of the efficient implementations of RAM in verifiable computation ~\cite{C:BCGTV13, NDSS:WSRBW15, SP:ZGKPP18}
    rely on the {\em address-ordered transcript} to check that a sequence of $\load$s and $\store$s are {\em consistent} with some initial state
    of the RAM. Using the tuple $(t,\op,\ind, v)$ to denote a RAM instruction with $t$ being the execution {\em timestamp}, $\op\in \{\load,\store\}$ being
    the operation type, $\ind$ being the index referenced and $v$ denoting the value read/stored, a time ordered execution transcript $\mathsf{tr} = (\ins_1,\ldots,\ins_m)$ is a sequence
    of instructions where $\ins_i=(t_i,\op_i,\ind_i, v_i)$ and $t_i < t_{i+1}$ for all $1\leq i<m$. The transcript $\mathsf{tr}$ is said to be consistent if
    for every $\load$ instruction $\ins=(t,\load,\ind,v)$, the value $v$ is the same as that for the most recent $\store$ instruction, which references the same index
    as $\ins$. This consistency can be efficiently checked by permuting the transcript $\mathsf{tr}$ to produce the sequence
    $\mathsf{tr}^\ast=(\ins_1^\ast,\ldots,\ins_m^\ast)$ with $\ins_i^\ast=(t_i^\ast,\op_i^\ast,\ind_i^\ast,v_i^\ast)$ which is sorted by index, with
    timestamp used to break the ties, i.e, $\ind_i^\ast\leq \ind_{i+1}^\ast$ and whenever $\ind_i^\ast=\ind_{i+1}^\ast$, we have $t_i^\ast < t_{i+1}^\ast$.
    We call the resulting transcript $\mathsf{tr}^\ast$ as address ordered transcript. On such a transcript, the consistency is enforced simply by
    checking that each $\load$ instruction involves the same value $v$ as the previous instruction, provided the referenced indices are the same.
    The above approach is easily extended to verify that result of executing a sequence of operations $\{(\op_i,\ind_i,v_i)\}_{i=1}^m$ on
    RAM $\vecT_0\in \F^n$ is $\vecT_1\in F^n$. In this case, we define a transcript $\mathsf{tr}=(\ins_1,\ldots,\ins_{2n+m})$ where the
    first $n$ instructions $\ins_i,i\in [n]$ are defined as $\ins_i=(i,\store,i,\vecT_0[\,i\,])$, the next $m$ instructions are defined
    as $\ins_{n+i},i\in [m]$ as $\ins_{n+i}=(n+i, \op_i, \ind_i, v_i)$ and the last $n$ instructions $\ins_{n+m+i}, i\in [n]$ as
    $\ins_{n+m+i}=(n+m+i,\load,i,\vecT_1[\,i\,])$. The consistency of the transcript $\mathsf{tr}$ is checked via address ordering permutation as
    described before. Intuitively, the first $n$ instructions in $\mathsf{tr}$ copy the initial contents of the RAM, the next $m$ instructions essentially
    capture the sequence of RAM operations executed on the initial RAM, while the final $n$ instructions read out the entire contents of the RAM which
    is expected to match the contents of $\vecT_1[\,i\,]$. One of the contributions of this work is a polynomial protocol that proves that RAM state
    $\vecT_1\in \F^n$ is obtained from RAM state $\vecT_0\in \F^n$ as a result of sequence of operations $\vec{\op}=\{(\op_i,\ind_i,v_i)\}_{i=1}^m$, where each
    of the vectors $\vecT_0,\vecT_1$ and $\vec{\op}$ are represented via polynomials. Informally, we show:
    \begin{theorem}[informal]\label{thm:pp-for-ram-informal}
    For $m,n\in \mathbb{N}$, there exists a polynomial protocol to prove that sequence of operations $\vec{\op}=\{(\op_i,\ind_i,v_i)\}_{i=1}^m$ updates a
    RAM state $\vecT_0\in \F^n$ to RAM state $\vecT_1\in \F^n$ with the prover-complexity of $\tilde{O}(m+n)$.
    \end{theorem}

    \noindent{\bf Improved RAM from Lookup Arguments}: The approach of the previous paragraph incurs a linear cost in
    the size of the RAM, even though the number of operations $m$ could be much smaller $m\ll n$. To circumvent the
    dependence on RAM size $n$, we use the efficient lookup arguments to isolate the part of RAM which is involved
    in the operation sequence $\vec{\op}=\{(\op_i,\ind_i,v_i)\}_{i=1}^m$.
    Concretely, from the vectors $\vecT_0,\vecT_1\in \F^n$ denoting the initial and final RAM states, we claim
    vectors $\vec{t}_0, \vec{t}_1\in \F^m$ such that $\vec{t}_j[\,i\,] = \vecT_j[\,\ind_i\,]$ for all $i\in [m]$ and
    $j=0,1$. Intuitively (modulo certain technicalities), we need to show that (i) the operation sequence $\vec{\op}$
    transforms $\vec{t}_0$ to $\vec{t}_1$ and (ii) the RAMs $\vecT_0$ and $\vecT_1$ are identical at indices not
    involved in $\vec{\op}$, i.e $\vecT_0[\,i\,]=\vecT_1[\,i\,]$ for $i\not\in \{\ind_i: i\in [m]\}$. In later
    sections, we show that the latter check can be accomplished with $\tilde{O}(m)$ prover complexity. The former
    check involves an adaptation of address ordered transcript approach to incorporate additional subtleties such
    as the possibility of multiple rows of $\vec{t}_0$ and $\vec{t}_1$ referring to the same index of their parent RAMs $\vecT_0$ and
    $\vecT_1$ respectively.
    In later sections, we show that this is efficiently handled by using the inherited indices in constructing the transcript showing
    consistency of $\vec{t}_0$ and $\vec{t}_1$. Since the lookup protocol is $O(m^2)$ and the memory consistency check involves transcript
    of size $O(m)$, the overall complexity of this step is $O(m^2)$. Figures ~\ref{fig:subtable-lookup} and ~\ref{fig:subtable-consistency}
    illustrate the overall idea for the first check with a concrete example.

    \noindent{\bf Incrementally Verifiable RAM with Deferred Pre-processing}: The approach in the previous paragraph yields an argument for RAM with
    table-independent prover-complexity ($O(m^2)$), {\em provided} the prover has access to the table-dependent pre-processing
    $\pp_{T,0}$ for the table $\vecT_0$. We also require lookup from table $\vecT_1$, but it differs from $\vecT_0$ in at most $m$
    positions, so by Theorem \ref{thm:pp-for-ram-informal}, this lookup also requires $O(m^2)$ prover time. Next, let's suppose we apply an
    update of $m$ operations to obtain $\vecT_2$ from $\vecT_1$, and incrementally obtain $\vecT_3,\ldots,\vecT_k$. By Theorem ~\ref{thm:pp-for-ram-informal},
    the update $\vecT_k\rightarrow \vecT_{k+1}$ costs $O(m^2 + (mk)\log^2(mk))$. If we choose to re-compute the pre-processing parameters after $k$
    batches (of $m$ updates each), the average cost per batch is $O(N\log N/k + m^2 + mk\log^2(mk))$. Setting $k\approx \sqrt{N/m}$ yields the
    average cost of $m$ updates as $O(m^2+\sqrt{mN})$, which scales sub-linearly with the size of the RAM. While the preceding analysis considers
    worst case, in specific applications (such as account transactions, where few accounts contribute a large volume of transactions), it may be
    possible to further defer the computation of pre-processing parameters. Thus we have:

    \begin{theorem}[Informal]\label{thm:inc-ver-ram-informal}
    Given $m,N\in \mathbb{N}$, there exists a polynomial protocol which incrementally proves updates of batch size $m$ on RAM of size $N$
    with amortized prover complexity of $O(m^2 + \sqrt{mN})$.
    \end{theorem}

    
    \tikzset{
        table/.style={
            matrix of nodes,
            row sep=-\pgflinewidth,
            column sep=-\pgflinewidth,
            nodes={
                rectangle,
                draw=black,
                align=center
            },
            minimum height=1.5em,
            text depth=0.5ex,
            text height=1.5ex,
            nodes in empty cells,
%%
            every even row/.style={
                nodes={fill=gray!20}
            },
            column 1/.style={
                nodes={text width=2em}
            },
            row 1/.style={
                nodes={
                    fill=white,
                    text=black,
                    font=\bfseries
                },
            }
        }
    }

    \tikzset{
        MyArrow/.style={
            single arrow, draw=black, minimum width=10mm, minimum height=30mm, inner sep=0mm, single arrow head extend=1mm, double arrow head extend=1mm
        }
    }

    \begin{figure}[t]
        \centering
        %\subcaptionbox{Lookup to get sub-tables \label{fig:lookup}}[\textwidth]{
        \resizebox{0.8\textwidth}{0.3\textheight}{
        \begin{tikzpicture}
            %\draw[step=1.0,black,thin] (-5,-5) grid (5,5);
        \matrix (initial) [table,text width=2em, width=2cm] at (-5,1)
        {
            Idx & Val\\
            1  & 10 \\
            2  & 20 \\
            3  & 15 \\
            4  & 25 \\
            5  & 40 \\
            6 & 50 \\
            7 & 60 \\
        };

        \matrix (ops1) [table,text width=2em, width=2cm] at (0,0)
            {
            Op & Idx & Val \\
            \store  & 7 & 40 \\
            \load  & 1 & 10 \\
            \store & 3 & 20 \\
            \load  & 3 & 20 \\
        };


        \matrix (final) [table,text width=2em, width=2cm] at (5,1)
            {
            Idx & Val\\
            1  & 10 \\
            2  & 20 \\
            3  & 20 \\
            4  & 25 \\
            5  & 40 \\
            6 & 50 \\
            7 & 40 \\
        };


        % -------------------------------- Subtables ------------------------------------- %
        \matrix (stinitial) [table,text width=2em, width=2cm] at (-5,-5)
            {
            Idx & Val\\
            7 & 60 \\
            1 & 10 \\
            3 & 15 \\
            3 & 15 \\
        };

        \matrix (ops2) [table,text width=2em, width=2cm] at (0,-5)
            {
            Op & Idx & Val \\
            \store  & 7 & 40 \\
            \load  & 1 & 10 \\
            \store & 3 & 20 \\
            \load  & 3 & 20 \\
        };


        \matrix (stfinal) [table,text width=2em, width=2cm] at (5,-5)
            {
            Idx & Val\\
            7 & 40 \\
            1 & 10 \\
            3 & 20 \\
            3 & 20 \\
        };

            \draw (-5,3.6) node[above]  {\bf RAM $\vecT_0$};
            \draw (0, 1.6) node[above] {\bf Operations};
            \draw (5,3.6) node[above] {\bf RAM $\vecT_1$};
            \draw (-5, -3.4) node[above] {\bf $\vec{t}_0$};
            \draw (5, -3.4) node[above] {\bf $\vec{t}_1$};

            % lookup arrows
            \draw [-Latex, thick] (-4.25,-1.75) -- (-4.25,-3.25);
            \draw [-Latex, thick] (4.25, -1.75) -- (4.25, -3.25);
            \draw [fill=grey!30!white] (-2,-2) rectangle (2, -3) node[midway] {Lookup on indices 7,1,3,3};

            \draw (-2, -2.5) -- (-4.25, -2.5) (2, -2.5) -- (4.25, -2.5);
            \draw [-Latex, dashed] (-4, 0) -- (-1.6,0);
            \draw [-Latex, dashed] (1.6,0) -- (4,0);
            \draw [-Latex, dashed] (-4, -5) -- (-1.6,-5);
            \draw [-Latex, dashed] (1.6,-5) -- (4,-5);
        \end{tikzpicture}
        }
        \caption{Using lookup arguments to reduce the consistency check between large RAMs $\vecT_0$
        and $\vecT_1$ to that between small tables $\vec{t}_0$ and $\vec{t}_1$. It is seperately shown that RAMs $\vecT_0$ and
        $\vecT_1$ have identical values outside the involved indices. Figure ~\ref{fig:subtable-consistency} illustrates address
        ordered transcript to check consistency of $\vec{t}_0$ and $\vec{t}_1$.}
        \label{fig:subtable-lookup}
        \end{figure}

        \begin{figure}[htbp]
            \centering
            \resizebox{0.8\textwidth}{0.4\textheight}{
        \begin{tikzpicture}
        % ------------------------------------------------------------------------------------------- %
            %\draw[step=1.0,black,thin] (-5,-5) grid (5,5);
        \matrix (trts) [table,text width=2em] at (-4,0)
            {
            Ts & Op & Idx & Val\\
            1 & \store & 7 & 60 \\
            2 & \store & 1 & 10 \\
            3 & \store & 3 & 15 \\
            4 & \store & 3 & 15 \\
            5 & \store  & 7 & 40 \\
            6 & \load  & 1 & 10 \\
            7 & \store & 3 & 20 \\
            8 & \load  & 3 & 20 \\
            9 & \load & 7 & 40 \\
            10 & \load & 1 & 10 \\
            11 & \load & 3 & 20 \\
            12 & \load & 3 & 20 \\
        };

        \matrix (trts) [table,text width=2em, width=2cm] at (4,0)
            {
            Ts & Op & Idx & Val\\
            2 & \store & 1 & 10 \\
            6 & \load & 1 & 10 \\
            10 & \load & 1 & 10 \\
            3 & \store & 3 & 15 \\
            4 & \store  & 3 & 15 \\
            7 & \store  & 3 & 20 \\
            8 & \load & 3 & 20 \\
            11 & \load  & 3 & 20 \\
            12 & \load & 3 & 20 \\
            1 & \store & 7 & 60 \\
            5 & \store & 7 & 40 \\
            9 & \load & 7 & 40 \\
        };

            \path (-2.1,0) -- (1.9, 0)  node[midway, MyArrow, text width=3cm, align=center] {Address-sorting permutation};
            \draw (-4, 4.3) node[above] {\bf \small Time ordered transcript ($\mathsf{tr}$)};
            \draw (4, 4.3) node[above] {\bf \small Address ordered transcript ($\mathsf{tr}^\ast$)};

        \end{tikzpicture}
            }
            \caption{Address ordered transcript for checking consistency of tables $\vec{t}_0$ and $\vec{t}_1$
                with respect to the operation sequence in Figure \ref{fig:subtable-lookup}. On the transcript
            $\mathsf{tr}^\ast$, one checks that the load instructions return the same value as the preceeding
            instruction if their indices are the same.}

        \label{fig:subtable-consistency}
    \end{figure}

    \noindent{\bf Our Contributions}: (Summarise later)

    \section{Related Work}\label{sec:rel-work}

    \section{Polynomial Protocol for RAM}\label{sec:poly-proto-ram}
    In this section, we briefly review and formalize existing memory-checking techniques to ensure
correctness of RAM operations. The formal definitions for various relations involved in memory checking
will be used to describe polynomial protocol for RAM in Appendix~\ref{sec:poly-proto-ram-app}.

\subsection{Correctness of RAM Update}\label{subsec:ram-update}
The versatility of the RAM primitive stems from its updatability. While a load operation leaves the RAM unchanged, the store operation
updates the value in the RAM associated with the referenced index. We model the update via the function
$\Rupd{I}$ which takes RAM $\vecT\in \RAM{I}{n}$, operation
$o=(\op,a,v)\in \RAMOp{I}$ as inputs and returns an updated RAM $\vecT'\in \RAM{I}{n}$.
The updated RAM $\vecT'=\Rupd{I}(\vecT,o)$ satisfies
$\vecT'=\vecT$ if $\op=0$ while for $\op=1$ it satisfies $\vecT'[\,a\,]=v$  and $\vecT'[\,x\,]=\vecT[\,x\,]$ for $x\neq a$. The central problem
in verifiable RAM protocols is to establish that a sequence of operations $\vec{o}=(o_1,\ldots,o_m)$ are correct with
respect to the initial RAM state $\vecT$ and the final RAM state $\vecT'$. This involves ensuring
that all load operations read the value which is consistent with updates to the RAM as a result of preceding
store operations, and that $\vecT'$ is the final state. We say that an operation $o=(\op,a,v)$ is {\em load-consistent}
with respect to RAM $\vec{T}$ if $v=\vecT[a]$ whenever $o$ is a load operation (store operations are vacuously defined to be load-consistent).
We formally define the notion of consistency below:

\begin{definition}[Consistent Operations]\label{defn:consistent-operations}
    Let $n\in \N$ and $\vecT,\vecT'\in \RAM{I}{n}$ for some index set $\setind$. We say that a sequence of operations
    $\vec{o}=(o_1,\ldots,o_k)\in \RAMOp{I}^k$ over $\setind$ is {\em consistent} with RAM states
    $\vecT,\vecT'$ if for all $i\in [k]$, $\vecT_{i}=\Rupd{I}(\vecT_{i-1},o_i)$ and operation $o_i$ is load-consistent with respect to $\vecT_{i-1}$. Here
    we assume $\vecT_0=\vecT$ and $\vecT_k=\vecT'$.
\end{definition}

For $m,n\in \N$, let $\LRAM{I}{m}{n}$ denote the language consisting of tuples $(\vecT,\vec{o},\vecT')$ with $\vecT,\vecT'\in \RAM{I}{n}$ and $\vec{o}\in (\RAMOp{I})^m$
such that $\vec{o}$ is consistent with $\vecT,\vecT'$.
Next, we formalize the folklore technique of checking correctness of RAM operations
using {\em address-ordered transcripts}.

\subsection{Consistency Check via Transcripts}\label{subsec:transcripts}
A {\em transcript} is time-stamped sequence of operations executed on a RAM.
More formally, given a RAM $\vecT=(\vec{a},\vec{v})\in \RAM{I}{n}$,
operation sequence $\vec{o}=(o_1,\ldots,o_m)$ with $o_i=(\bar{\op_i},\bar{a}_i,\bar{v}_i)\in \RAMOp{I}$ and RAM $\vecT'=(\vec{a}', \vec{v}')\in \RAM{I}{n}$,
the {\em time ordered transcript} for the tuple $(\vecT,\vec{o},\vecT')$ is given by the table $\tr$ with $k=2n+m$ rows and four columns
$\tr = (\vec{t},\vec{\op},\vec{A},\vec{V})$  defined as follows: (i) $\vec{t}=\setind_{k}=(1,\ldots,k)$,
(ii) $\vec{\op}=0^n||(\bar{\op}_1,\ldots,\bar{\op}_m)||0^n$,
(iii) $\vec{A}=\vec{a}||(\bar{a}_1,\ldots,\bar{a}_m)||\vec{a}'$ and
(iv) $\vec{V}=\vec{v}||(\bar{v}_1,\ldots,\bar{v}_m)||\vec{v}'$. The $i^{th}$ row of the table $\tr$ is
$(t_i,\op_i,A_i,V_i)$ for $i\in [k]$. The first $n$ records in $\tr$ correspond to the contents of $\vecT$,
the next $m$ records correspond to the operations in $\vec{o}$ and final $n$
records correspond to contents of $\vecT'$. The timestamp column $\vec{t}$ is added to order operations with the same index.
Notationally, we write $\tr=\TOT(\vecT,\vec{o},\vecT')$.

We call a transcript $\tr=(\vec{t},\vec{\op},\vec{A},\vec{V})$ to be {\em address ordered} if $A_i\leq A_{i+1}$ for $i\in [k-1]$ and
$t_i < t_{i+1}$ whenever $A_i=A_{i+1}$. For a transcript $\tr=(\vec{t},\vec{\op},\vec{A},\vec{V})$ with $k$ records and a
permutation $\sigma:[k]\rightarrow [k]$, we use $\sigma(\tr)$ to denote the transcript
$(\sigma(\vec{t}),\sigma(\vec{\op}),\sigma(\vec{A}),\sigma(\vec{V}))$
obtained by permuting the records of $\tr$ according to the permutation $\sigma$.
An address ordered transcript for tuple $(\vecT,\vec{o},\vecT')$ is defined as $\tr^\ast=\sigma(\tr)$ where $\tr=\TOT(\vecT,\vec{o},\vecT')$ and $\sigma$ is
a permutation such that $\tr^\ast$ is address ordered. We denote it by $\tr^\ast=\AOT(\vecT,\vec{o},\vecT')$.
We say that an address ordered transcript $\tr=(\vec{t},\vec{\op},\vec{A},\vec{V})$ satisfies {\em load-store correctness}
if for all pairs of consecutive records $(t_i,\op_i,A_i,V_i)$ and $(t_{i+1},\op_{i+1},A_{i+1},V_{i+1})$ we have $V_{i+1}=V_i$
whenever $\op_{i+1}=0$ (load operation) and
$A_i=A_{i+1}$, i.e, a load operation does not change the value at an index.
We formally state the folklore technique for enforcing memory consistency in our setting.
\begin{lemma}\label{lem:consistency-check}
Let $\F$ be a finite field, $m,n\in \N$ be positive integers and $\setind\subseteq \F$. Then $(\vecT,\vec{o},\vecT')\in \LRAM{I}{n}{m}$ if and only if
    the address ordered transcript $\tr^\ast=\AOT(\vecT,\vec{o},\vecT')$ satisfies load-store correctness.
\end{lemma}

The consistency check in Lemma \ref{lem:consistency-check} can be encoded as an arithmetic circuit of size $\wt{O}(m+n)$, thus
yielding an argument of knowledge for the language $\LRAM{I}{n}{m}$ with prover complexity quasi-linear in
$m+n$. For completeness, we present a self-contained argument of knowledge for
$\LRAM{I}{m}{m}$ ($m=n$) based on the ``polynomial protocol'' framework defined in ~\cite{Gabizon2019PLONKPO}.



\begin{comment}
\begin{enumerate}[leftmargin=2em]
\item $\prover\rightarrow\verifier$: For the transcript $\tr^\ast=(\vec{t^\ast},\vec{\op^\ast},\vec{A^\ast},\vec{V^\ast})$,
the prover computes encoding polynomials $\wt{\tr^\ast}=({t^\ast}(X),\op^\ast(X), A^\ast(X),V^\ast(X))$. Next, the prover
computes sets $I_1$ and $I_2$ as in Section \ref{subsec:encoded-relations}, and then computes polynomials
$Z_b(X)=\prod_{i\in I_b}(X-\omega^i)$ for $b=1,2$.
The prover also interpolates polynomials $\delta^\ast_A(X)$ and $\delta^\ast_T(X)$ satisfying
$\delta^\ast_A(\omega^i) = A^\ast_{i+1} - A^\ast_i$ for $i\in I_1$ and $\delta^\ast_T(\omega^i)=t^\ast_{i+1}-t^\ast_i$ for $i\not\in I_1$.
The prover sends all the above polynomials to the verifier.
\item $\verifier$ computes: The verifier checks relations (C1)-(C6) in Lemma \ref{lem:addr-ordered-transcript}. The range constraints
in (C7) can be verified using polynomial protocols in sub-vector lookup arguments such as Caulk+ ~\cite{EPRINT:PosKat22}, CQ
~\cite{EPRINT:EagFioGab22}.
\end{enumerate}


\subsection{Succinct Argument for Verifiable RAM}\label{subsec:succ-args}
Polynomial protocols can be compiled into succinct arguments of knowledge using an extractable polynomial commitment scheme. At a
high level the compilation involves the prover sending ``commitments'' to the message polynomials, whereas the verifier checks
the polynomial identities probabilistically by requesting evaluation proofs of the polynomials at random points.
We complile the polynomial protocol for RAM to a succinct argument of knowledge in the Algebraic Group Model (AGM)
 using $\kzg$ as the extractable polynomial commitment scheme. We also leverage homomorphism of $\kzg$ scheme
to avoid sending commitments and evaluations for polynomials which are linear combinations of previously committed polynomials.
We now present an argument for verifiable RAM, which combines the polynomial protocols in this section, and instantiates
them using $\kzg$ commitments. We make standard optimisations of batching several identity checks into one where applicable.
Let $\srs=\{\gone{\tau^i},\gtwo{\tau^i}\}_{i=0}^d$
denote the $\kzg$ setup parameters for degree $d\geq k$ for the bilinear group $\mathsf{BG}=(\Gone,\Gtwo,\gone{1},\gtwo{1},e)$. Let
$\F=\F_p$ be the finite field where $p$ is the order of the groups.


\begin{figure}[t]
\begin{subfigure}{\linewidth}
\centering
{\footnotesize
\begin{enumerate}[leftmargin=1em]
    \item $\prover\rightarrow\verifier$: Send polynomial commitments $\gone{a(X)}$, $\gone{v(X)}$, $\gone{a'(X)}$, $\gone{v'(X)}$,
    $\gone{\bar{\op}(X)}$, $\gone{\bar{a}(X)}, \gone{\bar{v}(X)}$, $\gone{\op(X)}$, $\gone{A(X)}$, $\gone{V(X)}$, $\gone{t^\ast(X)}$,
    $\gone{\op^\ast(X)}$, $\gone{A^\ast(X)}$, $\gone{V^\ast(X)}$, $\gone{\delta_A^\ast(X)}$, $\gone{\delta_T^\ast(X)}$.
    \item $\verifier\rightarrow\prover$: Send $\alpha,\beta,\gamma\gets \F$.
    \item $\prover$ computes: Compute sets $I_1=\{i\in [k]: A^\ast_i\neq A^\ast_{i+1}\}$, and $I_2=[k]\setminus I_1$. Next, compute
    polynomials:
    \begin{align*}
        &Z_1(X)=\prod_{i\in I_1}(X-\omega^i),\quad Z_2(X)=\prod_{i\in I_2}(X-\omega^i), \\
        &H(X) = \begin{bmatrix} 1 & \gamma & \gamma^2 \end{bmatrix}
        \begin{bmatrix}
            a(X) & v(X) & 0 \\
            a'(X) & v'(X) & 0 \\
            \bar{a}(X) & \bar{v}(X) & \bar{\op}(X)
        \end{bmatrix}
        \begin{bmatrix}
        1 \\
        \beta \\
        \beta^2
        \end{bmatrix}, \\
        &G(X) = A(X) + \beta V(X) + \beta^2 \op(X), \\
        &Q(X) = \big(H(X^3) - G(X) - \gamma G(\omega^m X) - \gamma^2 G(\omega^{2m} X)\big)/Z(X), \\
        &f(X) = G(X) + \beta^3 t(X),\, g(X) = A^{\ast}(X) + \beta V^{\ast}(X) + \beta^2 \op^{\ast}(X) + \beta^3 t^{\ast}(X), \\
        &z(X) = \sum_{i=1}^k \lambda_i(X)\prod_{j=1}^{i-1} (\alpha - g(\omega^j))/(\alpha - f(\omega^j)), \\
        &Q_1(X) = \frac{1}{\mathbb{Z}_\setH(X)}\Big( (\alpha - g(X))z(\omega X) - (\alpha - f(X))z(X) \\
        &\qquad\qquad + \beta Z_2(X)(A^\ast(\omega X) - A^\ast(X) - \delta_A^\ast(X))\Big), \\
        &Q_2(X) = \frac{1}{Z_2(X)}\Big((A^\ast(\omega X) - A^\ast(X))
         + \beta(t^\ast(\omega X) - t^\ast(X)-\delta_T^\ast(X)) \\
        &\qquad\qquad + \beta^2 (\op^\ast(X) - 1)(V^\ast(\omega X) - V^\ast(X))\Big)
    \end{align*}
    \item $\prover\rightarrow \verifier$: Send $\gone{Z_1(X)}$, $\gone{Z_2(X)}$, $\gone{Q(X)}$, $\gone{z(X)}$,
    $\gone{Q(X)}$, $\gone{Q_1(X)}$, $\gone{Q_2(X)}$.
    \item $\verifier\rightarrow\prover$: $s\gets \F$.
    \item $\prover\rightarrow\verifier$: Send evaluations $\val{Z_1}{s}=Z_1(s)$, $\val{Z_2}{s}=Z_2(s)$,
    $\val{G}{s}=G(s)$, $\val{Q}{s}=Q(s)$, $\val{Z}{s}=Z(s)$, $\val{A^\ast}{s}=A^\ast(s)$, $\val{V^\ast}{s}=V^\ast(s)$,
    $\val{\op^\ast}{s}=\op^\ast(s)$, $\val{t}{s}=t(s)$, $\val{t^\ast}{s}=t^\ast(s)$, $\val{z}{s}=z(s)$, $\val{Q_1}{s}=Q_1(s)$,
    $\val{Q_2}{s}=Q_2(s)$, $\val{H}{s^3}=H(s^3)$, $\val{G}{\omega^m s}=G(\omega^m s)$, $\val{\delta^\ast_A}{s}=\delta_A^\ast(s)$,
    $\val{\delta_T^\ast}{s}=\delta_T^\ast(s)$
    $\val{G}{\omega^{2m} s}=G(\omega^{2m} s)$,
    $\val{A^\ast}{\omega s}=A^\ast(\omega s)$, $\val{V^\ast}{\omega s}=V^\ast(s)$, $\val{t^\ast}{\omega s}=t^\ast(\omega s)$,
    $\val{z}{\omega s}=z(\omega s)$, $\val{A^\ast}{\omega}=A^\ast(\omega)$.

    \item $\verifier\rightarrow\prover$: Send $r \gets \F$.
    \item $\prover$ computes: Compute batched $\kzg$ proofs:
    \begin{alignat*}{3}
        &P_1 && &\quad=\quad  && &Z_1+r Z_2 + r^2 G + r^3 Q + r^4 Z + r^5 A^{\ast} + r^6 V^{\ast} + r^7 \op^{\ast} \\
        & && & && &\quad  + r^8 t + r^9 t^{\ast} + r^{10} z + r^{11} Q_1 + r^{12} Q_2 + r^{13} \delta_A^{\ast} + r^{14} \delta_T^{\ast} \\
        &P_2 && &\quad=\quad && &A^{\ast} + r V^{\ast} + r^2 z \\
        &\Pi_s && &\quad=\quad && &\kzgprove(P_1, s) \\
        &\Pi_{\omega s} && &\quad=\quad && &\kzgprove(P_2, \omega s) \\
        &\Pi_{s^3} && &\quad=\quad && &\kzgprove(H, s^3) \\
        &\Pi_{\{\omega^m s, \omega^{2m} s\}} && &\quad=\quad && &\kzgprove(G, (\omega^m s, \omega^{2m} s))
    \end{alignat*}
    \item $\prover\rightarrow \verifier$: Send $\Pi_s,\Pi_{\omega s}, \Pi_{s^3},\Pi_{\{\omega^m s, \omega^{2m} s\}}$ and
    $\val{H}{s^3},\val{G}{\omega^m s}, \val{G}{\omega^{2m} s}$.

    \item $\verifier$ computes: Compute commitments and evaluations for linear combinations.
    \begin{align*}
        \gone{G(X)} &= \gone{A(X)}+\beta\gone{V(X)}+\beta^2\gone{\op(X)}, \\
        \gone{H(X)} &= \begin{bmatrix} 1 & \gamma & \gamma^2 \end{bmatrix}
        \begin{bmatrix}
            \gone{a(X)} & \gone{v(X)} & \gone{0} \\
            \gone{a'(X)} & \gone{v'(X)} & \gone{0} \\
            \gone{\bar{a}(X)} & \gone{\bar{v}(X)} & \gone{\bar{\op}(X)}
        \end{bmatrix}
        \begin{bmatrix}
            1 \\
            \beta \\
            \beta^2
        \end{bmatrix}, \\
        \gone{P_1(X)} &= \gone{Z_1(X)}+r \gone{Z_2(X)} + r^2 \gone{G(X)} + r^3 \gone{Q(X)} + r^4 \gone{Z(X)} + r^5 \gone{A^{\ast}(X)} \\
         &\qquad + r^6 \gone{V^{\ast}(X)} + r^7 \gone{\op^{\ast}(X)} + r^8 \gone{t(X)} + r^9 \gone{t^{\ast}(X)} + r^{10} \gone{z(X)} \\
         &\qquad + r^{11} \gone{Q_1(X)} + r^{12} \gone{Q_2(X)} + r^{13} \gone{\delta_A^{\ast}(X)} + r^{14} \gone{\delta_T^{\ast}(X)}, \\
        \gone{P_2(X)} &= \gone{A^\ast(X)} + r\gone{V^\ast(X)} + r^2\gone{z(X)},\\
        \val{P_1}{s} &= \val{Z_1}{s}+r \val{Z_2}{s} + r^2 \val{G}{s} + r^3 \val{Q}{s} + r^4 \val{Z}{s} + r^5 \val{A^{\ast}}{s} \\
         &\qquad + r^6 \val{V^{\ast}}{s} + r^7 \val{\op^{\ast}}{s} + r^8 \val{t}{s} + r^9 \val{t^{\ast}}{s} + r^{10} \val{z}{s} \\
         &\qquad + r^{11} \val{Q_1}{s} + r^{12} \val{Q_2}{s} + r^{13} \val{\delta_A^{\ast}}{s} + r^{14} \val{\delta_T^{\ast}}{s}, \\
        \val{P_2}{\omega s} &= \val{A^\ast}{s} + r\val{V^\ast}{s} + r^2\val{z}{s}. \\
        \val{g}{s} &= \val{A^\ast}{s} + \beta \val{V^\ast}{s} + \beta^2 \val{\op^\ast}{s} + \beta^3 \val{t^\ast}{s}, \\
        \val{f}{s} &= \val{G}{s} + \beta^3 \val{t}{s}.
    \end{align*}

    \item $\verifier$ checks polynomial identities at $s$:
    \begin{alignat*}{3}
        &\val{Q}{s}\val{Z}{s} && &\quad \stackrel{?}{=}\quad  && &\val{H}{s^3} - \val{G}{s} - \gamma \val{G}{\omega^m s} - \gamma^2 \val{G}{\omega^{2m} s} \\
        &\val{Q_1}{s}(s^k-1) && &\quad \stackrel{?}{=}\quad && &(\alpha - \val{g}{s})\val{z}{\omega s} - (\alpha - \val{f}{s})\val{z}{s}
        +\beta \val{Z_2}{s}(\val{A^\ast}{\omega s}-\val{A^\ast}{s}-\val{\delta_A^\ast}{s}). \\
        &\val{Q_2}{s}\val{Z_2}{s} && &\quad \stackrel{?}{=}\quad && &(\val{A^\ast}{\omega s} - \val{A^\ast}{s}) + \beta (\val{t^\ast}{s} - \val{t}{s} - \val{\delta}{s}) \\
        & && & && &\qquad + \beta^2 (\val{\op^\ast}{s} - 1)(\val{V^\ast}{\omega s}-\val{V^\ast}{s}) \\
        &\val{Z_1}{s}\val{Z_2}{s} && &\quad \stackrel{?}{=}\quad && &s^k - 1 \\
    \end{alignat*}
    \item $\verifier$ checks evaluation proofs:
    \begin{alignat*}{3}
        &\kzgverify(\gone{P_1(X)},\val{P_1}{s},s, \Pi_s) && &\quad\stackrel{?}{=}\quad && &1 \\
        &\kzgverify(\gone{P_2(X)},\val{P_2}{\omega s},\Pi_{\omega s}) && &\quad\stackrel{?}{=}\quad && &1 \\
        &\kzgverify(\gone{H(X)},\val{H}{s^3},\Pi_{s^3}) && &\quad\stackrel{?}{=}\quad && &1 \\
        &\kzgverify(\gone{G(X)},(\val{G}{\omega^m s},\val{G}{\omega^{2m} s}),
        (\omega^m s, \omega^{2m} s), \Pi_{\{\omega^m s, \omega^{2m} s\}}) && &\quad\stackrel{?}{=}\quad && &1 \\
        &\kzgverify(\gone{A^\ast(X)}, \val{A^\ast}{\omega}, \omega, \Pi_\omega) && &\quad\stackrel{?}{=}\quad && &1
    \end{alignat*}
    \item $\verifier$ enforces range checks: Invoke the lookup argument $\varPi_{\mathsf{lookup}}$ to check that
    polynomials $\delta_A^\ast(X)$, $\delta_T^\ast(X)$ and $t^\ast(X)$ encode values in $[0,N]$.
    \item $\verifier$ outputs: The verifier outputs accept (1) if all the preceeding checks pass, else it outputs reject (0).
\end{enumerate}
}
\end{subfigure}
\caption{Argument of Knowledge for Verifiable RAM}
\label{fig:aok-vram}
\end{figure}
\end{comment}

\begin{comment}
Let $\LLconcat$ denote the language consisting of polynomial tuples $(a(X)$, $b(X)$, $c(X)$, $v(X))$ satisfying the identities
in Lemma ~\ref{lem:vec-concatenation}. To check membership of $(T,\vec{o},T',\tr)$ in the language $\LTr$, let $\wt{T}=(a(X),v(X))$, $\wt{T'}=(a'(X),v'(X))$,
$\wt{O}=(\bar{\op}(X),\bar{a}(X),\bar{v}(X))$ and $\wt{\tr}=(t(X),o(X),A(X),V(X))$ be the polynomial
encodings of $T,T',\vec{o}$ and $\tr$ respectively.
From the construction of time ordered transcript $\tr$ outlined
in Section \ref{subsec:transcripts}, we see that the equivalent constraints on the encodings are given by:
\begin{align*}\label{eq:tot-poly-constraints}
t(X) &= \sum_{i=1}^k i\lambda_i(X) \quad (= \mathsf{Enc}(\setind_k)) \\
(0, \bar{\op}(X), 0, o(X)) &\in \LLconcat \\
(a(X), \bar{a}(X), a'(X), A(X)) &\in \LLconcat \\
(v(X), \bar{v}(X), v'(X), V(X)) &\in \LLconcat
\end{align*}

We can probabilistically combine the different polynomial identities into one and obtain the following:
\begin{lemma}
    Let the polynomial $Z(X)=\prod_{i\in [m]}(X-\omega^i)$ and $\rho=\omega^m$ be as before.
    Then, we have $(T,\vec{o},T',\tr)\in \LTr$ if and only if the encoding polynomials as defined above satisfy
    the identity $G_\gamma(X) = 0  \text{ mod } Z(X)$ and $\evalH{t}=(1,\ldots,k)$ with overwhelming
    probability over the choice of $\gamma\gets \F$. Here the polynomial $G_\gamma(X)$ is defined as:
    \begin{multline*}
        G_\gamma(X) = o(X) + \gamma(\bar{\op}(X^3) - o(\rho X)) + \gamma^2 o(\rho^2 X) \\
        + \gamma^3(a(X^3) - A(X)) + \gamma^4(\bar{a}(X^3) - A(\rho X)) + \gamma^5(a'(X^3) - A(\rho^2 X)) \\
        + \gamma^6(v(X^3) - V(X)) + \gamma^7(\bar{v}(X^3) - V(\rho X)) + \gamma^8(v'(X^3) - V(\rho^2 X))
    \end{multline*}
\end{lemma}

Next, we consider the language $\Lperm$ consisting of pairs $(\tr, \tr^\ast)$ of $k$-length transcripts such
that $\tr^\ast=\sigma(\tr)$ for some permutation $\sigma:[k]\rightarrow [k]$. We now describe constraints
to check the same using polynomial encodings of the transcripts. Let encodings $\wt{\tr},\wt{\tr}^\ast$ be
given by polynomials as below:
\begin{align*}\label{eq:tr-encodings}
\wt{\tr} &= (t(X),\op(X),A(X),V(X)) \\
\wt{\tr}^\ast &= (t^\ast(X), \op^\ast(X), A^\ast(X), V^\ast(X))
\end{align*}
We need to show that for some $\sigma:[k]\rightarrow [k]$, $\evalH{p^\ast}=\sigma(\evalH{p})$ for $p\in \{t,\op,A,V\}$.
Again, for $\gamma\gets \F$, with overwhelming probability it is equivalent to establishing that polynomial
$f^\ast(X)=t^\ast(X)+\gamma \op^\ast(X) + \gamma^2 A^\ast(X) + \gamma^3 V^\ast(X)$ encodes a permutation of
vector encoded by $f(X)=t(X)+\gamma \op(X) + \gamma^2 A(X) + \gamma^3 V(X)$. We recall the following variation
of the grand product argument to show that two polynomials encode vectors which are permutations of each other.
\end{comment}












    \bibliographystyle{plain}
    \bibliography{crypto_crossref, main}
\end{document}
