%! Author = nitin
%! Date = 03/11/23

% Preamble
\documentclass[11pt]{article}

% Packages
\usepackage{llncsconf}
\usepackage{times}
\usepackage{amsmath,amsthm,amsfonts,enumitem,tikz, subcaption, mdframed}
\usepackage{amssymb}
\usepackage{tcolorbox}
\usepackage{comment}
\usepackage{graphicx}
\usepackage{caption}
\usepackage{subcaption}
\usetikzlibrary{arrows,arrows.meta,shapes.arrows, calc, positioning,matrix}


\newtheorem{theorem}{Theorem}[section]
\newtheorem{lemma}{Lemma}[section]
\newtheorem{definition}{Definition}[section]


%! Author = nitin
%! Date = 03/11/23

\newcommand{\npol}{\mathsf{NP}}
\newcommand{\enc}[2]{\mathsf{Enc}_{{}_{#2}}(\vec{#1})}
\newcommand{\vecA}{\vec{a}}
\newcommand{\vecB}{\vec{b}}
\newcommand{\vecC}{\vec{c}}
\newcommand{\vecD}{\vec{d}}
\newcommand{\vecP}{\vec{p}}
\newcommand{\vecQ}{\vec{q}}
\newcommand{\vecS}{\vec{s}}
\newcommand{\vecT}{\vec{T}}
\newcommand{\vecX}{\vec{x}}
\newcommand{\vecY}{\vec{y}}
\newcommand{\vecZ}{\vec{z}}
\newcommand{\vecW}{\vec{w}}
\newcommand{\vecI}{\vec{I}}
\newcommand{\vecind}{\overrightarrow{\ind}}
\newcommand{\vecop}{\overrightarrow{\op}}
\newcommand{\vecval}{\overrightarrow{\val}}
\newcommand{\setind}{\mathcal{I}}
\newcommand{\ins}{\mathsf{ins}}
\newcommand{\ind}{\mathsf{ind}}
\newcommand{\op}{\mathsf{op}}
\newcommand{\vecins}{\overrightarrow{\ins}}

%%% Base table and cache table
\newcommand{\vecTbase}{\vec{T}_{\mathsf{b}}}
\newcommand{\vecTcache}{\vec{T}_{\mathsf{ch}}}
\newcommand{\Tbasepoly}[1]{T_{\mathsf{b}}({#1})}
\newcommand{\Tcachepoly}[1]{T_{\mathsf{ch}}({#1})}
\newcommand{\Qbasepoly}[1]{Q_{2,\mathsf{b}}({#1})}
\newcommand{\Qcachepoly}[1]{Q_{2,\mathsf{ch}}({#1})}
\newcommand{\Qcachepolyone}[1]{Q'_{2,\mathsf{ch}}({#1})}
\newcommand{\Qcachepolytwo}[1]{Q''_{2,\mathsf{ch}}({#1})}
\newcommand{\Zinv}[1]{(\mathbb{Z}_#1)^{\times}}
\newcommand{\Zn}{\mathbb{Z}_N}
\newcommand{\setH}{\mathbb{K}}
\newcommand{\setV}{\mathbb{V}}
\newcommand{\setN}{\mathbb{H}}
\newcommand{\vpolyN}{Z_\mathbb{H}}
\newcommand{\mroots}{\mathbb{V}}
\newcommand{\nroots}{\mathbb{H}}
\newcommand{\iroots}{\mathbb{H}_I}
\newcommand{\BG}{\mathsf{BG}}
\newcommand{\F}{\mathbb{F}}
\newcommand{\Fp}{{\mathbb{F}_p}}
\newcommand{\Zp}{{\mathbb{Z}_p}}
\newcommand{\Zq}{{\mathbb{Z}_q}}
\newcommand{\G}{\mathbb{G}}
\newcommand{\N}{\mathbb{N}}
\newcommand{\V}{\mathsf{V}}

\newcommand{\Gone}{\mathbb{G}_1}
\newcommand{\Gtwo}{\mathbb{G}_2}
\newcommand{\GT}{\mathbb{G}_T}
\newcommand{\srs}{\mathsf{srs}}
\newcommand{\cm}{\mathsf{cm}}
\newcommand{\wt}[1]{\widetilde{#1}}
\newcommand{\outp}{\rightarrow}
\newcommand{\pp}{\mathsf{pp}}
\newcommand{\pk}{\ensuremath{\mathsf{pk}}}
\newcommand{\bal}{\ensuremath{\mathsf{bal}}}
\newcommand{\txno}{\ensuremath{\mathsf{txno}}}
\newcommand{\load}{\ensuremath{\mathsf{LD}}}
\newcommand{\store}{\ensuremath{\mathsf{STR}}}
\newcommand{\RAM}[2]{\ensuremath{\mathsf{RAM}}_{\mathcal{#1}, {#2}}}
\newcommand{\RAMOp}[1]{\ensuremath{\mathcal{O}_{{}_\mathcal{#1}}}}
\newcommand{\Rupd}[1]{\ensuremath{\mathsf{Upd}_{\mathcal{#1}}}}
\newcommand{\LRAM}[3]{\ensuremath{\mathsf{LRAM}_{\mathcal{#1},#2,#3}}}
\newcommand{\LTr}{\ensuremath{\mathcal{L}_{\mathsf{Tr}}}}
\newcommand{\tr}{\ensuremath{\mathsf{tr}}}
\newcommand{\TOT}{\ensuremath{\mathsf{TimeTr}}}
\newcommand{\AOT}{\ensuremath{\mathsf{AddrTr}}}
\newcommand{\evalH}[1]{\ensuremath{\vec{#1}_{|_\setH}}}
\newcommand{\evalV}[1]{\ensuremath{\vec{#1}_{|_\setV}}}
\newcommand{\LLconcat}{\ensuremath{\tilde{\mathcal{L}}_{\mathsf{concat}}}}
\newcommand{\Lperm}{\ensuremath{\mathcal{L}_{\mathsf{perm}}}}
\newcommand{\RRAM}[1]{\ensuremath{R}_{\mathsf{ram},#1}}
\newcommand{\RLOOK}{\ensuremath{\mathcal{R}^{\mathsf{lookup}}_{\srs,N,m}}}
\newcommand{\CRAM}{\ensuremath{\mathcal{R}^{\mathsf{ram}}_{\srs,N,m}}}
\newcommand{\CLRAM}{\ensuremath{\mathcal{R}^{\mathsf{LRAM}}_{\srs,m}}}
\newcommand{\val}[2]{\ensuremath{\mathsf{V}_{{#2},{#1}}}}
\newcommand{\uniq}[1]{\ensuremath{\mathsf{uniq}(\vec{{#1}})}}
% group encodings
\newcommand{\gone}[1]{\ensuremath{\left[{#1}\right]_1}}
\newcommand{\gtwo}[1]{\ensuremath{\left[{#1}\right]_2}}

% provers, verifiers, adversaries
\newcommand{\prover}{\ensuremath{\mathcal{P}}}
\newcommand{\verifier}{\ensuremath{\mathcal{V}}}
\newcommand{\Adv}{\ensuremath{\mathcal{A}}}
\newcommand{\kzg}{\ensuremath{\mathsf{KZG}}}
\newcommand{\kzgprove}{\ensuremath{\mathsf{KZG.Prove}}}
\newcommand{\kzgverify}{\ensuremath{\mathsf{KZG.Verify}}}
\newcommand{\kzgcommit}{\ensuremath{\mathsf{KZG.Commit}}}



\title{Faster Rollups from Improved Persistent RAM}

% Document
\begin{document}
    \maketitle
    \begin{abstract}
        Persistent RAM (random access memory) is a useful primitive in verifiable computation.
        The persistent RAM primitive allows one to efficiently verify that execution of a program $\mathcal{P}$
        with read/write access to memory state $\mathsf{st}_0$ results in the final state $\mathsf{st}_1$.
        The efficiency requires that verification is cheaper than naive execution of $\mathcal{P}$
        (typically poly-logarithmic) and only requires access to {\em succinct} digests (e.g. Merkle hash) of the state(s).
        Persistence refers to the fact that state digests allow proving updates to the state across several computations
        in an incremental fashion.

        Current approaches realizing persistent RAM in verifiable computation generally use Merkle-trees
        or cryptographic accumulators based on unknown-order groups (RSA, class-groups) to compress the state. The
        Merkle-tree based approaches suffer from high concrete costs due to poor batching properties, leading to
        verification of several hash function evaluations inside VC circuits. While the accumulator based approaches
        offer significant improvement for bigger batch sizes, they still incur computations linear in the size of the
        accumulated set to compute witnesses for inclusion proofs.

        In this paper, we present a polynomial protocol that realizes an improved persistent RAM primitive, building
        on the techniques used in recent lookup arguments with table size independent prover complexity.
        The RAM state and updates are suitably represented as polynomials, and proving correctness of the state update
        reduces to proving identities over related polynomials.
        These can be compiled into succinct non-interactive arguments
        of knowledge (SNARK) using a polynomial commitment scheme like \textsf{KZG} under the Algebraic Group Model (AGM) and
        the Random Oracle model. Amortized cost of proving correctness of a batch of $m$ updates for RAM of size $N$ is
        $O(m^2)$ field operations and a multi-exponentiation of size $O(\sqrt{mN})$, thus achieving sublinear $(\sqrt{N})$
        dependence on the RAM size. We also show that our approach outperforms existing approaches in practice and
        experimentally validate superior performance of our primitive when applied to blockchain rollups.

    \end{abstract}

    \section{Introduction}\label{sec:introduction}
    General purpose {\em Succinct Non-interactive Arguments of Knowledge} (SNARKs) enable one to generate succinct
    proofs of membership of a statement in an $\npol$ relation expressed as an arithmetic circuit. These proofs are
    extremely cheap to verify, which makes them useful for {\em Verifiable Computation} (VC), where a low-powered
    client (e.g. mobile phone), can outsource an expensive computation to an untrusted server, and later
    verify the correctness of the results at a minimal cost.
    However,
    arithmetic circuit based representations are inefficient in expressing relations involving the result of
    a program execution on memory/state, which frequently arise in the context of verifiable computation.
    These include scenarios such as proving the correctness of a query execution against a database,
    proving the correctness of inference
    from a decision tree, or proving correct update of table of account balances when a batch of transactions (e.g. transfers)
    are applied to it.
    In aforementioned examples, database tables, decision trees and accounts table are naturally
    modelled as addressable memory and one needs to prove that it is accessed/updated in accordance with the correct execution
    of the computation. Accordingly, there is a rich and expanding body of work to efficiently model the abstraction of
    addressable memory in verifiable computations. While complete acknowledgement of this vast body of work is beyond
    the scope of this paper (a fairly recent survey in \cite{WB15} is a good starting point) we summarise key approaches towards modelling RAM, and more pertinently persistent RAM
    in verifiable computations in the next section.


    The final example above is especially relevant in the context of recent efforts to scale blockchains
    by moving expensive computation off the blockchain to the so-called {\em layer two} or L2 chains. The blockchain
    only needs to verify the succinct proofs attesting to the correctness of the off-chain computation. This approach
    is popularly called {\em rollup}, as it allows verifying the result of several transactions modifying the L2 state (
    rolled up transactions), as part of one transaction verified on the main chain. This simultaneously helps the scalability
    and lowers the cost (e.g. gas fees) per transaction (due to succinct verification). The specific rollup scenario
    is discussed later in the paper, to illustrate the efficacy of our approach to achieving much more efficient rollups.

    \subsection{Our work}\label{subsec:ourwork}
    We present a persistent RAM construction, which advances the efforts
    towards achieving {\em verifiable outsourcing of state update} such as in ~\cite{EPRINT:BFRSBW13}
    and more recently in ~\cite{USENIX:OWWB20, CCS:CFHKKO22}. The most popular approaches to succinctly represent
    state involve accumulators based on Merkle-trees ~\cite{C:Merkle87}, or ones based on groups of unknown order
    (e.g. RSA, class-groups) ~\cite{C:CamLys02,C:BonBunFis19,USENIX:OWWB20, CCS:CFHKKO22}.
    The updates to the state are effected by insertions or deletions in the  accumulated set.
    By contrast, in this work we
    model the state as an addressable memory (RAM) described by vector $\vecT$, which stores a values $v_i$ at addresses $i$.
    We denote this as $\vecT[\,i\,]=v_i$. The RAM supports two operations (i) {\em indexed lookups} ($v_i := \vecT[\,i\,]$),
    denoted by the tuple $(\load, i, v_i)$ and (ii) {\em indexed update} ($\vecT[\,i\,] := v_i$), denoted
    by the tuple $(\store, i, v_i)$. We think of addresses $i\in [0,N]$ for $N\in \mathbb{Z}$ while the
    values $v_i\in \F$ for some finite field $\F$. We encode both the RAM and operations as polynomials, and
    use appropriate polynomial commitment schemes to obtain succinct commitments (digests) to them.
    In this paper, we do not require commitments to be {\em hiding}, as our focus is on succinctness.
    We consider privacy as an orthogonal goal, one we believe is easily achievable via small adaptations to our construction.

    \noindent{\bf Application to Blockchain Rollups}: We consider the application of persistent RAM primitive to a common rollup scenario.
    Our application is a simplified adaptation of rollup protocols such as ZkSync ~\cite{ZkSync}.
    We assume a layer two (L2) chain \textsf{DemoChain}, which issues its clients \textsf{DemoCoins}. The clients have accounts on
    L2, and they transfer \textsf{DemoCoins} to each other using L2 transactions. Such a system will also have mechanism to transfer
    funds to and from main-chain (e.g. Ethereum) accounts, but we
    ignore those details here.
    Each client is assigned a unique identifier $i$, and associated account information  is maintained as
    a tuple $(\pk_i,\bal_i,\txno_i)$ which refer respectively to the public key for verifying client's signature on transactions,
    account balance (number of coins owned by the client) and total number of transactions made from the account (to prevent replay attacks).
    The L2 state thus consists of the above information for all accounts. We model an L2 chain supporting upto $N$ accounts as a RAM $\vecT$
    of size $N$, where $\vecT[\,i\,]=(\pk_i,\bal_i,\txno_i)$ denotes client $i$'s account information. A transfer transaction of amount $v$
    from client $i$ to client $j$ involves two load operations, and two store operations on the RAM (for reading and updating the referenced
    accounts). These transactions are submitted by clients directly to a designated L2 chain operator \textsf{DemoOperator}, who batches $m$
    such transactions and updates the RAM state accordingly. The operator maintains the entire state off-chain and locally computes the updates for
    each batch of transactions. Only the succinct digest of the state (polynomial commitments) is stored on the main chain, and the proof of
    validity of the state update is verified by a main-chain transaction. In later sections, we provide more details on the application and the
    performance achieved when instantiated using our primitive.

    \subsection{Techniques}\label{subsec:techniques}
    \noindent{\bf Improved lookup from updatable tables}: Starting point for our work are the recent lookup arguments which prove that a vector of size $m$ appears as
    a sub-vector in a large fixed vector (table) of size $N$ with succinct proof sizes and verification, but most notably
    ensuring that prover runs in time sub-linear in the size of the table ($N$). The pioneering work ~\cite{CCS:ZBKMNS22}
    obtained prover complexity of $O(m^2+m\log N)$, which was improved in subsequent works to $O(m^2)$ ~\cite{EPRINT:PosKat22},
    $O(m\log^2 m)$ ~\cite{EPRINT:ZGKMR22}, and $O(m\log m)$ ~\cite{EPRINT:EagFioGab22}. However, the sub-linear prover
    complexity requires table-dependent $O(N\log N)$ pre-processing and $O(N)$ storage. This table-dependent
    pre-processing implies that while
    the aforementioned lookup arguments can be used to obtain efficient ROM (read only memory) semantics
    \footnote{The protocols for sub-vector relation in ~\cite{CCS:ZBKMNS22, EPRINT:ZGKMR22} also implicitly support indexed lookup semantics},
    they cannot be used as is for RAM (which supports update operations). Moreover, an update involving even a single
    index renders the entire $O(N)$ pre-processing unusable for further lookups, thus necessitating entire $O(N\log N)$
     re-computation. An important contribution of this work is modified pre-processing and prover for lookup scheme
    in ~\cite{EPRINT:PosKat22} which continues to ensure efficient lookups if updates to the table are small since the pre-processing.
    Informally, we achieve the following:

    \begin{theorem}[Informal]\label{thm:pre-process}
        There exists a deterministic $O(N\log N)$ time algorithm $\textsf{Params}(\vecT)\outp \pp_T$
        which on input $\vecT\in \F^N$, outputs parameters $\pp_T$ of size $O(N)$ such
        that: Given $\pp_T$, vectors $\vecT'\in \F^N$, $\vec{t}\in \F^m$ with $\vec{t}$ being a sub-vector of $\vecT'$
        an argument of knowledge for the same can be computed in time
        $O((m+\delta)\log^2 (m+\delta) + m^2)$ where $\delta=\Delta(\vecT, \vecT')$
    is the number of positions where $\vecT$ and $\vecT'$ differ.
    \end{theorem}
    For the construction in ~\cite{EPRINT:EagFioGab22}, a corresponding version of the above theorem holds with the prover complexity
    of $O((m+\delta)\log^2(m+\delta))$. We also remark that above theorem also holds for the indexed lookups that we briefly discussed before.\smallskip

    \noindent{\bf Polynomial Protocol for RAM}: Most of the efficient implementations of RAM in verifiable computation ~\cite{C:BCGTV13, NDSS:WSRBW15, SP:ZGKPP18}
    rely on the {\em address-ordered transcript} to check that a sequence of $\load$s and $\store$s are {\em consistent} with some initial state
    of the RAM. Using the tuple $(t,\op,\ind, v)$ to denote a RAM instruction with $t$ being the execution {\em timestamp}, $\op\in \{\load,\store\}$ being
    the operation type, $\ind$ being the index referenced and $v$ denoting the value read/stored, a time ordered execution transcript $\mathsf{tr} = (\ins_1,\ldots,\ins_m)$ is a sequence
    of instructions where $\ins_i=(t_i,\op_i,\ind_i, v_i)$ and $t_i < t_{i+1}$ for all $1\leq i<m$. The transcript $\mathsf{tr}$ is said to be consistent if
    for every $\load$ instruction $\ins=(t,\load,\ind,v)$, the value $v$ is the same as that for the most recent $\store$ instruction, which references the same index
    as $\ins$. This consistency can be efficiently checked by permuting the transcript $\mathsf{tr}$ to produce the sequence
    $\mathsf{tr}^\ast=(\ins_1^\ast,\ldots,\ins_m^\ast)$ with $\ins_i^\ast=(t_i^\ast,\op_i^\ast,\ind_i^\ast,v_i^\ast)$ which is sorted by index, with
    timestamp used to break the ties, i.e, $\ind_i^\ast\leq \ind_{i+1}^\ast$ and whenever $\ind_i^\ast=\ind_{i+1}^\ast$, we have $t_i^\ast < t_{i+1}^\ast$.
    We call the resulting transcript $\mathsf{tr}^\ast$ as address ordered transcript. On such a transcript, the consistency is enforced simply by
    checking that each $\load$ instruction involves the same value $v$ as the previous instruction, provided the referenced indices are the same.
    The above approach is easily extended to verify that result of executing a sequence of operations $\{(\op_i,\ind_i,v_i)\}_{i=1}^m$ on
    RAM $\vecT_0\in \F^n$ is $\vecT_1\in F^n$. In this case, we define a transcript $\mathsf{tr}=(\ins_1,\ldots,\ins_{2n+m})$ where the
    first $n$ instructions $\ins_i,i\in [n]$ are defined as $\ins_i=(i,\store,i,\vecT_0[\,i\,])$, the next $m$ instructions are defined
    as $\ins_{n+i},i\in [m]$ as $\ins_{n+i}=(n+i, \op_i, \ind_i, v_i)$ and the last $n$ instructions $\ins_{n+m+i}, i\in [n]$ as
    $\ins_{n+m+i}=(n+m+i,\load,i,\vecT_1[\,i\,])$. The consistency of the transcript $\mathsf{tr}$ is checked via address ordering permutation as
    described before. Intuitively, the first $n$ instructions in $\mathsf{tr}$ copy the initial contents of the RAM, the next $m$ instructions essentially
    capture the sequence of RAM operations executed on the initial RAM, while the final $n$ instructions read out the entire contents of the RAM which
    is expected to match the contents of $\vecT_1[\,i\,]$. We illustrate this approach in Figure \ref{fig:permutated-transcripts}. One of the contributions of this work is a polynomial protocol that proves that RAM state
    $\vecT_1\in \F^n$ is obtained from RAM state $\vecT_0\in \F^n$ as a result of sequence of operations $\vec{\op}=\{(\op_i,\ind_i,v_i)\}_{i=1}^m$, where each
    of the vectors $\vecT_0,\vecT_1$ and $\vec{\op}$ are represented via polynomials. Informally, we show:
    \begin{theorem}[informal]\label{thm:pp-for-ram-informal}
    For $m,n\in \mathbb{N}$, there exists a polynomial protocol to prove that sequence of operations $\vec{\op}=\{(\op_i,\ind_i,v_i)\}_{i=1}^m$ updates a
    RAM state $\vecT_0\in \F^n$ to RAM state $\vecT_1\in \F^n$ with the prover-complexity of $\tilde{O}(m+n)$.
    \end{theorem}

    \noindent{\bf Improved RAM from Lookup Arguments}: The approach of the previous paragraph incurs a linear cost in
    the size of the RAM, even though the number of operations $m$ could be much smaller $m\ll n$. To circumvent the
    dependence on RAM size $n$, we use the efficient lookup arguments to isolate the part of RAM which is involved
    in the operation sequence $\vec{\op}=\{(\op_i,\ind_i,v_i)\}_{i=1}^m$.
    Concretely, from the vectors $\vecT_0,\vecT_1\in \F^n$ denoting the initial and final RAM states, we claim
    vectors $\vec{t}_0, \vec{t}_1\in \F^m$ such that $\vec{t}_j[\,i\,] = \vecT_j[\,\ind_i\,]$ for all $i\in [m]$ and
    $j=0,1$. Intuitively (modulo certain technicalities), we need to show that (i) the operation sequence $\vec{\op}$
    transforms $\vec{t}_0$ to $\vec{t}_1$ and (ii) the RAMs $\vecT_0$ and $\vecT_1$ are identical at indices not
    involved in $\vec{\op}$, i.e $\vecT_0[\,i\,]=\vecT_1[\,i\,]$ for $i\not\in \{\ind_i: i\in [m]\}$. In later
    sections, we show that the latter check can be accomplished with $\tilde{O}(m)$ prover complexity. The former
    check involves an adaptation of folklore approach using address ordered transcript (we consider additional subtleties such
    as the possibility of multiple rows of $\vec{t}_0$ and $\vec{t}_1$ referring to the same index of their parent RAMs $\vecT_0$ and
    $\vecT_1$ respectively).
    In later sections, we show that this is efficiently handled by using the inherited indices in constructing the transcript showing
    consistency of $\vec{t}_0$ and $\vec{t}_1$. Since the lookup protocol is $O(m^2)$ and the memory consistency check involves transcript
    of size $O(m)$, the overall complexity of this step is $O(m^2)$. Figure ~\ref{fig:subtable-lookup} illustrates the above idea on
    a concrete example.
    %and ~\ref{fig:subtable-consistency}
    %illustrate the overall idea for the first check with a concrete example.

    \noindent{\bf Incrementally Verifiable RAM with Deferred Pre-processing}: The approach in the previous paragraph yields an argument for RAM with
    table-independent prover-complexity ($O(m^2)$), {\em provided} the prover has access to the table-dependent pre-processing
    $\pp_{T,0}$ for the table $\vecT_0$. We also require lookup from table $\vecT_1$, but it differs from $\vecT_0$ in at most $m$
    positions, so by Theorem \ref{thm:pp-for-ram-informal}, this lookup also requires $O(m^2)$ prover time. Next, let's suppose we apply an
    update of $m$ operations to obtain $\vecT_2$ from $\vecT_1$, and incrementally obtain $\vecT_3,\ldots,\vecT_k$. By Theorem ~\ref{thm:pp-for-ram-informal},
    the update $\vecT_k\rightarrow \vecT_{k+1}$ costs $O(m^2 + (mk)\log^2(mk))$. If we choose to re-compute the pre-processing parameters after $k$
    batches (of $m$ updates each), the average cost per batch is $O(N\log N/k + m^2 + mk\log^2(mk))$. Setting $k\approx \sqrt{N/m}$ yields the
    average cost of $m$ updates as $O(m^2+\sqrt{mN})$, which scales sub-linearly with the size of the RAM. While the preceding analysis considers
    worst case, in specific applications (such as account transactions, where few accounts contribute a large volume of transactions), it may be
    possible to further defer the computation of pre-processing parameters. Thus we have:

    \begin{theorem}[Informal]\label{thm:inc-ver-ram-informal}
    Given $m,N\in \mathbb{N}$, there exists a polynomial protocol which incrementally proves updates of batch size $m$ on RAM of size $N$
    with amortized prover complexity of $O(m^2 + \sqrt{mN})$.
    \end{theorem}

    \begin{figure}[t]
    \begin{subfigure}{\textwidth}
    \centering
    \includegraphics[width=0.8\textwidth]{example-lookup}
    \caption{Isolating subtables using lookup argument}
    \label{fig:subtable-consistency}
    \end{subfigure}
    \begin{subfigure}{\textwidth}
    \centering
        \includegraphics[height=0.3\textheight]{Address-ordered}
        \caption{Checking memory consistency using address ordering}
        \label{fig:permuted-transcripts}
    \end{subfigure}
    \end{figure}

    %
    \tikzset{
        table/.style={
            matrix of nodes,
            row sep=-\pgflinewidth,
            column sep=-\pgflinewidth,
            nodes={
                rectangle,
                draw=black,
                align=center
            },
            minimum height=1.5em,
            text depth=0.5ex,
            text height=1.5ex,
            nodes in empty cells,
%%
            every even row/.style={
                nodes={fill=gray!20}
            },
            column 1/.style={
                nodes={text width=2em}
            },
            row 1/.style={
                nodes={
                    fill=white,
                    text=black,
                    font=\bfseries
                },
            }
        }
    }

    \tikzset{
        MyArrow/.style={
            single arrow, draw=black, minimum width=10mm, minimum height=30mm, inner sep=0mm, single arrow head extend=1mm, double arrow head extend=1mm
        }
    }

    \begin{figure}[t]
        \centering
        %\subcaptionbox{Lookup to get sub-tables \label{fig:lookup}}[\textwidth]{
        \resizebox{0.8\textwidth}{0.3\textheight}{
        \begin{tikzpicture}
            %\draw[step=1.0,black,thin] (-5,-5) grid (5,5);
        \matrix (initial) [table,text width=2em, width=2cm] at (-5,1)
        {
            Idx & Val\\
            1  & 10 \\
            2  & 20 \\
            3  & 15 \\
            4  & 25 \\
            5  & 40 \\
            6 & 50 \\
            7 & 60 \\
        };

        \matrix (ops1) [table,text width=2em, width=2cm] at (0,0)
            {
            Op & Idx & Val \\
            \store  & 7 & 40 \\
            \load  & 1 & 10 \\
            \store & 3 & 20 \\
            \load  & 3 & 20 \\
        };


        \matrix (final) [table,text width=2em, width=2cm] at (5,1)
            {
            Idx & Val\\
            1  & 10 \\
            2  & 20 \\
            3  & 20 \\
            4  & 25 \\
            5  & 40 \\
            6 & 50 \\
            7 & 40 \\
        };


        % -------------------------------- Subtables ------------------------------------- %
        \matrix (stinitial) [table,text width=2em, width=2cm] at (-5,-5)
            {
            Idx & Val\\
            7 & 60 \\
            1 & 10 \\
            3 & 15 \\
            3 & 15 \\
        };

        \matrix (ops2) [table,text width=2em, width=2cm] at (0,-5)
            {
            Op & Idx & Val \\
            \store  & 7 & 40 \\
            \load  & 1 & 10 \\
            \store & 3 & 20 \\
            \load  & 3 & 20 \\
        };


        \matrix (stfinal) [table,text width=2em, width=2cm] at (5,-5)
            {
            Idx & Val\\
            7 & 40 \\
            1 & 10 \\
            3 & 20 \\
            3 & 20 \\
        };

            \draw (-5,3.6) node[above]  {\bf RAM $\vecT_0$};
            \draw (0, 1.6) node[above] {\bf Operations};
            \draw (5,3.6) node[above] {\bf RAM $\vecT_1$};
            \draw (-5, -3.4) node[above] {\bf $\vec{t}_0$};
            \draw (5, -3.4) node[above] {\bf $\vec{t}_1$};

            % lookup arrows
            \draw [-Latex, thick] (-4.25,-1.75) -- (-4.25,-3.25);
            \draw [-Latex, thick] (4.25, -1.75) -- (4.25, -3.25);
            \draw [fill=grey!30!white] (-2,-2) rectangle (2, -3) node[midway] {Lookup on indices 7,1,3,3};

            \draw (-2, -2.5) -- (-4.25, -2.5) (2, -2.5) -- (4.25, -2.5);
            \draw [-Latex, dashed] (-4, 0) -- (-1.6,0);
            \draw [-Latex, dashed] (1.6,0) -- (4,0);
            \draw [-Latex, dashed] (-4, -5) -- (-1.6,-5);
            \draw [-Latex, dashed] (1.6,-5) -- (4,-5);
        \end{tikzpicture}
        }
        \caption{Using lookup arguments to reduce the consistency check between large RAMs $\vecT_0$
        and $\vecT_1$ to that between small tables $\vec{t}_0$ and $\vec{t}_1$. It is seperately shown that RAMs $\vecT_0$ and
        $\vecT_1$ have identical values outside the involved indices. Figure ~\ref{fig:subtable-consistency} illustrates address
        ordered transcript to check consistency of $\vec{t}_0$ and $\vec{t}_1$.}
        \label{fig:subtable-lookup}
        \end{figure}

        \begin{figure}[htbp]
            \centering
            \resizebox{0.8\textwidth}{0.4\textheight}{
        \begin{tikzpicture}
        % ------------------------------------------------------------------------------------------- %
            %\draw[step=1.0,black,thin] (-5,-5) grid (5,5);
        \matrix (trts) [table,text width=2em] at (-4,0)
            {
            Ts & Op & Idx & Val\\
            1 & \store & 7 & 60 \\
            2 & \store & 1 & 10 \\
            3 & \store & 3 & 15 \\
            4 & \store & 3 & 15 \\
            5 & \store  & 7 & 40 \\
            6 & \load  & 1 & 10 \\
            7 & \store & 3 & 20 \\
            8 & \load  & 3 & 20 \\
            9 & \load & 7 & 40 \\
            10 & \load & 1 & 10 \\
            11 & \load & 3 & 20 \\
            12 & \load & 3 & 20 \\
        };

        \matrix (trts) [table,text width=2em, width=2cm] at (4,0)
            {
            Ts & Op & Idx & Val\\
            2 & \store & 1 & 10 \\
            6 & \load & 1 & 10 \\
            10 & \load & 1 & 10 \\
            3 & \store & 3 & 15 \\
            4 & \store  & 3 & 15 \\
            7 & \store  & 3 & 20 \\
            8 & \load & 3 & 20 \\
            11 & \load  & 3 & 20 \\
            12 & \load & 3 & 20 \\
            1 & \store & 7 & 60 \\
            5 & \store & 7 & 40 \\
            9 & \load & 7 & 40 \\
        };

            \path (-2.1,0) -- (1.9, 0)  node[midway, MyArrow, text width=3cm, align=center] {Address-sorting permutation};
            \draw (-4, 4.3) node[above] {\bf \small Time ordered transcript ($\mathsf{tr}$)};
            \draw (4, 4.3) node[above] {\bf \small Address ordered transcript ($\mathsf{tr}^\ast$)};

        \end{tikzpicture}
            }
            \caption{Address ordered transcript for checking consistency of tables $\vec{t}_0$ and $\vec{t}_1$
                with respect to the operation sequence in Figure \ref{fig:subtable-lookup}. On the transcript
            $\mathsf{tr}^\ast$, one checks that the load instructions return the same value as the preceeding
            instruction if their indices are the same.}

        \label{fig:subtable-consistency}
    \end{figure}

    \noindent{\bf Our Contributions}: (Summarise later)

    \section{Related Work}\label{sec:rel-work}

    \section{Model for RAM}\label{sec:model-for-ram}
    In this section, we formalize the notions of RAM, RAM operations and
execution transcripts etc., and subsequently introduce relevant relations over them.
We have earlier used a vector $\vecT\in \F^n$ to model a RAM of size $n$, where the $i^{th}$ entry $\vecT[\,i\,]$
implicitly corresponds to index (address) $i$. Here, we will consider a generalisation that will be useful later.
We will allow a RAM to explicitly associate a value $v\in \F$ to an index $a$ from an {\em index space}
$\cal{I}\subseteq \F$, such that the association is unambiguous.
\begin{definition}[RAM]\label{defn:RAM}
    Given $n\in \N$, finite field $\F$ and a set $\cal{I}\subseteq \F$, a RAM of size $n$ over indices $\cal{I}$
    is a tuple $T=(\vec{a},\vec{v})\in \mathcal{I}^n\times \F^n$ such that $\forall\, i,j\in [n]$  $v_i=v_j$ whenever $a_i=a_j$.
    We think of $T$ as a table with vectors $\vec{a}$ and $\vec{v}$ denoting its columns. The set of all such
    tables will be denoted by $\RAM{I}{n}$.
    %\begin{align*}
    %        T[\,a\,] = \begin{cases}
    %                   v \text{ if } \, \exists i\in [n] \text{ s.t. } (a,v) = (a_i,v_i), \\
     %                  \bot \text{ otherwise }
    %                \end{cases}
    %\end{align*}
\end{definition}

For a table $T=(\vec{a},\vec{v})\in \RAM{I}{n}$, we refer to tuples $(a_i,v_i)$, $i\in [n]$ as records of the table $T$.
We use the access notation $v=T[a]$ to mean that $(a,v)$ is a record of $T$.
Note that our definition allows identical records to be repeated. When we refer to a vector $\vecT\in \F^n$ as a RAM, we implicitly
assume $T=(\setind_n,\vecT)$ where $\setind_n=(1,2,\ldots,n)$ and we have $\vecT[i]=t_i$.
For a RAM $T\in \RAM{I}{n}$, a RAM operation is a three tuple $(\op,a,v)$ with $\op\in \{0,1\}$,
$a\in \setind$ and $v\in \F$. An operation with $\op=0$ is called a {\em load} operation which denotes reading a value $v$
mapped to index $a$ in the RAM. Similarly, an operation with $\op=1$ is called a {\em store} operation,
which denotes associating the value $v$ with index $a$ in the RAM.
We use $\RAMOp{I}$ to denote the set of all RAM operations with index set $\setind$.
%We say than an operation
%$(\op,a,v)\in \{0,1\}\times \setind\times \F$ is {\em admissible} with respect to RAM $T\in \RAM{I}{n}$ if
%$T[\,a\,]=v$ whenever $\op=0$ (i.e, the operation loads a value which agrees with the value in the RAM at the specified index).

\subsection{Correctness of RAM Update}\label{subsec:ram-update}
The versatility of the RAM primitive stems from its updatability. While a load operation leaves the RAM unchanged, the store operation
updates the value in the RAM associated with the referenced index. We model the update via the function
$\Rupd{I}$ which takes RAM $T\in \RAM{I}{n}$, operation
$o=(\op,a,v)\in \RAMOp{I}$ as inputs and returns an updated RAM $T'\in \RAM{I}{n}$.
The updated RAM $T'=\Rupd{I}(T,o)$ satisfies
$T'=T$ if $\op=0$ while for $\op=1$ it satisfies $T'[\,a\,]=v$  and $T'[\,x\,]=T[\,x\,]$ for $x\neq a$. The central problem
in verifiable RAM protocols is to establish that a sequence of operations $\vec{o}=(o_1,\ldots,o_m)$ are correct with
respect to the initial RAM state $T$ and the final RAM state $T'$. This involves ensuring
that all load operations read the value which is consistent with updates to the RAM as a result of preceding
store operations, and that $T'$ is the final state. We say that an operation $o=(\op,a,v)$ is {\em load-consistent}
with respect to RAM $T$ if $v=T[a]$ whenever $o$ is a load operation (store operations are vacuously defined to be load-consistent).
We formally define the notion of consistency below:

\begin{definition}[Consistent Operations]\label{defn:consistent-operations}
    Let $n\in \N$ and $T,T'\in \RAM{I}{n}$ for some index set $\setind$. We say that a sequence of operations
    $\vec{o}=(o_1,\ldots,o_k)\in \RAMOp{I}^k$ over $\setind$ is {\em consistent} with RAM states
    $T,T'$ if for all $i\in [k]$, $T_{i}=\Rupd{I}(T_{i-1},o_i)$ and operation $o_i$ is load-consistent with respect to $T_{i-1}$. Here
    we assume $T_0=T$ and $T_k=T'$.
\end{definition}

For $m,n\in \N$, let $\LRAM{I}{m}{n}$ denote the language consisting of tuples $(T,\vec{o},T')$ with $T,T'\in \RAM{I}{n}$ and $\vec{o}\in (\RAMOp{I})^m$
such that $\vec{o}$ is consistent with $T,T'$.
Next, we formalize the folklore technique of checking correctness of RAM operations
using {\em address-ordered transcripts}.

\subsection{Consistency Check via Transcripts}\label{subsec:transcripts}
A {\em transcript} is time-stamped sequence of operations executed on a RAM.
More formally, given a RAM $T=(\vec{a},\vec{v})\in \RAM{I}{n}$,
operation sequence $\vec{o}=(o_1,\ldots,o_m)$ with $o_i=(\bar{\op_i},\bar{a}_i,\bar{v}_i)\in \RAMOp{I}$ and RAM $T'=(\vec{a}', \vec{v}')\in \RAM{I}{n}$,
the {\em time ordered transcript} for the tuple $(T,\vec{o},T')$ is given by the table $\tr$ with $k=2n+m$ rows and four columns
$\tr = (\vec{t},\vec{\op},\vec{A},\vec{V})$  defined as follows: (i) $\vec{t}=\setind_{k}=(1,\ldots,k)$,
(ii) $\vec{\op}=0^n||(\bar{\op}_1,\ldots,\bar{\op}_m)||0^n$,
(iii) $\vec{A}=\vec{a}||(\bar{a}_1,\ldots,\bar{a}_m)||\vec{a}'$ and
(iv) $\vec{V}=\vec{v}||(\bar{v}_1,\ldots,\bar{v}_m)||\vec{v}'$. The $i^{th}$ row of the table $\tr$ is
$(t_i,\op_i,A_i,V_i)$ for $i\in [k]$. The first $n$ records in $\tr$ correspond to the contents of $T$,
the next $m$ records correspond to the operations in $\vec{o}$ and final $n$
records correspond to contents of $T'$. The timestamp column $\vec{t}$ is added to order operations with the same index.
Notationally, we write $\tr=\TOT(T_0,\vec{\op},T_1)$.

We call a transcript $\tr=(\vec{t},\vec{\op},\vec{A},\vec{V})$ to be {\em address ordered} if $A_i\leq A_{i+1}$ for $i\in [k-1]$ and
$t_i < t_{i+1}$ whenever $A_i=A_{i+1}$. For a transcript $\tr=(\vec{t},\vec{\op},\vec{A},\vec{V})$ with $k$ records and a
permutation $\sigma:[k]\rightarrow [k]$, we use $\sigma(\tr)$ to denote the transcript
$(\sigma(\vec{t}),\sigma(\vec{\op}),\sigma(\vec{A}),\sigma(\vec{V}))$
obtained by permuting the records of $\tr$ according to the permutation $\sigma$.
An address ordered transcript for tuple $(T,\vec{o},T')$ is defined as $\tr^\ast=\sigma(\tr)$ where $\tr=\TOT(T,\vec{o},T')$ and $\sigma$ is
a permutation such that $\tr^\ast$ is an address ordered. We denote it by $\tr^\ast=\AOT(T,\vec{o},T')$.
We say that an address ordered transcript $\tr=(\vec{t},\vec{\op},\vec{A},\vec{V})$ satisfies {\em load-store correctness}
if for all pairs of consecutive records $(t_i,\op_i,A_i,V_i)$ and $(t_{i+1},\op_{i+1},A_{i+1},V_{i+1})$ we have $V_{i+1}=V_i$
whenever $\op_{i+1}=0$ (load operation) and
$A_i=A_{i+1}$, i.e, a load operation does not change the value at an index.
We formally state the folklore technique for enforcing memory consistency in our setting.
\begin{lemma}\label{lem:consistency-check}
Let $\F$ be a finite field, $m,n\in \N$ be positive integers and $\setind\subseteq \F$. Then $(T,\vec{o},T')\in \LRAM{I}{n}{m}$ if and only if
    the address ordered transcript $\tr^\ast=\AOT(T,\vec{o},T')$ satisfies load-store correctness.
\end{lemma}

\subsection{Polynomial Encoding}\label{subsec:poly-encoding}
The aim of this section is to encode artefacts such as RAM state, operations and transcripts as polynomials, and
translate checking memory consistency (equivalently, checking membership in $\LRAM{I}{n}{m}$) to checking polynomial identities
over the encoded polynomials. For
simplicity and its usefulness later, we consider the case $m=n$, and accordingly check the membership in the language $\LRAM{I}{m}{m}$.
Let $k=3m$ and let $\omega$ be the $k^{th}$ root of unity in $\F$.
Let $\nu=\omega^3$, and thus $\nu$ is an $m^{th}$ root of unity in $\F$ (We assume, these roots exist in $\F$).
We define sets $\setH$ and $\setV$ below consisting of $k^{th}$ and $m^{th}$ roots of unity respectively.
\begin{gather}\label{eq:interpolation-sets}
    \setH = \{\omega,\ldots,\omega^k\},\quad \setV = \{\nu,\ldots,\nu^m\}
\end{gather}
Let $\{\lambda_i(X)\}_{i=1}^k$ be lagrange basis polynomials
for the set $\setH$ and $\{\tau_i(X)\}_{i=1}^m$ be the lagrange polynomials for the set $\setV$ satisfying
$\lambda_i(\omega^j)=\delta_{ij}$ for $i,j\in [k]$ and $\tau_i(\nu^j)=\delta_{ij}$ for $i,j\in [m]$.
We use $\setH$ to encode a vector $\vec{f}=(f_1,\ldots,f_k)$ of size $k$ as the polynomial $f(X)\in \F_{<k}[X]$ such that $f(\omega^i)=f_i$ for $i\in [k]$.
Similarly, we use $\setV$ to encode a vector
$\vec{g}=(g_1,\ldots,g_m)$ of size $m$ as the polynomial $g(X)\in F_{<m}[X]$ such that $g(\nu^i)=g_i$ for $i\in [m]$. We use
the notation $\enc{f}{\setH}$ and $\enc{g}{\setV}$ to denote polynomial encodings of vectors $\vec{f}\in \F^k$ and $\vec{g}\in \F^m$
respectively.
In the other direction, for a polynomial $f(X)\in \F[X]$,
we use $\vec{f}_{|_\setH}$ and $\vec{f}_{|_\setV}$ to denote the vectors $(f(\omega^1),\ldots,f(\omega^k))$ and $(f(\nu^1),\ldots,f(\nu^m))$ respectively.
The encoding of vectors as polynomials can be succinctly described using lagrange basis polynomials.
Then we have $\enc{f}{\setH}=\sum_{i=1}^k f_i\lambda_i(X)$ and $\enc{g}{\setV}=\sum_{i=1}^m g_i\tau_i(X)$.

We extend the encoding of vectors to encode RAM, operations and transcripts.
For a RAM $T=(\vec{a},\vec{v})\in \RAM{I}{m}$, we define its encoding
$\wt{T}=(a(X),v(X))$ where $a(X),v(X)\in \F_{<m}[X]$ encode vectors $\vec{a}, \vec{v}$ respectively.
Given an operation sequence
$\vec{o}=(o_1,\ldots,o_m)$ with $o_i=(\bar{\op}_i,\bar{a}_i,\bar{v}_i)$ we encode $\vec{o}$ as $\wt{O}=(\bar{\op}(X),\bar{a}(X),\bar{v}(X))$
where $\bar{\op}(X)$ encodes the
vector $\vec{\op}=(\bar{\op}_1,\ldots,\bar{\op}_m)$, $\bar{a}(X)$ encodes the vector $(\bar{a}_1,\ldots,\bar{a}_m)$ and $\bar{v}(X)$
encodes the vector $(\bar{v}_1,\ldots,\bar{v}_m)$.
Finally, a transcript $\tr=(\vec{t},\vec{\op},\vec{A},\vec{V})$ for tuples $(T,\vec{o},T')$ where $T,T'$ are RAMs of size $m$, and $\vec{o}$ is an
operation sequence of size $m$ is encoded as $\wt{\tr}$ $=$ $(t(X)$, $\op(X)$, $A(X)$, $V(X))$ where the polynomials
encode the respective vectors in $\F^k$.

\subsection{Relations over Polynomial Encodings}\label{subsec:encoded-relations}
In this section, we describe polynomial checks for two important relations we need in subsequent sections, viz,
(i) checking concatenation of vectors and (ii) checking monotonicity and load-store consistency of a transcript. Let
$k,m$ be integers with $k=3m$ as in previous section. The lemma below specifies the polynomial checks for verifying
that vector $\vec{v}\in \F^k$ is concatenation of vectors $\vec{a},\vec{b},\vec{c}$ in $\F^m$.

\begin{lemma}\label{lem:vec-concatenation}
Let $\vec{a},\vec{b},\vec{c}\in \F^m$  and $v\in \F^k$ be vectors encoded by polynomials
$a(X),b(X),c(X)$ and $v(X)$ respectively. Then,
{\small
\begin{alignat}{3}
    &a(X^3) - v(X) && = 0  &&  &\quad \text{ mod $Z(X)$ } \tag{A1}\label{eq:A1}\\
    &b(X^3) - v(\omega^m X) && = 0 &&  &\quad \text{ mod $Z(X)$ } \tag{A2}\label{eq:A2}\\
    &c(X^3) - v(\omega^{2m} X) && = 0 && &\quad \text{ mod $Z(X)$ } \tag{A3}\label{eq:A3}
\end{alignat}
}
for $Z(X)=\prod_{i=1}^m (X-\omega^i)$ if and only if $\vec{v}=\vec{a}||\vec{b}||\vec{c}$.
\end{lemma}
\begin{proof}
    Assume that the polynomial identities hold. Substituting $X=\omega^i$ for $i\in [m]$ in above equations implies
    for $i\in [m]$: $a_i=v_i$ (Eq \eqref{eq:A1}), $b_i=v_{m+i}$ (Eq \eqref{eq:A2}) and $c_i=v_{2m+i}$ (Eq \eqref{eq:A3}),
    which together imply $\vec{v}=\vec{a}||\vec{b}||\vec{c}$. Converse follows by observing that $\vec{v}=\vec{a}||\vec{b}||\vec{c}$
    implies that $v(X) = a(X^3)$, $v(\omega^m X)=b(X^3)$ and $v(\omega^{2m} X)=c(X^3)$ holds for all $X=\omega^i, i\in [m]$.
    Thus, the equalities hold modulo the polynomial $Z(X)$ as defined above.
\end{proof}

Next, we specify polynomial checks on the encoding of a transcript to ensure it satisfies address-ordering and load-store consistency.
Let $\tr=(\vec{t},\vec{\op},\vec{A},\vec{V})$ be a transcript encoded as
$\wt{\tr}=(t(X),\op(X),A(X),V(X))$. Recall that we need to check two conditions on $\tr$, viz (i) {\em monotonicty}:
the transcript is sorted by address and timestamp respectively, i.e, $A_i\leq A_{i+1}$ for all $i < k$ and
$t_i < t_{i+1}$ when $A_i=A_{i+1}$, (ii) {\em load-store consistency}: whenever $\op_{i+1}=0$ and $A_i=A_{i+1}$,
we have $V_i=V_{i+1}$. To do so, we exhibit disjoint sets $I_1,I_2$ with $I_1\uplus I_2=[k]$ such that: (i) for all
$i\in I_1$, $A_i < A_{i+1}$, (ii) for all $i\in I_2$, $(A_i = A_{i+1})\wedge (t_i < t_{i+1})$ and (iii) for all $i\in I_2$,
$(\op_i=1)\vee (V_i = V_{i+1})$. We note that the conditions on the sets $I_1$ and $I_2$ ensures monotonicity.
Moreover it can be seen that load-store consistency requirements are satisfied for all $i\in I_1$ (as $A_i\neq A_{i+1}$).
Similarly,load-store consistency also holds for all $i\in I_2$.
It remains to exhibit the sets and show that they satisfy the above invariants using polynomials, as in the following
lemma:
\begin{lemma}\label{lem:addr-ordered-transcript}
Let $\wt{\tr}$ be a polynomial encoding of transcript $\tr$ of size $k$, given by polynomials $t(X),\op(X),A(X)$ and $V(X)$.
Then for $kN<|\F|$, $\tr$ is address ordered and satisfies load-store consistency if and only if there exist polynomials
 $Z_1,Z_2,\delta_T,\delta_A$
such that the following hold:
{\small
\begin{alignat}{2}\label{eq:load-store-consitency-constraints}
&& & Z_2(X)(A(\omega X) - A(X) - \delta_A(X)) = 0 && & \text{ mod } \mathbb{Z}_\setH(X) \tag{C1} \label{eq:C1} \\
&& & A(\omega X) - A(X) = 0 && & \text{ mod } Z_2(X) \tag{C2} \label{eq:C2} \\
&& & t(\omega X) - t(X) - \delta_T(X) = 0 && & \text{ mod } Z_2(X) \tag{C3} \label{eq:C3} \\
&& & (\op(X) - 1)(V(\omega X) - V(X)) = 0 && & \text{ mod } Z_2(X) \tag{C4} \label{eq:C4} \\
&& & Z_1(X)\cdot Z_2(X) = \mathbb{Z}_\setH(X) && &\quad  \tag{C5} \label{eq:C5} \\
&& & 1\leq A(\omega) < N && &\quad \tag{C6} \label{eq:C6} \\
&& & 1\leq t(\omega^i) < N, 1\leq \delta_A(\omega^i)<N,\ 1\leq \delta_T(\omega^i)<N && &\text{ for } i\in [k] \tag{C7} \label{eq:C7}
\end{alignat}
}
\end{lemma}

\section{Succinct Argument for Verifiable RAM}\label{sec:poly-proto-ram}
The polynomial encodings in the previous section can be used to polynomial protocol for
checking the membership in the language $\LRAM{I}{m}{m}$ for $m\in \N$. The polynomial protocol can be subsequently
be compiled into a succinct argument using an extractable polynomial commitment scheme.
In this section, we use $\kzg$ polynomial commitment scheme to obtain a succinct argument for checking membership in $\LRAM{I}{m}{m}$
in the Algebraic Group Model (AGM).
At a high level, to prove $(\vec{T},\vec{o},\vec{T'})\in \LRAM{I}{m}{m}$, the prover
constructs time ordered transcript $\tr$ and then permutes it to obtain the address sorted transcript $\tr^\ast$.
It then sends the polynomial encodings of $\vec{T},\vec{o},\vec{T'},\tr$ and $\tr^\ast$ to the verifier, who verifies that:
\begin{enumerate}[leftmargin=1em]
\item The time ordered transcript is correctly constructed, i.e, $\tr=\TOT(\vec{T},\vec{o},\vec{T'})$.
\item The transcript $\tr^\ast$ is a permutation of the transcript $\tr$, i.e, $\tr^\ast=\sigma(\tr)$ for some permutation $\sigma$ of $[k]$.
\item The transcript $\tr^\ast$ is address ordered and satisfies load-store consistency.
\end{enumerate}
In fact, we consider memebership in the above language under commitments. Let $\srs$ denote a $\kzg$ setup over a bilinear group, with
prime order groups $\Gone, \Gtwo$ and $\GT$. We canonically commit to RAM, operation sequences and transcripts by committing to their
polynomial encodings. Commitment of an encoding represented as tuple of polynomials is simply the tuple consisting of commitments of the component
polynomials. We now define
the relation $\CLRAM$ below, and present a succinct argument for the same.
\begin{definition}\label{defn:committed-vram}
Let $\CLRAM$ consist of tuples $((c_T, c_o, c_T'), (T, \vec{o},T'))$ where $c_T$ $=$ $\kzgcommit(\srs, \wt{T})$,
$c_T'$ $=$ $\kzgcommit(\srs, \wt{T'})$,
$c_o$ $=$ $\kzgcommit(\srs,\wt{O})$ commit to $T$, $T'$ and $\vec{o}$ with $(T,\vec{o},T')\in \LRAM{I}{m}{m}$.
\end{definition}
\noindent In the above definition we have $c_T=(c_a,c_v)$ where $c_a$ and $c_v$ are $\kzg$ commitments to polynomials $a(X)$ and $v(X)$ in
the encoding $\wt{T}=(a(X), v(X))$. Similarly we parse $c_T'=(c_a', c_v')$ and $c_o=(\bar{c}_\op, \bar{c}_a, \bar{c}_v)$ (see Section
\ref{subsec:poly-encoding} for polynomial encodings).
For proving relation \ref{defn:committed-vram}, prover's input consists of initial RAM state $T=(\vec{a},\vec{v})$,
final RAM state $T'=(\vec{a'},\vec{v'})$, operation sequence $\vec{o}=(o_1,\ldots,o_m)$ with $o_i=(\bar{\op}_i,\bar{a}_i,\bar{v}_i)$,
time-ordered transcript $\tr=(\vec{t},\vec{\op},\vec{A},\vec{V})$ and address-ordered transcript $\tr^\ast=(\vec{t^\ast},\vec{\op^\ast},
\vec{A^\ast},\vec{V^\ast})$ obtained from $\tr$ using a permutation $\sigma:[k]\rightarrow [k]$. Verifier's input consists of the
commitments $c_T, c_o$ and $c_T'$ as described above.

The prover starts the protocol by sending commitments $c_\tr$ and $c^\ast_\tr$ to the transcripts $\tr$ and $\tr^\ast$ respectively.
To show that $\tr$ is correctly formed, the prover needs to prove the concatenations:
(i) $\vec{\op}=0^m||(\bar{\op}_1,\ldots,\bar{\op}_m)||0^m$, (ii) $\vec{A}=\vec{a}||(\bar{a}_1,\ldots,\bar{a}_m)||\vec{a'}$
and (iii) $\vec{V}=\vec{v}||(\bar{v}_1,\ldots,\bar{v}_m)||\vec{v'}$. Note that the time-stamp column $\vec{t}$ is implicitly assumed
to be $(1,\ldots,k)$.
The verifier checks the concatenations using Lemma \ref{lem:vec-concatenation}.
It uses a random challenge $\beta$ to reduce the three concatenations to one concatenation, and uses another challenge $\gamma$
to reduce the three polynomial checks in Lemma \ref{lem:vec-concatenation} to a single check.
The complete polynomial protocol is detailed in Figure \ref{fig:time-ordered-transcript}.

%\begin{tcolorbox}
\begin{figure}[htbp]
\centering
\begin{mdframed}
{\footnotesize
\noindent{\bf Common Input}: Commitments $c_T=(c_a,c_v)$, $c_o=(\bar{c}_\op, \bar{c}_a, \bar{c}_v)$, $c_T'=(c_a', c_v')$
and $c_\tr=(c_t, c_\op, c_A, c_V)$ to $T,o,T'$ and $\tr$ respectively. Commitment $\gone{Z}$ to the polynomial
$Z(X)=\prod_{i=1}^m (X-\omega^i)$.
\begin{enumerate}[leftmargin=2em]
\item $\verifier\rightarrow\prover$: Verifier sends $\beta,\gamma\gets \F$.
\item $\prover$ computes:
\begin{align}\label{eq:poly-constraints}
    & G_1(X) = a(X) + \beta v(X),\, G_2(X) = \bar{a}(X) + \beta \bar{v}(X) + \beta^2 \bar{\op}(X) \tag{A1} \\
    & G_3(X) = a'(X) + \beta v'(X),\, G(X) = A(X) + \beta V(X) + \beta^2 \op(X) \tag{A2} \\
    & H(X) = G_1(X) + \gamma G_2(X) + \gamma^2 G_3(X), \tag{A3} \\
    & Q(X) = (H(X^3) - G(X) - \gamma G(\omega^m X) - \gamma^2 G(\omega^{2m} X))/Z(X) \tag{A4}
\end{align}
\item $\prover\rightarrow\verifier$: The prover sends commitment $\gone{Q}$ to $Q(X)$.
\item $\verifier\rightarrow\prover$: Sends $s\gets\F$.
\item $\prover\rightarrow\verifier$: Sends evaluations $\val{s}{G}=G(s)$, $\val{\omega^m s}{G}=G(\omega^m s)$,
$\val{\omega^{2m}s}{G}=G(\omega^{2m} s)$, $\val{s^3}{H}=H(s^3)$, $\val{s}{Q}=Q(s)$ and $\val{s}{Z}=Z(s)$.
\item $\verifier\rightarrow\prover$: Sends $r\gets\F$.
\item $\prover\rightarrow\verifier$: Sends $\kzg$ proofs:
\begin{itemize}[leftmargin=2em]
\item $\Pi_G=\kzgprove(\srs,G,(s,\omega^m s, \omega^{2m}s))$.
\item $\Pi_H=\kzgprove(\srs,H,s^3)$.
\item $\Pi_F=\kzgprove(\srs,F,s)$ where $F(X)=Z(X) + rQ(X)$.
\end{itemize}
\item $\verifier$ computes: Compute commitments $\gone{G},\gone{H}$ and $\gone{F}$ using homomorphism. It then checks:
\begin{itemize}[leftmargin=2em]
\item $\kzgverify(\srs,\gone{G},(s,\omega^m s, \omega^{2m}s), (\val{s}{G}, \val{\omega^m s}{G}, \val{\omega^{2m} s}{G}),\Pi_G)$.
\item $\kzgverify(\srs,\gone{H}, s^3, \val{s^3}{H},\Pi_H)$.
\item $\kzgverify(\srs,\gone{F}, s, \val{s}{Z} + r\val{s}{Q}, \Pi_F)$.
\item $\val{s}{Q}\cdot \val{s}{Z} =? \val{s^3}{H}-\val{s}{G}-\gamma \val{\omega^m s}{G}-\gamma^2\val{\omega^{2m}s}{G}$.
\end{itemize}
\item $\verifier$ outputs: Accept if all the above checks succeeds, otherwise it rejects.
\end{enumerate}
}
\end{mdframed}
\vspace*{-5mm}
\caption{Protocol: Check correctness of time-ordered transcript}
\label{fig:time-ordered-transcript}
\end{figure}
%\end{tcolorbox}

Next, we show a polynomial protocol for proving that the transcript $\tr^\ast$ is a permutation of the transcript $\tr$.
We first recall the permutation argument for vectors from ~\cite{EPRINT:GabWilCio19}.
\begin{lemma}[Permutation Check \cite{EPRINT:GabWilCio19}]\label{lem:perm-argument}
Let $f(X), g(X)$ be polynomials in $\F[\,X\,]$. Then, the vectors $\vec{f}, \vec{g}\in \F^k$ encoded by the polynomials
are permutations of each other if and only if with overwhelming probability over the choice of $\alpha\gets \F$,
there exists a polynomial $z(X)$ satisfying the polynomial constraints:
{\small
\begin{align}\label{eq:perm-constraints}
\lambda_1(X)(z(X) -1) &= 0 \text{ mod } Z_\setH(X) \tag{B1} \label{eq:B1} \\
(\alpha - g(X))z(\omega X) &= (\alpha - f(X))z(X) \text{ mod } Z_\setH(X) \tag{B2} \label{eq:B2}
\end{align}
}
\end{lemma}
The polynomial protocol in Figure \ref{fig:permutated-transcripts} essentially invokes the above argument on
the random linear combination of the columns of the respective transcripts.
%\begin{tcolorbox}
\begin{figure}[htbp]
\centering
\begin{mdframed}
    {\footnotesize
    \noident{\bf Common Input}: Commitments $c_\tr=(c_t,c_\op,c_A, c_V)$ and $c_\tr^\ast=(c_t^\ast,c_\op^\ast, c_A^\ast, c_V^\ast)$
    of transcripts $\tr$ and $\tr^\ast$ respectively.
    \begin{enumerate}[leftmargin=2em]
        \item $\verifier\rightarrow\prover$: Sends $\alpha,\beta, \chi\gets \F$
        \item $\prover$ computes $f(X)=t(X) + \beta \op(X) + \beta^2 A(X) + \beta^3 V(X)$, $g(X) = t^\ast(X) + \beta \op^\ast(X)$
        $+ \beta^2 A^\ast(X) + \beta^3 V^\ast(X)$. It then computes polynomials $z(X),q(X)$ as:
        \begin{itemize}[leftmargin=1em]
        \item Interpolate polynomial $z(X)$ of degree $k-1$ such that $z(\omega)=1$ and
         $z(\omega^{i+1})=\prod_{j=1}^i (\alpha - f(\omega^j))/(\alpha - g(\omega^j))$ for $1\leq i\leq k$.
        \item Compute $q(X) = ((\alpha - g(X))z(\omega X) - (\alpha - f(X))z(X) + \chi\lambda_1(X)(z(X) - 1))/\mathbb{Z}_{\setH}(X)$.
        \end{itemize}
        \item $\prover\rightarrow\verifier$: Sends commitments $\gone{z}$ and $\gone{q}$ to polynomials $z(X)$ and $q(X)$ respectively.
        \item $\verifier$ computes: The verifier checks that $q(X)Z_\setH(X)$ = $(\alpha - g(X))z(\omega X)-(\alpha - f(X))z(X) + \chi\lambda_1(X)(z(X) - 1)$
        by requesting evaluations and $\kzg$ proofs of polynomials $f,g$ and $q$ at a random point. It accepts if the check succeeds.
    \end{enumerate}
}
\end{mdframed}
\vspace*{-5mm}
\caption{Protocol: Check permutation of transcripts}
\label{fig:permutated-transcripts}
\end{figure}
%\end{tcolorbox}
Finally, we see that Lemma \ref{lem:addr-ordered-transcript} implies a polynomial protocol to check that the transcript
$\tr^\ast$ is address ordered and satisfies load-store consistency. We skip the formal protocol details, which
essentially involves the prover identifying sets
$I_1, I_2$ as described in Section \ref{subsec:encoded-relations} and sending auxiliary polynomials $Z_1(X),Z_2(X),\delta^\ast_A(X)$ and $\delta^\ast_T(X)$ to the verifier.
The verifier then checks the identities (C1)-(C6) in Lemma \ref{lem:addr-ordered-transcript}.
The range checks in (C7) can be checked using polynomial protocols in sub-vector lookup arguments such as Caulk+ ~\cite{EPRINT:PosKat22}, CQ
~\cite{EPRINT:EagFioGab22}. For completeness, we also include a protocol for enforcing range checks using committed index lookup
described in Section \ref{subsec:committed-index-lookup}.

\begin{comment}
\begin{enumerate}[leftmargin=2em]
\item $\prover\rightarrow\verifier$: For the transcript $\tr^\ast=(\vec{t^\ast},\vec{\op^\ast},\vec{A^\ast},\vec{V^\ast})$,
the prover computes encoding polynomials $\wt{\tr^\ast}=({t^\ast}(X),\op^\ast(X), A^\ast(X),V^\ast(X))$. Next, the prover
computes sets $I_1$ and $I_2$ as in Section \ref{subsec:encoded-relations}, and then computes polynomials
$Z_b(X)=\prod_{i\in I_b}(X-\omega^i)$ for $b=1,2$.
The prover also interpolates polynomials $\delta^\ast_A(X)$ and $\delta^\ast_T(X)$ satisfying
$\delta^\ast_A(\omega^i) = A^\ast_{i+1} - A^\ast_i$ for $i\in I_1$ and $\delta^\ast_T(\omega^i)=t^\ast_{i+1}-t^\ast_i$ for $i\not\in I_1$.
The prover sends all the above polynomials to the verifier.
\item $\verifier$ computes: The verifier checks relations (C1)-(C6) in Lemma \ref{lem:addr-ordered-transcript}. The range constraints
in (C7) can be verified using polynomial protocols in sub-vector lookup arguments such as Caulk+ ~\cite{EPRINT:PosKat22}, CQ
~\cite{EPRINT:EagFioGab22}.
\end{enumerate}


\subsection{Succinct Argument for Verifiable RAM}\label{subsec:succ-args}
Polynomial protocols can be compiled into succinct arguments of knowledge using an extractable polynomial commitment scheme. At a
high level the compilation involves the prover sending ``commitments'' to the message polynomials, whereas the verifier checks
the polynomial identities probabilistically by requesting evaluation proofs of the polynomials at random points.
We complile the polynomial protocol for RAM to a succinct argument of knowledge in the Algebraic Group Model (AGM)
 using $\kzg$ as the extractable polynomial commitment scheme. We also leverage homomorphism of $\kzg$ scheme
to avoid sending commitments and evaluations for polynomials which are linear combinations of previously committed polynomials.
We now present an argument for verifiable RAM, which combines the polynomial protocols in this section, and instantiates
them using $\kzg$ commitments. We make standard optimisations of batching several identity checks into one where applicable.
Let $\srs=\{\gone{\tau^i},\gtwo{\tau^i}\}_{i=0}^d$
denote the $\kzg$ setup parameters for degree $d\geq k$ for the bilinear group $\mathsf{BG}=(\Gone,\Gtwo,\gone{1},\gtwo{1},e)$. Let
$\F=\F_p$ be the finite field where $p$ is the order of the groups.


\begin{figure}[t]
\begin{subfigure}{\linewidth}
\centering
{\footnotesize
\begin{enumerate}[leftmargin=1em]
    \item $\prover\rightarrow\verifier$: Send polynomial commitments $\gone{a(X)}$, $\gone{v(X)}$, $\gone{a'(X)}$, $\gone{v'(X)}$,
    $\gone{\bar{\op}(X)}$, $\gone{\bar{a}(X)}, \gone{\bar{v}(X)}$, $\gone{\op(X)}$, $\gone{A(X)}$, $\gone{V(X)}$, $\gone{t^\ast(X)}$,
    $\gone{\op^\ast(X)}$, $\gone{A^\ast(X)}$, $\gone{V^\ast(X)}$, $\gone{\delta_A^\ast(X)}$, $\gone{\delta_T^\ast(X)}$.
    \item $\verifier\rightarrow\prover$: Send $\alpha,\beta,\gamma\gets \F$.
    \item $\prover$ computes: Compute sets $I_1=\{i\in [k]: A^\ast_i\neq A^\ast_{i+1}\}$, and $I_2=[k]\setminus I_1$. Next, compute
    polynomials:
    \begin{align*}
        &Z_1(X)=\prod_{i\in I_1}(X-\omega^i),\quad Z_2(X)=\prod_{i\in I_2}(X-\omega^i), \\
        &H(X) = \begin{bmatrix} 1 & \gamma & \gamma^2 \end{bmatrix}
        \begin{bmatrix}
            a(X) & v(X) & 0 \\
            a'(X) & v'(X) & 0 \\
            \bar{a}(X) & \bar{v}(X) & \bar{\op}(X)
        \end{bmatrix}
        \begin{bmatrix}
        1 \\
        \beta \\
        \beta^2
        \end{bmatrix}, \\
        &G(X) = A(X) + \beta V(X) + \beta^2 \op(X), \\
        &Q(X) = \big(H(X^3) - G(X) - \gamma G(\omega^m X) - \gamma^2 G(\omega^{2m} X)\big)/Z(X), \\
        &f(X) = G(X) + \beta^3 t(X),\, g(X) = A^{\ast}(X) + \beta V^{\ast}(X) + \beta^2 \op^{\ast}(X) + \beta^3 t^{\ast}(X), \\
        &z(X) = \sum_{i=1}^k \lambda_i(X)\prod_{j=1}^{i-1} (\alpha - g(\omega^j))/(\alpha - f(\omega^j)), \\
        &Q_1(X) = \frac{1}{\mathbb{Z}_\setH(X)}\Big( (\alpha - g(X))z(\omega X) - (\alpha - f(X))z(X) \\
        &\qquad\qquad + \beta Z_2(X)(A^\ast(\omega X) - A^\ast(X) - \delta_A^\ast(X))\Big), \\
        &Q_2(X) = \frac{1}{Z_2(X)}\Big((A^\ast(\omega X) - A^\ast(X))
         + \beta(t^\ast(\omega X) - t^\ast(X)-\delta_T^\ast(X)) \\
        &\qquad\qquad + \beta^2 (\op^\ast(X) - 1)(V^\ast(\omega X) - V^\ast(X))\Big)
    \end{align*}
    \item $\prover\rightarrow \verifier$: Send $\gone{Z_1(X)}$, $\gone{Z_2(X)}$, $\gone{Q(X)}$, $\gone{z(X)}$,
    $\gone{Q(X)}$, $\gone{Q_1(X)}$, $\gone{Q_2(X)}$.
    \item $\verifier\rightarrow\prover$: $s\gets \F$.
    \item $\prover\rightarrow\verifier$: Send evaluations $\val{Z_1}{s}=Z_1(s)$, $\val{Z_2}{s}=Z_2(s)$,
    $\val{G}{s}=G(s)$, $\val{Q}{s}=Q(s)$, $\val{Z}{s}=Z(s)$, $\val{A^\ast}{s}=A^\ast(s)$, $\val{V^\ast}{s}=V^\ast(s)$,
    $\val{\op^\ast}{s}=\op^\ast(s)$, $\val{t}{s}=t(s)$, $\val{t^\ast}{s}=t^\ast(s)$, $\val{z}{s}=z(s)$, $\val{Q_1}{s}=Q_1(s)$,
    $\val{Q_2}{s}=Q_2(s)$, $\val{H}{s^3}=H(s^3)$, $\val{G}{\omega^m s}=G(\omega^m s)$, $\val{\delta^\ast_A}{s}=\delta_A^\ast(s)$,
    $\val{\delta_T^\ast}{s}=\delta_T^\ast(s)$
    $\val{G}{\omega^{2m} s}=G(\omega^{2m} s)$,
    $\val{A^\ast}{\omega s}=A^\ast(\omega s)$, $\val{V^\ast}{\omega s}=V^\ast(s)$, $\val{t^\ast}{\omega s}=t^\ast(\omega s)$,
    $\val{z}{\omega s}=z(\omega s)$, $\val{A^\ast}{\omega}=A^\ast(\omega)$.

    \item $\verifier\rightarrow\prover$: Send $r \gets \F$.
    \item $\prover$ computes: Compute batched $\kzg$ proofs:
    \begin{alignat*}{3}
        &P_1 && &\quad=\quad  && &Z_1+r Z_2 + r^2 G + r^3 Q + r^4 Z + r^5 A^{\ast} + r^6 V^{\ast} + r^7 \op^{\ast} \\
        & && & && &\quad  + r^8 t + r^9 t^{\ast} + r^{10} z + r^{11} Q_1 + r^{12} Q_2 + r^{13} \delta_A^{\ast} + r^{14} \delta_T^{\ast} \\
        &P_2 && &\quad=\quad && &A^{\ast} + r V^{\ast} + r^2 z \\
        &\Pi_s && &\quad=\quad && &\kzgprove(P_1, s) \\
        &\Pi_{\omega s} && &\quad=\quad && &\kzgprove(P_2, \omega s) \\
        &\Pi_{s^3} && &\quad=\quad && &\kzgprove(H, s^3) \\
        &\Pi_{\{\omega^m s, \omega^{2m} s\}} && &\quad=\quad && &\kzgprove(G, (\omega^m s, \omega^{2m} s))
    \end{alignat*}
    \item $\prover\rightarrow \verifier$: Send $\Pi_s,\Pi_{\omega s}, \Pi_{s^3},\Pi_{\{\omega^m s, \omega^{2m} s\}}$ and
    $\val{H}{s^3},\val{G}{\omega^m s}, \val{G}{\omega^{2m} s}$.

    \item $\verifier$ computes: Compute commitments and evaluations for linear combinations.
    \begin{align*}
        \gone{G(X)} &= \gone{A(X)}+\beta\gone{V(X)}+\beta^2\gone{\op(X)}, \\
        \gone{H(X)} &= \begin{bmatrix} 1 & \gamma & \gamma^2 \end{bmatrix}
        \begin{bmatrix}
            \gone{a(X)} & \gone{v(X)} & \gone{0} \\
            \gone{a'(X)} & \gone{v'(X)} & \gone{0} \\
            \gone{\bar{a}(X)} & \gone{\bar{v}(X)} & \gone{\bar{\op}(X)}
        \end{bmatrix}
        \begin{bmatrix}
            1 \\
            \beta \\
            \beta^2
        \end{bmatrix}, \\
        \gone{P_1(X)} &= \gone{Z_1(X)}+r \gone{Z_2(X)} + r^2 \gone{G(X)} + r^3 \gone{Q(X)} + r^4 \gone{Z(X)} + r^5 \gone{A^{\ast}(X)} \\
         &\qquad + r^6 \gone{V^{\ast}(X)} + r^7 \gone{\op^{\ast}(X)} + r^8 \gone{t(X)} + r^9 \gone{t^{\ast}(X)} + r^{10} \gone{z(X)} \\
         &\qquad + r^{11} \gone{Q_1(X)} + r^{12} \gone{Q_2(X)} + r^{13} \gone{\delta_A^{\ast}(X)} + r^{14} \gone{\delta_T^{\ast}(X)}, \\
        \gone{P_2(X)} &= \gone{A^\ast(X)} + r\gone{V^\ast(X)} + r^2\gone{z(X)},\\
        \val{P_1}{s} &= \val{Z_1}{s}+r \val{Z_2}{s} + r^2 \val{G}{s} + r^3 \val{Q}{s} + r^4 \val{Z}{s} + r^5 \val{A^{\ast}}{s} \\
         &\qquad + r^6 \val{V^{\ast}}{s} + r^7 \val{\op^{\ast}}{s} + r^8 \val{t}{s} + r^9 \val{t^{\ast}}{s} + r^{10} \val{z}{s} \\
         &\qquad + r^{11} \val{Q_1}{s} + r^{12} \val{Q_2}{s} + r^{13} \val{\delta_A^{\ast}}{s} + r^{14} \val{\delta_T^{\ast}}{s}, \\
        \val{P_2}{\omega s} &= \val{A^\ast}{s} + r\val{V^\ast}{s} + r^2\val{z}{s}. \\
        \val{g}{s} &= \val{A^\ast}{s} + \beta \val{V^\ast}{s} + \beta^2 \val{\op^\ast}{s} + \beta^3 \val{t^\ast}{s}, \\
        \val{f}{s} &= \val{G}{s} + \beta^3 \val{t}{s}.
    \end{align*}

    \item $\verifier$ checks polynomial identities at $s$:
    \begin{alignat*}{3}
        &\val{Q}{s}\val{Z}{s} && &\quad \stackrel{?}{=}\quad  && &\val{H}{s^3} - \val{G}{s} - \gamma \val{G}{\omega^m s} - \gamma^2 \val{G}{\omega^{2m} s} \\
        &\val{Q_1}{s}(s^k-1) && &\quad \stackrel{?}{=}\quad && &(\alpha - \val{g}{s})\val{z}{\omega s} - (\alpha - \val{f}{s})\val{z}{s}
        +\beta \val{Z_2}{s}(\val{A^\ast}{\omega s}-\val{A^\ast}{s}-\val{\delta_A^\ast}{s}). \\
        &\val{Q_2}{s}\val{Z_2}{s} && &\quad \stackrel{?}{=}\quad && &(\val{A^\ast}{\omega s} - \val{A^\ast}{s}) + \beta (\val{t^\ast}{s} - \val{t}{s} - \val{\delta}{s}) \\
        & && & && &\qquad + \beta^2 (\val{\op^\ast}{s} - 1)(\val{V^\ast}{\omega s}-\val{V^\ast}{s}) \\
        &\val{Z_1}{s}\val{Z_2}{s} && &\quad \stackrel{?}{=}\quad && &s^k - 1 \\
    \end{alignat*}
    \item $\verifier$ checks evaluation proofs:
    \begin{alignat*}{3}
        &\kzgverify(\gone{P_1(X)},\val{P_1}{s},s, \Pi_s) && &\quad\stackrel{?}{=}\quad && &1 \\
        &\kzgverify(\gone{P_2(X)},\val{P_2}{\omega s},\Pi_{\omega s}) && &\quad\stackrel{?}{=}\quad && &1 \\
        &\kzgverify(\gone{H(X)},\val{H}{s^3},\Pi_{s^3}) && &\quad\stackrel{?}{=}\quad && &1 \\
        &\kzgverify(\gone{G(X)},(\val{G}{\omega^m s},\val{G}{\omega^{2m} s}),
        (\omega^m s, \omega^{2m} s), \Pi_{\{\omega^m s, \omega^{2m} s\}}) && &\quad\stackrel{?}{=}\quad && &1 \\
        &\kzgverify(\gone{A^\ast(X)}, \val{A^\ast}{\omega}, \omega, \Pi_\omega) && &\quad\stackrel{?}{=}\quad && &1
    \end{alignat*}
    \item $\verifier$ enforces range checks: Invoke the lookup argument $\varPi_{\mathsf{lookup}}$ to check that
    polynomials $\delta_A^\ast(X)$, $\delta_T^\ast(X)$ and $t^\ast(X)$ encode values in $[0,N]$.
    \item $\verifier$ outputs: The verifier outputs accept (1) if all the preceeding checks pass, else it outputs reject (0).
\end{enumerate}
}
\end{subfigure}
\caption{Argument of Knowledge for Verifiable RAM}
\label{fig:aok-vram}
\end{figure}
\end{comment}

\begin{comment}
Let $\LLconcat$ denote the language consisting of polynomial tuples $(a(X)$, $b(X)$, $c(X)$, $v(X))$ satisfying the identities
in Lemma ~\ref{lem:vec-concatenation}. To check membership of $(T,\vec{o},T',\tr)$ in the language $\LTr$, let $\wt{T}=(a(X),v(X))$, $\wt{T'}=(a'(X),v'(X))$,
$\wt{O}=(\bar{\op}(X),\bar{a}(X),\bar{v}(X))$ and $\wt{\tr}=(t(X),o(X),A(X),V(X))$ be the polynomial
encodings of $T,T',\vec{o}$ and $\tr$ respectively.
From the construction of time ordered transcript $\tr$ outlined
in Section \ref{subsec:transcripts}, we see that the equivalent constraints on the encodings are given by:
\begin{align*}\label{eq:tot-poly-constraints}
t(X) &= \sum_{i=1}^k i\lambda_i(X) \quad (= \mathsf{Enc}(\setind_k)) \\
(0, \bar{\op}(X), 0, o(X)) &\in \LLconcat \\
(a(X), \bar{a}(X), a'(X), A(X)) &\in \LLconcat \\
(v(X), \bar{v}(X), v'(X), V(X)) &\in \LLconcat
\end{align*}

We can probabilistically combine the different polynomial identities into one and obtain the following:
\begin{lemma}
    Let the polynomial $Z(X)=\prod_{i\in [m]}(X-\omega^i)$ and $\rho=\omega^m$ be as before.
    Then, we have $(T,\vec{o},T',\tr)\in \LTr$ if and only if the encoding polynomials as defined above satisfy
    the identity $G_\gamma(X) = 0  \text{ mod } Z(X)$ and $\evalH{t}=(1,\ldots,k)$ with overwhelming
    probability over the choice of $\gamma\gets \F$. Here the polynomial $G_\gamma(X)$ is defined as:
    \begin{multline*}
        G_\gamma(X) = o(X) + \gamma(\bar{\op}(X^3) - o(\rho X)) + \gamma^2 o(\rho^2 X) \\
        + \gamma^3(a(X^3) - A(X)) + \gamma^4(\bar{a}(X^3) - A(\rho X)) + \gamma^5(a'(X^3) - A(\rho^2 X)) \\
        + \gamma^6(v(X^3) - V(X)) + \gamma^7(\bar{v}(X^3) - V(\rho X)) + \gamma^8(v'(X^3) - V(\rho^2 X))
    \end{multline*}
\end{lemma}

Next, we consider the language $\Lperm$ consisting of pairs $(\tr, \tr^\ast)$ of $k$-length transcripts such
that $\tr^\ast=\sigma(\tr)$ for some permutation $\sigma:[k]\rightarrow [k]$. We now describe constraints
to check the same using polynomial encodings of the transcripts. Let encodings $\wt{\tr},\wt{\tr}^\ast$ be
given by polynomials as below:
\begin{align*}\label{eq:tr-encodings}
\wt{\tr} &= (t(X),\op(X),A(X),V(X)) \\
\wt{\tr}^\ast &= (t^\ast(X), \op^\ast(X), A^\ast(X), V^\ast(X))
\end{align*}
We need to show that for some $\sigma:[k]\rightarrow [k]$, $\evalH{p^\ast}=\sigma(\evalH{p})$ for $p\in \{t,\op,A,V\}$.
Again, for $\gamma\gets \F$, with overwhelming probability it is equivalent to establishing that polynomial
$f^\ast(X)=t^\ast(X)+\gamma \op^\ast(X) + \gamma^2 A^\ast(X) + \gamma^3 V^\ast(X)$ encodes a permutation of
vector encoded by $f(X)=t(X)+\gamma \op(X) + \gamma^2 A(X) + \gamma^3 V(X)$. We recall the following variation
of the grand product argument to show that two polynomials encode vectors which are permutations of each other.
\end{comment}









    \section{Improved Batching Efficient RAM}\label{sec:batch-efficient-ram}
    
The construction in the previous section results in prover complexity which is quasi-linear in both the
size of the RAM and the number of operations.
Our goal in this section is to achieve prover complexity which is {\em sublinear} in the size of the RAM.
In what follows, let $N$ denote the size of the RAM (upper-case to signify it's large) and $m$ denote the number
of operations in a batch. We will use a vectors in $\F^N$ to denote the ``large'' RAMs, where index column is implicitly
assumed to be $(1,\ldots,N)$.
Let $\vec{T},\vec{T'}\in \F^N$ denote the initial and final RAM states, and let $\vec{o}$ be
a sequence of $m$ operations which updates $\vec{T}$ to $\vec{T}'$. Let $\vec{a}\in \F^m$ denote the vector
of RAM indices referenced by the operations in $\vec{o}$, i.e, $a_i$ is the index referenced by $i^{th}$ operation.
To prove the transformation of $\vec{T}$ to $\vec{T}'$ via operation sequence $\vec{o}$, we proceed as follows:
\begin{itemize}[leftmargin=2em, label=-]
    \item We isolate sub-tables $S=(\vec{a},\vec{v})$ and $S'=(\vec{a},\vec{a'})$ of $T$ and $T'$ consisting of
    rows corresponding to indices in $\vec{a}$. This requires proving $\vec{v}=\vecT[\vec{a}]$ and $\vec{v'}=
    \vecT'[\vec{a}]$, which we show using {\em committed index lookup} discussed in Section ~\ref{subsec:committed-index-lookup}.

    \item On the isolated sub-table $S$ and $S'$ of size $m$, we use the standard memory checking arguments (c.f. argument
    presented in Section \ref{sec:poly-proto-ram-app}) to prove that sequence $\vec{o}$ correctly updates $S$ to $S'$ with
    prover complexity of $\wt{O}(m)$.

    \item Finally, we show that the RAMs $T$ and $T'$ are identical outside indices in $\vec{a}$. We call them $\vec{a}$-identical
    and describe the protocol for proving the same in Section ~\ref{subsec:proximity-ram}.
\end{itemize}

\begin{figure}[htbp]
    \centering
    \includegraphics[width=0.4\textwidth]{RAM-Lookup}
    \caption{Illustrating different steps of sub-linear lookup protocol between large RAMs $\vecT$ and $\vecT'$.}
    \label{fig:blueprint}
\end{figure}

The blueprint for the above approach is illustrated in Figure ~\ref{fig:blueprint}.
The efficiency of the above approach relies crucially on the efficiency of committed index lookup used to
reduce the size of the RAMs for quasi-linear memory checking methods. It is tempting to use the recent lookup arguments
in ~\cite{CCS:ZBKMNS22,EPRINT:PosKat22,EPRINT:EagFioGab22} to prove the correctness of the first step with prover complexity dependent only on $m$.
However, employing them directly is difficult; their table-independent efficiency relies on
table specific expensive pre-computation, which does not help when the table is updatable. This is the problem we solve
in Section ~\ref{sec:update-protocol}, where we modify the prover algorithm for the lookup arguments to remain efficient
with access to pre-computed parameters for an ``approximate'' table.
\begin{comment} %repetition
In particular, we show that lookup protocols in
~\cite{CCS:ZBKMNS22,EPRINT:PosKat22,EPRINT:EagFioGab22} can be used to show $m$ lookups from a table $\vecT'$, given pre-computed parameters
for a table $\vecT$ with additional overhead of $O((m+\delta)\log^2 (m+\delta))$, where $\delta$ denotes the hamming distance
of tables $\vecT$ and $\vecT'$. By optimally deferring the $O(N\log N)$ re-computation till we
accumulate $\delta \approx \sqrt{mN}$ updates, we achieve an amortized prover overhead $O(\sqrt{mN})$ over the lookup protocols for non-updatable tables.
This modification described in ~\ref{sec:update-protocol} applies to all the aforementioned lookup protocols.
\end{comment}

\noindent{\bf Additional Notation}:
Before proceeding, we introduce the subgroup $\setN=\{\xi,\ldots,\xi^N\}$ consisting of $N^{th}$ roots of unity,
over which we encode vectors in $\F^N$ as polynomials of degree less than $N$. Let $\{\mu_i(X)\}_{i=1}^N$ be the associated
lagrange basis polynomials over the set $\setN$. We also recall the set $\setV$ consisting of $m^{th}$ roots of unity
$\nu,\ldots,\nu^m$ with associated lagrange polynomials as $\{\tau_i(X)\}_{i=1}^m$. For $\vec{f}\in \F^N$, let
$\enc{f}{\setN}$ denote the polynomial encoding of $\vec{f}$ over $\setN$ given by $\sum_{i=1}^N f_i\mu_i(X)$. Similarly,
for $\vec{g}\in \F^m$, let $\enc{g}{\setV}$ denote its polynomial encoding over $\setV$ given by $\sum_{i=1}^m g_i\tau_i(X)$. For
vectors $\vec{t}$, we sometimes use the index notation $\vec{t}[a_i]$ to denote $a_i^{th}$ element of the vector. For vectors
$\vec{t}$ and $\vec{a}$ we use the notation $\vec{t}[\,\vec{a}\,]$ to denote the vector $\vec{v}$ such that $v_i=\vec{t}[\,a_i\,]$ for all $i$.
\chaya{above can be pruned. some of it is already in prelims.}


\subsection{Committed Index Lookup}\label{subsec:committed-index-lookup}
Let $m,N\in \N$ be fixed parameters with $m < N$ and let $\srs$ denote a $\kzg$ setup of degree $d\geq N$
over bi-linear group $(\F$, $\Gone$, $\Gtwo$, $\GT$, $e$, $\gone{1}$, $\gtwo{1}$, $[1]_t)$. Recall that the committed index
lookup relation in Definition ~\ref{defn:comm-index-lookup} involves the prover showing knowledge of vectors $\vecT\in \F^N$,
$\vec{a}\in \F^m$ and $\vec{v}\in \F^m$ corresponding to public commitments $c_T, c_a$ and $c_v$ such that they
satisfy $v_i = \vecT[\,a_i\,] = T_{a_i}$.
We present a polynomial protocol for the same, which is an adaptation of the lookup protocol from Caulk+ ~\cite{EPRINT:PosKat22}.
However, here we do not aim for zero-knowledge. Let $T(X)=\enc{t}{\setN}$, $a(X)=\enc{a}{\setV}$ and
$v(X)=\enc{v}{\setV}$ denote the polynomials encoding the vectors $\vec{t},\vec{a}$ and $\vec{v}$ respectively.
The verifier knows commitments to these polynomials at the start of the protocol.
Now $v_i = \vec{t}[a_i]$ for $i\in [m]$ is equivalent to $v(\nu^i) = T(\xi^{a(\nu^i)})$ for $i\in [m]$. To
obtain a polynomial protocol, the prover interpolates a polynomial $h(X)=\sum_{i=1}^m \xi^{a_i}\tau_i(X)$, which satisfies
$h(\nu^i)=\xi^{a(\nu^i)}$. To show that polynomial $h$ correctly ``exponentiates'' evaluations of $a(X)$, we consider the
inverting polynomial $\ell(X)=\sum_{i=1}^N i\mu_i(X)$ which behaves like ``log'' over $\setN$ by evaluating to $i$ on $\xi^i$. Now, we see
that all constraints are encoded as polynomial identities below:
\begin{equation}
    \begin{aligned}
        \ell(h(X)) &= a(X) \quad \text{mod } Z_{\setV}\\  % & \quad\text{ encodes } & \quad \forall i\in [m]:& h(\nu^i) = \xi^{a(\nu^i)}  \\
        T(h(X)) &= v(X) \quad \text{mod } Z_\setV \\ % \quad\text{ encodes } & \quad \forall i\in [m]:& v_i = \vec{t}[a_i] \\
        Z_{\setN}(h(X)) &= 0 \qquad \text{mod } Z_\setV  %&\quad\text{ encodes } & \quad \forall i \in [m]:& h(\nu^i)\in \setN
    \end{aligned}
    \label{eq:comm-index-lookup}
\end{equation}
The last polynomial identity ensures that evaluations of $h$ on $\setV$ lie in $\setN$ (the set of roots of $\vpolyN$). Since the polynomial $\ell$ is one-one
over $\setN$, the first equation implies $h(\nu^i)=\xi^{a_i}$ for all $i\in [m]$. The desired relation $v_i=T_{a_i}$ now follows from the second identity.
The above formulation involves composition with polynomials $\ell,T$ and $\vpolyN$ of degree $O(N)$, which is inefficient. We use the trick from
\cite{EPRINT:PosKat22}, where we work with low-degree restrictions of $O(N)$-degree polynomials such as $T, \ell$ over the set
$\setN_I=\{{h(\nu^i)}: i\in [m]\}=\{\xi^{a_i}:i\in I\}\subseteq \setN$, where $I=\{a_i: i\in [m]\}$. The prover
commits to the polynomials $Z_I(X)=\prod_{i\in I}(X-\xi^i)$, $h(X)$ and low degree ($<m$) restrictions $T_I, \ell_I$ of $T$ and $\ell$
on the $\setN_I$ respectively. The polynomial protocol then checks the following:
\begin{equation}
    \begin{alignedat}{3}
        T(X) - T_I(X) &= 0 \quad \text{ mod } Z_I ,&\quad& T_I(h(X)) &= v(X) \quad \text{ mod } Z_{\setV} \\
        \ell(X) - \ell_I(X) &= 0 \quad \text{ mod } Z_I ,&\quad& \ell_I(h(X)) &= a(X) \quad \text{ mod } Z_{\setV} \\
        Z_{\setN}(X) &= 0 \quad \text{ mod } Z_I ,&\quad& Z_I(h(X)) &= 0 \quad \text{ mod } Z_{\setV}
    \end{alignedat}
    \label{eq:poly-comm-index}
\end{equation}
It must be noted that the above identities imply the earlier polynomial identities in \eqref{eq:comm-index-lookup}. This is so because evaluations
of $h$ on $\setV$ are roots of $Z_I$, which implies $T_I(h(\nu^i))=T(h(\nu^i))$, $\ell_I(h(\nu^i))=\ell(h(\nu^i))$ and $\vpolyN(h(\nu^i))=0$ over $\setV$.
While the identities on the left still involve a degree $N$ polynomial, we can use the $\srs$ to check the polynomial
identity at the point $\tau$ encoded in the $\srs$. For example, we can evaluate the encoded quotient $\gtwo{Q(X)} =$
$\gtwo{\frac{(T(X) - T_I(X)}{Z_I(X)}}$ using the relation:
\begin{equation*}
    \gtwo{\frac{T(X)-T_I(X)}{Z_I(X)}} = \sum_{i\in I}\frac{1}{Z_I'(\xi^i)}\gtwo{\frac{T(X)-t_i}{X-\xi^i}}
\end{equation*}
By pre-computing the $\kzg$ proofs $W_1^i=\gtwo{\frac{T(X)-t_i}{X-\xi^i}}$ for all $i\in [N]$, the encoded quotient can be
evaluated using $O(m)$ $\Gtwo$-operations and $O(m\log^2 m)$ $\F$-operations.
The identity is then checked using a real pairing check
$$e(\gone{T(X)}-\gone{T_I(X)},\gtwo{1})=e(\gone{Z_I(X)},\gtwo{Q(X)}).$$
Similarly, we also pre-compute the encoded
quotients $W_2^i=\gtwo{\frac{\ell(X) - i}{X-\xi^i}}$ and $W_3^i=\gtwo{\frac{\vpolyN(X)}{X-\xi^i}}$ for all $i\in [N]$.
The quotients can be computed in time $O(N\log N)$ using the techniques in ~\cite{EPRINT:FeiKho23}. Using $\kzg$ commitment
scheme the polynomial relations over $Z_\setV$ can be checked in a standard manner
by having the prover send evaluation proofs for the committed polynomials at a random point chosen by the verifier.
The total prover effort incurred is $O(m^2)$ group and field operations.
Thus, we have:
\begin{lemma}\label{lem:comm-index-lookup}
Assuming $\kzg$ is extractable polynomial commitment scheme, there exists a succinct argument of knowledge for
the relation $\RLOOK$ with prover complexity of $O(m^2)$, given access to pre-computed parameters of size $O(N)$.
\end{lemma}

\subsubsection{Committed Index Lookup: Generic Transformation}\label{subsubsec:generic-transformation}
The protocol for committed index lookup using Caulk+ ~\cite{EPRINT:PosKat22} can become prohibitive for higher values of
$m$ due to the quadratic dependence on it. Here we describe a generic transformation, which realizes a committed index lookup
argument from any sub-vector argument such as ~\cite{CCS:ZBKMNS22,EPRINT:PosKat22,EPRINT:ZGKMR22,EPRINT:EagFioGab22} using a homo-morphic
vector commitment scheme. We recall that $\vec{a}\in \F^m$ is called a sub-vector of $\vec{b}\in \F^n$ if each element of $\vec{a}$
also occurs in $\vec{b}$. We use $\vec{a}\leq \vec{b}$ to denote $\vec{a}$ is a sub-vector of $\vec{b}$.
The transformation follows from the observation in the following lemma:
\begin{lemma}\label{lem:generic-transformation}
Let $\vec{t}\in \F^n$ and let $\vec{a},\vec{v}\in \F^m$ for some positive integers $m,n$. Let $\vec{I}_n$ denote the vector $(1,\ldots,n)$.
Then for $\gamma\gets \F$, $\vec{a}\leq \vec{I}_n$, $\vec{v}\leq \vec{t}$ and $(\vec{v}+\gamma \vec{a})\leq (\vec{t} + \gamma \vec{I}_n)$ implies
$\vec{v}=\vec{t}[\,\vec{a}\,]$ except with probability $O(n/|\F|)$.
\end{lemma}
The proof of the Lemma appears in Section ~\ref{sec:generic-transformation-app}.
Using Lemma ~\ref{lem:generic-transformation}, allows construction of argument for proving $\vec{v}=\vec{t}[\,\vec{a}\,]$ using three instantiations
of a sub-vector argument. We require homomorphism of the commitment scheme to enable the verifier to compute commitments for $\vec{v}+\gamma \vec{a}$ and
$\vec{t} + \gamma \vec{I}_n$ for the final instantiation of sub-vector argument. In particular, we also benchmark committed index lookup
protocol using lookup argument in ~\cite{EPRINT:EagFioGab22}, which incurs prover complexity of $O(m\log m)$.
~

\subsection{Almost Identical RAM States}\label{subsec:proximity-ram}
For a vector $\vec{a}\in [N]^m$, let $\uniq{a}=\{a_i: i\in [m]\}$ denote the subset of unique values in $\vec{a}$. We call two
RAM states $\vecT, \vecT'\in \F^N$ to be $\vec{a}$-{\em identical} if $\vecT[i]=\vecT'[i]$ for all $i\not\in\uniq{a}$. As before,
let $T(X),T'(X)$ and $a(X)$ be polynomials encoding the vectors $\vecT,\vecT'$ (over $\setN$) and $\vec{a}$ (over $\setV$). Given
commitments $c_T, c_T'$ and $c_a$ polynomial protocol to prove that committed vectors $\vecT,\vecT'\in \F^N$ and $\vec{a}\in \F^m$
are such that $\vecT,\vecT'$ are $\vec{a}$-identical involves proving the relation $Z_I(X)(T(X) - T'(X)) = 0$ over $Z_\setN$ where
$I=\uniq{a}$ and $Z_I(X)=\prod_{i\in I}(X-\xi^i)$ denotes the vanishing polynomial for the set $\setN_I=\{\xi^i: i\in I\}$.
The prover commits to polynomial $Z_I$ and proves (i) $Z_I(T - T') = 0 \text{ mod } Z_\setN$ and (ii) $Z_I$ is the vanishing
polynomial of the set $\setN_I$ as defined. To prove the first relation, the prover computes the polynomial $Q(X)$ as below:
\begin{align}\label{eq:poly-q}
D(X) &= \frac{(T(X)-T'(X))\cdot Z_I(X)}{Z_\setN(X)} \nonumber \\
&= \sum_{i\in I}\frac{(t_i - t_i')\mu_i(X)}{Z_\setN(X)} Z_I(X) \nonumber \\
\intertext{ Substituting, $\Delta_i=t_i-t_i'$, $\mu_i(X)=\vpolyN(X)/(\vpolyN'(\xi^i)(X-\xi^i))$ }
&=\sum_{i\in I}\frac{\Delta_i}{Z_\setN'(\xi^i)}\left(\frac{Z_I(X)}{X-\xi^i}\right) = \sum_{i\in I}\frac{\Delta_i Z_I'(\xi^i)}{Z_\setN'(\xi^i)}\kappa_i(X)
\end{align}
In the above, the summation only runs over indices in $I$, as $\Delta i = 0$ for $i\not\in I$. In the final equality, we use
$\kappa_i(X) = Z_I(X)/(Z_I'(\xi^i)(X-\xi^i))$ for $i\in I$ which we recognize as the lagrange basis polynomials for the set
$\{\xi^i: i\in I\}$. Thus, Equation \eqref{eq:poly-q} implies that $D(X)$ is at most degree $|I|-1$ polynomial, with
$D(\xi^i)=\Delta_i Z_I'(\xi^i)/\vpolyN'(\xi^i)$ for $i\in I$.
The prover can therefore interpolate $D(X)$ (in power basis)
in $O(|I|\log^2 |I|)$ $\F$-operations and compute $\gone{D(X)}$ in $O(|I|)$ $\Gone$-operations. The prover sends the
commitment $\gone{D(X)}$ to the verifier. Finally, the verifier can
check the identity $Z_I(T - T') = D\cdot Z_\setN$ by a pairing check. For this, since the tables are committed in $\Gone$, prover will need to send $\elttwo{Z_I(X)}$.

Next, the prover needs to show that $Z_I(X)$ is indeed the vanishing polynomial of $\setN_I$.
We again use the polynomial $h(X)=\sum_{i=1}^m \xi^{a_i}\tau_i(X)$ which interpolates the vector $(\xi^{a_1},\ldots,\xi^{a_m})$.
The correctness of the $h$ polynomial can be established by checking the polynomial identities in the last two rows of Equation
~\eqref{eq:poly-comm-index}. The aforementioned identities show that $Z_I(h(X)) = 0$ over $Z_\setV$ which shows that $Z_I$ vanishes over
entire vector interpolated by $h$ over $\setV$. To assert that $Z_I$ has no additional roots, the prover commits to the product polynomial
$K(X)=\prod_{i=1}^m (X - h(\nu^i))$ and the quotient polynomial $q(X)=K(X)/Z_I(X)$. The verifier checks the polynomial identities
at $\alpha$, i.e $K(\alpha)=q(\alpha)Z_I(\alpha)$ and $K(\alpha)=\prod_{i=1}^m(\alpha - h(\nu^i))$.
The former is easily accomplished
using evaluation proofs for $K,q$ and $Z_I$ at $\alpha$
For checking the latter, the prover commits to another polynomial
$u(X)$ satisfying $u(\nu^i)=\prod_{j=1}^{i-1}\big((\alpha - h(\nu^j))/(1 + \beta\tau_1(\nu^j))\big)$ for $i\in [m]$
where $\beta=K(\alpha) - 1$.
The verifier ensures the correctness of $u(X)$ by
\begin{equation}
    \begin{aligned}
        \tau_1(X)(u(X) - 1) &= 0 \text{ mod } Z_{\setV} \\
        u(\nu X)(1+\beta \tau_1(X))-u(X)(\alpha - h(X)) &= 0 \text{ mod } Z_\setV.
    \end{aligned}
    \label{eq:kh-check}
\end{equation}
We prove that the above constraints imply that $K(\alpha)=\prod_{i\in [m]}(\alpha - h(\nu^i))$.
Note that in this protocol we require commitment to the polynomial $Z_I$ in both $\Gone$ and $\Gtwo$,
and thus another pairing check is required to show that the $Z_I(X)$ committed in $\Gone$
(whose well formation is shown in the protocol as described above) is the same as the $Z_I(X)$ committed in $\Gtwo$,
used for the real pairing check.
\begin{lemma}\label{lem:kh-check}
There exists a polynomial $u(X)\in \F[X]$ satisfying the identities in Equation ~\eqref{eq:kh-check}
if and only if $K(\alpha)=1+\beta=\prod_{i\in [m]} (\alpha - h(\nu^i))$.
\end{lemma}
\begin{proof}
    Assume that the identitites hold for some polynomial $u(X)$.
    The first identity implies $u(\nu)=1$. From the second identity, we conclude that for all $i\in [m]$, we have
    $u(\nu^{i+1})=u(\nu^i)\cdot ((\alpha - h(\nu^i))/(1+\beta \tau_1(\nu^i)))$, and thus:
    $$1 = u(\nu^{m+1})/u(\nu) = \prod_{i\in [m]}\left(\frac{\alpha - h(\nu^i)}{1+\beta \tau_1(\nu^i)}\right).$$
    We observe that the product of denominators in the above equation is simply $1+\beta$ as $\tau_1(\nu^i)$
    is $0$ for all $i\neq 1$, and thus $1+\beta = \prod_{i=1}^m (\alpha - h(\nu^i))$. In the other direction,
    it is easy to check that $u(X)$ as defined for an honest prover, satisfies the identities in Equation ~\ref{eq:kh-check}.
\end{proof}

\noindent{\em Remark}: Note that we incur $O(m^2)$ cost if we show the correctness of polynomial $h$ using techniques of Section ~\ref{subsec:committed-index-lookup}.
However, we can use more $O(m\log m)$ complexity committed index lookup obtained from ~\cite{EPRINT:EagFioGab22}, which allows us to show correctness of
$h(X)$ by proving that it interpolates vector $\vec{h}$ on $\setV$ obtained as committed index lookup using vector $\vec{a}$ on the vector $\vecT_{exp}=(\xi,\xi^2,\ldots,\xi^N)$
i.e., $\vec{h}=\vecT_{exp}[\, \vec{a}\,]$.

\subsection{Batching-Efficient RAM: Combined Protocol}\label{subsec:all-together}
We put the entire protocol together now. Let $\setind$ denote the set of indices $\{1,\ldots,N\}$, and $\mathcal{I}_N$
denote the vector $(1,\ldots,N)$. We formally define the committed RAM relation for which we present an argument of
knowledge in this section.
\begin{definition}\label{defn:committed-ram}
We define the {\em committed ram} relation
$\CRAM$ to consist of tuples $((c_T, c_T', c_\op, c_a, c_w),(\vecT, \vecT',\vec{\op},\vec{a},\vec{w}))$
such that:
\begin{itemize}[leftmargin=1em]
    \item $(T,\vec{o},T')\in \LRAM{I}{N}{m}$ for $T=(\setind_N,\vecT)$, $T'=(\setind_N,\vecT')$ and $\vec{o}=(o_1,\ldots,o_m)$
    where $o_i=(\op_i, a_i, w_i)\in \RAMOp{I}$ for all $i\in [m]$.
    \item $c_T=\KZGcommit(\srs, T(X))$, $c_T'=\KZGcommit(\srs, T'(X))$, $c_\op = \KZGcommit(\srs,\op(X))$,  $c_a=\KZGcommit(\srs, a(X))$,
    $c_w = \KZGcommit(\srs, w(X))$ where polynomials $T(X), T'(X)$ encode vectors $\vecT, \vecT'$ over $\setN$, while $\op(X), a(X)$ and
    $w(X)$ encode vectors $\vec{\op}=(\op_1,\ldots,\op_m)$, $\vec{a}$ and $\vec{w}$ over $\setV$.
\end{itemize}
\end{definition}
As outlined in the blueprint, the prover first commits to ``smaller'' RAMs $S=(\vec{a},\vec{v})$ and $S'=(\vec{a},\vec{v}')$
where $\vec{v}=\vecT[\vec{a}]$ and $\vec{v}'=\vecT'[\vec{a}]$. The prover commits to $S$ and $S'$ by sending commitments
$c_v$ and $c_v'$ to $\vec{v}$ and $\vec{v}'$. Then the prover and verifier execute the committed index lookup protocol to
prove:
\begin{equation}
(c_T, c_a, c_v)\in \RLOOK\, \wedge\, (c_T', c_a, c_v')\in \RLOOK
\end{equation}
The verifier uses a random challenge $\chi\gets \F$ to reduce two instances of $\RLOOK$ to one instance
$(c_T + \chi c_T', c_a, c_v + \chi c_v')\in \RLOOK$. Then, we show that
RAMs $\vecT$ and $\vecT'$ are $\vec{a}$-identical using the protocol in Section \ref{subsec:proximity-ram}.
All that remains is to prove is that the operation sequence $\vec{o}$ is consistent with small RAMs $S$ and $S'$.
We check this using the argument in Section ~\ref{sec:poly-proto-ram}. Specifically, the prover and the verifier set
$c_S = (c_a, c_v)$, $c_S'=(c_a, c_v')$ and $c_o = (c_\op, c_a, c_w)$, and execute the argument of knowledge for
showing $(c_S, c_o, c_S')\in \CLRAM$ (see Definition ~\ref{defn:committed-ram}). We provide the complete protocol
listing in Figure ~\ref{fig:complete-listing}. The protocol in Figure ~\ref{fig:complete-listing} assumes pre-computed parameters
for the tables $T$ and $T'$. In the next section, we discuss how to efficiently maintain table-dependent parameters in the setting
requiring updates to the table.

\begin{theorem}\label{thm:committed-ram}
The protocol in Figure~\ref{fig:complete-listing} (continued in Figures~\ref{fig:complete-listing-2} and
~\ref{fig:complete-listing-3}) is a succinct argument of knowledge for the relation $\CLRAM$ in
the AGM, under the $Q$-DLOG assumption for the bilinear group $(\F,\Gone,\Gtwo,\GT,e,g_1,g_2)$.
\end{theorem}

%%% complete protocol listing %%%
\begin{figure}[t!]
    \begin{mdframed}

        \underline{Setup $(1^\secp,N,m, \vecT, \vecT')$}:
        \begin{itemize}[leftmargin=1em]
            \item $\srs = (\{\gone{\tau^i}\}_{i=0}^N, \{\gtwo{\tau^i}\}_{i=0}^N)$ for $\tau\gets \F$.
            \item $W_2^i=\gtwo{(\ell(X) - i)/(X-\xi^i)}$, $i\in [N]$(needed by prover)
            \item $W_3^i=\gtwo{\vpolyN(X)/(X-\xi^i)}$, $i\in [N]$(needed by prover)
            \item $\gone{\ell(X)}, \gone{\vpolyN(X)}, \gtwo{\vpolyN(X)}$(known by both)
        \end{itemize}

        \underline{Precompute $(\vecT, \vecT')$}:
        \begin{itemize}[leftmargin=1em]
            \item $W_1^i=\gtwo{(T(X)-T(\xi^i))/(X-\xi^i)}$, $i\in [N]$,
            \item ${W_1^i}'=\gtwo{(T'(X) - T'(\xi^i))/(X-\xi^i)}$, $i\in [N]$.
        \end{itemize}

        {\bf Common Input}: $\srs$, $c_T, c_T', c_\op, c_a, c_w\in \Gone$.\\
        {\bf Prover's Input}: Vectors $\vecT,\vecT',\vec{\op},\vec{a},\vec{w}$ and their encoding polynomials.\\

        {\bf Round 1}: Commit to sub RAMs.
        \begin{enumerate}[leftmargin=1em, label=\arabic*.]
            \item $\prover$ computes $\vec{v}=\vecT[\,\vec{a}\,]$, $\vec{v}'=\vecT'[\,\vec{a}\,]$ and the encoding
            polynomials $v(X)$ and $v'(X)$.
            \item $\prover$ sends $c_v = \gone{v(X)}$, $c_v'=\gone{v'(X)}$.
            \item $\verifier$ sends $\chi\gets \F$.
        \end{enumerate}

        {\bf Round 2}: Execute committed index lookup.
        \begin{enumerate}[leftmargin=1em, label=\arabic*.]
            \item $\prover$ and $\verifier$ compute $C_T=c_T + \chi c_T'$, $C_V=c_v + \chi c_v'$.
            \item $\prover$ computes $P(X) = T(X) + \chi T'(X)$, $V(X)=v(X) + \chi v'(X)$.
            \item $\prover$ computes $I=\{a_i: i\in [m]\}$, $\setN_I=\{\xi^i: i\in I\}$.
            \item $\prover$ computes polynomials:
            \begin{itemize}[leftmargin=1em, label=-]
                \item Vanishing polynomial $Z_I(X)$ of $\setN_I$.
                \item Polynomial $h(X)=\sum_{i\in [m]}\xi^{a_i}\tau_i(X)$.
                \item Restrictions $P_I(X),\ell_I(X)$ of $P(X),\ell(X)$ on set $I$.
                \item $K(X)=\prod_{i\in [m]}(X-\xi^{a_i})$, $q(X)=K(X)/Z_I(X)$
                \item $D(X)= \sum_{i\in I}\frac{\Delta_i Z_I'(\xi^i)}{Z_\setN'(\xi^i)}\kappa_i(X)$ by interpolation as described in section 5.2
            \end{itemize}

            \item $\prover$ sends $c_p = \gone{P_I(X)}$, $c_z=\gone{Z_I(X)}$, $c_{z2}=\gtwo{Z_I(X)}$, $c_h=\gone{h(X)}$, $c_l = \gone{\ell_I(X)}$,
            $c_k = \gone{K(X)}$, $c_q = \gone{q(X)}$, $c_d=\gone{D(X)}$
            \item $\verifier$ sends $\gamma\gets \F$.
        \end{enumerate}

        {\bf Round 3}: Prover send aggregated quotients.
        \begin{enumerate}[leftmargin=1em, label=\arabic*.]
            \item $\prover$ computes $g(X)=P_I(X) + \gamma \ell_I(X) + \gamma^2 Z_I(X)$.
            \item $\prover$ computes $Q(X) = (g(h(X)) - v(X) -\gamma a(X))/Z_\setV(X)$.
            \item $\prover$ computes: $W = \sum_{i\in [m]} \frac{1}{Z_I'(\xi^i)} (W_1^i + \chi {W_1^i}' + \gamma W_2^i + \gamma^2 W_3^i)$.
            \item $\prover$ sends $W\in \Gtwo$, $c_Q=\gone{Q(X)}$.
            \item $\verifier$ computes $c_g = c_p + \gamma c_l + \gamma^2 c_z$, $C_G = C_T + \gamma\gone{\ell(X)}+\gamma^2\gone{\vpolyN(X)}$.
            \item $\verifier$ checks: $e(C_G - c_g, \gtwo{1})=e(c_z, W)$.
            \item $\verifier$ checks: $e(c_T-c_{T'}, c_{z2})=e(c_d, \gtwo{\vpolyN(X)})$
            \item $\verifier$ checks: $e(c_z, [1]_2)=e([1]_1, c_{z2})$
            \item $\verifier$ sends $\alpha\gets \F$.
        \end{enumerate}

        Continued in Figure ~\ref{fig:complete-listing-2}
    \end{mdframed}
    \caption{Batching-Efficient RAM Protocol}
    \label{fig:complete-listing}
\end{figure}

\begin{figure}[t!]
    \begin{mdframed}
        \begin{center}
            Continued from Figure \ref{fig:complete-listing}
        \end{center}
        {\bf Round 4}: Prover sends evaluations.
        \begin{enumerate}[leftmargin=1em, label=\arabic*.]
            \item $\prover$ computes $\val{\alpha}{v}=v(\alpha)$, $\val{\alpha}{a}=a(\alpha)$, $\val{\alpha}{h}=h(\alpha)$, $\val{\alpha}{K}=K(\alpha)$,
            $\val{h(\alpha)}{g}=g(h(\alpha))$, $\val{\alpha}{Q}=Q(\alpha)$, $\val{\alpha}{q}=q(\alpha)$, $\val{\alpha}{Z}=Z_I(\alpha)$.
            \item $\prover$ sends $\val{\alpha}{v}$, $\val{\alpha}{a}$, $\val{\alpha}{h}$, $\val{\alpha}{K}$, $\val{h(\alpha)}{g}$, $\val{\alpha}{Q}$,
            $\val{\alpha}{q}$, $\val{\alpha}{Z}$
            \item $\prover$ computes polynomial $u(X)$ as in Section ~\ref{subsec:proximity-ram}
            and sends $c_u=\gone{u(X)}$.
            \item $\verifier$ checks $\val{\alpha}{Q}(\alpha^m-1)=\val{h(\alpha)}{g}-\val{\alpha}{v}-\gamma \val{\alpha}{a}$.
            \item $\verifier$ checks $\val{\alpha}{Z}\val{\alpha}{q}=\val{\alpha}{K}$.
            \item $\verifier$ sets $\beta=\val{\alpha}{K}-1$ and sends $\theta\gets\F$.
        \end{enumerate}

        {\bf Round 5}: Check correctness of $K$.
        \begin{enumerate}[leftmargin=1em, label=\arabic*.]
            \item $\prover$ computes:
            \begin{align*}
                Q'(X) &= \big(u(\nu X)(1+\beta \tau_1(X))-u(X)(\alpha - h(X)) \\
                &\quad + \theta \tau_1(X)(u(X)-1)\big)/Z_\setV(X).
            \end{align*}
            \item $\prover$ sends $c_Q'=\gone{Q'(X)}$.
            \item $\verifier$ sends $x\gets \F$.
        \end{enumerate}

        {\bf Round 6}: Prover sends more evaluations.
        \begin{enumerate}[leftmargin=1em, label=\arabic*.]
            \item $\prover$ computes $\val{x}{u}=u(x)$, $\val{\nu x}{u}=u(\nu x)$, $\val{x}{h}=h(x)$, $\val{x}{Q'}=Q'(x)$
            \item $\prover$ sends $\val{x}{u}$, $\val{\nu x}{u}$, $\val{x}{h}$, $\val{x}{Q'}$.
            \item $\verifier$ checks $\val{x}{Q'}(x^m-1)=\val{\nu x}{u}(1+\beta \tau_1(x))-\val{x}{u}(\alpha - \val{x}{h})$.
            \item $\verifier$ sends $r_a, r_h, r_q, r_v, r_K, r_Q, r_Z\gets \F$ and $r_h', r_u', r_Q'\gets \F$.
        \end{enumerate}

        \begin{center}
        Continue in Figure ~\ref{fig:complete-listing-3}.
        \end{center}


    \end{mdframed}
    \caption{Batching-Efficient RAM Protocol: Continued}
    \label{fig:complete-listing-2}
\end{figure}

\begin{figure}[t!]
    \begin{mdframed}
        \begin{center}
            Continued from Figure \ref{fig:complete-listing-2}
        \end{center}
        \item {\bf Round 7}: Check aggregated evaluation.
        \begin{enumerate}[leftmargin=1em, label=\arabic*.]
            \item $\prover$ computes:
            \begin{align*}
                \Phi_\alpha(X) &= r_a a(X)+ r_h h(X) + r_q q(X) + r_v v(X) \\
                &\quad + r_K K(X) + r_Q Q(X) + r_Z Z_I(X) \\
                \Phi_x(X) &= r_h' h(X) + r_u' u(X)+r_Q'Q'(X)
            \end{align*}
            \item $\prover$ computes $\Pi_\alpha = \KZGopen(\srs, \Phi_\alpha(X), \alpha)$.
            \item $\prover$ computes $\Pi_x = \KZGopen(\srs, \Phi_x(X), x)$.
            \item $\prover$ computes $\Pi_g = \KZGopen(\srs, g(X), \val{\alpha}{h})$.
            \item $\prover$ computes $\Pi_u = \KZGopen(\srs, u(X), \nu x)$.
            \item $\prover$ sends $\Pi_\alpha$, $\Pi_x$, $\Pi_g$ and $\Pi_u$.
            \item $\verifier$ computes:
            \begin{align*}
                \gone{\Phi_\alpha} &= r_a c_a + r_h c_h + r_q c_q + r_v c_v + r_z c_z + r_K c_K + r_Q c_Q. \\
                \gone{\Phi_x} &= r_h' c_h + r_u' c_u + r_Q' c_Q' \\
                V_{\alpha} &= r_a \val{\alpha}{a} + r_h \val{\alpha}{h} + r_q \val{\alpha}{q} + r_v \val{\alpha}{v} \\
                &\quad + r_z\val{\alpha}{Z} r_K \val{\alpha}{K} + r_Q \val{\alpha}{Q}. \\
                V_x &= r_h' \val{x}{h} + r_u' \val{x}{u}+r_Q' \val{x}{Q'}
            \end{align*}
            \item $\verifier$ checks:
            \begin{itemize}[leftmargin=1em]
                \item $\KZGverify(\srs, \gone{\Phi_\alpha}, V_\alpha, \alpha, \Pi_\alpha)$.
                \item $\KZGverify(\srs, \gone{\Phi_x}, V_x, x, \Pi_x)$.
                \item $\KZGverify(\srs, c_g, \val{h(\alpha)}{g}, \val{\alpha}{h}, \Pi_g)$.
                \item $\KZGverify(\srs, c_u,\val{\nu x}{u}, \nu x, \Pi_u)$.
            \end{itemize}
            \item $\prover$ and $\verifier$ set $c_S=(c_a, c_v)$, $c_S'=(c_a, c_v')$, $c_o=(c_\op, c_a, c_w)$.
            \item $\prover$ and $\verifier$ execute argument for $(c_S, c_o, c_S')\in \CLRAM$ (Section ~\ref{sec:poly-proto-ram}).
        \end{enumerate}
    \end{mdframed}
    \caption{Batching-Efficient RAM Protocol-Continued}
    \label{fig:complete-listing-3}
\end{figure}





\section{Fast Lookups from Approximate Pre-Processing}\label{sec:update-protocol}

The recent progress and interest in lookup arguments has been stellar, as exemplified by the series of works
both in uni-variate setting ~\cite{CCS:ZBKMNS22,EPRINT:PosKat22,EPRINT:ZGKMR22,EPRINT:EagFioGab22} and multi-variate
setting ~\cite{lasso}. However, the excellent online efficiency of current constructions relies on expensive
$\wt{O}(|T|)$ table-specific  pre-computation for a table $T$, or on tables exhibiting tensor structure as in ~\cite{lasso}.
This limits their application in settings where such assumptions are not viable, for example when tables model account balances in
a layer 2 (L2) blockchain network. We make the first attempt in this direction. Our key idea is to extend the utiltiy of pre-computed
parameters for a table $\vecT$, to proving lookups from tables $\vecT'\neq \vecT$. Essentially, we show that for $\delta=\Delta(\vecT, \vecT')$,
an argument for $m$ lookups from $\vecT'$ incurs an additional prover overhead of $(m+\delta)\log^2(m+\delta)$. We note that overhead is additive
in $\delta$ and that too only {\em quasi} linear. Our competitive overhead rests on several innovative applications of algebraic
algorithms, which are summarised in Section ~\ref{subsec:comp-algebra-app}.

\noindent{\bf Naive approaches are inadequate}: We note that the aforementioned constructions of lookup arguments require encoded quotients
of the form $\gany{(T(X)-T(\xi^i))/(X-\xi^i)}$ for upto $m$ values of $i$ during the proof generation. While constructions ~\cite{CCS:ZBKMNS22,EPRINT:PosKat22}
consider quotients encoded in the group $\Gtwo$, the protocol in ~\cite{EPRINT:EagFioGab22} encodes them in $\Gone$. We use a generic $[\,\cdot\,]_g$ to
account for protocol-specific choices. We also see that even a small change to the table requires one to update all the quotients (the polynomial $T(X)$ is
common to all quotients). Updating the $|\vecT|$ quotients for each batch is clearly infeasible. One could consider delaying the updation of the quotients, till
the time they are actually required in a proof, which happens when the corresponding index in the table is involved in lookup. However, each of the $m$ quotients
is now potentially ``lagging'' by $\delta$ updates, so we would need $\Omega(m\delta)$ group operations to refresh all of them. This gives us multiplicative degradation
with $\delta$, and is clearly unsustainable for reasonable values of $\delta$. We abandon the idea of computing individual encoded quotients, and instead attempt
to directly compute the aggregate encoding $\gany{(T(X)-T_I(X))/Z_I(X)}$, which as seen earlier is given by the summation below:
\begin{equation}\label{eq:encoded-quotient}
\gany{\frac{T(X)-T_I(X)}{Z_I(X)}} = \sum_{i\in I}\frac{1}{Z_I'(\xi^i)}\gany{\frac{T(X)-T(\xi^i)}{X-\xi^i}}
\end{equation}
We now describe our approach.
%Recall that our lookup protocol in section 5.1 involves certain precomputations by the prover namely $W_1^i, W_2^i, W_3^i$. $W_2^i$ and $W_3^i$ do not depend on the table. However, $W_1^i$ depends on the lookup table and their values will change even if the table changes by a small amount. It is expensive to recompute all the $W_1^i$ for every small change in the table and this will affect the efficiency of our lookup protocol in the long run.\\\\
%In this section, we show how to achieve efficient lookups even when the table is changing frequently, as long as the cumulative change in the table is small. \\
%In particular, we show how the prover can compute $[Q(X)]_2=\gtwo{\frac{T(X)-T_I(X)}{Z_I(X)}}$ without computing all the $W_1^i$(thus minimizing the overhead).\\
%The overhead(as long as the table doesn't change too much) will be much lower than the time needed for the lookup and so is very practical.

\subsection{Base + Cache approach}\label{subsec:base-cache}
The key idea we employ is to express the current table $\vecT\in \F^N$ as $\vecTbase + \vecTcache$, where $\vecTbase$ is the table for which we assume that
the encoded quotients are available (via the $O(N\log N)$ computation), and $\vecTcache$ captures the changes to the table since. We will periodically update (say
after $s$ batch updates) $\vecTbase$ to current table state, and re-compute all the quotients (we call it the {\em offline} phase).
We will revisit the question on choosing $s$ optimally later. Let $I\subseteq [N]$ denote the set of indices in the current batch of $m$ lookups. The {\em online}
phase of our proof generation involves computing the sum in Equation \eqref{eq:encoded-quotient} for the table $\vecT$.
%that we do not compute $W_1^i$ after each change of the table. Instead, this expensive computation will be done periodically for all $i \in [N]$ after say $s \in \mathbb{Z}$ batches.
%Let current table $\vecT$ can be represented as $\vecTbase + \vecTcache$ where the vector
%$\vecTbase$ denotes the base table (with respect to which $W_1^i$ was last computed for all $i \in [N]$) and the vector $\vecTcache$ corresponds to the changes
%that have happened to the base table since the last rebasing (rebasing denotes computation of all $W_1^i$)\\\\
%Thus, there is an \textbf{online} phase which happens after every batch (which includes computation of $[Q(X)]_2$ among other things) and an \textbf{offline} phase which consists of the rebasing(this is all prover computations) \\\\
\begin{comment}
\noindent{\bf Offline Phase}: This computation is executed once after every $s$ rounds. Here, the prover updates the base vector $\vecTbase$ with the changes in the cache vector
$\vecTcache$ by setting $\vecTbase := \vecTbase + \vecTcache$ and simultaneously clears the cache vector by setting
$\vecTcache = 0$.\\
It computes the commitment of $T_b$ as well\\
It also re-computes the $\mathsf{KZG}$ opening proofs $[W_1^i(X)]_2$ for $i\in [N]$ where
$W_1^i(X) = (\Tbasepoly{X} - t_i)/(X-\xi^i)$. Here, $t_i=\Tbasepoly{\xi^i}$ are the coordinates
of the updated base vector $\vecTbase$.\\
As mentioned in section 5.1 this can be done in $O(N\log N)$ group and field operations.\\\\
\noindent{\bf Online Phase}:
The online phase happens for every batch because the purpose of this phase is to ensure that all the things needed for the current execution of the lookup protocol are available. We show how the prover computes the next table $T'$ from the current table $T$ and the new Cache vector from the old cache vector (by an inductive argument this suffices)
\begin{enumerate}[leftmargin=1em]
    \item Prover has the $T$ for the current round and the commitment $[T(X)]_1$ as well(because these are just the $T'$, $[T'(x)]_1$ of the last round)
    \item The $\vecTcache$ and $\vecTcache(X)$ is also updated to the start of the current round (contains information till previous round:$\vecT=\vecTbase+\vecTcache$)
    \item The prover updates the cache using the current batch: $\vec{T'}_{\mathsf{ch}}[i] = \vec{T}_{\mathsf{ch}}[i] + \Delta_i$ for $i\in I$ in $O(m)$ $\F$ operations
    \item Here $\Delta_i$ for all $i \in I$ is the change that will happen to $\vecT$ \textbf{during the current round}
    \item Prover computes the commitment to the new cache polynomial:
    $$[\vec{T'}_{\mathsf{ch}}(X)]_1=[\vec{T}_{\mathsf{ch}}(X)]_1+\sum_{i\in I}\Delta_i[\mu_i(X)]_1$$ in
    $O(m)$ $\Gone$ operations.
    \item Prover also gets ${T}'$ as ${T'}[i]=T_b[i]+ \vec{T'_{\text{ch}}}[i]$ using the $T_b$ and the latest cache
    \item Prover computes the commitment to the new table $\vecT'$: $[T'(X)]_1=[\Tbasepoly{X}]_1+[\vec{T'}_{\mathsf{ch}}(X)]_1$

    \item In addition, the other things (apart from $[Q(X)]_2$) needed for the current round of the lookup protocol are also computed by the prover as described in the lookup protocol in section 5.1 as it is just naive computation

\end{enumerate}
\subsection{Computation of $[Q(X)]_2$}
Clearly, it suffices to efficiently compute $[Q(X)]_2$ where $[Q(X)]_2=\gtwo{\frac{T(X)-T_I(X)}{Z_I(X)}}$. We have the information of $[\Tbasepoly{X}-\Tbasepoly{\xi^i}/(X-\xi^i)]_2$. For this, we have the following lemma:
\end{comment}
The following Theorem determines the efficiency of the online phase of our prover.
\begin{theorem}\label{thm:approx-setup}
Let $N,\xi$ be as defined previously. Suppose we are given
$\kzg$ proofs $\{W_i\}_{i=1}^N$ with $W_i=\gtwo{\Tbasepoly{X} - \Tbasepoly{\xi^i}/(X-\xi^i)}$, where
$\Tbasepoly{X}=\enc{T_{\mathsf{b}}}{\setN}$ encodes a vector $\vecTbase\in \F^N$.
Let $I \subset [N]$, $\setN_I=\{\xi^i:i\in I\}$, $Z_I(X)$ denote the vanishing polynomial of $\setN_I$ and
$T_I(X)$ be the restriction of polynomial $T(X)$ on $\setN_I$.
Then, there exists an algorithm to compute $\kzg$ multi-opening proof
$\gany{Q(X)}=\gany{(T(X) - T_I(X)/Z_I(X)}$ for encoding $T(X)=\enc{T}{\setN}$ of vector $\vecT\in \F^N$ using $O((\delta + |I|) \log^2 (\delta + |I|))$ $\F$-operations
and $O(\delta + |I|)$ $\mathbb{G}$-operations. Here, $\delta$ denotes the hamming distance
between vectors $\vecTbase$ and $\vecT$.
\end{theorem}
\begin{proof}
    Let $\vecT=\vecTbase+\vecTcache$ and thus $T(X)=\Tbasepoly{X}+\Tcachepoly{X}$.
    Define $K=I\cup \{j\in [N]: \vecTcache[\,j\,]\neq 0\}$ as a set which captures the indices where the current table $\vecT$ differs from the base $\vecTbase$,
    where we explicitly also include the lookup indices $I$ in $K$. For $j\in K$, let $\vecTcache[j]=\Delta t_j$. Then $\Tcachepoly{X}=\sum_{j\in K}\Delta t_j\mu_j(X)$.
    %By definition of $K$, $|K|\leq \delta +|I|$. So, we need to bound $\Gtwo$ operations by $O(|K|)$ and field operations by $O(|K| \log^2|K|)$\\
    %First of all note that:
    We write the quotient $Q(X)$ as:
        {\small
    \begin{equation*}
    \begin{aligned}
    Q(X) = \sum_{i\in \setind}\frac{1}{z_I'(\xi^i)}\left(\frac{\Tbasepoly{X} - \Tbasepoly{\xi^i}}{X-\xi^i}\right)
     + \sum_{i\in \setind}\frac{1}{z_I'(\xi^i)}\left(\frac{\Tcachepoly{X} - \Tcachepoly{\xi^i}}{X-\xi^i}\right)
    \end{aligned}
    %\label{eq:Q2-upd}
    \end{equation*}
    }

    From above, we have $\gany{Q(x)}=\gany{\Qbasepoly{x}}+\gany{\Qcachepoly{x}}$ where
    \begin{gather*}
        \Qbasepoly{X}=\sum_{i\in \setind}(Z_I'(\xi^i))^{-1} (\Tbasepoly{X}-\Tbasepoly{\xi^i})/(X-\xi^i) \\
        \Qcachepoly{X}=\sum_{i\in \setind}(Z_I'(\xi^i))^{-1} (\Tcachepoly{X}-\Tcachepoly{\xi^i})/(X-\xi^i)
    \end{gather*}
    We can compute
    $\elttwo{\Qbasepoly{X}}$ from the pre-computed KZG openings of $\Tbasepoly{X}$ at points $\xi^i,i\in I$ using $O(|I|)$ group operations and
    $O(|I|\log^2 |I|)$ field operations. Therefore, it suffices to compute $\gany{\Qcachepoly{X}}$ efficiently.
    %\textbf{Thus, it suffices to describe the computation for $\elttwo{\Qcachepoly{X}}$. }\\
    Using $\Tcachepoly{X}=\sum_{j\in K}\Delta t_j\mu_j(X)$ and setting $c_i=(1/z_I'(\xi^i))$ for $i\in I$,
    we write $\Qcachepoly{X}$ as linear combination of table-independent polynomials:
    \begin{align*}
        \Qcachepoly{X} &= \sum_{i\in \setind} c_i\sum_{j\in K} \Delta t_j\frac{\mu_j(X)-\mu_j(\xi^i)}{X-\xi^i} \\
        &= \sum_{i\in \setind} c_i\Delta t_i\frac{\mu_i(X) - 1}{X-\xi^i} + \sum_{i\in \setind}\sum_{j\in K\setminus\{i\}}c_i\Delta t_j\frac{\mu_j(X)}{X-\xi^i}
    \end{align*}
    Now, we can write $\gany{\Qcachepoly{X}}=\elany{\Qcachepolyone{X}} + \elany{\Qcachepolytwo{X}}$ where:
        {\small
    \begin{gather*}
        \Qcachepolyone{X}=\sum_{i\in \setind}c_i\Delta t_i\frac{\mu_i(X)-1}{X-\xi^i},\,
        \Qcachepolytwo{X}=\sum_{i\in \setind}\sum_{j\in K\setminus \{i\}} c_i\Delta t_j\frac{\mu_j(X)}{X-\xi^i}
    \end{gather*}
    }
    The term $\elttwo{\Qcachepolyone{X}}$ can be computed using $O(|I|)$ group operations by augmenting the setup with pre-computed
    $\kzg$ opening proofs of polynomials $\mu_i(X)$ at $\xi^i$ for $i\in [N]$. This adds $O(N)$ to the setup parameters, while the computation
    can be done in $O(N\log N)$ time with methods similar to existing pre-computed parameters. This eventually leaves us with $\elany{\Qcachepolytwo{X}}$.
    %That is by precomputing $[\frac{\mu_i(X)-1}{X-\xi^i}]_2$. This requires just $N$ more precomputations and can be done along with the other precomputations which are done in the lookup protocol\\
    Next, we synthesize the polynomial $\Qcachepolytwo{X}$ in a form that reduces group operations required to compute its encoding.
    %\textbf{Thus, it suffices to describe the computation for $\elttwo{\Qcachepolytwo{X}}$}:
    \begin{align}\label{eq:Qcachepoly2}
    &\Qcachepolytwo{X} = \sum_{i\in \setind}c_i\sum_{j\in K\setminus \{i\}} \Delta t_j\mu_j(X)/(X-\xi^i) \nonumber \\
    &\quad = \sum_{i\in\setind}c_i\sum_{j\in K\setminus \{i\}}\frac{\Delta t_j}{Z_{\nroots}'(\xi^j)} \frac{Z_{\nroots}(X)}{(X-\xi^i)(X-\xi^j)} \nonumber \\
    %\intertext{Above, we expanded $\mu_j(X)$. Now using $Z_\nroots'(\xi^j)=N\xi^{-j}$ and using partial fractions}
    &\quad = N^{-1}\sum_{i\in\setind}c_i\sum_{j\in K\setminus \{i\}}\frac{\xi^j\Delta t_j}{\xi^i-\xi^j}
    \left(\frac{Z_\nroots(X)}{X-\xi^i} - \frac{Z_\nroots(X)}{X-\xi^j}\right) \nonumber \\
    &\quad = N^{-1}\sum_{i\in\setind}\left(c_i\cdot \sum_{j\in K\setminus \{i\}} \frac{\xi^j\Delta t_j}{\xi^i-\xi^j}\right)\frac{Z_\nroots(X)}{X-\xi^i} \nonumber \\
    &\qquad + \sum_{j\in K}\left(\xi^j\Delta t_j\cdot \sum_{i\in \setind\setminus \{j\}}\frac{c_i}{\xi^j-\xi^i}\right)\frac{Z_{\nroots}(X)}{X-\xi^j}
    \end{align}
    In the first step, we substituted $\mu_j(X)$, while in the final step we re-arranged the summation to accumulate the scalar factor for
    each distinct polynomial of the form $\vpolyN(X)/(X-\xi^i)$. Define scalars $a_i$, $i\in I$ and $b_j$, $j\in K$ as below:
    %this last equality, the first term is just the first term of the distributive property in finite fields.\\
    %The second term is just the second term of the distributive property in finite fields except that the order of the sums is reversed. This follows from the following fact \\
    %\begin{fact}
    %    $\sum_{i \in I} \sum_{j \in K \setminus \{i\}} f(i,j)=\sum_{j \in K} \sum_{i \in I \setminus \{j\}} f(i,j) $
    %\end{fact}
    %In the above equation \eqref{eq:Qcachepoly2}, let us define:
    \begin{gather}\label{eq:scalars}
        a_i = \sum_{j\in K\setminus \{i\}}\frac{\xi^j\Delta t_j}{\xi^i-\xi^j}, i\in \setind\quad
        b_j=  \sum_{i\in \setind\setminus \{j\}}\frac{c_i}{\xi^j - \xi^i}, j\in K
        %W_3^i(X) = \frac{Z_\nroots(X)}{X-\xi^i}, \text{ for } i\in [N]
    \end{gather}
    Now, recalling that
    $W_3^j=\gany{\vpolyN(X)/(X-\xi^i)}$, we see that $\elany{\Qcachepolytwo{X}}$ can be written as linear combination of $O(|K|+|I|)$ group elements.
    \begin{equation}\label{eq:Qcachepoly2commit}
    \elttwo{\Qcachepolytwo{X}} = N^{-1}\left(\sum_{i\in\setind}(c_ia_i)\cdot W_3^i + \sum_{j\in K}(\xi^j \Delta t_j b_j)\cdot W_3^j\right)
    \end{equation}
    Now $c_i=(z_I(\xi^i))^{-1}$ for all $i \in \setind$ can be determined in $O(|I|log^2 |I|)$ field operations by evaluating $Z_{\setind}'(X)$ on
    $H_{\setind}$. So, given $\{a_i\}_{i\in I}, \{b_j\}_{j\in K}$, $\elttwo{\Qcachepolytwo{x}}$ can be computed in $O(|\setind|+|K|)$ group operations.
    While we have diligently reduced the group operations, we still seem to need $O(|I||K|)=O(m\delta)$ field operations. We clearly need better than
    naive way of computing the scalars in \eqref{eq:scalars} to obtain additive overhead in $\delta$. This is what we consider next.
    Routine calculation shows that we can write $a_i$ as:
    \begin{equation*}
        a_i = -\Delta T + \Delta t_i + \xi^i\sum_{j\in K\setminus\{i\}}\frac{\Delta t_j}{\xi^i-\xi^j}, i\in I
    \end{equation*}
    where in the above, we have $\Delta T=\sum_{j\in K}\Delta t_j$.
    %a_i = -\sum_{j\in K\setminus \{i\}}\Delta t_j + \xi^i\sum_{j\in K\setminus \{i\}}\frac{\Delta t_j}{\xi^i-\xi^j} $$
    %This is because:
    %$$a_i+\sum_{j\in K\setminus \{i\}}\Delta t_j= \sum_{j \in K \setminus \{i\}}\frac{\xi^j\Delta t_j}{\xi^i-\xi^j}+\Delta t_j$$
    %$$=\sum_{j \in K \setminus \{i\}}\frac{\xi^i\Delta t_j}{\xi^i-\xi^j} = \xi^i\sum_{j\in K\setminus \{i\}}\frac{\Delta t_j}{\xi^i-\xi^j}$$
    %Now, define $\Delta T=\sum_{j\in K}\Delta t_j$\\

    %Here computing $\Delta T$ is a one time computation (per batch). It can be computed from the knowledge of $T_{\text{ch}}$ in the online phase.
    %We have:
    %$$  $$
    %Suppose we get $\sum_{j\in K\setminus\{i\}}\frac{\Delta t_j}{\xi^i-\xi_j}$ for all $i \in I$ efficiently. Then $a_i$ for all $i \in I$ can be obtained in $O(|I|)$ field operations. \\
    %\textbf{Thus, to get $a_i$ for all $i \in I$ it suffices to describe the computation of $e_i=\sum_{j\in K\setminus\{i\}}\frac{\Delta t_j}{\xi^i-\xi^j}$ for all $i \in I$}\\\\
    Above implies that to compute $a_i, i\in I$ efficiently, it is sufficient to efficiently
    compute $e_i=\sum_{j\in K\setminus\{i\}}\frac{\Delta t_j}{\xi^i-\xi^j}$ for all $i \in I$. Our next lemma claims that
    such {\em reciprocal sums} can be computed efficiently. The computation of $b_j,j\in K$ can also be reduced a similar computation.
    We defer this reduction and the full proof of Lemma ~\ref{lem:sum-computation} to the Appendix, but illustrate the key ideas in the proof.
    %Our requirement is now to bound the number of field operations for $e_i$ and for $b_j$. For this, we invoke the following lemma with the proof in the appendix.
    \begin{lemma}\label{lem:sum-computation}
    Let $I\subset K\subset [N]$ and let $e_i$ for all $i \in \setind$ and $b_j$ for all $j \in K$ be as described above.
    Then, $e_i$ for all $i \in I$ and $b_j$ for all $j \in K$ can be computed in $O(|K|\log^2|K|)\, \mathbb{F}$ operations
    \end{lemma}
    \begin{proof}[Proof-Sketch]
        First, we mention that the special case of the lemma when $\Delta t_j=1$ for all $j\in K$ admits an efficient computation due to the following identity
        proved in Lemma ~\ref{lem:sumtoder}.
        \begin{equation*}
            \frac{Z_K''(\xi^i)}{Z_K'(\xi^i)} = 2\sum_{j\in K\setminus \{i\}}\frac{1}{\xi^i-\xi^j}
        \end{equation*}
        for $Z_K(X)=\prod_{i\in K}(X-\xi^i)$. The polynomial $Z_K$ can be computed in $O(|K|\log^2|K|)$ and subsequent evaluations of its first two
        derivatives can also be evaluated on the set $\{\xi^i: i\in I\}$ with the same complexity. However, to deal with arbitrary values of $\Delta t_j$ we
        need more ingenuity. We will {\em imagine} $\Delta t_j$ to be $p(\xi^j)$ for some polynomial $p(X)$. Moreover, we demand that $p(\xi^j)=0$ for $j\not\in K$.
        We will not compute such a polynomial $p$, as it has degree $O(N)$, but view it as an ``oracle'' which we can hopefully query at the points we need.
        Then it can be seen that $e_i=g_i(\xi^i) - r_i(\xi^i)$ for rational functions $g_i(X)$ and $r_i(X)$ defined by:
        \begin{align}\label{eq:rat-fun-f}
        g_i(X) &=\sum_{j\in [N]\setminus i}\frac{p(X)}{X-\xi^j} \nonumber \\
        r_i(X) &=\sum_{j\in [N]\setminus i} \frac{p(X)-p(\xi^j)}{X-\xi^j}
        \end{align}
        Now, $g_i(\xi^i)$ for $i\in I$ turns out to be (using the special case above):
        $$p(\xi^i)\sum_{j\in K\setminus \{i\}} 1/(\xi^i-\xi^j)=\Delta t_i (Z_K''(\xi^i)/Z_K'(\xi^i))/2$$

    \end{proof}

    From the lemma \ref{lem:sum computation}, we have shown that the field operations needed to get $e_i$ and $b_j$ and thus $\elttwo{\Qcachepolytwo{X}}$ is $O(|K|\log^2|K|)$. \\

    This completes the proof of Lemma \ref{lem:approx-setup}.
\end{proof}

\subsection{Amortized Analysis of the update protocol}
Recall that we were able to get $[Q(X)]_2$ in $O(|K|)$ group operations and $O(|K|\log^2|K|)$ field operations. \\
For concrete analysis, let $s$ be the period after which the rebasing takes place. Also, the lookup happens at maximum of $m$ indices during a single batch. Thus, $|I|\leq m$.\\
This gives an upper bound on $\delta$, that is $ms$ and an upper bound on $K$, that is $ms+m=m(s+1)$.\\

Clearly $O(K)=O(ms)$ and $O(|K|\log^2|K|)=O(ms \log^2(ms))$ so group operations are $O(ms)$ and field operations are $O(ms\log^2(ms))$ \\

Moreover, after every $s$ batches, the rebasing(offline phase) is done which we know takes $O(N\log N)$ group and field operations.\\

So, the amortized number of operations for the offline and online phase in total is:
$O(ms \log^2(ms)+\frac{N\log N}{s})$ $\mathbb{F}$ operations and $O(ms +\frac{N\log N}{s})$ $\Gtwo$ operations\\

The value of $s$ which minimizes the group operations is $\sqrt{\frac{N}{m}}$. For this value of $s$:\\
\textbf{The amortized group operations needed are $\tilde{O}(\sqrt{mN})$}\\
\textbf{The amortized field operations needed are also $\tilde{O}(\sqrt{mN})$}\\
Here $\tilde{O}$ denotes that the polylog factors have been neglected\\\\

















\appendix
\section{More Preliminaries}

\subsection{Polynomial Commitment Scheme}
\label{sec:pcs_def}
The notion of a polynomial commitment scheme that allows the prover to open evaluations of the committed polynomial succinctly was introduced in~\cite{AC:KatZavGol10} who gave a construction under the trusted setup assumption.
A polynomial commitment scheme over $\F$ is a tuple $\pc = 
(\pcsetup,\pccommit,\pcopen,\pceval)$ where:

\begin{itemize}
    \item $\pcsetup(1^\secp,D) \rightarrow \pp$. On input security parameter $\secp$, and an upper bound $D \in \mathbb{N}$ on the degree, $\pcsetup$ generates public parameters $\pp$.
    
    \item $\pccommit(\pp, f(X),d) \rightarrow (C,\mathbf{\tilde{c}})$. On input the public parameters $\pp$, and a univariate polynomial $f(X) \in \F[X]$ with degree at most $d \leq D$, $\pccommit$ outputs a commitment to the polynomial $C$, and additionally an opening hint $\mathbf{\tilde{c}}$.
	
	\item $\pcopen(\pp,f(X),d,C,\mathbf{\tilde{c}}) \rightarrow b$. On input the public parameters $\pp$, the commitment $C$ and the opening hint $\mathbf{\tilde{c}}$, a polynomial $f(X)$ of degree $d \leq D$, $\pcopen$ outputs a bit indicating accept or reject. 
    
	    \item $\pceval(\pp,C,d,x,v;f(X)) \rightarrow b $. A public coin interactive protocol 
	    $\langle P_{\mathsf{eval}}(f(X)), V_{\mathsf{eval}} \rangle(\pp,C,d,z,v)$ between a PPT prover and a PPT verifier. The parties have as common input public parameters $\pp$, commitment $C$, degree $d$, evaluation point $x$, and claimed evaluation $v$. The prover has, in addition, the opening $f(X)$ of $C$, with $\deg(f) \leq d$. At the end of the protocol, the verifier outputs $1$ indicating accepting the proof that $f(x)=v$, or outputs $0$ indicating rejecting the proof.
		
		\end{itemize}
		
A polynomial commitment scheme must satisfy completeness, binding and extractability.

\begin{definition}[Completeness]
\label{def:pcs-comp}
For all polynomials $f(X) \in \F[X]$ of degree $d \leq D$, for all $x \in \F$,
\[
\Pr \left( 
\begin{matrix}
 %\pccheck(\pp,C,d,z,v,\pi) = 1 
 b=1
\end{matrix}
 \,:\,
 \begin{matrix}
\pp \leftarrow \pcsetup(1^\secp,D) \\
 (C,\mathbf{\tilde{c}}) \leftarrow \pccommit(\pp, f(X),d) \\
 v \leftarrow f(x) \\
 b \leftarrow \pceval(\pp,C,d,x,v;f(X))
 %\pi \leftarrow \pceval(\pp,f(X),d,z) \\
\end{matrix}
 \right) = 1.
 \]
\end{definition}

\begin{definition}[Binding] 
\label{def:pcs-binding-app}
A polynomial commitment scheme $\pc$ is binding if for all PPT $\Adv$, the following probability is negligible in $\secp$:
\[
\Pr \left( 
\begin{matrix}
 \pcopen(\pp,f_0,d,C,\mathbf{\tilde{c}_0}) =1 \wedge \\
 \pcopen(\pp,f_1,d,C,\mathbf{\tilde{c}_1}) = 1 \wedge \\
 f_0 \neq f_1
\end{matrix}
 \,:\,
 \begin{matrix}
\pp \leftarrow \pcsetup(1^\secp,D) \\
 (C,f_0,f_1,\mathbf{\tilde{c}_0}, \mathbf{\tilde{c}_1},d) \leftarrow \Adv(\pp) 
\end{matrix}
 \right).
 \]
\end{definition}

\begin{definition}[Extractability]
\label{def:pcs-ext-app}
For any PPT adversary $\Adv = (\Adv_{1},\Adv_{2})$, there exists a PPT algorithm $\ext$ such that the following probability is negligible in $\secp$:
	 \[
    \Pr\left(
      \begin{matrix}
      %\pccheck(\pp,C,d,z,v,\pi) = 1 \wedge
      b = 1 \wedge
      \R_{\pceval}(\pp,C,x,v; \tilde{f},\mathbf{\tilde{c}}) = 0
      \end{matrix}
      \,:\,
      \begin{matrix}
         \pp \leftarrow \pcsetup(1^\secp,D) \\
          (C,d,x,v,\mathsf{st}) \leftarrow \Adv_{1}(\pp) \\
         (\tilde{f},\mathbf{\tilde{c}}) \leftarrow \ext^{\Adv_{2}}(\pp)\\
         b \leftarrow \langle \Adv_{2}(\mathsf{st}), V_{\mathsf{eval}} \rangle(\pp,C,d,x,v)
      \end{matrix}
    \right).
  \]
  where the relation $\R_{\pceval}$ is defined as follows:
\begin{align*}
        \R_{\pceval} &= \{\left((\pp,C \in \mathbb{G},\; x \in \F, \; v \in \F);\; (f(X), \mathbf{\tilde{c}}) \right) : \\
        &\qquad (
        \pcopen(\pp,f,d,C,\mathbf{\tilde{c}}) = 1 )
         \land v = f(x)  \}
    \end{align*} 
    
\end{definition}

We denote by $\mathsf{Prove}, \mathsf{Verify}$, the non-interactive prover and verifier algorithms obtained by applying FS to the 
$\pceval$ public-coin interactive protocol.

\begin{definition}[Succinctness]
\label{def:pcs-succinct}
We require the commitments and the evaluation proofs to be of size independent of the degree of the polynomial, that is the scheme is \emph{proof succinct} if
$|C|$ is $\mathsf{poly}(\secp)$, $|\pi|$ is $\mathsf{poly}(\secp)$ where $\pi$ is the transcript obtained by applying FS to $\pceval$. Additionally, the scheme is \emph{verifier succinct} if $\pceval$ runs in time
$\mathsf{poly}(\secp) \cdot \allowbreak \mathsf{log} (d)$ for the verifier.
\end{definition}


\subsection{Succinct Argument of Knowledge}
\label{sec:aok}
\section{Argument for RAM From Polynomial Protocols}\label{sec:poly-proto-ram-app}
In this section, we give a self-contained argument of knowledge for membership in the language
$\LRAM{I}{m}{m}$ introduced in Section \ref{sec:model-for-ram}. We first consider the polynomial encoding
of different RAM artefacts.
\subsection{Polynomial Encoding}\label{subsec:poly-encoding}
%The aim of this section is to encode artefacts such as RAM state, operations and transcripts as polynomials, and
%translate checking memory consistency (equivalently, checking membership in $\LRAM{I}{n}{m}$) to checking polynomial identities
%over the encoded polynomials. For
%simplicity and its usefulness later, we consider the case $m=n$, and accordingly check the membership in the language $\LRAM{I}{m}{m}$.
Let $k=3m$ and let $\omega$ be a primitive $k^{th}$ root of unity in $\F$.
Let $\nu=\omega^3$, and thus $\nu$ is a primitive $m^{th}$ root of unity in $\F$ (We assume, these roots exist in $\F$).
We recall $\setV$ as the subgroup consisting of $m^{th}$ roots of unity with associated Lagrange basis polynomials
$\{\tau_i(X)\}_{i\in [m]}$, while we additionally introduce the set $\setH$ of $k^{th}$ roots of unity with
$\{\lambda_i(X)\}_{i\in [k]}$ as the associated Lagrange polynomials.
\begin{gather}\label{eq:interpolation-sets}
\setH = \{\omega,\ldots,\omega^k\},\quad \setV = \{\nu,\ldots,\nu^m\}
\end{gather}
As before, we define the encoding of vectors in $\vec{f}\in \F^k$ as $\enc{f}{\setH}=\sum_{i\in [k]}f_i\lambda_i(X)$.
%Let $\{\lambda_i(X)\}_{i=1}^k$ be lagrange basis polynomials
%for the set $\setH$ and $\{\tau_i(X)\}_{i=1}^m$ be the lagrange polynomials for the set $\setV$ satisfying
%$\lambda_i(\omega^j)=\delta_{ij}$ for $i,j\in [k]$ and $\tau_i(\nu^j)=\delta_{ij}$ for $i,j\in [m]$.
%We use $\setH$ to encode a vector $\vec{f}=(f_1,\ldots,f_k)$ of size $k$ as the polynomial $f(X)\in \F_{<k}[X]$ such that $f(\omega^i)=f_i$ for $i\in [k]$.
%Similarly, we use $\setV$ to encode a vector
%$\vec{g}=(g_1,\ldots,g_m)$ of size $m$ as the polynomial $g(X)\in F_{<m}[X]$ such that $g(\nu^i)=g_i$ for $i\in [m]$. We use
%the notation $\enc{f}{\setH}$ and $\enc{g}{\setV}$ to denote polynomial encodings of vectors $\vec{f}\in \F^k$ and $\vec{g}\in \F^m$
%respectively.
%In the other direction, for a polynomial $f(X)\in \F[X]$,
%we use $\vec{f}_{|_\setH}$ and $\vec{f}_{|_\setV}$ to denote the vectors $(f(\omega^1),\ldots,f(\omega^k))$ and $(f(\nu^1),\ldots,f(\nu^m))$ respectively.
%The encoding of vectors as polynomials can be succinctly described using lagrange basis polynomials.
%Then we have $\enc{f}{\setH}=\sum_{i=1}^k f_i\lambda_i(X)$ and $\enc{g}{\setV}=\sum_{i=1}^m g_i\tau_i(X)$.
We canonically extend the encoding of vectors to encode RAM, operations and transcripts by encoding their component vectors.
Thus, for a RAM $\vecT=(\vec{a},\vec{v})\in \RAM{I}{m}$, we define its encoding
$\wt{T}=(a(X),v(X))$ where $a(X),v(X)\in \F_{<m}[X]$ encode vectors $\vec{a}, \vec{v}$ respectively.
Given an operation sequence
$\vec{o}=(o_1,\ldots,o_m)$ with $o_i=(\bar{\op}_i,\bar{a}_i,\bar{v}_i)$ we encode $\vec{o}$ as
$\wt{O}$ $=$ $(\bar{\op}(X)$ ,$\bar{a}(X)$ ,$\bar{v}(X))$
where $\bar{\op}(X)$ encodes the
vector $\vec{\op}=(\bar{\op}_1,\ldots,\bar{\op}_m)$, $\bar{a}(X)$ encodes the vector $(\bar{a}_1,\ldots,\bar{a}_m)$ and
$\bar{v}(X)$encodes the vector $(\bar{v}_1,\ldots,\bar{v}_m)$.
Finally, a transcript $\tr=(\vec{t},\vec{\op},\vec{A},\vec{V})$ for tuples $(\vecT,\vec{o},\vecT')$ where $\vecT,\vecT'$ are RAMs of size $m$,
and $\vec{o}$ is an operation sequence of size $m$ is encoded as $\wt{\tr}$ $=$ $(t(X)$, $\op(X)$, $A(X)$, $V(X))$
where the polynomials $t(X),\op(X),V(X)$ and $A(X)$ encode the respective vectors in $\F^k$ (See Section \ref{sec:model-for-ram}).

\subsection{Relations over Polynomial Encodings}\label{subsec:encoded-relations}
In this section, we describe polynomial checks for two important relations we need in subsequent sections, viz,
(i) checking concatenation of vectors and (ii) checking monotonicity and load-store consistency of a transcript.
The lemma below specifies the polynomial identities for verifying
that vector $\vec{v}\in \F^k$ is concatenation of vectors $\vec{a},\vec{b},\vec{c}$ in $\F^m$.

\begin{lemma}\label{lem:vec-concatenation}
Let $\vec{a},\vec{b},\vec{c}\in \F^m$  and $v\in \F^k$ be vectors encoded by polynomials
$a(X),b(X),c(X)$ and $v(X)$ respectively. Then,
\begin{align}
    a(X^3) - v(X)  &= 0  \quad \text{ mod $Z(X)$ } \tag{A1}\label{eq:A1}\\
    b(X^3) - v(\omega^m X)  &= 0   \quad \text{ mod $Z(X)$ } \tag{A2}\label{eq:A2}\\
    c(X^3) - v(\omega^{2m} X) &= 0 \quad \text{ mod $Z(X)$ } \tag{A3}\label{eq:A3}
\end{align}
for $Z(X)=\prod_{i=1}^m (X-\omega^i)$ if and only if $\vec{v}=\vec{a}||\vec{b}||\vec{c}$.
\end{lemma}
\begin{proof}
    Assume that the polynomial identities hold. Substituting $X=\omega^i$ for $i\in [m]$ in above equations implies
    for $i\in [m]$: $a_i=v_i$ (Eq \eqref{eq:A1}), $b_i=v_{m+i}$ (Eq \eqref{eq:A2}) and $c_i=v_{2m+i}$ (Eq \eqref{eq:A3}),
    which together imply $\vec{v}=\vec{a}||\vec{b}||\vec{c}$. Converse follows by observing that $\vec{v}=\vec{a}||\vec{b}||\vec{c}$
    implies that $v(X) = a(X^3)$, $v(\omega^m X)=b(X^3)$ and $v(\omega^{2m} X)=c(X^3)$ holds for all $X=\omega^i, i\in [m]$.
    Thus, the equalities hold modulo the polynomial $Z(X)$ as defined above.
\end{proof}


Next, we specify polynomial checks on the encoding of a transcript to ensure it satisfies address-ordering and load-store consistency.
Let $N$ be an upper bound on the values of $\vec{A}$, i.e, the index set $\setind\subseteq [N]$.
Let $\tr=(\vec{t},\vec{\op},\vec{A},\vec{V})$ be a transcript encoded as
$\wt{\tr}=(t(X),\op(X),A(X),V(X))$. Recall that we need to check two conditions on $\tr$, viz, (i) {\em monotonicty}:
the transcript is sorted by address and timestamp respectively, i.e, $A_i\leq A_{i+1}$ for all $i < k$ and
$t_i < t_{i+1}$ whenever $A_i=A_{i+1}$, (ii) {\em load-store consistency}: whenever $\op_{i+1}=0$ and $A_i=A_{i+1}$,
we have $V_i=V_{i+1}$.
To do so, we exhibit disjoint sets $I_1,I_2$ with $I_1\uplus I_2=[k-1]$ such that: (i) for all
$i\in I_1$, $A_i < A_{i+1}$, (ii) for all $i\in I_2$, $(A_i = A_{i+1})\wedge (t_i < t_{i+1})$ and (iii) for all $i\in I_2$,
$(\op_i=1)\vee (V_i = V_{i+1})$.
Note that the conditions on the sets $I_1$ and $I_2$ ensures monotonicity.
Moreover, it can be seen that load-store consistency requirements are satisfied for all $i\in I_1$ (as $A_i\neq A_{i+1}$).
Similarly,load-store consistency also holds for all $i\in I_2$.
It remains to exhibit the sets and show that they satisfy the above invariants using polynomials, as in the following
lemma:
\begin{lemma}\label{lem:addr-ordered-transcript}
Let $\wt{\tr}$ be a polynomial encoding of transcript $\tr$ of size $k$, given by polynomials $t(X),\op(X),A(X)$ and $V(X)$,
with index set $[N]$. Then assuming $kN<|\F|$, $\tr$ is address ordered and satisfies load-store consistency if and only if there exist polynomials
$Z_1,Z_2,\delta_T,\delta_A$
such that the following hold:
\begin{align}%\label{eq:loadstore-consitency-constraints}
    & A(\omega X) - A(X) - \delta_A(X)= 0 \text{ mod }\, \mathbb{Z}_1(X) \tag{C1} \label{eq:C1} \\
    & A(\omega X) - A(X) = 0  \text{ mod } Z_2(X) \tag{C2} \label{eq:C2} \\
    & t(\omega X) - t(X) - \delta_T(X) = 0  \text{ mod } Z_2(X) \tag{C3} \label{eq:C3} \\
    & (\op(X) - 1)(V(\omega X) - V(X)) = 0  \text{ mod } Z_2(X) \tag{C4} \label{eq:C4} \\
    & Z_1(X)\cdot Z_2(X)\cdot (X-1) = \mathbb{Z}_\setH(X) \quad  \tag{C5} \label{eq:C5} \\
    & 1\leq A(\omega^i) \leq N  \quad \tag{C6} \label{eq:C6} \\
    & 1\leq t(\omega^i) \leq N, 1\leq \delta_A(\omega^i)\leq N,\ 1\leq \delta_T(\omega^i)\leq N \text{ for } i\in [k] \tag{C7} \label{eq:C7}
\end{align}
\end{lemma}
\begin{proof}
    Suppose there exist polynomials $Z_1(X),Z_2(X),\delta_T(X)$ and $\delta_A(X)$ satisfying above identities. From Equation
    ~\eqref{eq:C5}, we conclude that their exist sets $I_1,I_2$ with $I_1\uplus I_2=[k-1]$ such that $Z_b(X)$, $b\in \{1,2\}$ is the
    vanishing polynomial of the set $\{\omega^i: i\in I_b\}$. We now note that the following are true for $i\in I_1$:
    \begin{itemize}[leftmargin=1em]
        \item $A(\omega^{i+1})-A(\omega^i)=\delta_A(\omega^i)$. Since $1\leq \delta_A(\omega^i)\leq N$, this ensures $A_i < A_{i+1}$ for the vector $\vec{A}$ encoded
        by $A(X)$. We note that $kN < |\F|$ implies there is no overflow modulo the field characteristic.
    \end{itemize}
    Similarly, it can be seen that for $i\in I_2$, we must have (i) $A_i=A_{i+1}\wedge t_i < t_{i+1}$
    and (ii) $\op_i=1\,\vee\,V_i=V_{i+1}$. Together these imply that the encoded transcript is address-ordered.
\end{proof}

Protocols facilitating the checks mentioned in Lemma~\ref{lem:vec-concatenation} and Lemma~\ref{lem:addr-ordered-transcript} are presented in Figure \ref{fig:concatenation} and \ref{fig:encoded-relations} respectively.

\begin{figure}[htbp]
	\begin{mdframed}
		{
			{\bf Common Input}: Commitments $c_a$, $c_b$, $c_c$, $c_v$, and $\gone{Z}$ (to the polynomial
			$Z(X)=\prod_{i=1}^m (X-\omega^i)$). \\
			{\bf Prover's Input}: Vectors $\vec{a},\vec{b},\vec{c}\in\F^m$ and $\vec{v}\in \F^k$.
			\begin{enumerate}[leftmargin=1em, label=\arabic*.]
				\item $\verifier$ sends $\gamma\gets \F$.
				\item $\prover$ computes:
				\begin{align}
					& h(X) = a(X) + \gamma b(X) + \gamma^2 c(X),\\
					& Q(X) = (h(X^3) - v(X) - \gamma v(\omega^m X) - \gamma^2 v(\omega^{2m} X))/Z(X).
				\end{align}
				\item $\prover$ sends commitment $\gone{Q}$ = $\gone{Q(X)}$.
				\item $\verifier$ sends $s\gets\F$.
				\item $\prover$ sends evaluations $\val{s}{v}=v(s)$, $\val{\omega^m s}{v}=v(\omega^m s)$,
				$\val{\omega^{2m}s}{v}=v(\omega^{2m} s)$, $\val{s^3}{h}=h(s^3)$, $\val{s}{Q}=Q(s)$ and $\val{s}{Z}=Z(s)$.
				\item $\verifier$ sends $r\gets\F$.
				\item $\prover$ computes $\kzg$ proofs:
				\begin{itemize}[leftmargin=1em]
					\item $\Pi_v=\kzgprove(\srs,v,(s,\omega^m s, \omega^{2m}s))$.
					\item $\Pi_h=\kzgprove(\srs,h,s^3)$.
					\item $\Pi_f=\kzgprove(\srs,f,s)$ where $f(X)=Z(X) + rQ(X)$.
				\end{itemize}
				\item $\prover$ sends $\Pi_v$, $\Pi_h$ and $\Pi_f$.
				\item $\verifier$ Computes commitments $\gone{h}$ and $\gone{f}$ using homomorphism.
				\item $\verifier$ checks:
				\begin{itemize}[leftmargin=1em]
					\item $\kzgverify(\srs,\gone{v}, \vec{e}_v, \vec{p}_v, \Pi_v)$ where
					$\vec{p}_v$ = $(s,\omega^m s, \omega^{2m}s)$ and $\vec{e}_v$ = $(\val{s}{v}, \val{\omega^m s}{v}, \val{\omega^{2m} s}{v})$.
					\item $\kzgverify(\srs,\gone{h},  \val{s^3}{h}, s^3,\Pi_h)$.
					\item $\kzgverify(\srs,\gone{f},  \val{s}{Z} + r\val{s}{Q}, s,\Pi_f)$.
					\item $\val{s}{Q}\cdot \val{s}{Z} = \val{s^3}{h}-\val{s}{v}-\gamma \val{\omega^m s}{v}-\gamma^2\val{\omega^{2m}s}{v}$.
				\end{itemize}
				\item $\verifier$ outputs accept if all the above checks succeed, else it rejects.
			\end{enumerate}
		}
	\end{mdframed}
	%\vspace*{-5mm}
	\caption{Check concatenation over committed vectors.}
	\label{fig:concatenation}
\end{figure}

\begin{figure}[htbp]

    \begin{mdframed}
    {
            {\bf Common Input}: Commitments $c_t$, $c_{op}$, $c_A$ and $c_V$ to $\vec{t},\vec{op},\vec{A}$ and $\vec{V}$ constituting the transcript $\tr$.

            {\bf Prover's Input}: $\tr=(\vec{t}, \vec{op}, \vec{A}, \vec{V})$ and its polynomial encoding $\wt{\tr}=(t(X), \op(X), A(X), V(X))$.
        \begin{enumerate}[leftmargin=1em, label=\arabic*]
            \item Prover determines sets $I_1, I_2$ as described in Appendix A.2.
            \item Prover computes polynomials $Z_1(X)$, $Z_2(X)$, $\delta_T(X)$, $\delta_A(X)$.
            \item $\prover$ sends $\gone{Z_1(X)}$, $\gone{Z_2(X)}$, $\gone{\delta_T(X)}$, $\gone{\delta_A(X)}$.
            \item $\verifier$ sends $\gamma\gets \F$.
            \item $\prover$ computes:
            $$Q_1(X) =  (A(\omega X) - A(X) - \delta_A(X)) / \Z_1(X)$$
            $$Q_2(X) =  [(A(\omega X) - A(X))+\gamma(t(\omega X) - t(X) - \delta_T(X))+$$
            $$ \gamma^2 (\op(X) - 1)(V(\omega X) - V(X))] / Z_2(X) $$


            \item $\prover$ sends commitments $\gone{Q_1(X)}$, $\gone{Q_2(X)}$.
            \item $\verifier$ sends $s\gets\F$.
            \item $\prover$ sends evaluations $\val{s}{A}=A(s)$, $\val{\omega s}{A}=A(\omega s)$, $\val{s}{\delta_A} = \delta_A(s)$, $\val{s}{t}=t(s)$, $\val{\omega s}{t}=t(\omega s)$, $\val{s}{\delta_T} = \delta_T(s)$, $\val{s}{\op}=\op(s)$, $\val{s}{V}=V(s)$, $\val{\omega s}{V}=V(\omega s)$, $\val{s}{Q_1}=Q_1(s)$, $\val{s}{Q_2}=Q_2(s)$, $\val{s}{Z_1}=Z_1(s)$, $\val{s}{Z_2}=Z_2(s)$.
            \item $\verifier$ checks:
            \begin{itemize}
                \item $\val{s}{Q_1}\cdot \val{s}{Z_1} = (\val{\omega s}{A} - \val{s}{A} - \val{s}{\delta_A})$.
                \item  $\val{s}{Q_2}\cdot \val{s}{Z_2} = (\val{\omega s}{A} - \val{s}{A})+\gamma (\val{\omega s}{t} - \val{s}{t} - \val{s}{\delta_T})+\gamma^2 (\val{s}{\op}-1)(\val{\omega s}{V} - \val{s}{V}) $.
                \item $\val{s}{Z_1} \cdot \val{s}{Z_2}=s^{3m}-1$.
            \end{itemize}
            \item $\verifier$ sends $r_1, r_2\gets\F$.
            \item $\prover$ computes:
            \begin{itemize}
                \item $\Phi_{ws}(X)= A(X)+r_1 t(X) +r_1^2 V(X)$.
                \item $\Phi_s(X)=A(X)+r_2 \delta_A(X)+r_2^2 t(X)+ r_2^3 \delta_T(X)+r_2^4 \op(X) +r_2^5 V(X)+r_2^6 Q_1(X)+ r_2^7 Q_2(x)+r_2^8 Z_1(X)+r_2^9 Z_2(X)$.
                \item $\prover$ computes $\Pi_{\omega s} = \KZGopen(\srs, \Phi_{\omega s}(X), \omega s)$.
                \item $\prover$ computes $\Pi_{s} = \KZGopen(\srs, \Phi_{s}(X),s)$.

            \end{itemize}
            \item $\prover$ sends $\Pi_{\omega s}, \Pi_{s}$.
            \item $\verifier$ computes:
            \begin{itemize}
                \item $[\Phi_{\omega s}(X)]_1= c_A +r_1 c_t + r_1^2 c_V$.
                \item $[\Phi_{s}(X)]_1= c_A +r_2 [\delta_A(X)]_1 + r_2^2 c_t +r_2^3 [\delta_T(X)]_1 +r_2^4 c_{\op} +r_2^5 c_V +r_2^6 [Q_1(X)]_1 +r_2^7 [Q_2(X)]_1 +r_2^8 [Z_1(X)]_1 + r_2^9 [Z_2(X)]_1 $.
                \item $V_{ws}= \val{\omega s}{A}+r_1 \val{\omega s}{t}+ r_1^2 \val{\omega s}{V}$.
                \item $V_s=\val{s}{A}+r_2 \val{s}{\delta_A}+r_2^2 \val{s}{t}+r_2^3 \val{s}{\delta_T}+r_2^4 \val{s}{\op}
                +r_2^5 \val{s}{V} +r_2^6 \val{s}{Q_1} +r_2^7 \val{s}{Q_2}+r_2^8 \val{s}{Z_1} +r_2^9 \val{s}{Z_2}$.
            \end{itemize}
            \item $\verifier$ checks:
            \begin{itemize}
                \item $\KZGverify(\srs, \gone{\Phi_{ws}}, V_{ws}, \omega s, \Pi_{\omega s})$.
                \item $\KZGverify(\srs, \gone{\Phi_{s}}, V_{s}, s, \Pi_{s})$.
            \end{itemize}

            \item $\prover$ and $\verifier$ invoke sub-vector arguments $(\subvecprover,\subvecverifier)$ (eg. ~\cite{EPRINT:EagFioGab22})
            to prove that $(\srs,c_A,c_I)$, $(\srs,c_t,c_I)$, $(\srs,\gone{\delta_A(X)},c_I)$ and $(\srs,\gone{\delta_T(X)},c_I)$
            are in $\RSVEC$.

            \item $\verifier$ accepts if all checks succeed and the sub-vector arguments accept. Otherwise it rejects.
        \end{enumerate}
    }
    \end{mdframed}

    \caption{Check that transcript is address ordered and load-store consistent.}
    \label{fig:encoded-relations}
\end{figure}

\section{Succinct Argument for Verifiable RAM}\label{sec:poly-proto-ram}
The polynomial encodings in the previous section can be used to polynomial protocol for
checking the membership in the language $\LRAM{I}{m}{m}$ for $m\in \N$. The polynomial protocol can be subsequently
be compiled into a succinct argument using an extractable polynomial commitment scheme.
In this section, we use $\kzg$ polynomial commitment scheme to obtain a succinct argument for checking membership in $\LRAM{I}{m}{m}$
in the Algebraic Group Model (AGM).
At a high level, to prove $(\vecT,\vec{o},\vecT')\in \LRAM{I}{m}{m}$, the prover
constructs time ordered transcript $\tr$ and then permutes it to obtain the address sorted transcript $\tr^\ast$.
It then sends the polynomial encodings of $\vecT,\vec{o},\vecT',\tr$ and $\tr^\ast$ to the verifier, who verifies that:
\begin{enumerate}[leftmargin=2em]
    \item The time ordered transcript is correctly constructed, i.e, $\tr=\TOT(\vecT,\vec{o},\vecT')$.
     This is achieved using the protocol in Figure~\ref{fig:time-ordered-transcript}.
    \item The transcript $\tr^\ast$ is a permutation of the transcript $\tr$, i.e, $\tr^\ast=\sigma(\tr)$ for some permutation $\sigma$ of $[k]$.
    The protocol for this check appears in Figure~\ref{fig:permutated-transcripts}.
    \item The transcript $\tr^\ast$ is address ordered and satisfies load-store consistency. We describe the protocol
    to check this property of transcripts in Figure~\ref{fig:encoded-relations}.
\end{enumerate}
We check above conditions over commitments. Let $\srs$ denote a $\kzg$ setup over a bi-linear group, with
prime order groups $\Gone, \Gtwo$ and $\GT$. We canonically commit to RAM, operation sequences and transcripts by committing to their
polynomial encodings. Commitment of an encoding represented as tuple of polynomials is simply the tuple consisting of commitments of the component
polynomials. We now define
the relation $\CLRAM$ below, and present a succinct argument for the same.
\begin{definition}\label{defn:committed-vram}
Let $\CLRAM$ consist of tuples $((c_T, c_o, c_T')$, $(\vecT, \vec{o},\vecT'))$ where $c_T$ $=$ $\kzgcommit(\srs, \wt{T})$,
$c_T'$ $=$ $\kzgcommit(\srs, \wt{T'})$,
$c_o$ $=$ $\kzgcommit(\srs,\wt{O})$ commit to $\vecT$, $\vecT'$ and $\vec{o}$ with $(\vecT,\vec{o},\vecT')\in \LRAM{I}{m}{m}$.
\end{definition}
\noindent In the above definition we have $c_T=(c_a,c_v)$ where $c_a$ and $c_v$ are $\kzg$ commitments to polynomials $a(X)$ and $v(X)$ in
the encoding $\wt{T}=(a(X), v(X))$. Similarly we parse $c_T'=(c_a', c_v')$ and $c_o=(\bar{c}_\op, \bar{c}_a, \bar{c}_v)$ (see Appendix
\ref{subsec:poly-encoding} for polynomial encodings).
For proving relation~\ref{defn:committed-vram}, prover's input consists of initial RAM state $\vecT=(\vec{a},\vec{v})$,
final RAM state $\vecT'=(\vec{a'},\vec{v'})$, operation sequence $\vec{o}=(o_1,\ldots,o_m)$ with $o_i=(\bar{\op}_i,\bar{a}_i,\bar{v}_i)$,
time-ordered transcript $\tr=(\vec{t},\vec{\op},\vec{A},\vec{V})$ and address-ordered transcript $\tr^\ast=(\vec{t^\ast},\vec{\op^\ast},
\vec{A^\ast},\vec{V^\ast})$ obtained from $\tr$ using a permutation $\sigma:[k]\rightarrow [k]$.
Verifier's input consists of the commitments $c_T, c_o$ and $c_T'$ as described above.

The prover starts the protocol by sending commitments $c_\tr$ and $c^\ast_\tr$ to the transcripts $\tr$ and $\tr^\ast$ respectively.
To show that $\tr$ is correctly formed, the prover needs to prove the concatenations:
(i) $\vec{\op}=0^m||(\bar{\op}_1,\ldots,\bar{\op}_m)||0^m$, (ii) $\vec{A}=\vec{a}||(\bar{a}_1,\ldots,\bar{a}_m)||\vec{a'}$
and (iii) $\vec{V}=\vec{v}||(\bar{v}_1,\ldots,\bar{v}_m)||\vec{v'}$. Note that the time-stamp column $\vec{t}$ is implicitly assumed
to be $(1,\ldots,k)$.
The verifier checks the concatenations using Lemma \ref{lem:vec-concatenation}.
It uses a random challenge $\beta$ to reduce the three concatenations to one concatenation, and uses another challenge $\gamma$
to reduce the three polynomial checks in Lemma \ref{lem:vec-concatenation} to a single check.
The complete polynomial protocol is detailed in Figure \ref{fig:time-ordered-transcript}.


\begin{table*}[bt]
    \begin{tabular}{|l|l|l|l|l|}
        \hline
        {\bf Component  }                                                                                      & {\bf Protocol} & {\bf Prover Work}      & {\bf Verifier Work} & {\bf Communication}   \\ \hline
        Concatenation of transcripts                                                                    & Figure ~\ref{fig:time-ordered-transcript}    & $O(m \log m)\,\F,\, O(m)\,\Gone$      & $2P$            & $4\Gone$, $6\F$         \\ \hline
        Permutation of transcripts                                                                      & Figure ~\ref{fig:permutated-transcripts}   & $O(m \log m)\,\F,\, O(m)\,\Gone$       & $2P$            & $4\Gone$, $5\F$         \\ \hline
        \begin{tabular}[c]{@{}l@{}}Memory consistency \& \\ address ordering of transcript\end{tabular} & Figure ~\ref{fig:encoded-relations}    & $O(m \log m)\,\F,\, O(m)\,\Gone$       & $6P$            & $20\Gone$,$19\F$        \\ \hline
        \rowcolor{lightgray}
        Polynomial Protocol for RAM                                                                     & Figure ~\ref{fig:covering-protocol}   & $O(m \log m)\,\F,\, O(m)\,\Gone$      & $7P$            & $36\Gone$, $30\F$       \\ \hline

    \end{tabular}
    \caption{Efficiency parameters for components of polynomial protocol for RAM. Here $m$ denotes both the size of
    the RAM and number of operations (the special case we consider). $P$ denotes a pairing evaluation, while $\Gone$
    $\Gtwo$ and $\F$ denote the groups and the scalar field of the bilinear group used for instantiating the protocol.
    }
    \label{tbl:components-poly-ram}
    \vspace*{-5mm}
\end{table*}


%\begin{tcolorbox}
\begin{figure}[htbp]

    \begin{mdframed}
    {
            {\bf Common Input}: Commitments $c_T=(c_a,c_v)$, $c_o=(\bar{c}_\op, \bar{c}_a, \bar{c}_v)$, $c_T'=(c_a', c_v')$
        and $c_\tr=(c_t, c_\op, c_A, c_V)$ to $\vecT,\vec{o},\vecT'$ and $\tr$(which is supposed to be the time ordered transcript) respectively. Commitment $\gone{Z(X)}$ to the polynomial
        $Z(X)=\prod_{i=1}^m (X-\omega^i)$.\\
    {\bf Prover's Input}: $\tr, \vecT, \vecT', \vec{o}$ and their polynomial encodings, $Z(X)$.
        \begin{enumerate}[leftmargin=1em, label=\arabic*.]
            \item $\verifier$ sends $\beta,\gamma\gets \F$.
            \item $\prover$ computes:
            \begin{align}\label{eq:poly-constraints}
            & G_1(X) = a(X) + \beta v(X),\, G_2(X) = \bar{a}(X) + \beta \bar{v}(X) + \beta^2 \bar{\op}(X) \tag{A1} \\
            & G_3(X) = a^*(X) + \beta v^*(X),\, G(X) = A(X) + \beta V(X) + \beta^2 \op(X) \tag{A2} \\
            & H(X) = G_1(X) + \gamma G_2(X) + \gamma^2 G_3(X), \tag{A3} \\
            & Q(X) = [(H(X^3) - G(X) - \gamma G(\omega^m X) - \gamma^2 G(\omega^{2m} X))]/Z(X) \tag{A4}
            \end{align}
            \item $\prover$ sends commitment $\gone{Q}$ to $Q(X)$.
            \item $\verifier$ sends $s\gets\F$.
            \item $\prover$ sends evaluations $\val{s}{G}=G(s)$, $\val{\omega^m s}{G}=G(\omega^m s)$,
            $\val{\omega^{2m}s}{G}=G(\omega^{2m} s)$, $\val{s^3}{H}=H(s^3)$, $\val{s}{Q}=Q(s)$ and $\val{s}{Z}=Z(s)$.
            \item $\verifier$ sends $r\gets\F$.
            \item $\prover$ sends $\kzg$ proofs:
            \begin{itemize}[leftmargin=1em]
                \item $\Pi_G=\kzgprove(\srs,G(X),(s,\omega^m s, \omega^{2m}s))$.
                \item $\Pi_H=\kzgprove(\srs,H(X),s^3)$.
                \item $\Pi_F=\kzgprove(\srs,F(X),s)$ where $F(X)=Z(X) + rQ(X)$.
            \end{itemize}
            \item $\verifier$ computes $\gone{G(X)},\gone{H(X)}$ and $\gone{F(X)}$ using homomorphism.
            \item $\verifier$ checks:
            \begin{itemize}[leftmargin=1em]
                \item $\kzgverify(\srs,\gone{G}, (\val{s}{G}, \val{\omega^m s}{G}, \val{\omega^{2m} s}{G}),\\(s,\omega^m s, \omega^{2m}s),\Pi_G)$.
                \item $\kzgverify(\srs,\gone{H}, \val{s^3}{H}, s^3, \Pi_H)$.
                \item $\kzgverify(\srs,\gone{F}, \val{s}{Z} + r\val{s}{Q}, s, \Pi_F)$.
                \item $\val{s}{Q}\cdot \val{s}{Z} = \val{s^3}{H}-\val{s}{G}-\gamma \val{\omega^m s}{G}-\gamma^2\val{\omega^{2m}s}{G}$.
            \end{itemize}
            \item $\verifier$ accepts if all the above checks succeeds, otherwise it rejects.
        \end{enumerate}
    }
    \end{mdframed}
%    \vspace*{-5mm}
    \caption{Check the correctness of time-ordered transcript.}
    \label{fig:time-ordered-transcript}
\end{figure}
%\end{tcolorbox}

Next, we show a polynomial protocol for proving that the transcript $\tr^\ast$ is a permutation of the transcript $\tr$.
We first recall the permutation argument for vectors from ~\cite{EPRINT:GabWilCio19}.
\begin{lemma}[Permutation Check \cite{EPRINT:GabWilCio19}]\label{lem:perm-argument}
Let $f(X), g(X)$ be polynomials in $\F[\,X\,]$. Then, the vectors $\vec{f}, \vec{g}\in \F^k$ encoded by the polynomials
are permutations of each other if and only if with overwhelming probability over the choice of $\alpha\gets \F$,
there exists a polynomial $z(X)$ satisfying the polynomial constraints:

\begin{align}
    \lambda_1(X)(z(X) -1) &= 0 \text{ mod } Z_\setH(X) \tag{B1} \label{eq:B1} \\
    (\alpha - g(X))z(\omega X) &= (\alpha - f(X))z(X) \text{ mod } Z_\setH(X) \tag{B2} \label{eq:B2}
\end{align}

\end{lemma}
The polynomial protocol in Figure \ref{fig:permutated-transcripts} essentially invokes the above argument on
the random linear combination of the columns of the respective transcripts.
%\begin{tcolorbox}
\begin{figure}[htbp]

    \begin{mdframed}
    {
            {\bf Common Input}: Commitments $c_\tr=(c_t,c_\op,c_A, c_V)$ and $c_\tr^\ast=(c_t^\ast,c_\op^\ast, c_A^\ast, c_V^\ast)$
        of transcripts $\tr$ and $\tr^\ast$ respectively.\\
    {\bf Prover's Input}: Transcripts $\tr, \tr^{*}$ and their polynomial encodings, permutation $\sigma$ such that $\tr^{*}=\sigma(\tr)$.
        \begin{enumerate}[leftmargin=1em, label=\arabic*]
            \item $\verifier$ sends $\alpha,\beta, \chi\gets \F$.
            \item $\prover$ computes $f(X)=t(X) + \beta \op(X) + \beta^2 A(X) + \beta^3 V(X)$, $g(X) = t^\ast(X) + \beta \op^\ast(X)$
            $+ \beta^2 A^\ast(X) + \beta^3 V^\ast(X)$. It then computes polynomials $z(X),q(X)$ as:
            \begin{itemize}[leftmargin=1em]
                \item Interpolate polynomial $z(X)$ of degree $k-1$ such that $z(\omega)=1$ and
                $z(\omega^{i+1})=\prod_{j=1}^i (\alpha - f(\omega^j))/(\alpha - g(\omega^j))$ for $1\leq i\leq k-1$.
                \item Compute $q(X) = ((\alpha - g(X))z(\omega X) - (\alpha - f(X))z(X) + \chi\lambda_1(X)(z(X) - 1))/\mathbb{Z}_{\setH}(X)$.
            \end{itemize}
            \item $\prover$ sends commitments $\gone{z(X)}$ and $\gone{q(X)}$ to polynomials $z(X)$ and $q(X)$ respectively.
            \item $\verifier$ computes commitments $[f]_1, [g]_1$ by homomorphism.
            \item $\verifier$ checks that $q(X)Z_\setH(X)$ = $(\alpha - g(X))z(\omega X)-(\alpha - f(X))z(X) + \chi\lambda_1(X)(z(X) - 1)$
            by requesting evaluations and $\kzg$ proofs of polynomials $f,g, q, z$ at a random point, say $s$ and evaluation and $\kzg$ proof of $z$ at $\omega s$.
            \item $\verifier$ accepts if all the checks succeed, else it rejects.
        \end{enumerate}
    }
    \end{mdframed}
%    \vspace*{-5mm}
    \caption{Check that transcripts are permutations of each other.}
    \label{fig:permutated-transcripts}
\end{figure}

Finally, we see that Lemma~\ref{lem:addr-ordered-transcript} implies a polynomial protocol to check that the transcript
$\tr^\ast$ is address ordered and satisfies load-store consistency, which
essentially involves the prover identifying sets
$I_1, I_2$ as described in Appendix \ref{subsec:encoded-relations} and sending auxiliary polynomials $Z_1(X),Z_2(X),\delta^\ast_A(X)$ and $\delta^\ast_T(X)$ to the verifier.
The verifier then checks the identities (C1)-(C6) in Lemma \ref{lem:addr-ordered-transcript}.
The range checks in (C7) can be checked using polynomial protocols in sub-vector lookup arguments such as
~\cite{EPRINT:PosKat22,EPRINT:EagFioGab22,PKC:CFFLL24,PKC:ZSG24}.
The protocol (compiled using KZG commitments in AGM) can be found in Figure~\ref{fig:encoded-relations}.
The overall protocol for $\CLRAM$ which combines invokes protocols in Figures~\ref{fig:time-ordered-transcript},\ref{fig:permutated-transcripts}
and \ref{fig:encoded-relations} as sub-protocols is presented in Figure~\ref{fig:covering-protocol}.
\begin{figure}[htbp]

    \begin{mdframed}
    {
            {\bf Common Input}: Commitments $c_T=(c_a,c_v)$, $c_o=(\bar{c}_\op, \bar{c}_a, \bar{c}_v)$, $c_T'=(c_a', c_v')$.

            {\bf Prover's Input}: $ \vecT, \vecT', \vec{o}$ and their polynomial encodings.
        \begin{enumerate}[leftmargin=1em, label=\arabic*.]
            \item $\prover$ computes:
            \begin{itemize}
                \item $\tr$, which is the time ordered transcript corresponding to $\vecT, \vec{o}, \vecT'$, and its polynomial encoding.
                \item $Z(X)=\prod_{i=1}^m (X-\omega^i)$.
                \item $c_\tr=(c_t, c_\op, c_A, c_V)$ which is the commitment of $\tr$.
                \item $[Z(X)]_1$.
            \end{itemize}

            \item $\prover$ sends  $c_\tr=(c_t, c_\op, c_A, c_V)$ and $[Z(X)]_1$.
            \item $\prover$ and $\verifier$ run the protocol for checking correctness of time ordered transcript as in Figure~\ref{fig:time-ordered-transcript}.
            \item $\prover$ computes the address ordered transcript $\tr^{*}$ from the time ordered transcript $\tr$ by permuting suitably. It also computes its polynomial encoding.
            \item $\prover$ computes the permutation $\sigma$ such that $\tr^{*}= \sigma(\tr)$ (this is computed along with the $\tr^{*}$).
            \item $\prover$ computes the commitment $c_\tr^\ast=(c_t^\ast,c_\op^\ast, c_A^\ast, c_V^\ast)$ of $\tr^{*}$.
            \item $\prover$ sends  $c_\tr^\ast$.
            \item $\prover$ and $\verifier$ run the protocol for checking that the two transcripts are permutations of each other
            as in Figure~\ref{fig:permutated-transcripts}.
            \item $\prover$ and $\verifier$ run  the protocol for checking the constraints given in Lemma~\ref{lem:addr-ordered-transcript}
            as in Figure~\ref{fig:encoded-relations}.
            \item $\verifier$ accepts if all the three sub-protocols accept. $\verifier$ rejects otherwise.
        \end{enumerate}
    }
    \end{mdframed}
%    \vspace*{-5mm}
    \caption{Overall protocol for the relation $\CLRAM$}
    \label{fig:covering-protocol}
\end{figure}

\smallskip

\noindent{\bf Efficiency.}
We provide a break-up of costs incurred by different components involved in construction
of RAM based on memory-checking techniques in Table ~\ref{tbl:components-poly-ram}. To reduce pairing
checks we use standard technique of batching pairing checks involving common generators. In addition, to reduce
communication, instead of naively invoking four instances of sub-vector argument in Step 15 of the protocol
in Figure~\ref{fig:encoded-relations}, we concatenate the four vectors using a variant of protocol for
concatenation of vectors in Figure~\ref{fig:concatenation}, and then use the sub-vector argument to show that
the concatenated vector is a sub-vector of the vector $(1,\ldots,N)$. For CQ~\cite{EPRINT:EagFioGab22} based instantiation,
this reduces the total communication of this
check from $4\times (8\Gone + 3\F)$ to $(4\Gone+6\F) + (8\Gone + 3\F)$, a saving of $\approx 20\Gone$. The reported
overheads in Table ~\ref{tbl:components-poly-ram} take into account such optimizations.

\section{Proof of Lemma ~\ref{lem:sum-computation}}
%Recall the statement of lemma \ref{lem:sum computation}. We want to compute $e_i$ for all $i \in \setind$ and $b_j$ for all $j \in K$ in $O(|K|\log^2|K|)$ $\mathbb{F}$ operations.\\\\
Before starting the proof, we collect some preliminaries which will be useful in the proof.

\subsection{Computational Algebra Preliminaries}\label{subsec:comp-algebra-app}
Let $\F$ be a finite field of prime order $p$ and $\G$ be a cyclic additive group of order $p$ with generator $g$. For $s \in \F$, we use
the notation $\elt{s}$ to denote the group element $s\cdot g$. Assume that $\F$ contains $n^{th}$ root of unity $\xi$
satisfying $\xi^n=1$ for large $n$ (All polynomial degrees are assumed less than $n$).

\begin{fact}[\textsf{Fast Evaluation}]\label{fc:fft}
Let $f\in \F[X]$ be a polynomial of degree $<d$ and $(\xi_1,\ldots,\xi_r)\in \F^r$ be distinct points in $\F$.
Then the vector $(f(\xi_1),\ldots,f(\xi_r))$ can be computed in $O((d+r)\log (d+r))$ $\F$ operations if $\xi_1,\ldots,\xi_r$ form roots
of unity, and in $O((d+r)\log^2(d+r))$ $\F$ operations otherwise.
\end{fact}

\begin{fact}[\textsf{Fast Interpolation}]\label{fc:ifft}
Let $\xi_1,\ldots,\xi_d$ be distinct points in $\F$ and $(v_1,\ldots,v_d)\in \F^d$. Then $(f_0,\ldots,f_{d-1})\in \F^d$
can be computed in $O(d\log^2 d)$ operations in $\F$ such that $f(\xi_i)=v_i$ for all $i\in [d]$ where
$f(X)=\sum_{i=0}^{d-1}f_iX^i$.
\end{fact}

\begin{fact}[\textsf{Fast Multiplication}]\label{fc:mult}
Let $\xi_1,\ldots,\xi_r$ be distinct points in $\F$. Then coefficients of $f(X)=\prod_{i=1}^r (X-\xi_i)$
can be computed in $O(r\log^2 r)$ operations in $\F$.
\end{fact}

\begin{fact}[\textsf{Multi KZG proofs} ~\cite{EPRINT:FeiKho23}]\label{fc:multkzg}
Let $\{\elt{x^i}\}_{i=1}^d$ be given for some $x\in \F$. Then for set of $r$ distinct points $\xi_1,\ldots,\xi_r$,
and a polynomial $f(X)\in \F[X]$ of degree $<d$,the vector $(\elt{h_1(x)},\ldots,\elt{h_r(x)})$,
where $h_i(X) = (f(X) - f(\xi_i))/(X - \xi_i)$ can be computed in
$O((r+d)\log(r+d))$ group and field operations when $\xi_1,\ldots,\xi_r$ are roots of unity, and in
$O(rlog^2 r + d\log d)$ group and field operations otherwise.
\end{fact}

\begin{fact}[\textsf{Lagrange Polynomials}]\label{fc:lagrange}
Let $\mathbb{S}=\{\xi_1,\ldots,\xi_r\}$ be a set of $r$ distinct points and let $\tau_1(X),\ldots,\tau_r(X)$ be
the corresponding lagrange polynomials of degree $r-1$ each. Let $Z_{\mathbb{S}}(X)=\prod_{i=1}^r (X-\xi_r)$ denote the vanishing polynomial
for $\mathbb{S}$. Then we have:
\begin{gather*}
    \sum_{i=1}^r \tau_i(X) = 1 \\
    \tau_i(X) = \frac{Z_{\mathbb{S}}(X)}{Z_{\mathbb{S}}'(\xi_i)(X-\xi_i)} \text{ for all } i\in [r]
\end{gather*}
\end{fact}

\noindent{\bf Formal Derivative}: For a polynomial $p(X) \in \F[X]$, we define the formal derivative of $p(X)$ as the polynomial
$u(X,X)$ where $u(X,Y)=\frac{p(X)-p(Y)}{X-Y}$. It can be seen that $u(X,X)$ equals the polynomial $p'(X)$ obtained by differentiating
$p(X)$ according to regular rules of calculus. Thus, this definition agrees with the one given earlier in the preliminaries.
%Let us denote the formal derivative of $p(X)$ as $p^f(X)$.\\
%Then, $p^f(X)= u(X,X)$ where $u(X,Y)=\frac{p(X)-p(Y)}{X-Y}$\\\\
%It is well known that the formal derivative of a polynomial satisfies all the properties that we want the derivative of a polynomial to satisfy, like linearity, product rule, quotient rule etc.\\\\
%As a result, the formal derivative $p^f(X)$ defined as above can be computed from $p(X)$ in the \textit{usual} way, that is $\frac{d}{dx}X^k=k X^{k-1}$ and extend using linearity of derivative.\\
%Henceforth, we will represent the formal derivative of $p(X)$ as $p'(X)$ and because of the above statements, we will compute it in the usual way and freely use all the properties of derivatives on it.\\

\subsection{Some Useful Results}\label{subsec:sub-results}
We state and prove some facts which are used later throughout the proof.

\begin{lemma}\label{lem:sumtoder}
For $K\subset [N]$, define $H_K$ to be $\{\xi^i:i \in K\}$. Let $p(X)$ be the vanishing polynomial of $H_{K}$.
Let $p'(X)$ and $p''(X)$ denote the formal first derivative and second derivative of $p(X)$ respectively.
Then, $p''(\xi^i)/p'(\xi^i)=2 \cdot \sum_{j\in K\setminus \{i\}} 1/(\xi^i-\xi^j)$ for all $i \in K$
\end{lemma}

\begin{proof}
    Observe that $p'(X)=\sum_{i\in K}\prod_{j\in K\setminus \{i\}}(X-\xi^j)$ and \\
    $p''(X)=\sum_{i\in K}\sum_{j\in K\setminus \{i\}}\prod_{k\in K\setminus \{i,j\}}(X-\xi^k)$. Thus for $r\in K$,
    we have:
    \begin{align*}
        p'(\xi^r) &= \prod_{j \in K \setminus \{r\}}(\xi^r-\xi^j), \\
        p''(\xi^r)&=\sum_{j\in K\setminus \{r\}}\prod_{k\in K\setminus \{r,j\}} (\xi^r-\xi^k)
        +\sum_{i\in K\setminus \{r\}}\prod_{k\in K\setminus \{r,i\}} (\xi^r-\xi^k)
    \end{align*}
    Note that only non-zero products in the expansion of $p''(\xi^r)$ occur when $i=r$ or $j=r$, resulting in the two summands for the same in the above equation.
    Moreover, we notice that both summands are the same, giving us $p''(\xi^r)=2\sum_{i\in K\setminus \{r\}}\prod_{k\in \setminus \{r,i\}}(\xi^r-\xi^k)$. One may
    now verify that $p'(\xi^r)/p''(\xi^r)$ gives the desired result claimed in the lemma.
\end{proof}
%To compute $p''(\xi^r)$, note that $p''(X)$ is a (double) sum of many products. When one substitutes $\xi^r$ in a product, the product is non zero if and only if $r=i\, \text{or}\, r=j$\\\\

%    Putting $\xi^r$ gives:
%    $$ p''(\xi^r)=\sum_{j\in K\setminus \{r\}}\prod_{k\in K\setminus \{r,j\}} (\xi^r-\xi^k)+$$
%    $$ \sum_{i\in K\setminus \{r\}}\prod_{k\in K\setminus \{r,i\}} (\xi^r-\xi^k) $$
%    But this is just,
%    $$2\cdot \sum_{j\in K\setminus \{r\}}\prod_{k\in K\setminus \{r,j\}} (\xi^r-\xi^k) $$

%    Now, we are left to show that $\frac{\frac{p''(\xi^r)}{2}}{p'(\xi^r})=\sum_{j\in K\setminus \{r\}} 1/(\xi^r-\xi^j)$, or
%    $$\sum_{j\in K\setminus \{r\}}\prod_{k\in K\setminus \{r,j\}} (\xi^r-\xi^k)=\prod_{j \in K \setminus \{r\}}(\xi^r-\xi^j) \sum_{j\in K\setminus \{r\}} 1/(\xi^r-\xi^j)$$

%    But we observe that:
%    $$\sum_{j\in K\setminus \{r\}}\prod_{k\in K\setminus \{r,j\}} (\xi^r-\xi^k)=\sum_{j\in K\setminus \{r\}}\frac{\prod_{k\in K\setminus \{r\}} (\xi^r-\xi^k)}{(\xi^r-\xi^j)}$$

%    But $\prod_{k\in K\setminus \{r\}} (\xi^r-\xi^k)$ is independent of $j$, so it can be pulled outside the sum:
%    $$\sum_{j\in K\setminus \{r\}}\prod_{k\in K\setminus \{r,j\}} (\xi^r-\xi^k)=\prod_{k\in K\setminus \{r\}} (\xi^r-\xi^k) \sum_{j\in K\setminus \{r\}} 1/(\xi^r-\xi^j) $$
%    But $k$ is just a dummy variable. We can replace $k$ by $j$ on the RHS:
%    $$\sum_{j\in K\setminus \{r\}}\prod_{k\in K\setminus \{r,j\}} (\xi^r-\xi^k)=\prod_{j\in K\setminus \{r\}} (\xi^r-\xi^j) \sum_{j\in K\setminus \{r\}} 1/(\xi^r-\xi^j) $$
%    This completes the proof.
%\end{proof}

\begin{comment}
    ************ ALTERNATIVE PROOF **************
    \textbf{An alternative proof of \ref{lem:sumtoder}}:\\
    An alternative proof is possible by inducting on the cardinality of $K\subset N$.\\
    The base case where $K$ has one element, say $i$ is trivial. This is because in that case $p$ is a linear polynomial $X-\xi^i$ and LHS=$\frac{p''}{p'}=\frac{0}{1}=0$ and on RHS we are summing over the empty set so the sum is $0$.\\\\
    Now let $|K|=k$ and the elements of $K$ be $a_1, a_2, \cdots, a_k$.
    Consequently $p(X)=\prod_{i \in K}(X-\xi^i)$\\
    We assume that the result holds for this case. Thus,
    $$p''(\xi^i)/p'(\xi^i)=2 \cdot \sum_{j\in K\setminus \{i\}} 1/(\xi^i-\xi^j)$$ for all $i \in K$.\\
    Suppose we introduce another element in the set $K$, call it $\alpha$
    The vanishing polynomial for the new $K$(call it $K')$ is $q(X)=p(X)(X-\xi^{\alpha})$\\
    Then by the product rule for derivatives, we have:
    $$q'(X)=p'(X)(X-\xi^{\alpha})+p(X)$$ and
    $$q''(X)=2p'(X)+p''(X)(X-\xi^{\alpha})$$

    It suffices to show that $$\frac{q''(\xi^i)}{q'(\xi^i)}=2\sum_{j\in K'\setminus \{i\}}\frac{1}{\xi^i-\xi^j}$$
    for all $i \in K'$\\
    For this we make 2 cases:\\
    \textbf{Case 1-}$i \in K$
    Then
    $$q'(\xi^i)=p'(\xi^i)(\xi^i-\xi^{\alpha})$$ and
    $$q''(\xi^i)=2p'(\xi^i)+p''(\xi^i)(\xi^i-\xi^{\alpha})$$
    This used that $p(\xi^i)=0$\\
    Thus, $$\frac{q''(\xi^i)}{q'(\xi^i)}=\frac{2p'(\xi^i)+p''(\xi^i)(\xi^i-\xi^{\alpha})}{p'(\xi^i)(\xi^i-\xi^{\alpha})}$$
    Dividing numerator and denominator by $p'(\xi^i)(\xi^i-\xi^{\alpha})$ gives:
    $$\frac{q''(\xi^i)}{q'(\xi^i)}=\frac{2}{\xi^i-\xi^\alpha}+\frac{p''(\xi^i)}{p'(\xi^i)}$$
    By induction hypothesis this becomes:
    $$\frac{q''(\xi^i)}{q'(\xi^i)}=\frac{2}{\xi^i-\xi^\alpha}+2 \cdot \sum_{j\in K\setminus \{i\}} 1/(\xi^i-\xi^j)$$
    Note that $$\frac{1}{\xi^i-\xi^\alpha}+ \sum_{j\in K\setminus \{i\}} 1/(\xi^i-\xi^j)=\sum_{j\in K'\setminus \{i\}}\frac{1}{\xi^i-\xi^j}$$
    Using that $K'=K \cup \{\alpha\}$\\
    Thus, $$\frac{q''(\xi^i)}{q'(\xi^i)}=2\sum_{j\in K'\setminus \{i\}}\frac{1}{\xi^i-\xi^j}$$ which completes case 1. \\\\
    \textbf{Case 2-}$i=\alpha$. Then
    $$q'(\xi^{\alpha})=p(\xi^{\alpha})$$ and
    $$q''(\xi^{\alpha})=2p'(\xi^{\alpha})$$
    These we get by replacing $X$ by $\xi^{\alpha}$ in the expressions of $q', q''$
    So, it is sufficient to prove that:
    $$\frac{2p'(\xi^{\alpha})}{p(\xi^{\alpha})}=2\sum_{j\in K'\setminus \{\alpha\}}\frac{1}{\xi^{\alpha}-\xi^j}$$

    But $$p(\xi^{\alpha})=\prod_{i \in K}(\xi^{\alpha}-\xi^i)=\prod_{i \in K'\setminus \{\alpha\}}(\xi^{\alpha}-\xi^i)$$ and $$p'(\xi^{\alpha})=\sum_{j \in K} \prod_{i \in K \setminus \{j\}}(\xi^{\alpha}-\xi^i)=\sum_{j \in K'\setminus \{\alpha\}} \prod_{i \in K' \setminus \{j, \alpha\}}(\xi^{\alpha}-\xi^i)$$
    But now, note that $$\sum_{j \in K'\setminus \{\alpha\}} \prod_{i \in K' \setminus \{j, \alpha\}}(\xi^{\alpha}-\xi^i)=\prod_{i \in K' \setminus \{ \alpha\}}(\xi^{\alpha}-\xi^i)\sum_{j \in K'\setminus \{\alpha\}}\frac{1}{\xi^{\alpha}-\xi^j}$$
    So, $$\frac{2p'(\xi^{\alpha})}{p(\xi^{\alpha})}=\frac{2\cdot \prod_{i \in K' \setminus \{ \alpha\}}(\xi^{\alpha}-\xi^i)\sum_{j \in K'\setminus \{\alpha\}}\frac{1}{\xi^{\alpha}-\xi^j}}{\prod_{i \in K'\setminus \{\alpha\}}(\xi^{\alpha}-\xi^i)}=$$\\
    $$=2\cdot\sum_{j \in K'\setminus \{\alpha\}}\frac{1}{\xi^{\alpha}-\xi^j}$$

    This completes the proof of case 2 and thus the proof of the lemma.
\end{comment}

\begin{lemma}[Sumcheck]\label{lem:sumcheck}
Let $u(X,Y)$ be a bi-variate polynomial over a finite field $\F$ with degree less than $N$ in each of the variables and
$\nroots$ be defined as the group of $N^{\text{th}}$ roots of unity $(N<<|\F|)$
with generator $\xi \in \F$. Then $\sum_{i\in [N]}u(X,\xi^i)= Nu(X,0)$
\end{lemma}

\begin{proof}
    For some $d<N$, we write $u(X,Y)=a_0+a_1 Y+ a_2 Y^2+\cdots+ a_d Y^d$
    where each $a_i$ is a polynomial in $X$ of degree less than $N$.
    Now we write the sum:
    \begin{align*}
        \sum_{i\in [N]}u(X,\xi^i) &= N a_0+a_1(\xi^+\xi^2+\cdots+\xi^N) \\
        &\, +a_2(\xi^2+\xi^4+\cdots+\xi^{2N})+ \cdots + a_d(\xi^d+\cdots +\xi^{Nd})
    \end{align*}
    But for any $\alpha=\xi^k$ for $k<N$, $\alpha+\alpha^2+\cdots \alpha^N=0$. Thus, all terms vanish except the first term
    and hence $\sum_{i\in [N]}u(X,\xi^i)=N a_0$.
    The lemma follows by observing that $a_0=u(X, 0)$.
\end{proof}

\begin{lemma}\label{lem:zk-hat}
Let $Z_\nroots(X)$ be the vanishing polynomial for $\nroots$, let $\widehat{Z}_K(X)$ and  $Z_K(X)$ be the vanishing polynomials for $H_{[N]\setminus K}$ and $H_K$ respectively.
Let $\mu_1(X),\ldots,\mu_N(X)$ be Lagrange polynomials for the set $\nroots=\{\xi,\ldots,\xi^N\}$. Then:
\begin{gather}
    \widehat{Z}_K(X) = \sum_{j\in K}\frac{Z_\nroots'(\xi^j)}{Z_K'(\xi^j)}\mu_j(X), \\
    \widehat{Z}_K'(X)= \sum_{j\in K}\frac{Z_\nroots'(\xi^j)}{Z_K'(\xi^j)}\mu_j'(X)
\end{gather}
\end{lemma}
We use the following standard observation:
\begin{fact}
    If polynomials $f,g$ of degree $<N$ agree on $N$ points, then they are equal as polynomials, that is, $f(X)=g(X)$
\end{fact}

\begin{proof}
    Note that the second equation follows from the first by linearity of derivatives, so it suffices to prove the first equation.
    Both the sides of the identity are polynomials of degree $<N$, so it suffices to show their evaluations are identical over $N$ distinct points.
    In particular we show their evaluations are identical over $\nroots$.
    Consider evaluating LHS and RHS at $\xi^i$ for $i \in [N]\setminus K$.
    The left side is $0$ by definition of $\hat{Z}_K(X)$, while the right hand side is zero by the properties of Lagrange polynomials.
    Now consider evaluations LHS and RHS at $\xi^i$ for $i \in K$.
    The RHS is $\frac{Z_\nroots'(\xi^i)}{Z_K'(\xi^i)}$ by properties of Lagrange polynomials, while the
    LHS is $\prod_{j \in [N]\setminus K} (\xi^i-\xi^j)$\\
    Multiplying dividing by $\prod_{ j \in K \setminus \{i\}}(\xi^i-\xi^j)$ gives:
    $$LHS = \frac{\prod_{j \in [N] \setminus \{i\}} (\xi^i-\xi^j)}{\prod_{j \in K \setminus \{i\}}(\xi^i-\xi^j)}$$
    Which is $\frac{Z_\nroots'(\xi^i)}{Z_K'(\xi^i)}$, the same as the right hand side.
    This proves the claim.
\end{proof}

\begin{lemma}\label{lem:lamda-deriv}
Let $\mu_1,\ldots,\mu_N$ be the lagrange polynomials for the set $\nroots=\{\xi^i:i\in [N]\}$
of the $N^{th}$ roots of unity. Then we have:
\begin{equation*}
    \mu_i'(\xi^j) = \begin{cases}
                        \frac{(N-1)}{2\xi^{i}}  \text{ if } j=i\\
                        \frac{\xi^i}{\xi^j(\xi^j-\xi^i)} \text{ otherwise }
    \end{cases}
\end{equation*}
\end{lemma}

\begin{proof}
    Suppose first that $i \neq j$.
    We know that $\mu_i(X)= \frac{Z_H(X)}{Z_H'(\xi^i)(X-\xi^i)}$.
    Thus, by applying quotient rule(we can apply quotient rule because $\mu_i$ is defined at $\xi^j$ as $j\neq i$):
    $$\mu_i'(X) \cdot Z_H'(\xi^i)= \frac{(X-\xi^i)(N\cdot X^{N-1})-(X^N-1)}{(X-\xi^i)^2}$$
    Substituting $X$ by $\xi^j$, we get:
    $$\mu_i'(\xi^j) \cdot \frac{N}{\xi^i}= \frac{N(\xi^j-\xi^i)}{\xi^j (\xi^j-\xi^i)^2}$$
    Thus, we get:
    $$\mu_i'(\xi^j)=\frac{\xi^i}{\xi^j(\xi^j-\xi^i)}$$
    In particular, when $i=j$, we have:
    \begin{align*}
    \mu_i(X) &= \frac{\prod_{j \in [N] \setminus \{i\}}(X-\xi^j)}{Z_H'(\xi^i)}\\
    \textnormal{or,  }\mu_i(X)\cdot Z_H'(\xi^i) &= \prod_{j \in [N] \setminus \{i\}}(X-\xi^j)\\    
    \end{align*} 
	Differentiating the above equation on both sides, we get:
	\[\mu_i'(X) \cdot \frac{N}{\xi^i} = \sum_{j,k \in [N] \setminus \{i\}}\prod_{k \in [N]\setminus\{i,j\}}(X-\xi^k)\]  
%    Now consider $i=j$:
%    Then, $\mu_i(X)=\frac{\prod_{j \in [N] \setminus \{i\}}(X-\xi^j)}{Z_H'(\xi^i)}$\\
%    Thus,  $\mu_i(X)\cdot Z_H'(\xi^i)=\prod_{j \in [N] \setminus \{i\}}(X-\xi^j)$\\
%    Thus, $\mu_i'(X) \cdot \frac{N}{\xi^i}=\sum_{j \in [N] \setminus \{i\}}\prod_{k \in [N]\setminus\{i,j\}}(X-\xi^k)$\\
    Substituting $X=\xi^i$ in the above equation yields:
    \begin{align*}
        \mu_i'(\xi^i) \cdot \frac{N}{\xi^i} &= \sum_{j \in [N] \setminus \{i\}}\prod_{k \in [N]\setminus\{i,j\}}(\xi^i-\xi^k) \\
        &=\sum_{j \in [N] \setminus \{i\}}\frac{\prod_{k \in [N]\setminus\{i\}}(\xi^i-\xi^k)}{\xi^i-\xi^j} \\
        &= \prod_{k \in [N]\setminus\{i\}}(\xi^i-\xi^k)\sum_{j\in [N]\setminus \{i\}}\frac{1}{\xi^i-\xi^j} \\
        &=Z_H'(\xi^i)\sum_{j\in [N]\setminus \{i\}}\frac{1}{\xi^i-\xi^j}\\
        &= N/\xi^i\sum_{j\in [N]\setminus \{i\}}\frac{1}{\xi^i-\xi^j}
    \end{align*}
    We divide on both sides by $N/\xi^i$ in the above, and use Lemma ~\ref{lem:sumtoder} to obtain:
    \begin{align*}
        \mu_i'(\xi^i) &= \sum_{j\in [N]\setminus \{i\}}\frac{1}{\xi^i-\xi^j} =\frac{Z_H''(\xi^i)}{2 Z_H'(\xi^i)}
        =\frac{N-1}{2\xi^i}
    \end{align*}

\end{proof}

\begin{lemma}\label{lem:tau}
Let $K\subseteq \F$ be a set of cardinality $k$ and $\mathcal{X}=\{x_j:j\in K\}$ be a set
where $x_j$ for $j\in K$ are distinct elements of $\F$. Let $Z_\mathcal{X}(X)=z_k X^k+\cdots+z_0$ denote the vanishing polynomial of $\mathcal{X}$
and $\{\tau_j(X)\}_{j\in K}$ denote the Lagrange polynomials such that $\tau_i(x_j)=\delta_{ij}$ for $i,j\in K$. Then for all $j\in K$,
we have $\tau_j'(x_j)=F_K(x_j)/Z_\mathcal{X}'(x_j)$ where the polynomial
$F_K(X)$ is defined as
\[F_K(X)=\binom{k}{2}z_k X^{k-2}+\cdots+\binom{2}{2}z_2=\sum_{j=2}^k z_j\binom{j}{2}X^{j-2} \]
\end{lemma}
\begin{proof}
    For $j\in K$ we have: 
    \[\tau_j(X)=\frac{Z_\mathcal{X}(X)}{(X-x_j)Z_\mathcal{X}'(x_j)}= \frac{1}{Z_\mathcal{X}'(x_j)}\frac{Z_\mathcal{X}(X)}{X-x_j}\] 
    by definition of Lagrange polynomials. By long division of $Z_\mathcal{X}(X)$ by $(X-x_j)$, we have:
    \begin{align*}
        \tau_j(X) &= \frac{1}{Z_\mathcal{X}'(x_j)}\big(z_k X^{k-1} + (x_jz_k + z_{k-1})X^{k-2} + \cdots\\
        & \quad + (x_j^{k-1}z_k + \cdots + z_1)\big)\\
        &= \frac{1}{Z_\mathcal{X}'(x_j)}\sum_{p=0}^{k-1} \left(\sum_{q=p+1}^k z_q x_j^{q-p-1}\right)X^p
    \end{align*}
    Differentiating both sides, we have:
    \begin{align*}
    \tau_j'(X) &= \frac{1}{Z_\mathcal{X}'(x_j)}\sum_{p=0}^{k-1} \left(\sum_{q=p+1}^{k}z_q x_j^{q-p-1}\right)p X^{p-1}\\
%    In this step we just differentiated $X^p$ to get $p X^{p-1}$
    &= \frac{1}{Z_\mathcal{X}'(x_j)}\sum_{p=1}^{k-1} p  \sum_{q=p+1}^{k}z_q x_j^{q-p-1} X^{p-1}
    \end{align*}
    Substituting $X=x_j$, we get:
    \begin{align*}
    \tau_j'(x_j) &= \frac{1}{Z_\mathcal{X}'(x_j)} \sum_{p=1}^{k-1} p \sum_{q=p+1}^k z_q x_j^{q-2}\\
    &= \frac{1}{Z_\mathcal{X}'(x_j)}\sum_{q=2}^k z_q x_j^{q-2} \sum_{p=1}^{q-1} p\\
%    
%    Here, we reversed the order of the sum. $p$ going from 1 to $k-1$ and $q$ going from $p+1$ to $k$ is same as $q$ going from $2$ to $k$ and $p$ going from 1 to $q-1$
%    $$\tau_j'(x_j)
    &= \frac{1}{Z_\mathcal{X}'(x_j)}\sum_{q=2}^k z_q \binom{q}{2} x_j^{q-2}\\
%	\end{align*}
%%    Here we used that $\sum_{p=1}^{q-1}p=\frac{q(q-1)}{2}=\binom{q}{2}$
%	Replacing $q$~(which is a dummy variable) by $j$ above, we get
%	\begin{align*}
%		\tau_j'(x_j) &= \frac{1}{Z_\mathcal{X}'(x_j)}\sum_{j=2}^k z_j \binom{j}{2} x_j^{j-2}\\
		& = \frac{F_K(x_j)}{Z_\mathcal{X}'(x_j)}
	\end{align*}
	This completes the proof.
%
%    But $q$ is just a dummy variable. We might as well replace $q$ by $j$
%
%    Thus, $$\tau_j'(x_j)= \frac{1}{Z_\mathcal{X}'(x_j)}\sum_{j=2}^k z_j \binom{j}{2} x_j^{j-2} $$
%    If we compare this to the definition of $F_K(X)$ given in the lemma statement, we get:
%    $$\tau_j'(x_j)= \frac{F_K(x_j)}{Z_\mathcal{X}'(x_j)}$$
\end{proof}

\begin{comment}
    \subsection{Brief Proof Summary of lemma \ref{lem:sum-computation}}\label{subsec:sum-computational}
    What we want to compute is a sum over the set $K$ for all $i \in I$ or a sum over the set $I$ for all $j \in K$. Let us give a sketch for the first case.

    We interpolate a polynomial $p$ in such a way that its evaluations at $\nroots$ correspond to the numerators of the required sum. We take the full $\nroots$ and not subsets of it so that we can apply sumcheck easily later.
    Next we introduce some rational functions. These will be very useful to reduce the sum we need to compute to an evaluation of the rational functions at some points.
    These evaluations can be reduced to evaluations of polynomials using the simple form of $Z_{\nroots}$. By introducing some more polynomials and using formulas for sumcheck, finally, the entire task can be reduced to calculating $p(0)$ and $p'(X)$ at certain points.

    The polynomial $p$ turns out to be a product of polynomials: $p(X)=\widehat{Z}_K(X)\cdot q(X)$ and so the next step is to get the evaluations of $\widehat{Z}_K(X)$ and $q(X)$ at those points.
    For this, some further lemmas are used to get evaluations of $q(X)$ at degree $q$ many points and thus $q(X)$ is obtained by interpolation.
    Next $q(0)$ can be computed and thus $p(0)$ is also computed.
    $Z_K(X)$ and $q'(X)$ can also be computed easily(details in the proof)\\
    This leaves computation of $\widehat{Z}_K(X)$ at those points as the main task. For this another lemma is needed related to derivatives of Lagrange polynomials.
    Eventually the task reduces to needing to compute something which can be computed efficiently using another lemma.\\
    Finally, every single step involves a reduction which can be done in the number of operations allowed in the statement of the lemma. So, we get what we want.\\\\
    For the second case, the proof is very similar, except for some interchanging of $I$ and $K$, and interchanging of $i$ and $j$. \\
    This completes a brief summary of the proof.
\end{comment}

\subsection{Proof of lemma ~\ref{lem:sum-computation}}\label{subsec:sum-computation}

\begin{proof}

    We give the proof for computation of $a_i$ for all $i \in \setind$ in full detail and briefly mention the modifications needed to compute $b_j$ for all $j \in K$.

    \begin{equation}\label{eq:ei}
    a_i = \sum_{j\in K\setminus \{i\}} \frac{d_j}{\xi^i - \xi^j}
    \end{equation}
    Recall that $\setind \subset K$ in this case.
    To compute $a_i$, we first define a polynomial $p(X)$ of degree at most $N-1$ such that
    $p(\xi^j)=d_j$ for $j\in K$ and $p(\xi^j)=0$ for $j \in [N]\setminus K$.
    Then, the vanishing polynomial of $H_{[N]\setminus K}$ divides $p(X)$ and
    there exists a polynomial $q(X)$ of degree at most $|K|-1$ such that:
    \begin{equation}\label{eq:px}
    p(X) = \hat{Z}_K(X)\cdot q(X)
    \end{equation}
    where $\hat{Z}_K(X)=\prod_{i\in [N]\setminus K}(X-\xi^i)$ is the vanishing polynomial of $H_{[N]\setminus K}$. Now, we introduce the rational functions:
    \begin{gather}\label{eq:fgr}
    f_i(X) = \sum_{j\in [N]\setminus \{i\}}\frac{p(\xi^j)}{X-\xi^j}, i\in \setind \\
    g_i(X) = \sum_{j\in [N]\setminus \{i\}}\frac{p(X)}{X-\xi^j}, i\in \setind \\
    r_i(X) = \sum_{j\in [N]\setminus \{i\}}\frac{p(X) - p(\xi^j)}{X-\xi^j}, i\in \setind
    \end{gather}
    Note that, by the definition of $p(X)$, $f_i(\xi^i)=a_i \,\forall i$. Thus, it suffices to compute $f_i(\xi^i)$ for all $i \in I$. Since $f_i(X) = g_i(X) - r_i(X)$ for $i\in I$, we have that
    $a_i=g_i(\xi^i)-r_i(\xi^i)$.
    Thus, we need to compute $g_i(\xi^i)$ and $r_i(\xi^i)$ for all $i \in I \subset K$.
    \begin{align*}
        g_i(\xi^i) &= p(\xi^i)\sum_{j\in [N]\setminus \{i\}} \frac{1}{\xi^i-\xi^j} \\
        &= \frac{p(\xi^i)Z_{\nroots}''(\xi^i)}{2 Z_{\nroots}'(\xi^i)} \quad \text{(from lemma \ref{lem:sumtoder})} \\
        &= \frac{(N-1)d_i}{2\xi^i}
    \end{align*}
	In the above, we used $Z_H(X)=X^N-1$ and that $p(\xi^i)=d_i$. In other words, $g_i(\xi^i)$ for all $i$ can be obtained in $O(|\setind|)$ operations. Therefore, it suffices to compute $r_i(\xi^i)$ for all $i \in I$ efficiently. To this end, we write $r_i(X)$ as:
    \[r_i(X) = \sum_{j\in [N]}\frac{p(X)-p(\xi^j)}{X-\xi^j} - \frac{p(X)-p(\xi^i)}{X-\xi^i}\]
    By defining the bi-variate polynomial 
    \[u(X,Y)=(p(X) - p(Y))/(X-Y)\]
    we get
    \[r_i(X)= \sum_{j\in [N]}u(X,\xi^j) - u(X,\xi^i)\]
    Defining $r(X)=\sum_{j\in [N]}u(X,\xi^j)$, we have:
    $$r_i(X)=r(X)-u(X, \xi^i)$$
    Substituting $X = \xi^i$ in the above, we have: 
    $$r_i(\xi^i)=r(\xi^i) - u(\xi^i,\xi^i) = r(\xi^i) - p'(\xi^i)$$
    where $p'(\xi^i) =  u(\xi^i,\xi^i)$ by the definition of formal derivative. Now, using $r(X)=N u(X,0)$~(Lemma \ref{lem:sumcheck}), we have: 
    \[r(X)=N \frac{(p(X) - p(0))}{X}\] 
    Finally, substituting $X = \xi^i$ above, we have:
    $$r(\xi^i)=N \frac{(d_i - p(0))}{\xi^i}$$
    Thus, it remains to compute $p(0)$ and $p'(\xi^i)$ efficiently for each $i\in \setind$.
    
    \smallskip
    
    \noindent{\bf Computing the polynomial $q(X)$.} Recall from Equation~\eqref{eq:px} that
    \[q(\xi^j) = \frac{p(\xi^j)}{\widehat{Z}_K(\xi^j)}\]
	for all $j\in K$. Furthermore, by Lemma~\ref{lem:zk-hat}, we have: 
	\[\hat{Z}_K(\xi^j)=\frac{Z_{\nroots}'(\xi^j)}{Z_K'(\xi^j)}=\frac{\frac{N}{\xi^j}}{Z_K'(\xi^j)}\]
	for each $j\in K$. Observe that, given the set $K$, we can compute the polynomial $Z_K(X)$ in $O(|K|\log^2 |K|)$ operations using the fast multiplication, and we can then obtain $Z_K'(X)$ in additional $O(|K|)$ operations. Finally, $Z_K'(\xi^j)$ can be evaluated for $j\in K$ in additional $O(|K|\log^2 |K|)$ operations. Thus we can efficiently compute $q(\xi^j)$ for all $j\in K$  $O(|K|\log^2 |K|)$ operations. Since degree of $q(X)$ is strictly less than $|K|$, we can further interpolate to obtain the polynomial $q(X)$ in $O(|K|\log^2 |K|)$ field operations.
    
    \smallskip
    
    \noindent{\bf Computing $p(0)$.} From Equation~\eqref{eq:px}, we have
    \[p(0)=\widehat{Z}_K(0)\cdot q(0)\]
    Additionally, since we have 
    \[\hat{Z}_K(0)=\frac{Z_H(0)}{Z_K(0)}=\frac{-1}{Z_K(0)}\]
    this enables us to compute $p(0)$ since $q(0)$ and $Z_K(0)$ are just the constant terms of the known polynomials $q(X)$ and $Z_K(X)$. 
    
    \smallskip
    
    \noindent{\bf Computing $p'(xi^i)$.} We now show how to compute $p'(\xi^i)$ for each $i\in I$. First, by differentiating the polynomial $q(X)$ obtained above, we obtain $q'(X)$. Then, by fast evaluation, we get evaluations of $q'(X)$ at $\xi^i$ for all $i \in I$, again in $O(|K|\log^2 |K|)$ field operations. 
     
    
    
%    Let $p(X)=\widehat{Z}_K(X)\cdot q(X)$
%    as in Equation \eqref{eq:px} and let $Z_K(X)$ be the vanishing polynomial for $H_K$.
%    Note that $Z_K(X) \cdot \hat{Z}_K(X)=X^N-1$ and also that
%    $Z_\nroots'(\xi^j)= \frac{N}{\xi^j} $.
%    Also, $Z_K(X)$ can be computed by fast multiplication in $O(|K|\log^2 |K|)$ operations and then it can be differentiated monomial by monomial to get $Z_K'(X)$.\\
%    \textbf{Then, fast evaluation gives $Z_K'(X)$ at $\{\xi^j:j\in K\}$ in $O(|K|\log^2 |K|)$ operations}\\
%    Recall from Lemma \ref{lem:zk-hat} that $\hat{Z}_K(\xi^j)=\frac{Z_{\nroots}'(\xi^j)}{Z_K'(\xi^j)}$ and
%    thus $\hat{Z}_K(\xi^j)=\frac{\frac{N}{\xi^j}}{Z_K'(\xi^j)}$ for all $j \in K$.
%    Further recalling from \eqref{eq:px} that $q(\xi^j)=\frac{p(\xi^j)} {\hat{Z}_K(\xi^j)}=\frac{d_j}{\widehat{Z}_K(\xi^j)}$ enables us to compute
%    $q(\xi^j)$ for all $j \in K$.
%    As degree of $q(X)$ is strictly less than $|K|$ we can interpolate to get $q(X)$ in $O(|K|\log^2 |K|)$ field operations.
%    By differentiating monomial by monomial, we get $q'(X)$ and by fast evaluation we get evaluations of $q'(X)$ at $\xi^i$ for all $i \in I$, again in $O(|K|\log^2 |K|)$ field operations.

%    Since $p(0)=q(0)\cdot \hat{Z}_K(0)$, and
    

%    \textbf{Now it remains to compute $p'(X)$ at $\{\xi^i:i \in I\}=H_I$.}
    Using the  product rule for derivatives we have: $p'(X)=q(X) \hat{Z}_K'(X)+q'(X) \hat{Z}_K(X)$.

    Thus, to get $p'(X)$ at $H_I$, we need $q, q', \hat{Z}_K$ and $\hat{Z}_K'$ at $H_I$. Out of which, we already have the evaluations for $q, q', \hat{Z}_K$.

    \textbf{Thus, it suffices to obtain
        $\hat{Z}_K'(X)$ at the set $H_I$}

    Recall the second equation of lemma \ref{lem:zk-hat} and replace $X$ by $\xi^i$ to get:
    $$  \widehat{Z}_K'(\xi^i) = \sum_{j\in K\setminus \{i\}} \frac{Z_\nroots'(\xi^j)}{Z_K'(\xi^j)}\mu_j'(\xi^i) + \frac{Z_\nroots'(\xi^i)}{Z_K'(\xi^i)}\mu_i'(\xi^i)$$

    Using lemma \ref{lem:lamda-deriv}, this becomes:
    $$ \widehat{Z}_K'(\xi^i)= N\xi^{-i}\sum_{j\in K\setminus \{i\}}\frac{1}{Z_K'(\xi^j)(\xi^i-\xi^j)} + \frac{N(N-1)}{2\xi^{2i}Z_K'(\xi^i)}.$$


    Let $$\varphi_i=\sum_{j\in K\setminus \{i\}}\frac{1}{Z_K'(\xi^j)(\xi^i-\xi^j)}$$
    Thus,\textbf{ it suffices to compute $\varphi_i$ for all $i \in I$}\\

    Define:
    $$ \Phi_i(X) = \sum_{j\in K\setminus \{i\}} \frac{1}{Z_K'(\xi^j)(X-\xi^j)} $$
    at $X=\xi^i$. It is clear that $\varphi_i=\Phi_i(\xi^i)$.
    Let $\{\eta_i(X)\}_{i\in K}$ be the Lagrange polynomials for the set $\{\xi^i:i\in K\}= H_K$.
    Then, $\frac{\eta_j(X)}{Z_K(X)}=\frac{1}{Z_K'(\xi^j)(X-\xi^j)}$.\\
    Thus, $\Phi_i(X)$ can be rewritten as:
    \begin{align*}
        \Phi_i(X)&=\sum_{j\in K\setminus \{i\}} \frac{\eta_j(X)}{Z_K(X)}\\
        &= \sum_{j\in K\setminus \{i\}} \frac{\eta_j(X)/(X-\xi^i)}{Z_K(X)/(X-\xi^i)}
    \end{align*}

    Putting $X=\xi^i$ in the above, we have:

    $$\varphi_i = \Phi_i(\xi^i) = \left(\sum_{j\in K\setminus \{i\}}\frac{\eta_j(X)/(X-\xi^i)}{Z_K(X)/(X-\xi^i)}\right)(\xi^i)$$
    This can be simplified as:
    \begin{align*}\varphi_i = \Phi_i(\xi^i) &= \left(\sum_{j\in K\setminus \{i\}}\frac{\eta_j(X)/(X-\xi^i)}{\prod_{k \in K \setminus \{i\}}(X-\xi^k)}\right)(\xi^i)\\
    &= \sum_{j\in K\setminus \{i\}}\left(\frac{\eta_j(X)/(X-\xi^i)}{\prod_{k \in K \setminus \{i\}}(X-\xi^k)}\right)(\xi^i)\\
    &= \sum_{j\in K\setminus \{i\}}\frac{\left(\eta_j(X)/(X-\xi^i)\right)(\xi^i)}{\left(\prod_{k \in K \setminus \{i\}}(X-\xi^k)\right)(\xi^i)}\\
    &= \frac{1}{Z_K'(\xi^i)}\sum_{j\in K\setminus \{i\}}\left(\eta_j(X)/(X-\xi^i)\right)(\xi^i)
    \end{align*}

    Now, note that for all $j \neq i$, $\left(\eta_j(X)/(X-\xi^i)\right)(\xi^i)$ is just the evaluation of the polynomial $\frac{\eta_j(X)-\eta_j(\xi^i)}{X-\xi^i}$ at the point $\xi^i$.\\
    But, since $\eta_j$ is a (Lagrange) polynomial, this is just $\eta_j'(\xi^i)$ by definition of formal derivative of the polynomial $\eta_j(X)$.

  

    Thus, we get:


    $$\varphi_i =\Phi_i(\xi^i) =  \frac{1}{Z_K'(\xi^i)}\sum_{j\in K\setminus \{i\}}\eta_j'(\xi^i)$$
    But, $$\sum_{j\in K\setminus \{i\}}\eta_j'(\xi^i)= \sum_{j\in K}\eta_j'(\xi^i)-\eta_i'(\xi^i)=-\eta_i'(\xi^i)$$
    using that the sum of Lagrange polynomials is the constant $1$

    Thus,
    $$\varphi_i = \frac{-\eta_i'(\xi^i)}{Z_K'(\xi^i)}$$

    \textbf{Thus, it suffices to efficiently compute $\eta_i'(\xi^i)$ for $i\in \setind$}

    For this recall Lemma \ref{lem:tau} and observe that in our setting $\mathcal{X}$ is just $H_K$, $x_j$ are $\xi^j$ and $Z_\mathcal{X}(X)$ which is the vanishing polynomial of $\mathcal{X}=H_K$ is just $Z_K(X)$. The $\eta_i$ in the lemma are the Lagrange polynomials corresponding to $\mathcal{X}=H_K$ so they are same as the $\eta_i$ in our setting\\\\
    Thus, we get:
    $$\eta_i'(\xi^i)=\frac{F_K(\xi^i)}{Z_K'(\xi^i)}$$



    \textbf{Thus, it suffices to compute $F_K(\xi^i)$ for all $i \in I$}\\

    From the definition $F_K$ (Lemma ~\ref{lem:tau}), it can be computed once we know
    coefficients  $z_0, z_1, z_2, \cdots, z_k$ of the polynomial $Z_K(X)$.
    But, we know the polynomial $Z_K$ (via fast multiplication) hence we know $z_0, z_1, \cdots, z_{k}$. Thus, we can easily compute $F_K(X)$.

    Now, by fast evaluation, we can get $F_K(\xi^i)$ for all $i \in I$ as required. This will take $O(|K| \log^2|K|)$ field operations.


    This completes the proof for $a_i$ (Note that in the proof we throughout used that $|\setind| \leq |K|$ so that the various $O(|\setind|)$ operations do not contribute to the final complexity).\\
    For the case of $b_j$, notice first that it is very similar to the case of $a_i$ except that the roles of $\setind$ and $K$ are reversed and $i$ and $j$ are reversed.

    First of all define $c_i$ for $i$ outside $\setind$ to be 0. For $i \in \setind$, $c_i$ is given by the specific lookup argument and so is known.
    Next, define polynomial $p$ of degree atmost $N-1$ such that $p(\xi^i)=c_i$ for all $i \in [N]$.
    From here on, we proceed exactly as we did for the $a_i$ case, replacing every instance of $\setind$ with $K$, every $K$ with $\setind$, every $i$ with $j$ and every $j$ with $i$.
    All the required lemmas are also modified appropriately in this way.

    We reach till
    $$\Phi_j(X)=\sum_{i\in I\setminus \{j\}} \frac{\eta_i(X)}{Z_I(X)}$$
    and we need to compute $\varphi_j=\Phi_j(\xi^j)$ for all $j \in K$.\\

    Here, we deviate: For $j \in K\setminus I$, we can very easily compute $\varphi_j=\Phi_j(\xi^j)$ as
    \begin{align*}
        \Phi_j(\xi^j)&=\sum_{i\in I\setminus \{j\}} \frac{\eta_i(\xi^j)}{Z_I(\xi^j)}\\
        &=\frac{1}{Z_I(\xi^j)} \sum_{i\in I}\eta_i(\xi^j)\\
        &=\frac{1}{Z_I(\xi^j)}
    \end{align*}

    \textbf{So, it suffices to compute $\varphi_j$ for all $j \in \setind$}

    Now, again continue exactly as in the $a_i$ case until we reach:
    $$\varphi_j =\Phi_j(\xi^j) = \frac{-\eta_j'(\xi^j)}{Z_I'(\xi^j)}$$
    for all $j \in \setind$.
    Note that $Z_I'(\xi^j)$ we would have already computed for all $j \in \setind$ by the time we reach this stage \\
    \textbf{So, it suffices to compute $-\eta_j'(\xi^j)$ for all $j \in \setind$\\}
    But we computed $\eta_i'(\xi^i)$ for $i\in \setind$ during the computation of $a_i$. We reuse whatever we got!
    This completes the computation of $b_j$ for all $j \in K$
    and finishes the proof of lemma \ref{lem:sum-computation}.
\end{proof}


%\section{Reducing indexed lookups to subvector lookups}\label{sec:generic-transformation-app}
%We give the proof of Lemma ~\ref{lem:generic-transformation} here.
%\begin{proof}[Proof of Lemma ~\ref{lem:generic-transformation}]
%Assume that there exists $i\in [m]$ such that $v_i\neq \vec{t}[\,a_i\,]$.
% We bound the probability
%that the check $\vec{v} + \gamma \vec{a}\leq \vec{t}+\gamma \vec{I}_n$ succeeds for
%$\gamma\gets \F$ chosen uniformly. The probability is clearly upper bound by the probability
%that $v_i + \gamma a_i\in \vec{t}+\gamma \vec{I}_n$, or $v_i+\gamma a_i=t_j + \gamma j$ for
%some $j\in [n]$, or $v_i + \gamma (a_i - j)=t_j$. We note that for $a_i\neq j$ (by assumption),
%the left hand side is uniform in $\F$, and hence occurs in $\vec{t}$ with probability at most
%$n/|\F|$. This concludes the proof of the lemma.
%\end{proof}

\begin{comment}
Let us say we have a subvector lookup argument $L$ which takes as input two vectors say $a, b$ and determines whether $b$ is a subvector of $a$. It outputs $1$ if $b$ is a subvector of $a$ and $0$ if $b$ is not a subvector of $a$.
Recall that $b$ is said to be a subvector of $a$ if every element of $b$ is in $a$\\
Indexed lookup argument $L'$ takes in 3 vectors $a, b, c$ where $c$ is a vector of indices with $|b|=|c|$ and checks that $b$ is a subvector of $a$ and that $a[c_i]=b_i$ for all $i$. That is, it does $L(a, b)$ first and further checks that $a$ restricted to the indices in $c$(in a particular order) gives $b$(in that particular order).
If $b$ is a subvector of $a$ and $a[c_i]=b_i$ for all $i$ then it outputs $1$ else it outputs $0$.\\

\textbf{Note:} Given $a,b$ with $b$ subvector of $a$, there will always exist a $c$ with $|c|=|b|$ such that $a[c_i]=b_i$ for all $i$. The $c$ may not be unique though. \\\\
So, we are given $T, v, a$ with $|a|=|v|$ and we want to compute $L'(T, v, a)$ using $L(\cdot, \cdot)$.
Let $|T|=N$. Let $|v|=m=|a|$.
Procedure to compute $L'(T, v, a)$ is as follows:
\begin{itemize}
    \item First compute $L([N], a)$. This checks that $a$ is indeed a vector of indices. Call the output of this as $b_1$
    \item Next compute $L(T, v)$. This checks that $v$ is indeed a subvector of $T$. Call the output of this as $b_2$
    \item Choose a $\gamma$ uniformly at random from $\F^*$
    \item Compute $L(T+\gamma[N], v+\gamma a)$ which checks that $v+\gamma a$ is a subvector of $T+\gamma [N]$. Call the output $b_3$.
    \item $L'(T, v, a)$ is then computed as $b_1 \wedge b_2 \wedge b_3$
\end{itemize}
This procedure works for any $T,v,a$ with $|v|=|a|$. So, this gives a procedure for calculating $L'(\cdot, \cdot, \cdot)$.

\begin{lemma}\label{lem:reduction}
Let the procedure for getting $L'(\cdot, \cdot, \cdot)$ be as above. Then, if $L(\cdot, \cdot)$ is a valid subvector lookup argument then the procedure yields a valid indexed lookup argument $L'(\cdot, \cdot, \cdot)$ with an error probability upper bounded by $\frac{O(1)}{|\F|}$.
\end{lemma}

\begin{proof}
    To show that the procedure yields a valid indexed lookup argument $L'(T, v, a)$, we need to show the following:
    \begin{enumerate}

        \item If $v$ is indeed a subvector of $T$ with $T$ restricted to $a$ giving $v$, then $b_1=b_2=b_3=1$ with probability lower bounded by $1-\frac{O(1)}{|\F|}$
        \item If either $v$ is not a subvector of $T$ or if $v$ is a subvector of $T$ but restriction of $T$ to $a$ is not $v$, then at least one of $b_1, b_2, b_3$ is $0$ with probability lower bounded by $1-\frac{O(1)}{|\F|}$
    \end{enumerate}

    For (1), suppose $v$ is a subvector of $T$ with $T$ restricted to $a$ giving $v$. Then $L(T, v)=b_2=1$.
    Moreover, $a$ is a subvector of $[N]$(else restricting $T$ to $a$ won't even make sense).
    So, $L([N], a)=b_1=1$.
    Lastly, since $v_i=T[a_i]$ for all $i \in m$, we have that for any $\gamma \in \F^{*}$,
    $$T[a_i]+\gamma a_i=v_i+\gamma a_i$$ for all $i$ \\
    But $T[a_i]+\gamma a_i$ is of course an element of $T+\gamma [N] $ for all $i \in [m]$.(Every $a_i \in [N]$).
    So, $\{T[a_i]+\gamma a_i:i \in [m]\}$ is a subvector of $T+\gamma [N] $.
    Thus, $\{v_i+\gamma a_i:i \in [m]\}$ is a subvector of $T+\gamma [N] $.
    So, $v+\gamma a$ is a subvector of $T+\gamma [N] $.
    Thus, $L(T+\gamma[N], v+\gamma a)=b_3=1$

    This completes proof of (1).(In this case the probability is actually 1) \\\\
    For proof of (2). if $v$ is not a subvector of $T$ then $L(T, v)=b_2=0$ with probability 1 and we are done.
    So let us assume that $v$ is a subvector of $T$ and restriction of $T$ to $a$ is not $v$.
    If $a$ is not a vector of indices (that is, some $a_i \notin [N]$) then $L([N], a)=b_1=0$ with probability 1 and we are done.
    So, let us assume that $a$ is a vector of indices (of course by definition of $L'$, we have that $|a|=|v|=m$).\\
    \textbf{It suffices to show that $L(T+\gamma[N], v+\gamma a)=b_3$ is 0 with probability lower bounded by $1-\frac{O(1)}{|\F|}$.}

    Since $v$ is a subvector of $T$ there is a restriction of $T$ which gives $v$. Say the restriction is $b$. Then $|b|=|v|=m$ and $T[b_i]=v_i$ for all $i \in [m]$.
    Moreover, by our assumption that restriction of $T$ to $a$ is not $v$, we have that $a \neq b$.
    But $a\neq b$ implies there exists a $k \in [m]$ such that $a_k\neq b_k$.
    As restriction of $T$ to $a$ is not $v$, there exists a $k \in [m]$ such that $T[a_k]\neq v_k$\\
    Let $I=\{k \in [m]: T[a_k]\neq v_k\}$.
    By the fact that restriction of $T$ to $a$ is not $v$ , $I \neq \phi$.
    Also, $I \subset [N]$.
    By well ordering principle, $I$ has a smallest element, say $i$.
    Then for this $i$, $T[a_i]\neq v_i$. Moreover, as $v_i=T[b_i]$; $a_i \neq b_i$ \\

    Let $\gamma$ be picked uniformly at random from $\F^*$.
    Consider the vector $T+\gamma[N]$. Once $\gamma$ has been chosen, this is a fixed vector..
    Also, consider the vector $v+\gamma a$. One of the elements in this vector is $v_i +\gamma a_i$.
    We will upper bound the probability that this element is an element of the vector $T+\gamma[N]$. \\
    $$\Pr[v_i +\gamma a_i \in T+\gamma[N]]=\Pr[\exists j \in [N] \ni v_i +\gamma a_i=T[j]+\gamma j]$$
    But by union bound:
    $$\Pr[\exists j \in [N] \ni v_i +\gamma a_i=T[j]+\gamma j] \leq \sum_{j=1}^N \Pr[v_i +\gamma a_i=T[j]+\gamma j]$$
    Thus,
    $$\Pr[v_i +\gamma a_i \in T+\gamma[N]] \leq \sum_{j=1}^N \Pr\left[\gamma = \frac{T[j]-v_i}{a_i-j}\right] $$

    Let $M'$ be defined as $M'=\{j\in [N] \ni T[j]=v_i \}$.
    Let $M$ be defined as $M' \cup \{a_i\}$. Clearly $M \subset [N]$.\\
    But now, consider $\frac{T[j]-v_i}{a_i-j}$. For $j=a_i$ this is not defined. And for $j$ such that $T[j]=v_i$, this is 0. So, for such values of $j$, probability that $\gamma$ equals to it is $0$.
    For other values of $j$, $\frac{T[j]-v_i}{a_i-j}$ is an element of $\F^{*}$ and since $\gamma$ is a random element of $F^{*}$, the probability that $\gamma$ equals to it is $\frac{1}{|\F|-1}$\\
    In other words,
    \begin{gather}
        \Pr\left[\gamma = \frac{T[j]-v_i}{a_i-j}\right]=\frac{1}{|\F|-1}\, \forall j \in [N]\setminus M\\
        \Pr\left[\gamma = \frac{T[j]-v_i}{a_i-j}\right]=0 \, \forall j \in M
    \end{gather}

    Thus, we get that:
    $$\Pr[v_i +\gamma a_i \in T+\gamma[N]] \leq
    \sum_{j \in [N]\setminus M} \Pr\left[\gamma = \frac{T[j]-v_i}{a_i-j}\right]+$$
    $$\sum_{j \in M} \Pr\left[\gamma = \frac{T[j]-v_i}{a_i-j}\right]=\sum_{j \in [N]\setminus M} \Pr\left[\gamma = \frac{T[j]-v_i}{a_i-j}\right]$$
    $$=\sum_{j \in [N]\setminus M} \frac{1}{|\F|-1}=\frac{|[N]\setminus M|}{|\F|-1} \leq \frac{N-2}{|\F|-1}$$
    We used first that $[N]\setminus M$ and $M$ are disjoint and union is $[N]$, then we used the above equalities for $\Pr\left[\gamma = \frac{T[j]-v_i}{a_i-j}\right]$ and finally used that $|M| \geq 2$ as $M$ atleast contains $a_i$ and $b_i$.\\
    So we have obtained:
    $$\Pr[v_i +\gamma a_i \in T+\gamma[N]] \leq \frac{N-2}{|\F|-1}$$
    All this was done for the smallest element $i$ of the set $I$.
    For $v+\gamma a$ to be a subvector of $T+\gamma [N]$, we need that:
    $$v_i +\gamma a_i \in T+\gamma[N]$$ for all $i \in I$.
    But observe that,
    $$\Pr[v_i +\gamma a_i \in T+\gamma[N] \, \text{for all} \, i \in I] \leq \Pr[v_i +\gamma a_i \in T+\gamma[N]] \leq \frac{N-2}{|\F|-1}$$
    So, we have that $$\Pr[v_i +\gamma a_i \in T+\gamma[N] \, \text{for all} \,i \in I] \leq \frac{N-2}{|\F|-1}$$
    So, the probability that $v+\gamma a$ is a subvector of $T+\gamma [N]$ is upper bounded by $\frac{N-2}{|\F|-1}$.
    But $N <<|\F|$. So, $\frac{N-2}{|\F|-1}$ is $\frac{O(1)}{|\F|}$. \textbf{Thus, the error probability is upper bounded by $\frac{O(1)}{|\F|}$}.\\
    So, with a probability of atleast $(1-\frac{N-2}{|\F|-1})$, $v+\gamma a$ is not a subvector of $T+\gamma [N]$.
    So, $L(T+\gamma[N], v+\gamma a)=b_3$ is 0 with probability lower bounded by $1-\frac{O(1)}{|\F|}$. \\
    This completes the proof of lemma \ref{lem:reduction}
\end{proof}
\end{comment}

%%% complete protocol listing %%%
%\begin{figure}[t!]
%	\begin{mdframed}
%		
%		\underline{Setup $(1^\secp,N,m, \vecT, \vecT')$}:
%		\begin{itemize}[leftmargin=1em]
%			\item $\srs = (\{\gone{\tau^i}\}_{i=0}^N, \{\gtwo{\tau^i}\}_{i=0}^N)$ for $\tau\gets \F$.
%			\item $W_2^i=\gtwo{(\ell(X) - i)/(X-\xi^i)}$, $i\in [N]$(needed by prover)
%			\item $W_3^i=\gtwo{\vpolyN(X)/(X-\xi^i)}$, $i\in [N]$(needed by prover)
%			\item $\gone{\ell(X)}, \gone{\vpolyN(X)}, \gtwo{\vpolyN(X)}$(known by both)
%		\end{itemize}
%		
%		\underline{Precompute $(\vecT, \vecT')$}:
%		\begin{itemize}[leftmargin=1em]
%			\item $W_1^i=\gtwo{(T(X)-T(\xi^i))/(X-\xi^i)}$, $i\in [N]$,
%			\item ${W_1^i}'=\gtwo{(T'(X) - T'(\xi^i))/(X-\xi^i)}$, $i\in [N]$.
%		\end{itemize}
%		
%		{\bf Common Input}: $\srs$, $c_T, c_T', c_\op, c_a, c_w\in \Gone$.\\
%		{\bf Prover's Input}: Vectors $\vecT,\vecT',\vec{\op},\vec{a},\vec{w}$ and their encoding polynomials.\\
%		
%		{\bf Round 1}: Commit to sub RAMs.
%		\begin{enumerate}[leftmargin=1em, label=\arabic*.]
%			\item $\prover$ computes $\vec{v}=\vecT[\,\vec{a}\,]$, $\vec{v}'=\vecT'[\,\vec{a}\,]$ and the encoding
%			polynomials $v(X)$ and $v'(X)$.
%			\item $\prover$ sends $c_v = \gone{v(X)}$, $c_v'=\gone{v'(X)}$.
%			\item $\verifier$ sends $\chi\gets \F$.
%		\end{enumerate}
%		
%		{\bf Round 2}: Execute committed index lookup.
%		\begin{enumerate}[leftmargin=1em, label=\arabic*.]
%			\item $\prover$ and $\verifier$ compute $C_T=c_T + \chi c_T'$, $C_V=c_v + \chi c_v'$.
%			\item $\prover$ computes $P(X) = T(X) + \chi T'(X)$, $V(X)=v(X) + \chi v'(X)$.
%			\item $\prover$ computes $I=\{a_i: i\in [m]\}$, $\setN_I=\{\xi^i: i\in I\}$.
%			\item $\prover$ computes polynomials:
%			\begin{itemize}[leftmargin=1em, label=-]
%				\item Vanishing polynomial $Z_I(X)$ of $\setN_I$.
%				\item Polynomial $h(X)=\sum_{i\in [m]}\xi^{a_i}\tau_i(X)$.
%				\item Restrictions $P_I(X),\ell_I(X)$ of $P(X),\ell(X)$ on set $I$.
%				\item $K(X)=\prod_{i\in [m]}(X-\xi^{a_i})$, $q(X)=K(X)/Z_I(X)$
%				\item $D(X)= \sum_{i\in I}\frac{\Delta_i Z_I'(\xi^i)}{Z_\setN'(\xi^i)}\kappa_i(X)$ by interpolation as described in section 5.2
%			\end{itemize}
%			
%			\item $\prover$ sends $c_p = \gone{P_I(X)}$, $c_z=\gone{Z_I(X)}$, $c_{z2}=\gtwo{Z_I(X)}$, $c_h=\gone{h(X)}$, $c_l = \gone{\ell_I(X)}$,
%			$c_k = \gone{K(X)}$, $c_q = \gone{q(X)}$, $c_d=\gone{D(X)}$
%			\item $\verifier$ sends $\gamma\gets \F$.
%		\end{enumerate}
%		
%		{\bf Round 3}: Prover send aggregated quotients.
%		\begin{enumerate}[leftmargin=1em, label=\arabic*.]
%			\item $\prover$ computes $g(X)=P_I(X) + \gamma \ell_I(X) + \gamma^2 Z_I(X)$.
%			\item $\prover$ computes $Q(X) = (g(h(X)) - v(X) -\gamma a(X))/Z_\setV(X)$.
%			\item $\prover$ computes: $W = \sum_{i\in [m]} \frac{1}{Z_I'(\xi^i)} (W_1^i + \chi {W_1^i}' + \gamma W_2^i + \gamma^2 W_3^i)$.
%			\item $\prover$ sends $W\in \Gtwo$, $c_Q=\gone{Q(X)}$.
%			\item $\verifier$ computes $c_g = c_p + \gamma c_l + \gamma^2 c_z$, $C_G = C_T + \gamma\gone{\ell(X)}+\gamma^2\gone{\vpolyN(X)}$.
%			\item $\verifier$ checks: $e(C_G - c_g, \gtwo{1})=e(c_z, W)$.
%			\item $\verifier$ checks: $e(c_T-c_{T'}, c_{z2})=e(c_d, \gtwo{\vpolyN(X)})$
%			\item $\verifier$ checks: $e(c_z, [1]_2)=e([1]_1, c_{z2})$
%			\item $\verifier$ sends $\alpha\gets \F$.
%		\end{enumerate}
%		
%		Continued in Figure ~\ref{fig:complete-listing-3}
%	\end{mdframed}
%	\caption{Batching-Efficient RAM Protocol}
%	\label{fig:complete-listing-1}
%\end{figure}
%
%\begin{figure}[t!]
%    \begin{mdframed}
%        \begin{center}
%            Continued from Figure \ref{fig:complete-listing-1} %\moumita{merged the 2nd and 3rd figure.}
%        \end{center}
%        {\bf Round 4}: Prover sends evaluations.
%        \begin{enumerate}[leftmargin=1em, label=\arabic*.]
%            \item $\prover$ computes $\val{\alpha}{v}=v(\alpha)$, $\val{\alpha}{a}=a(\alpha)$, $\val{\alpha}{h}=h(\alpha)$, $\val{\alpha}{K}=K(\alpha)$,
%            $\val{h(\alpha)}{g}=g(h(\alpha))$, $\val{\alpha}{Q}=Q(\alpha)$, $\val{\alpha}{q}=q(\alpha)$, $\val{\alpha}{Z}=Z_I(\alpha)$.
%            \item $\prover$ sends $\val{\alpha}{v}$, $\val{\alpha}{a}$, $\val{\alpha}{h}$, $\val{\alpha}{K}$, $\val{h(\alpha)}{g}$, $\val{\alpha}{Q}$,
%            $\val{\alpha}{q}$, $\val{\alpha}{Z}$
%            \item $\prover$ computes polynomial $u(X)$ as in Section ~\ref{subsec:proximity-ram}
%            and sends $c_u=\gone{u(X)}$.
%            \item $\verifier$ checks $\val{\alpha}{Q}(\alpha^m-1)=\val{h(\alpha)}{g}-\val{\alpha}{v}-\gamma \val{\alpha}{a}$.
%            \item $\verifier$ checks $\val{\alpha}{Z}\val{\alpha}{q}=\val{\alpha}{K}$.
%            \item $\verifier$ sets $\beta=\val{\alpha}{K}-1$ and sends $\theta\gets\F$.
%        \end{enumerate}
%
%        {\bf Round 5}: Check correctness of $K$.
%        \begin{enumerate}[leftmargin=1em, label=\arabic*.]
%            \item $\prover$ computes:
%            \begin{align*}
%                Q'(X) &= \big(u(\nu X)(1+\beta \tau_1(X))-u(X)(\alpha - h(X)) \\
%                &\quad + \theta \tau_1(X)(u(X)-1)\big)/Z_\setV(X).
%            \end{align*}
%            \item $\prover$ sends $c_Q'=\gone{Q'(X)}$.
%            \item $\verifier$ sends $x\gets \F$.
%        \end{enumerate}
%
%        {\bf Round 6}: Prover sends more evaluations.
%        \begin{enumerate}[leftmargin=1em, label=\arabic*.]
%            \item $\prover$ computes $\val{x}{u}=u(x)$, $\val{\nu x}{u}=u(\nu x)$, $\val{x}{h}=h(x)$, $\val{x}{Q'}=Q'(x)$
%            \item $\prover$ sends $\val{x}{u}$, $\val{\nu x}{u}$, $\val{x}{h}$, $\val{x}{Q'}$.
%            \item $\verifier$ checks $\val{x}{Q'}(x^m-1)=\val{\nu x}{u}(1+\beta \tau_1(x))-\val{x}{u}(\alpha - \val{x}{h})$.
%            \item $\verifier$ sends $r_a, r_h, r_q, r_v, r_K, r_Q, r_Z\gets \F$ and $r_h', r_u', r_Q'\gets \F$.
%        \end{enumerate}
%
%        \begin{center}
%            Continue in Figure ~\ref{fig:complete-listing-3}.
%        \end{center}
%
%
%    \end{mdframed}
%    \caption{Batching-Efficient RAM Protocol: Continued}
%    \label{fig:complete-listing-2}
%\end{figure}
%
%\begin{figure}[t!]
%    \begin{mdframed}
%        \begin{center}
%            Continued from Figure \ref{fig:complete-listing-2}
%        \end{center}
%        \item {\bf Round 7}: Check aggregated evaluation.
%        \begin{enumerate}[leftmargin=1em, label=\arabic*.]
%            \item $\prover$ computes:
%            \begin{align*}
%                \Phi_\alpha(X) &= r_a a(X)+ r_h h(X) + r_q q(X) + r_v v(X) \\
%                &\quad + r_K K(X) + r_Q Q(X) + r_Z Z_I(X) \\
%                \Phi_x(X) &= r_h' h(X) + r_u' u(X)+r_Q'Q'(X)
%            \end{align*}
%            \item $\prover$ computes $\Pi_\alpha = \KZGopen(\srs, \Phi_\alpha(X), \alpha)$.
%            \item $\prover$ computes $\Pi_x = \KZGopen(\srs, \Phi_x(X), x)$.
%            \item $\prover$ computes $\Pi_g = \KZGopen(\srs, g(X), \val{\alpha}{h})$.
%            \item $\prover$ computes $\Pi_u = \KZGopen(\srs, u(X), \nu x)$.
%            \item $\prover$ sends $\Pi_\alpha$, $\Pi_x$, $\Pi_g$ and $\Pi_u$.
%            \item $\verifier$ computes:
%            \begin{align*}
%                \gone{\Phi_\alpha} &= r_a c_a + r_h c_h + r_q c_q + r_v c_v + r_z c_z + r_K c_K + r_Q c_Q. \\
%                \gone{\Phi_x} &= r_h' c_h + r_u' c_u + r_Q' c_Q' \\
%                V_{\alpha} &= r_a \val{\alpha}{a} + r_h \val{\alpha}{h} + r_q \val{\alpha}{q} + r_v \val{\alpha}{v} \\
%                &\quad + r_z\val{\alpha}{Z} r_K \val{\alpha}{K} + r_Q \val{\alpha}{Q}. \\
%                V_x &= r_h' \val{x}{h} + r_u' \val{x}{u}+r_Q' \val{x}{Q'}
%            \end{align*}
%            \item $\verifier$ checks:
%            \begin{itemize}[leftmargin=1em]
%                \item $\KZGverify(\srs, \gone{\Phi_\alpha}, V_\alpha, \alpha, \Pi_\alpha)$.
%                \item $\KZGverify(\srs, \gone{\Phi_x}, V_x, x, \Pi_x)$.
%                \item $\KZGverify(\srs, c_g, \val{h(\alpha)}{g}, \val{\alpha}{h}, \Pi_g)$.
%                \item $\KZGverify(\srs, c_u,\val{\nu x}{u}, \nu x, \Pi_u)$.
%            \end{itemize}
%            \item $\prover$ and $\verifier$ set $c_S=(c_a, c_v)$, $c_S'=(c_a, c_v')$, $c_o=(c_\op, c_a, c_w)$.
%            \item $\prover$ and $\verifier$ execute argument for $(c_S, c_o, c_S')\in \CLRAM$ \\
%            (Section ~\ref{sec:poly-proto-ram}).
%        \end{enumerate}
%    \end{mdframed}
%    \caption{Batching-Efficient RAM Protocol-Continued}
%    \label{fig:complete-listing-3}
%\end{figure}


\section{Committed Index Lookup from Caulk+}\label{app:committed-index-lookup}

In this section, we present an explicit (non-black-box) adaptation of~\cite{EPRINT:PosKat22} to obtain a committed index lookup, which again incurs costs comparable to a single instance of the underlying sub-vector protocol. Let $m,N\in \N$ be fixed parameters with $m < N$ and let $\srs$ denote a $\kzg$ setup of degree $d\geq N$ over bi-linear group $(\F$, $\Gone$, $\Gtwo$, $\GT$, $e$, $\gone{1}$, $\gtwo{1}$, $[1]_t)$. Recall that the committed index
lookup relation in Definition ~\ref{defn:comm-index-lookup} involves the prover showing knowledge of vectors $\vecT\in \F^N$,
$\vec{a}\in \F^m$ and $\vec{v}\in \F^m$ corresponding to public commitments $c_T, c_a$ and $c_v$ such that they
satisfy $v_i = \vecT[\,a_i\,] = T_{a_i}$.
We present a polynomial protocol for the same, which is an adaptation of the lookup protocol from Caulk+ ~\cite{EPRINT:PosKat22}.
However, here we do not aim for zero-knowledge. Let $T(X)=\enc{t}{\setN}$, $a(X)=\enc{a}{\setV}$ and
$v(X)=\enc{v}{\setV}$ denote the polynomials encoding the vectors $\vec{t},\vec{a}$ and $\vec{v}$ respectively.
The verifier knows commitments to these polynomials at the start of the protocol.
Now $v_i = \vec{t}[a_i]$ for $i\in [m]$ is equivalent to $v(\nu^i) = T(\xi^{a(\nu^i)})$ for $i\in [m]$. To
obtain a polynomial protocol, the prover interpolates a polynomial $h(X)=\sum_{i=1}^m \xi^{a_i}\tau_i(X)$, which satisfies
$h(\nu^i)=\xi^{a(\nu^i)}$. To show that polynomial $h$ correctly ``exponentiates'' evaluations of $a(X)$, we consider the
inverting polynomial $\ell(X)=\sum_{i=1}^N i\mu_i(X)$ which behaves like ``log'' over $\setN$ by evaluating to $i$ on $\xi^i$. Now, we see
that all constraints are encoded as polynomial identities below:
\begin{equation}
	\begin{aligned}
		\ell(h(X)) &= a(X) \quad \text{mod } Z_{\setV}\\  % & \quad\text{ encodes } & \quad \forall i\in [m]:& h(\nu^i) = \xi^{a(\nu^i)}  \\
		T(h(X)) &= v(X) \quad \text{mod } Z_\setV \\ % \quad\text{ encodes } & \quad \forall i\in [m]:& v_i = \vec{t}[a_i] \\
		Z_{\setN}(h(X)) &= 0 \qquad \text{mod } Z_\setV  %&\quad\text{ encodes } & \quad \forall i \in [m]:& h(\nu^i)\in \setN
	\end{aligned}
	\label{eq:comm-index-lookup}
\end{equation}
The last polynomial identity ensures that evaluations of $h$ on $\setV$ lie in $\setN$ (the set of roots of $\vpolyN$). Since the polynomial $\ell$ is one-one
over $\setN$, the first equation implies $h(\nu^i)=\xi^{a_i}$ for all $i\in [m]$. The desired relation $v_i=T_{a_i}$ now follows from the second identity.
The above formulation involves composition with polynomials $\ell,T$ and $\vpolyN$ of degree $O(N)$, which is inefficient. We use the trick from
\cite{EPRINT:PosKat22}, where we work with low-degree restrictions of $O(N)$-degree polynomials such as $T, \ell$ over the set
$\setN_I=\{{h(\nu^i)}: i\in [m]\}=\{\xi^{a_i}:i\in I\}\subseteq \setN$, where $I=\{a_i: i\in [m]\}$. The prover
commits to the polynomials $Z_I(X)=\prod_{i\in I}(X-\xi^i)$, $h(X)$ and low degree ($<m$) restrictions $T_I, \ell_I$ of $T$ and $\ell$
on the $\setN_I$ respectively. The polynomial protocol then checks the following:
\begin{equation}
	\begin{alignedat}{3}
		T(X) - T_I(X) &= 0 \quad \text{ mod } Z_I ,&\quad& T_I(h(X)) &= v(X) \quad \text{ mod } Z_{\setV} \\
		\ell(X) - \ell_I(X) &= 0 \quad \text{ mod } Z_I ,&\quad& \ell_I(h(X)) &= a(X) \quad \text{ mod } Z_{\setV} \\
		Z_{\setN}(X) &= 0 \quad \text{ mod } Z_I ,&\quad& Z_I(h(X)) &= 0 \quad \text{ mod } Z_{\setV}
	\end{alignedat}
	\label{eq:poly-comm-index}
\end{equation}
It must be noted that the above identities imply the earlier polynomial identities in \eqref{eq:comm-index-lookup}. This is so because evaluations
of $h$ on $\setV$ are roots of $Z_I$, which implies $T_I(h(\nu^i))=T(h(\nu^i))$, $\ell_I(h(\nu^i))=\ell(h(\nu^i))$ and $\vpolyN(h(\nu^i))=0$ over $\setV$.
While the identities on the left still involve a degree $N$ polynomial, we can use the $\srs$ to check the polynomial
identity at the point $\tau$ encoded in the $\srs$. For example, we can evaluate the encoded quotient $\gtwo{Q(X)} =$
$\gtwo{\frac{(T(X) - T_I(X)}{Z_I(X)}}$ using the relation:
\begin{equation*}
	\gtwo{\frac{T(X)-T_I(X)}{Z_I(X)}} = \sum_{i\in I}\frac{1}{Z_I'(\xi^i)}\gtwo{\frac{T(X)-t_i}{X-\xi^i}}
\end{equation*}
By pre-computing the $\kzg$ proofs $W_1^i=\gtwo{\frac{T(X)-t_i}{X-\xi^i}}$ for all $i\in [N]$, the encoded quotient can be
evaluated using $O(m)$ $\Gtwo$-operations and $O(m\log^2 m)$ $\F$-operations.
The identity is then checked using a real pairing check
$$e(\gone{T(X)}-\gone{T_I(X)},\gtwo{1})=e(\gone{Z_I(X)},\gtwo{Q(X)}).$$
Similarly, we also pre-compute the encoded
quotients $W_2^i=\gtwo{\frac{\ell(X) - i}{X-\xi^i}}$ and $W_3^i=\gtwo{\frac{\vpolyN(X)}{X-\xi^i}}$ for all $i\in [N]$.
The quotients can be computed in time $O(N\log N)$ using the techniques in ~\cite{EPRINT:FeiKho23}. Using $\kzg$ commitment
scheme the polynomial relations over $Z_\setV$ can be checked in a standard manner
by having the prover send evaluation proofs for the committed polynomials at a random point chosen by the verifier.
The total prover effort incurred is $O(m^2)$ group and field operations.
Thus, we have:
\begin{lemma}\label{lem:comm-index-lookup}
	Assuming $\kzg$ is extractable polynomial commitment scheme, there exists a succinct argument of knowledge for
	the relation $\RLOOK$ with prover complexity of $O(m^2)$, given access to pre-computed parameters of size $O(N)$.
\end{lemma}


    \bibliographystyle{plain}
    \bibliography{crypto_crossref, main}
\end{document}
