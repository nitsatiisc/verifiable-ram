\begin{figure}[htbp]
    \centering
    \includegraphics[width=\textwidth]{RAM-Lookup}
    \caption{Illustrating different steps of sub-linear lookup protocol between large RAMs $\vecT$ and $\vecT'$.}
    \label{fig:blueprint}
\end{figure}

The construction in the previous section results in prover complexity which is quasi-linear in both the
size of the RAM and the number of operations.
Our goal in this section is to achieve prover complexity which is {\em sublinear} in the size of the RAM.
In what follows, let $N$ denote the size of the RAM (upper-case to signify it's large) and $m$ denote the number
of operations in a batch. We will use a vectors in $\F^N$ to denote the ``large'' RAMs, where index column is implicitly
assumed to be $(1,\ldots,N)$.
Let $\vec{T},\vec{T'}\in \F^N$ denote the initial and final RAM states, and let $\vec{o}$ be
a sequence of $m$ operations which updates $\vec{T}$ to $\vec{T}'$. Let $\vec{a}\in \F^m$ denote the vector
of RAM indices referenced by the operations in $\vec{o}$, i.e, $a_i$ is the index referenced by $i^{th}$ operation.
To prove the transformation of $\vec{T}$ to $\vec{T}'$ via operation sequence $\vec{o}$, we proceed as follows:
\begin{itemize}[leftmargin=2em, label=-]
\item We isolate sub-tables $S=(\vec{a},\vec{v})$ and $S'=(\vec{a},\vec{a'})$ of $T$ and $T'$ consisting of
rows corresponding to indices in $\vec{a}$. This requires proving $\vec{v}=\vecT[\vec{a}]$ and $\vec{v'}=
\vecT'[\vec{a}]$, which we show using {\em committed index lookup} discussed in Section ~\ref{subsec:committed-index-lookup}.

\item On the isolated sub-table $S$ and $S'$ of size $m$, we use the standard memory checking arguments (c.f. argument
presented in Section \ref{sec:poly-proto-ram-app}) to prove that sequence $\vec{o}$ correctly updates $S$ to $S'$ with
prover complexity of $\wt{O}(m)$.

\item Finally, we show that the RAMs $T$ and $T'$ are identical outside indices in $\vec{a}$. We call them $\vec{a}$-identical
and describe the protocol for proving the same in Section ~\ref{subsec:proximity-ram}.
\end{itemize}
The above blueprint is illustrated in Figure ~\ref{fig:blueprint}.

The efficiency of the above approach relies crucially on the efficiency of committed index lookup used to
reduce the size of the RAMs for quasi-linear memory checking methods.
Using recently developed lookup arguments directly is difficult, as their efficiency relies on
table specific expensive pre-computation, which does not help when the table itself is being updated.
To circumvent this, we design a lookup argument, which efficiently proves lookup with access to pre-computed
parameters for a ``nearby'' table. In particular, we show that lookup protocols in
~\cite{CCS:ZBKMNS22,EPRINT:PosKat22,EPRINT:EagFioGab22} can be used to show $m$ lookups from a table $\vecT'$, given pre-computed parameters
for a table $\vecT$ with additional overhead of $O((m+\delta)\log^2 (m+\delta))$, where $\delta$ denotes the number of positions where
$\vecT$ and $\vecT'$ differ (hamming distance). By optimally deferring the $O(N\log N)$ re-computation till we
accumulate $\delta \approx \sqrt{mN}$ updates, we achieve an amortized prover overhead $O(\sqrt{mN})$ over the read-only protocol.
This modification, which applies to all the aforementioned lookup protocols is described in Section ~\ref{sec:update-protocol}.

\noindent{\bf Additional Notation}:
Before proceeding, we introduce the subgroup $\setN=\{\xi,\ldots,\xi^N\}$ consisting of $N^{th}$ roots of unity,
over which we encode vectors in $\F^N$ as polynomials of degree less than $N$. Let $\{\mu_i(X)\}_{i=1}^N$ be the associated
lagrange basis polynomials over the set $\setN$. We also recall the set $\setV$ consisting of $m^{th}$ roots of unity
$\nu,\ldots,\nu^m$ with associated lagrange polynomials as $\{\tau_i(X)\}_{i=1}^m$. For $\vec{f}\in \F^N$, let
$\enc{f}{\setN}$ denote the polynomial encoding of $\vec{f}$ over $\setN$ given by $\sum_{i=1}^N f_i\mu_i(X)$. Similarly,
for $\vec{g}\in \F^m$, let $\enc{g}{\setV}$ denote its polynomial encoding over $\setV$ given by $\sum_{i=1}^m g_i\tau_i(X)$.


\subsection{Committed Index Lookup}\label{subsec:committed-index-lookup}
Let $m,N\in \N$ with $m < N$ and let $\srs$ denote a $\kzg$ setup over bilinear group $(\Gone,\Gtwo,\GT,\gone{1},\gtwo{1},e)$
large enough to commit to polynomials of degree $<N$. Prior works on lookup arguments (~\cite{CCS:ZBKMNS22,EPRINT:PosKat22} etc.)
have considered proving sub-vector relation over committed vectors, i.e, given commitments $c_t$ and $c_v$ to vectors $\vec{t}\in \F^N$
and $\vec{v}\in \F^m$, one proves that for all $i\in [m]$, there exists $j\in [N]$ such that $v_i=t_j$ .
We consider a slightly modified relation,
called {\em committed index lookup}  which also commits to indices where $\vec{v}$ appears in $\vec{t}$.

\begin{definition}\label{defn:comm-index-lookup}
We define the {\em committed index lookup} relation $\RLOOK$ to consist of tuples
$((c_t,c_a,c_v),(\vec{t},\vec{a},\vec{v}))$ where $c_t,c_a,c_v\in \Gone$, $\vec{t}\in \F^N$, $\vec{a},\vec{v}\in \F^m$ such
that $v_i = \vec{t}[a_i]=t_{a_i}$ for all $i\in [m]$ and $c_t = \kzgcommit(\srs, \enc{t}{\setN})$, $c_a=\kzgcommit(\srs,\enc{a}{\setV})$
and $c_v=\kzgcommit(\srs,\enc{v}{\setV})$.
\end{definition}

We present a polynomial protocol for the above relation, which is an adaptation of the lookup protocol from Caulk+ ~\cite{EPRINT:PosKat22}
to the indexed lookup case. Moreover, we do not aim for zero-knowledge. Let $T(X)=\enc{t}{\setN}$, $a(X)=\enc{a}{\setV}$ and
$v(X)=\enc{v}{\setV}$ denote the polynomials encoding the vectors $\vec{t},\vec{a}$ and $\vec{v}$ respectively. The prover commits
to these polynomials. Now $v_i = \vec{t}[a_i]$ for $i\in [m]$ is equivalent to $v(\nu^i) = T(\xi^{a(\nu^i)})$ for $i\in [m]$. To
obtain a polynomial protocol, the prover interpolates a polynomial $h(X)=\sum_{i=1}^m \xi^{a_i}\tau_i(X)$, which satisfies
$h(\nu^i)=\xi^{a(\nu^i)}$. To show that polynomial $h$ correctly ``exponentiates'' evaluations of $a(X)$, we consider the
polynomial $\ell(X)=\sum_{i=1}^N i\mu_i(X)$ which behaves like ``log'' over $\setN$ by evaluating to $i$ on $\xi^i$. Now, we see
that all constraints are encoded as polynomial identities below:
\begin{alignat}{3}
\ell(h(X)) &= a(X) \quad \text{mod } Z_{\setV}(X) & \quad\text{ encodes } & \quad \forall i\in [m]:& h(\nu^i) = \xi^{a(\nu^i)}  \\
T(h(X)) &= v(X) \quad \text{mod } Z_\setV(X) & \quad\text{ encodes } & \quad \forall i\in [m]:& v_i = \vec{t}[a_i] \\
Z_{\setN}(h(X)) &= 0 \quad \text{mod } Z_\setV(X)  &\quad\text{ encodes } & \quad \forall i \in [m]:& h(\nu^i)\in \setN
\end{alignat}
The above formulation involves composition with polynomials $\ell,T$ and $\vpolyN$ of degree $O(N)$, which is inefficient. We use the trick from
\cite{EPRINT:PosKat22}, where we work with low-degree restrictions of polynomials such as $T, \ell$ over the set
$\setN_I=\{{h(\nu^i)}: i\in I\}=\{\xi^{a_i}:i\in I\}\subseteq \setN$, where $I=\{a_i: i\in [m]\}$. To this end, the prover
commits to the polynomial $Z_I(X)=\prod_{i\in I}(X-\xi^i)$, and low degree ($<m$) restrictions $T_I, \ell_I$ of $T$ and $\ell$
on $\setN_I$ respectively. The polynomial protocol then checks the following:
\begin{alignat}{3}\label{eq:poly-comm-index}
T(X) - T_I(X) &= 0 \quad \text{ mod } Z_I(X) &&,\quad T_I(h(X)) &= v(X) \quad \text{ mod } Z_{\setV}(X) \\
\ell(X) - \ell_I(X) &= 0 \quad \text{ mod } Z_I(X) &&,\quad \ell_I(h(X)) &= a(X) \quad \text{ mod } Z_{\setV}(X) \\
Z_{\setN}(X) &= 0 \quad \text{ mod } Z_I(X) &&,\quad Z_I(h(X)) &= 0 \quad \text{ mod } Z_{\setV}(X)
\end{alignat}
While the identities on the left still involve a degree $N$ polynomial, we can use the $\srs$ to check the polynomial
identity at the point $\tau$ encoded in the $\srs$. For example, we can evaluate the encoded quotient $\gtwo{Q(X)} =$
$\gtwo{\frac{(T(X) - T_I(X)}{Z_I(X)}}$ using the relation:
\begin{equation*}
\gtwo{\frac{T(X)-T_I(X)}{Z_I(X)}} = \sum_{i\in I}\frac{1}{Z_I'(\xi^i)}\gtwo{\frac{T(X)-t_i}{X-\xi^i}}
\end{equation*}
By pre-computing the $\kzg$ proofs $W_1^i=\gtwo{\frac{T(X)-t_i}{X-\xi^i}}$ for all $i\in [N]$, the encoded quotient can be
evaluated using $O(m)$ $\Gtwo$-operations and $O(m\log^2 m)$ $\F$-operations. The identity is then checked using a real
pairing check $e(\gone{T(X)}-\gone{T_I(X)},\gtwo{1})=e(\gone{Z_I(X)},\gtwo{Q(X)})$.
Similarly, we also pre-compute the encoded
quotients $W_2^i=\gtwo{\frac{\ell(X) - i}{X-\xi^i}}$ and $W_3^i=\gtwo{\frac{\vpolyN(X)}{X-\xi^i}}$ for all $i\in [N]$.
The quotients can be computed in time $O(N\log N)$ using the techniques in ~\cite{EPRINT:FeiKho23}.
The polynomial relations over $Z_\setV$ can be checked in a standard manner via evaluations at a random point with $O(m^2)$ prover effort.
Thus, we have:
\begin{lemma}\label{lem:comm-index-lookup}
Assuming $\kzg$ is extractable polynomial commitment scheme, there exists a succinct argument of knowledge for
the relation $\RLOOK$ with prover complexity of $O(m^2)$, given access to pre-computed parameters of size $O(N)$.
\end{lemma}

\subsection{Proximity of RAM States}\label{subsec:proximity-ram}
For a vector $\vec{a}\in [N]^m$, let $\uniq{a}=\{a_i: i\in [m]\}$ denote the subset of unique values in $\vec{a}$. We call two
RAM states $\vecT, \vecT'\in \F^N$ to be $\vec{a}$-{\em identical} if $\vecT[i]=\vecT'[i]$ for all $i\not\in\uniq{a}$. As before,
let $T(X),T'(X)$ and $a(X)$ be polynomials encoding the vectors $\vecT,\vecT'$ (over $\setN$) and $\vec{a}$ (over $\setV$). The
polynomial protocol involves proving the relation $Z_I(X)(T(X) - T'(X)) = 0$ over $Z_\setN$ where $Z_I(X)=\prod_{i\in I}(X-\xi^i)$
for $I=\uniq{a}$ denotes the vanishing polynomial over the support of $\vec{a}$.
The prover commits to polynomial $Z_I$ and proves (i) $Z_I(T - T') = 0 \text{ mod } Z_\setH$ and (ii) $Z_I$ is the vanishing
polynomial of the support of vector $\vec{a}$. To prove the first relation, the prover computes the polynomial $Q(X)$ as below:
\begin{align}\label{eq:poly-q}
Q(X) &= \frac{(T(X)-T'(X))\cdot Z_I(X)}{Z_\setN(X)} \nonumber \\
&= \sum_{i\in I}\frac{(t_i - t_i')\mu_i(X)}{Z_\setN(X)} Z_I(X) \nonumber \\
\intertext{ Substituting, $\Delta_i=t_i-t_i'$, $\mu_i(X)=\vpolyN(X)/(\vpolyN'(\xi^i)(X-\xi^i))$ }
&=\sum_{i\in I}\frac{\Delta_i}{Z_\setN'(\xi^i)}\left(\frac{Z_I(X)}{X-\xi^i}\right) = \sum_{i\in I}\frac{\Delta_i Z_I'(\xi^i)}{Z_\setN'(\xi^i)}\kappa_i(X)
\end{align}
In the above, the summation only runs over indices in $I$, as $t_i=t_i'$ for $i\not\in I$. In the final equality, we use
$\kappa_i(X) = Z_I(X)/(Z_I'(\xi^i)(X-\xi^i))$ for $i\in I$ which we recognize as the lagrange basis polynomials for the set
$\{\xi^i: i\in I\}$. Thus, Equation \eqref{eq:poly-q} implies that $Q$ is a degree $|I|-1$ polynomial, with
$Q(\xi^i)=\Delta_i Z_I'(\xi^i)/\vpolyN'(\xi^i)$ for $i\in I$. The prover can therefore interpolate $Q(X)$ (in power basis)
in $O(|I|\log^2 |I|)$ $\F$-operations and compute $\gtwo{Q(X)}$ in $O(|I|)$ $\Gtwo$-operations.

Next, the prover needs to show that $Z_I(X)$ is indeed the vanishing polynomial of $\setN_I=\{\xi^i: i\in I\}$ where $I=\uniq{a}$.
We again use the polynomial $h(X)=\sum_{i=1}^m \xi^{a_i}\tau_i(X)$ which interpolates the vector $(\xi^{a_1},\ldots,\xi^{a_m})$.
The correctness of the $h$ polynomial can be established using the restriction $\ell_I$ of ``log'' polynomial $\ell$ as before.
We show that $Z_I(h(X)) = 0$ over $Z_\setV$ which shows that $Z_I$ vanishes over entire vector interpolated
by $h$. To assert that $Z_I$ has no additional roots, the prover commits to the product polynomial
$K(X)=\prod_{i=1}^m (X - h(\nu^i))$ and the quotient polynomial $q(X)=K(X)/Z_I(X)$. The verifier checks the polynomial identities
at $\alpha$, i.e $K(\alpha)=q(\alpha)Z_I(\alpha)$ and $K(\alpha)=\prod_{i=1}^m(\alpha - h(\nu^i))$. The former is easily accomplished
using evaluation proofs for $K,q$ and $Z_I$ at $\alpha$. For checking the latter, the prover commits to another polynomial
$u(X)$ satisfying $u(\nu^i)=\prod_{j=1}^{i-1}\big((\alpha - h(\nu^j))/(1 + \beta\tau_1(\nu^j))\big)$ for $i\in [m]$
where $\beta=K(\alpha) - 1$.
The verifier ensures the correctness of $u(X)$ by checking:
\begin{align*}
\tau_1(X)(u(X) - 1) &= 0 \text{ mod } Z_{\setV} \\
u(\nu X)(1+\beta \tau_1(X))-u(X)(\alpha - h(X)) &= 0 \text{ mod } Z_\setV.
\end{align*}
Note that the constraints on $u(X)$ essentially
ensure that $K(\alpha)\cdot 1\cdots 1 = (\alpha - h(\nu))\cdots (\alpha - h(\nu^m))$, which follows from the fact that
$u(\nu)=u(\nu^{m+1})=1$.

\subsection{Batching Efficient RAM: Combined Protocol}\label{subsec:all-together}
We put the entire protocol together now. Let $\setind$ denote the set of indices $\{1,\ldots,N\}$, and $\mathcal{I}_N$
denote the vector $(1,\ldots,N)$. We formally define the committed RAM relation for which we present the argument of
knowledge in this section.
\begin{definition}\label{defn:committed-ram}
We define the {\em committed ram} relation
$\CRAM$ to consist of tuples $((c_T, c_T', c_\op, c_a, c_w),(\vecT, \vecT',\vec{\op},\vec{a},\vec{w}))$
such that:
\begin{itemize}[leftmargin=1em]
\item $(T,\vec{o},T')\in \LRAM{I}{N}{m}$ for $T=(\setind_N,\vecT)$, $T'=(\setind_N,\vecT')$ and $\vec{o}=(o_1,\ldots,o_m)$
where $o_i=(\op_i, a_i, w_i)\in \RAMOp{I}$ for all $i\in [m]$.
\item $c_T=\kzgcommit(\srs,\enc{T}{\setN})$, $c_T'=\kzgcommit(\srs,\enc{T'}{\setN})$, $c_\op=\kzgcommit(\srs,\enc{\op}{\setV})$,
$c_a=\kzgcommit(\srs,\enc{a}{\setV})$ and $c_w$ $=$ \\ $\kzgcommit(\srs,\enc{w}{\setV})$.
\end{itemize}
\end{definition}
As outlined in the blueprint, the prover first commits to ``smaller'' RAMs $S=(\vec{a},\vec{v})$ and $S'=(\vec{a},\vec{v}')$
where $\vec{v}=\vecT[\vec{a}]$ and $\vec{v}'=\vecT'[\vec{a}]$. The prover commits to $S$ and $S'$ by sending commitments
$c_v$ and $c_v'$ to $\vec{v}$ and $\vec{v}'$. Then the prover and verifier execute the committed index lookup protocol to
prove:
\begin{equation}
(c_T, c_a, c_v)\in \RLOOK\, \wedge\, (c_T', c_a, c_v')\in \RLOOK
\end{equation}
The verifier uses a random challenge $\chi\gets \F$ to reduce two instances of $\RLOOK$ to one instance
$(c_T + \chi c_T', c_a, c_v + \chi c_v')\in \RLOOK$. Thereafter, the prover and verifier execute the argument for
showing $(S,\vec{o},S')\in \LRAM{I}{m}{m}$ as described in Section \ref{subsec:succ-args}. Finally, we show that
RAMs $\vecT$ and $\vecT'$ are $\vec{a}$-identical using the protocol in Section \ref{subsec:proximity-ram}.

\section{Fast Lookups from Approximate Pre-Processing}\label{sec:update-protocol}

The recent progress and interest in lookup arguments has been stellar, as exemplified by the series of works
both in uni-variate setting ~\cite{CCS:ZBKMNS22,EPRINT:PosKat22,EPRINT:ZGKMR22,EPRINT:EagFioGab22} and multi-variate
setting ~\cite{lasso}. However, the excellent online efficiency of current constructions relies on expensive
$\wt{O}(|T|)$ table-specific  pre-computation for a table $T$, or on tables exhibiting tensor structure as in ~\cite{lasso}.
This limits their application in settings where such assumptions are not viable, for example when tables model account balances in
a layer 2 (L2) blockchain network. We make the first attempt in this direction. Our key idea is to extend the utiltiy of pre-computed
parameters for a table $\vecT$, to proving lookups from tables $\vecT'\neq \vecT$. Essentially, we show that for $\delta=\Delta(\vecT, \vecT')$,
an argument for $m$ lookups from $\vecT'$ incurs an additional prover overhead of $(m+\delta)\log^2(m+\delta)$. We note that overhead is additive
in $\delta$ and that too only {\em quasi} linear. Our competitive overhead rests on several innovative applications of algebraic
algorithms, which are summarised in Section ~\ref{subsec:comp-algebra-app}.

\noindent{\bf Naive approaches are inadequate}: We note that the aforementioned constructions of lookup arguments require encoded quotients
of the form $\gany{(T(X)-T(\xi^i))/(X-\xi^i)}$ for upto $m$ values of $i$ during the proof generation. While constructions ~\cite{CCS:ZBKMNS22,EPRINT:PosKat22}
consider quotients encoded in the group $\Gtwo$, the protocol in ~\cite{EPRINT:EagFioGab22} encodes them in $\Gone$. We use a generic $[\,\cdot\,]_g$ to
account for protocol-specific choices. We also see that even a small change to the table requires one to update all the quotients (the polynomial $T(X)$ is
common to all quotients). Updating the $|\vecT|$ quotients for each batch is clearly infeasible. One could consider delaying the updation of the quotients, till
the time they are actually required in a proof, which happens when the corresponding index in the table is involved in lookup. However, each of the $m$ quotients
is now potentially ``lagging'' by $\delta$ updates, so we would need $\Omega(m\delta)$ group operations to refresh all of them. This gives us multiplicative degradation
with $\delta$, and is clearly unsustainable for reasonable values of $\delta$. We abandon the idea of computing individual encoded quotients, and instead attempt
to directly compute the aggregate encoding $\gany{(T(X)-T_I(X))/Z_I(X)}$, which as seen earlier is given by the summation below:
\begin{equation}\label{eq:encoded-quotient}
\gany{\frac{T(X)-T_I(X)}{Z_I(X)}} = \sum_{i\in I}\frac{1}{Z_I'(\xi^i)}\gany{\frac{T(X)-T(\xi^i)}{X-\xi^i}}
\end{equation}
We now describe our approach.
%Recall that our lookup protocol in section 5.1 involves certain precomputations by the prover namely $W_1^i, W_2^i, W_3^i$. $W_2^i$ and $W_3^i$ do not depend on the table. However, $W_1^i$ depends on the lookup table and their values will change even if the table changes by a small amount. It is expensive to recompute all the $W_1^i$ for every small change in the table and this will affect the efficiency of our lookup protocol in the long run.\\\\
%In this section, we show how to achieve efficient lookups even when the table is changing frequently, as long as the cumulative change in the table is small. \\
%In particular, we show how the prover can compute $[Q(X)]_2=\gtwo{\frac{T(X)-T_I(X)}{Z_I(X)}}$ without computing all the $W_1^i$(thus minimizing the overhead).\\
%The overhead(as long as the table doesn't change too much) will be much lower than the time needed for the lookup and so is very practical.

\subsection{Base + Cache approach}\label{subsec:base-cache}
The key idea we employ is to express the current table $\vecT\in \F^N$ as $\vecTbase + \vecTcache$, where $\vecTbase$ is the table for which we assume that
the encoded quotients are available (via the $O(N\log N)$ computation), and $\vecTcache$ captures the changes to the table since. We will periodically update (say
after $s$ batch updates) $\vecTbase$ to current table state, and re-compute all the quotients (we call it the {\em offline} phase).
We will revisit the question on choosing $s$ optimally later. Let $I\subseteq [N]$ denote the set of indices in the current batch of $m$ lookups. The {\em online}
phase of our proof generation involves computing the sum in Equation \eqref{eq:encoded-quotient} for the table $\vecT$.
%that we do not compute $W_1^i$ after each change of the table. Instead, this expensive computation will be done periodically for all $i \in [N]$ after say $s \in \mathbb{Z}$ batches.
%Let current table $\vecT$ can be represented as $\vecTbase + \vecTcache$ where the vector
%$\vecTbase$ denotes the base table (with respect to which $W_1^i$ was last computed for all $i \in [N]$) and the vector $\vecTcache$ corresponds to the changes
%that have happened to the base table since the last rebasing (rebasing denotes computation of all $W_1^i$)\\\\
%Thus, there is an \textbf{online} phase which happens after every batch (which includes computation of $[Q(X)]_2$ among other things) and an \textbf{offline} phase which consists of the rebasing(this is all prover computations) \\\\
\begin{comment}
\noindent{\bf Offline Phase}: This computation is executed once after every $s$ rounds. Here, the prover updates the base vector $\vecTbase$ with the changes in the cache vector
$\vecTcache$ by setting $\vecTbase := \vecTbase + \vecTcache$ and simultaneously clears the cache vector by setting
$\vecTcache = 0$.\\
It computes the commitment of $T_b$ as well\\
It also re-computes the $\mathsf{KZG}$ opening proofs $[W_1^i(X)]_2$ for $i\in [N]$ where
$W_1^i(X) = (\Tbasepoly{X} - t_i)/(X-\xi^i)$. Here, $t_i=\Tbasepoly{\xi^i}$ are the coordinates
of the updated base vector $\vecTbase$.\\
As mentioned in section 5.1 this can be done in $O(N\log N)$ group and field operations.\\\\
\noindent{\bf Online Phase}:
The online phase happens for every batch because the purpose of this phase is to ensure that all the things needed for the current execution of the lookup protocol are available. We show how the prover computes the next table $T'$ from the current table $T$ and the new Cache vector from the old cache vector (by an inductive argument this suffices)
\begin{enumerate}[leftmargin=1em]
    \item Prover has the $T$ for the current round and the commitment $[T(X)]_1$ as well(because these are just the $T'$, $[T'(x)]_1$ of the last round)
    \item The $\vecTcache$ and $\vecTcache(X)$ is also updated to the start of the current round (contains information till previous round:$\vecT=\vecTbase+\vecTcache$)
    \item The prover updates the cache using the current batch: $\vec{T'}_{\mathsf{ch}}[i] = \vec{T}_{\mathsf{ch}}[i] + \Delta_i$ for $i\in I$ in $O(m)$ $\F$ operations
    \item Here $\Delta_i$ for all $i \in I$ is the change that will happen to $\vecT$ \textbf{during the current round}
    \item Prover computes the commitment to the new cache polynomial:
    $$[\vec{T'}_{\mathsf{ch}}(X)]_1=[\vec{T}_{\mathsf{ch}}(X)]_1+\sum_{i\in I}\Delta_i[\mu_i(X)]_1$$ in
    $O(m)$ $\Gone$ operations.
    \item Prover also gets ${T}'$ as ${T'}[i]=T_b[i]+ \vec{T'_{\text{ch}}}[i]$ using the $T_b$ and the latest cache
    \item Prover computes the commitment to the new table $\vecT'$: $[T'(X)]_1=[\Tbasepoly{X}]_1+[\vec{T'}_{\mathsf{ch}}(X)]_1$

    \item In addition, the other things (apart from $[Q(X)]_2$) needed for the current round of the lookup protocol are also computed by the prover as described in the lookup protocol in section 5.1 as it is just naive computation

\end{enumerate}
\subsection{Computation of $[Q(X)]_2$}
Clearly, it suffices to efficiently compute $[Q(X)]_2$ where $[Q(X)]_2=\gtwo{\frac{T(X)-T_I(X)}{Z_I(X)}}$. We have the information of $[\Tbasepoly{X}-\Tbasepoly{\xi^i}/(X-\xi^i)]_2$. For this, we have the following lemma:
\end{comment}
The following Theorem determines the efficiency of the online phase of our prover.
\begin{theorem}\label{thm:approx-setup}
Let $N,\xi$ be as defined previously. Suppose we are given
$\kzg$ proofs $\{W_i\}_{i=1}^N$ with $W_i=\gtwo{\Tbasepoly{X} - \Tbasepoly{\xi^i}/(X-\xi^i)}$, where
$\Tbasepoly{X}=\enc{T_{\mathsf{b}}}{\setN}$ encodes a vector $\vecTbase\in \F^N$.
Let $I \subset [N]$, $\setN_I=\{\xi^i:i\in I\}$, $Z_I(X)$ denote the vanishing polynomial of $\setN_I$ and
$T_I(X)$ be the restriction of polynomial $T(X)$ on $\setN_I$.
Then, there exists an algorithm to compute $\kzg$ multi-opening proof
$\gany{Q(X)}=\gany{(T(X) - T_I(X)/Z_I(X)}$ for encoding $T(X)=\enc{T}{\setN}$ of vector $\vecT\in \F^N$ using $O((\delta + |I|) \log^2 (\delta + |I|))$ $\F$-operations
and $O(\delta + |I|)$ $\mathbb{G}$-operations. Here, $\delta$ denotes the hamming distance
between vectors $\vecTbase$ and $\vecT$.
\end{theorem}
\begin{proof}
    Let $\vecT=\vecTbase+\vecTcache$ and thus $T(X)=\Tbasepoly{X}+\Tcachepoly{X}$.
    Define $K=I\cup \{j\in [N]: \vecTcache[\,j\,]\neq 0\}$ as a set which captures the indices where the current table $\vecT$ differs from the base $\vecTbase$,
    where we explicitly also include the lookup indices $I$ in $K$. For $j\in K$, let $\vecTcache[j]=\Delta t_j$. Then $\Tcachepoly{X}=\sum_{j\in K}\Delta t_j\mu_j(X)$.
    %By definition of $K$, $|K|\leq \delta +|I|$. So, we need to bound $\Gtwo$ operations by $O(|K|)$ and field operations by $O(|K| \log^2|K|)$\\
    %First of all note that:
    We write the quotient $Q(X)$ as:
        {\small
    \begin{equation*}
    \begin{aligned}
    Q(X) = \sum_{i\in \setind}\frac{1}{z_I'(\xi^i)}\left(\frac{\Tbasepoly{X} - \Tbasepoly{\xi^i}}{X-\xi^i}\right)
     + \sum_{i\in \setind}\frac{1}{z_I'(\xi^i)}\left(\frac{\Tcachepoly{X} - \Tcachepoly{\xi^i}}{X-\xi^i}\right)
    \end{aligned}
    %\label{eq:Q2-upd}
    \end{equation*}
    }

    From above, we have $\gany{Q(x)}=\gany{\Qbasepoly{x}}+\gany{\Qcachepoly{x}}$ where
    \begin{gather*}
        \Qbasepoly{X}=\sum_{i\in \setind}(Z_I'(\xi^i))^{-1} (\Tbasepoly{X}-\Tbasepoly{\xi^i})/(X-\xi^i) \\
        \Qcachepoly{X}=\sum_{i\in \setind}(Z_I'(\xi^i))^{-1} (\Tcachepoly{X}-\Tcachepoly{\xi^i})/(X-\xi^i)
    \end{gather*}
    We can compute
    $\elttwo{\Qbasepoly{X}}$ from the pre-computed KZG openings of $\Tbasepoly{X}$ at points $\xi^i,i\in I$ using $O(|I|)$ group operations and
    $O(|I|\log^2 |I|)$ field operations. Therefore, it suffices to compute $\gany{\Qcachepoly{X}}$ efficiently.
    %\textbf{Thus, it suffices to describe the computation for $\elttwo{\Qcachepoly{X}}$. }\\
    Using $\Tcachepoly{X}=\sum_{j\in K}\Delta t_j\mu_j(X)$ and setting $c_i=(1/z_I'(\xi^i))$ for $i\in I$,
    we write $\Qcachepoly{X}$ as linear combination of table-independent polynomials:
    \begin{align*}
        \Qcachepoly{X} &= \sum_{i\in \setind} c_i\sum_{j\in K} \Delta t_j\frac{\mu_j(X)-\mu_j(\xi^i)}{X-\xi^i} \\
        &= \sum_{i\in \setind} c_i\Delta t_i\frac{\mu_i(X) - 1}{X-\xi^i} + \sum_{i\in \setind}\sum_{j\in K\setminus\{i\}}c_i\Delta t_j\frac{\mu_j(X)}{X-\xi^i}
    \end{align*}
    Now, we can write $\gany{\Qcachepoly{X}}=\elany{\Qcachepolyone{X}} + \elany{\Qcachepolytwo{X}}$ where:
        {\small
    \begin{gather*}
        \Qcachepolyone{X}=\sum_{i\in \setind}c_i\Delta t_i\frac{\mu_i(X)-1}{X-\xi^i},\,
        \Qcachepolytwo{X}=\sum_{i\in \setind}\sum_{j\in K\setminus \{i\}} c_i\Delta t_j\frac{\mu_j(X)}{X-\xi^i}
    \end{gather*}
    }
    The term $\elttwo{\Qcachepolyone{X}}$ can be computed using $O(|I|)$ group operations by augmenting the setup with pre-computed
    $\kzg$ opening proofs of polynomials $\mu_i(X)$ at $\xi^i$ for $i\in [N]$. This adds $O(N)$ to the setup parameters, while the computation
    can be done in $O(N\log N)$ time with methods similar to existing pre-computed parameters. This eventually leaves us with $\elany{\Qcachepolytwo{X}}$.
    %That is by precomputing $[\frac{\mu_i(X)-1}{X-\xi^i}]_2$. This requires just $N$ more precomputations and can be done along with the other precomputations which are done in the lookup protocol\\
    Next, we synthesize the polynomial $\Qcachepolytwo{X}$ in a form that reduces group operations required to compute its encoding.
    %\textbf{Thus, it suffices to describe the computation for $\elttwo{\Qcachepolytwo{X}}$}:
    \begin{align}\label{eq:Qcachepoly2}
    &\Qcachepolytwo{X} = \sum_{i\in \setind}c_i\sum_{j\in K\setminus \{i\}} \Delta t_j\mu_j(X)/(X-\xi^i) \nonumber \\
    &\quad = \sum_{i\in\setind}c_i\sum_{j\in K\setminus \{i\}}\frac{\Delta t_j}{Z_{\nroots}'(\xi^j)} \frac{Z_{\nroots}(X)}{(X-\xi^i)(X-\xi^j)} \nonumber \\
    %\intertext{Above, we expanded $\mu_j(X)$. Now using $Z_\nroots'(\xi^j)=N\xi^{-j}$ and using partial fractions}
    &\quad = N^{-1}\sum_{i\in\setind}c_i\sum_{j\in K\setminus \{i\}}\frac{\xi^j\Delta t_j}{\xi^i-\xi^j}
    \left(\frac{Z_\nroots(X)}{X-\xi^i} - \frac{Z_\nroots(X)}{X-\xi^j}\right) \nonumber \\
    &\quad = N^{-1}\sum_{i\in\setind}\left(c_i\cdot \sum_{j\in K\setminus \{i\}} \frac{\xi^j\Delta t_j}{\xi^i-\xi^j}\right)\frac{Z_\nroots(X)}{X-\xi^i} \nonumber \\
    &\qquad + \sum_{j\in K}\left(\xi^j\Delta t_j\cdot \sum_{i\in \setind\setminus \{j\}}\frac{c_i}{\xi^j-\xi^i}\right)\frac{Z_{\nroots}(X)}{X-\xi^j}
    \end{align}
    In the first step, we substituted $\mu_j(X)$, while in the final step we re-arranged the summation to accumulate the scalar factor for
    each distinct polynomial of the form $\vpolyN(X)/(X-\xi^i)$. Define scalars $a_i$, $i\in I$ and $b_j$, $j\in K$ as below:
    %this last equality, the first term is just the first term of the distributive property in finite fields.\\
    %The second term is just the second term of the distributive property in finite fields except that the order of the sums is reversed. This follows from the following fact \\
    %\begin{fact}
    %    $\sum_{i \in I} \sum_{j \in K \setminus \{i\}} f(i,j)=\sum_{j \in K} \sum_{i \in I \setminus \{j\}} f(i,j) $
    %\end{fact}
    %In the above equation \eqref{eq:Qcachepoly2}, let us define:
    \begin{gather}\label{eq:scalars}
        a_i = \sum_{j\in K\setminus \{i\}}\frac{\xi^j\Delta t_j}{\xi^i-\xi^j}, i\in \setind\quad
        b_j=  \sum_{i\in \setind\setminus \{j\}}\frac{c_i}{\xi^j - \xi^i}, j\in K
        %W_3^i(X) = \frac{Z_\nroots(X)}{X-\xi^i}, \text{ for } i\in [N]
    \end{gather}
    Now, recalling that
    $W_3^j=\gany{\vpolyN(X)/(X-\xi^i)}$, we see that $\elany{\Qcachepolytwo{X}}$ can be written as linear combination of $O(|K|+|I|)$ group elements.
    \begin{equation}\label{eq:Qcachepoly2commit}
    \elttwo{\Qcachepolytwo{X}} = N^{-1}\left(\sum_{i\in\setind}(c_ia_i)\cdot W_3^i + \sum_{j\in K}(\xi^j \Delta t_j b_j)\cdot W_3^j\right)
    \end{equation}
    Now $c_i=(z_I(\xi^i))^{-1}$ for all $i \in \setind$ can be determined in $O(|I|log^2 |I|)$ field operations by evaluating $Z_{\setind}'(X)$ on
    $H_{\setind}$. So, given $\{a_i\}_{i\in I}, \{b_j\}_{j\in K}$, $\elttwo{\Qcachepolytwo{x}}$ can be computed in $O(|\setind|+|K|)$ group operations.
    While we have diligently reduced the group operations, we still seem to need $O(|I||K|)=O(m\delta)$ field operations. We clearly need better than
    naive way of computing the scalars in \eqref{eq:scalars} to obtain additive overhead in $\delta$. This is what we consider next.
    Routine calculation shows that we can write $a_i$ as:
    \begin{equation*}
        a_i = -\Delta T + \Delta t_i + \xi^i\sum_{j\in K\setminus\{i\}}\frac{\Delta t_j}{\xi^i-\xi^j}, i\in I
    \end{equation*}
    where in the above, we have $\Delta T=\sum_{j\in K}\Delta t_j$.
    %a_i = -\sum_{j\in K\setminus \{i\}}\Delta t_j + \xi^i\sum_{j\in K\setminus \{i\}}\frac{\Delta t_j}{\xi^i-\xi^j} $$
    %This is because:
    %$$a_i+\sum_{j\in K\setminus \{i\}}\Delta t_j= \sum_{j \in K \setminus \{i\}}\frac{\xi^j\Delta t_j}{\xi^i-\xi^j}+\Delta t_j$$
    %$$=\sum_{j \in K \setminus \{i\}}\frac{\xi^i\Delta t_j}{\xi^i-\xi^j} = \xi^i\sum_{j\in K\setminus \{i\}}\frac{\Delta t_j}{\xi^i-\xi^j}$$
    %Now, define $\Delta T=\sum_{j\in K}\Delta t_j$\\

    %Here computing $\Delta T$ is a one time computation (per batch). It can be computed from the knowledge of $T_{\text{ch}}$ in the online phase.
    %We have:
    %$$  $$
    %Suppose we get $\sum_{j\in K\setminus\{i\}}\frac{\Delta t_j}{\xi^i-\xi_j}$ for all $i \in I$ efficiently. Then $a_i$ for all $i \in I$ can be obtained in $O(|I|)$ field operations. \\
    %\textbf{Thus, to get $a_i$ for all $i \in I$ it suffices to describe the computation of $e_i=\sum_{j\in K\setminus\{i\}}\frac{\Delta t_j}{\xi^i-\xi^j}$ for all $i \in I$}\\\\
    Above implies that to compute $a_i, i\in I$ efficiently, it is sufficient to efficiently
    compute $e_i=\sum_{j\in K\setminus\{i\}}\frac{\Delta t_j}{\xi^i-\xi^j}$ for all $i \in I$. Our next lemma claims that
    such {\em reciprocal sums} can be computed efficiently. The computation of $b_j,j\in K$ can also be reduced a similar computation.
    We defer this reduction and the full proof of Lemma ~\ref{lem:sum-computation} to the Appendix, but illustrate the key ideas in the proof.
    %Our requirement is now to bound the number of field operations for $e_i$ and for $b_j$. For this, we invoke the following lemma with the proof in the appendix.
    \begin{lemma}\label{lem:sum-computation}
    Let $I\subset K\subset [N]$ and let $e_i$ for all $i \in \setind$ and $b_j$ for all $j \in K$ be as described above.
    Then, $e_i$ for all $i \in I$ and $b_j$ for all $j \in K$ can be computed in $O(|K|\log^2|K|)\, \mathbb{F}$ operations
    \end{lemma}
    \begin{proof}[Proof-Sketch]
        First, we mention that the special case of the lemma when $\Delta t_j=1$ for all $j\in K$ admits an efficient computation due to the following identity
        proved in Lemma ~\ref{lem:sumtoder}.
        \begin{equation*}
            \frac{Z_K''(\xi^i)}{Z_K'(\xi^i)} = 2\sum_{j\in K\setminus \{i\}}\frac{1}{\xi^i-\xi^j}
        \end{equation*}
        for $Z_K(X)=\prod_{i\in K}(X-\xi^i)$. The polynomial $Z_K$ can be computed in $O(|K|\log^2|K|)$ and subsequent evaluations of its first two
        derivatives can also be evaluated on the set $\{\xi^i: i\in I\}$ with the same complexity. However, to deal with arbitrary values of $\Delta t_j$ we
        need more ingenuity. We will {\em imagine} $\Delta t_j$ to be $p(\xi^j)$ for some polynomial $p(X)$. Moreover, we demand that $p(\xi^j)=0$ for $j\not\in K$.
        We will not compute such a polynomial $p$, as it has degree $O(N)$, but view it as an ``oracle'' which we can hopefully query at the points we need.
        Then it can be seen that $e_i=g_i(\xi^i) - r_i(\xi^i)$ for rational functions $g_i(X)$ and $r_i(X)$ defined by:
        \begin{align}\label{eq:rat-fun-f}
        g_i(X) &=\sum_{j\in [N]\setminus i}\frac{p(X)}{X-\xi^j} \nonumber \\
        r_i(X) &=\sum_{j\in [N]\setminus i} \frac{p(X)-p(\xi^j)}{X-\xi^j}
        \end{align}
        Now, $g_i(\xi^i)$ for $i\in I$ turns out to be (using the special case above):
        $$p(\xi^i)\sum_{j\in K\setminus \{i\}} 1/(\xi^i-\xi^j)=\Delta t_i (Z_K''(\xi^i)/Z_K'(\xi^i))/2$$

    \end{proof}

    From the lemma \ref{lem:sum computation}, we have shown that the field operations needed to get $e_i$ and $b_j$ and thus $\elttwo{\Qcachepolytwo{X}}$ is $O(|K|\log^2|K|)$. \\

    This completes the proof of Lemma \ref{lem:approx-setup}.
\end{proof}

\subsection{Amortized Analysis of the update protocol}
Recall that we were able to get $[Q(X)]_2$ in $O(|K|)$ group operations and $O(|K|\log^2|K|)$ field operations. \\
For concrete analysis, let $s$ be the period after which the rebasing takes place. Also, the lookup happens at maximum of $m$ indices during a single batch. Thus, $|I|\leq m$.\\
This gives an upper bound on $\delta$, that is $ms$ and an upper bound on $K$, that is $ms+m=m(s+1)$.\\

Clearly $O(K)=O(ms)$ and $O(|K|\log^2|K|)=O(ms \log^2(ms))$ so group operations are $O(ms)$ and field operations are $O(ms\log^2(ms))$ \\

Moreover, after every $s$ batches, the rebasing(offline phase) is done which we know takes $O(N\log N)$ group and field operations.\\

So, the amortized number of operations for the offline and online phase in total is:
$O(ms \log^2(ms)+\frac{N\log N}{s})$ $\mathbb{F}$ operations and $O(ms +\frac{N\log N}{s})$ $\Gtwo$ operations\\

The value of $s$ which minimizes the group operations is $\sqrt{\frac{N}{m}}$. For this value of $s$:\\
\textbf{The amortized group operations needed are $\tilde{O}(\sqrt{mN})$}\\
\textbf{The amortized field operations needed are also $\tilde{O}(\sqrt{mN})$}\\
Here $\tilde{O}$ denotes that the polylog factors have been neglected\\\\














