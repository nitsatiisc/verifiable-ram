\appendix
\section{Proof of Lemma 6.2}
Recall the statement of lemma \ref{lem:sum computation}. We want to compute $e_i$ for all $i \in \setind$ and $b_j$ for all $j \in K$ in $O(|K|\log^2|K|)$ $\mathbb{F}$ operations.\\\\

Before starting the proof, we list some preliminaries and some lemmas/facts which will be useful in the proof.\\

\subsection{Computational Algebra Preliminaries}
Let $\F$ be a finite field of prime order $p$ and $\G$ be a cyclic additive group of order $p$ with generator $g$. For $s \in \F$, we use
the notation $\elt{s}$ to denote the group element $s\cdot g$. Assume that $\F$ contains $n^{th}$ root of unity $\xi$
satisfying $\xi^n=1$ for large $n$ (All polynomial degrees are assumed less than $n$).

\begin{fact}[\textsf{Fast Evaluation}]\label{fc:fft}
Let $f\in \F[X]$ be a polynomial of degree $<d$ and $(\xi_1,\ldots,\xi_r)\in \F^r$ be distinct points in $\F$.
Then the vector $(f(\xi_1),\ldots,f(\xi_r))$ can be computed in $O((d+r)\log (d+r))$ $\F$ operations if $\xi_1,\ldots,\xi_r$ form roots
of unity, and in $O((d+r)\log^2(d+r))$ $\F$ operations otherwise.
\end{fact}

\begin{fact}[\textsf{Fast Interpolation}]\label{fc:ifft}
Let $\xi_1,\ldots,\xi_d$ be distinct points in $\F$ and $(v_1,\ldots,v_d)\in \F^d$. Then $(f_0,\ldots,f_{d-1})\in \F^d$
can be computed in $O(d\log^2 d)$ operations in $\F$ such that $f(\xi_i)=v_i$ for all $i\in [d]$ where
$f(X)=\sum_{i=0}^{d-1}f_iX^i$.
\end{fact}

\begin{fact}[\textsf{Fast Multiplication}]\label{fc:mult}
Let $\xi_1,\ldots,\xi_r$ be distinct points in $\F$. Then coefficients of $f(X)=\prod_{i=1}^r (X-\xi_i)$
can be computed in $O(r\log^2 r)$ operations in $\F$.
\end{fact}

\begin{fact}[\textsf{Multi KZG proofs}]\label{fc:multkzg}
Let $\{\elt{x^i}\}_{i=1}^d$ be given for some $x\in \F$. Then for set of $r$ distinct points $\xi_1,\ldots,\xi_r$,
and a polynomial $f(X)\in \F[X]$ of degree $<d$,the vector $(\elt{h_1(x)},\ldots,\elt{h_r(x)})$,
where $h_i(X) = (f(X) - f(\xi_i))/(X - \xi_i)$ can be computed in
$O((r+d)\log(r+d))$ group and field operations when $\xi_1,\ldots,\xi_r$ are roots of unity, and in
$O(rlog^2 r + d\log d)$ group and field operations otherwise.
\end{fact}

\begin{fact}[\textsf{Lagrange Polynomials}]\label{fc:lagrange}
Let $\mathbb{S}=\{\xi_1,\ldots,\xi_r\}$ be a set of $r$ distinct points and let $\tau_1(X),\ldots,\tau_r(X)$ be
the corresponding lagrange polynomials of degree $r-1$ each. Let $Z_{\mathbb{S}}(X)=\prod_{i=1}^r (X-\xi_r)$ denote the vanishing polynomial
for $\mathbb{S}$. Then we have:
\begin{gather*}
    \sum_{i=1}^r \tau_i(X) = 1 \\
    \tau_i(X) = \frac{Z_{\mathbb{S}}(X)}{Z_{\mathbb{S}}'(\xi_i)(X-\xi_i)} \text{ for all } i\in [r]
\end{gather*}
\end{fact}

\subsection{Notion of derivative in finite field setting}

\textbf{Formal Derivative}- For a polynomial $p(X) \in \F[X]$, we define the formal derivative of $p(X)$ as the polynomial $u(X,X)$ where $u(X,Y)=\frac{p(X)-p(Y)}{X-Y}$\\
Let us denote the formal derivative of $p(X)$ as $p^f(X)$.\\
Then, $p^f(X)= u(X,X)$ where $u(X,Y)=\frac{p(X)-p(Y)}{X-Y}$\\\\
It is well known that the formal derivative of a polynomial satisfies all the properties that we want the derivative of a polynomial to satisfy, like linearity, product rule, quotient rule etc.\\\\
As a result, the formal derivative $p^f(X)$ defined as above can be computed from $p(X)$ in the \textit{usual} way, that is $\frac{d}{dx}X^k=k X^{k-1}$ and extend using linearity of derivative.\\
Henceforth, we will represent the formal derivative of $p(X)$ as $p'(X)$ and because of the above statements, we will compute it in the usual way and freely use all the properties of derivatives on it.\\

\subsection{Some useful lemmas:}

We need the following lemma:\\
\begin{lemma}\label{lem:sumtoder}
For $K\subset [N]$, define $H_K$ to be $\{\xi^i:i \in K\}$. Let $p(X)$ be the vanishing polynomial of $H_{K}$. Let $p'(X)$ and $p''(X)$ denote the formal first derivative and second derivative of $p(X)$ respectively.\\
Then, $p''(\xi^i)/p'(\xi^i)=2 \cdot \sum_{j\in K\setminus \{i\}} 1/(\xi^i-\xi^j)$ for all $i \in K$
\end{lemma}

\begin{proof}

    We have for any $r \in K$:\\
    $p'(\xi^r)=\prod_{j \in K \setminus \{r\}}(\xi^r-\xi^j)$\\
    To compute $p''(\xi^r)$, note that $p''(X)$ is a (double) sum of many products. When one substitutes $\xi^r$ in a product, the product is non zero if and only if $r=i\, \text{or}\, r=j$\\\\

    Putting $\xi^r$ gives:
    $$ p''(\xi^r)=\sum_{j\in K\setminus \{r\}}\prod_{k\in K\setminus \{r,j\}} (\xi^r-\xi^k)+$$
    $$ \sum_{i\in K\setminus \{r\}}\prod_{k\in K\setminus \{r,i\}} (\xi^r-\xi^k) $$
    But this is just,
    $$2\cdot \sum_{j\in K\setminus \{r\}}\prod_{k\in K\setminus \{r,j\}} (\xi^r-\xi^k) $$

    Now, we are left to show that $\frac{\frac{p''(\xi^r)}{2}}{p'(\xi^r})=\sum_{j\in K\setminus \{r\}} 1/(\xi^r-\xi^j)$, or
    $$\sum_{j\in K\setminus \{r\}}\prod_{k\in K\setminus \{r,j\}} (\xi^r-\xi^k)=\prod_{j \in K \setminus \{r\}}(\xi^r-\xi^j) \sum_{j\in K\setminus \{r\}} 1/(\xi^r-\xi^j)$$

    But we observe that:
    $$\sum_{j\in K\setminus \{r\}}\prod_{k\in K\setminus \{r,j\}} (\xi^r-\xi^k)=\sum_{j\in K\setminus \{r\}}\frac{\prod_{k\in K\setminus \{r\}} (\xi^r-\xi^k)}{(\xi^r-\xi^j)}$$

    But $\prod_{k\in K\setminus \{r\}} (\xi^r-\xi^k)$ is independent of $j$, so it can be pulled outside the sum:
    $$\sum_{j\in K\setminus \{r\}}\prod_{k\in K\setminus \{r,j\}} (\xi^r-\xi^k)=\prod_{k\in K\setminus \{r\}} (\xi^r-\xi^k) \sum_{j\in K\setminus \{r\}} 1/(\xi^r-\xi^j) $$
    But $k$ is just a dummy variable. We can replace $k$ by $j$ on the RHS:
    $$\sum_{j\in K\setminus \{r\}}\prod_{k\in K\setminus \{r,j\}} (\xi^r-\xi^k)=\prod_{j\in K\setminus \{r\}} (\xi^r-\xi^j) \sum_{j\in K\setminus \{r\}} 1/(\xi^r-\xi^j) $$
    This completes the proof.
\end{proof}

\textbf{An alternative proof of \ref{lem:sumtoder}}:\\
A much shorter proof is possible by inducting on the cardinality of $K\subset N$.\\
The base case where $K$ has one element, say $i$ is trivial. This is because in that case $p$ is a linear polynomial $X-\xi^i$ and LHS=$\frac{p''}{p'}=\frac{0}{1}=0$ and on RHS we are summing over the empty set so the sum is $0$.\\\\
Now let $|K|=k$ and the elements of $K$ be $a_1, a_2, \cdots, a_k$.
Consequently $p(X)=\prod_{i \in K}(X-\xi^i)$\\
We assume that the result holds for this case. Thus,
$$p''(\xi^i)/p'(\xi^i)=2 \cdot \sum_{j\in K\setminus \{i\}} 1/(\xi^i-\xi^j)$$ for all $i \in K$.\\
Suppose we introduce another element in the set $K$, call it $\alpha$
The vanishing polynomial for the new $K$(call it $K')$ is $q(X)=p(X)(X-\xi^{\alpha})$\\
Then by the product rule for derivatives, we have:
$$q'(X)=p'(X)(X-\xi^{\alpha})+p(X)$$ and
$$q''(X)=2p'(X)+p''(X)(X-\xi^{\alpha})$$

It suffices to show that $$\frac{q''(\xi^i)}{q'(\xi^i)}=2\sum_{j\in K'\setminus \{i\}}\frac{1}{\xi^i-\xi^j}$$
for all $i \in K'$\\
For this we make 2 cases:\\
\textbf{Case 1-}$i \in K$
Then
$$q'(\xi^i)=p'(\xi^i)(\xi^i-\xi^{\alpha})$$ and
$$q''(\xi^i)=2p'(\xi^i)+p''(\xi^i)(\xi^i-\xi^{\alpha})$$
This used that $p(\xi^i)=0$\\
Thus, $$\frac{q''(\xi^i)}{q'(\xi^i)}=\frac{2p'(\xi^i)+p''(\xi^i)(\xi^i-\xi^{\alpha})}{p'(\xi^i)(\xi^i-\xi^{\alpha})}$$
Dividing numerator and denominator by $p'(\xi^i)(\xi^i-\xi^{\alpha})$ gives:
$$\frac{q''(\xi^i)}{q'(\xi^i)}=\frac{2}{\xi^i-\xi^\alpha}+\frac{p''(\xi^i)}{p'(\xi^i)}$$
By induction hypothesis this becomes:
$$\frac{q''(\xi^i)}{q'(\xi^i)}=\frac{2}{\xi^i-\xi^\alpha}+2 \cdot \sum_{j\in K\setminus \{i\}} 1/(\xi^i-\xi^j)$$
Note that $$\frac{1}{\xi^i-\xi^\alpha}+ \sum_{j\in K\setminus \{i\}} 1/(\xi^i-\xi^j)=\sum_{j\in K'\setminus \{i\}}\frac{1}{\xi^i-\xi^j}$$
Using that $K'=K \cup \{\alpha\}$\\
Thus, $$\frac{q''(\xi^i)}{q'(\xi^i)}=2\sum_{j\in K'\setminus \{i\}}\frac{1}{\xi^i-\xi^j}$$ which completes case 1. \\\\
\textbf{Case 2-}$i=\alpha$. Then
$$q'(\xi^{\alpha})=p(\xi^{\alpha})$$ and
$$q''(\xi^{\alpha})=2p'(\xi^{\alpha})$$
These we get by replacing $X$ by $\xi^{\alpha}$ in the expressions of $q', q''$
So, it is sufficient to prove that:
$$\frac{2p'(\xi^{\alpha})}{p(\xi^{\alpha})}=2\sum_{j\in K'\setminus \{\alpha\}}\frac{1}{\xi^{\alpha}-\xi^j}$$

But $$p(\xi^{\alpha})=\prod_{i \in K}(\xi^{\alpha}-\xi^i)=\prod_{i \in K'\setminus \{\alpha\}}(\xi^{\alpha}-\xi^i)$$ and $$p'(\xi^{\alpha})=\sum_{j \in K} \prod_{i \in K \setminus \{j\}}(\xi^{\alpha}-\xi^i)=\sum_{j \in K'\setminus \{\alpha\}} \prod_{i \in K' \setminus \{j, \alpha\}}(\xi^{\alpha}-\xi^i)$$
But now, note that $$\sum_{j \in K'\setminus \{\alpha\}} \prod_{i \in K' \setminus \{j, \alpha\}}(\xi^{\alpha}-\xi^i)=\prod_{i \in K' \setminus \{ \alpha\}}(\xi^{\alpha}-\xi^i)\sum_{j \in K'\setminus \{\alpha\}}\frac{1}{\xi^{\alpha}-\xi^j}$$
So, $$\frac{2p'(\xi^{\alpha})}{p(\xi^{\alpha})}=2\cdot \prod_{i \in K' \setminus \{ \alpha\}}(\xi^{\alpha}-\xi^i)\sum_{j \in K'\setminus \{\alpha\}}\frac{1}{\xi^{\alpha}-\xi^j}/\prod_{i \in K'\setminus \{\alpha\}}(\xi^{\alpha}-\xi^i)=2\cdot\sum_{j \in K'\setminus \{\alpha\}}\frac{1}{\xi^{\alpha}-\xi^j}$$

This completes the proof of case 2 and thus the proof of the lemma.
\begin{lemma}[Sumcheck]\label{lem:sumcheck}
Let $u(X,Y)$ be a bi-variate polynomial over a finite field $\F$ and $\nroots$ be defined as the group of $N^{\text{th}}$ roots of unity $(N<<|\F|)$ with generator $\xi \in \F$ \\
Then $\sum_{i\in [N]}u(X,\xi^i)= Nu(X,0)$
\end{lemma}

\begin{proof}
    First of all, by a basic result in algebra, $\F[X,Y]=\F[X][Y]$\\
    $u(X,Y) \in \F[X,Y]$ thus $u(X,Y) \in \F[X][Y]$\\
    Let it have degree $d<N$ ($u(X,Y)$ itself had degree less than $N$)
    Thus, $u(X,Y)=a_0+a_1 Y+ a_2 Y^2+\cdots+ a_d Y^d$\\
    Here each $a_i$ is a polynomial function in $X$\\
    Now,  $\sum_{i\in [N]}u(X,\xi^i)$
    $$N a_0+a_1(\xi^+\xi^2+\cdots+\xi^N)+a_2(\xi^2+\xi^4+\cdots+\xi^{2N})+ \cdots + a_d(\xi^d+\xi^{2d}+\cdots +\xi^{Nd})$$
    But for any $\alpha=\xi^k$ for $k<N$, $\alpha+\alpha^2+\cdots \alpha^N=0$ (basic property of complex numbers)\\
    Thus, all terms vanish except the first term\\
    So, $\sum_{i\in [N]}u(X,\xi^i)=N a_0$\\
    But $a_0=u(X, 0)$\\
    Thus, $\sum_{i\in [N]}u(X,\xi^i)=N u(X, 0)$
\end{proof}

\begin{lemma}\label{lem:zk-hat}
Let polynomials $Z_\nroots(X)$ be the vanishing polynomial for $\nroots$, let $\widehat{Z}_K(X)$ and $Z_K(X)$ be the vanishing polynomials for $H_{[N]\setminus K}$ and $H_K$ respectively. Let $\mu_1(X),\ldots,\mu_N(X)$
be Lagrange polynomials for the set $\nroots=\{\xi,\ldots,\xi^N\}$. Then:
\begin{gather}
    \widehat{Z}_K(X) = \sum_{j\in K}\frac{Z_\nroots'(\xi^j)}{Z_K'(\xi^j)}\mu_j(X), \\
    \widehat{Z}_K'(X)= \sum_{j\in K}\frac{Z_\nroots'(\xi^j)}{Z_K'(\xi^j)}\mu_j'(X)
\end{gather}
\end{lemma}
The following fact will be useful in proving the lemma.
\begin{fact}[\textsf{Standard result from Algebra}]
    If $2$ polynomials $f,g$ of degree $<N$ agree on $N$ points, then they are equal as polynomials, that is, $f(X)=g(X)$
\end{fact}

\begin{proof}
    The second equation follows from the first by simple differentiation (linearity of derivative)\\
    It suffices to prove the first equation\\
    Both LHS and RHS are of degree $<N$. We will show that LHS and RHS agree on $N$ points. That will finish proof of (11) by the fact\\
    Consider evaluating LHS and RHS at $\xi^i$ for $i \in [N]\setminus K$ \\
    LHS is 0 by definition of $\hat{Z}_K(X)$. RHS is also 0 by properties of Lagrange polynomials.\\
    Consider evaluating LHS and RHS at $\xi^i$ for $i \in K$\\
    RHS is $\frac{Z_\nroots'(\xi^i)}{Z_K'(\xi^i)}$ by properties of Lagrange polynomials.\\
    LHS is $\prod_{j \in [N]\setminus K} (\xi^i-\xi^j)$\\
    Multiplying dividing by $\prod_{ j \in K \setminus \{i\}}(\xi^i-\xi^j)$ gives:
    $$\frac{\prod_{j \in [N] \setminus \{i\}} (\xi^i-\xi^j)}{\prod_{j \in K \setminus \{i\}}(\xi^i-\xi^j)}$$
    Which is $\frac{Z_\nroots'(\xi^i)}{Z_K'(\xi^i)}$ =RHS\\
    Thus, LHS and RHS of (11) agree on $\nroots$. Thus, they agree on $N$ points
\end{proof}

\begin{lemma}\label{lem:lamda-deriv}
Let $\mu_1,\ldots,\mu_N$ be the lagrange polynomials for the set $\nroots=\{\xi^i:i\in [N]\}$
of the $N^{th}$ roots of unity. Then we have:
\begin{equation*}
    \mu_i'(\xi^j) = \begin{cases}
                        \frac{(N-1)}{2\xi^{i}}  \text{ if } j=i\\
                        \frac{\xi^i}{\xi^j(\xi^j-\xi^i)} \text{ otherwise }
    \end{cases}
\end{equation*}
\end{lemma}

\begin{proof}
    Suppose first that $i \neq j$:\\
    We know that $\mu_i(X)= \frac{Z_H(X)}{Z_H'(\xi^i)(X-\xi^i)}$\\
    Thus, by applying quotient rule(we can apply quotient rule because $\mu_i$ is defined at $\xi^j$ as $j\neq i$):
    $$\mu_i'(X) \cdot Z_H'(\xi^i)= \frac{(X-\xi^i)(N\cdot X^{N-1})-(X^N-1)}{(X-\xi^i)^2}$$
    Substituting $X$ by $\xi^j$, we get:
    $$\mu_i'(\xi^j) \cdot \frac{N}{\xi^i}= \frac{N(\xi^j-\xi^i)}{\xi^j (\xi^j-\xi^i)^2}$$
    Thus, we get:
    $$\mu_i'(\xi^j)=\frac{\xi^i}{\xi^j(\xi^j-\xi^i)}$$\\
    Now consider $i=j$:\\
    Then, $\mu_i(X)=\frac{\prod_{j \in [N] \setminus \{i\}}(X-\xi^j)}{Z_H'(\xi^i)}$\\
    Thus,  $\mu_i(X)\cdot Z_H'(\xi^i)=\prod_{j \in [N] \setminus \{i\}}(X-\xi^j)$\\
    Thus, $\mu_i'(X) \cdot \frac{N}{\xi^i}=\sum_{j \in [N] \setminus \{i\}}\prod_{k \in [N]\setminus\{i,j\}}(X-\xi^k)$\\
    Now, substituting $X$ by $\xi^i$ gives:
    $$\mu_i'(\xi^i) \cdot \frac{N}{\xi^i}=\sum_{j \in [N] \setminus \{i\}}\prod_{k \in [N]\setminus\{i,j\}}(\xi^i-\xi^k)$$
    or
    $$\mu_i'(\xi^i) \cdot \frac{N}{\xi^i}=\sum_{j \in [N] \setminus \{i\}}\frac{\prod_{k \in [N]\setminus\{i\}}(\xi^i-\xi^k)}{\xi^i-\xi^j}$$
    or$$\mu_i'(\xi^i) \cdot \frac{N}{\xi^i}= \prod_{k \in [N]\setminus\{i\}}(\xi^i-\xi^k)\sum_{j\in [N]\setminus \{i\}}\frac{1}{\xi^i-\xi^j}$$
    or $$\mu_i'(\xi^i) \cdot \frac{N}{\xi^i}=Z_H'(\xi^i)\sum_{j\in [N]\setminus \{i\}}\frac{1}{\xi^i-\xi^j}$$
    or $$\mu_i'(\xi^i)=\sum_{j\in [N]\setminus \{i\}}\frac{1}{\xi^i-\xi^j}$$
    or $$\mu_i'(\xi^i)=\frac{Z_H''(\xi^i)}{2 Z_H'(\xi^i)} \text{ From the lemma \ref{lem:sumtoder}}$$
    or $$ \mu_i'(\xi^i)=\frac{N-1}{2\xi^i}$$

\end{proof}

\begin{lemma}\label{lem:tau}
Let $K\subseteq \F$ be a set of cardinality $k$ and $\mathcal{X}=\{x_j:j\in K\}$ be a set
where $x_j$ for $j\in K$ are distinct elements of $\F$. Let $Z_\mathcal{X}(X)=z_k X^k+\cdots+z_0$ denote the vanishing polynomial of $\mathcal{X}$
and $\{\eta_j(X)\}_{j\in K}$ denote the lagrange polynomials such that $\eta_i(x_j)=\delta_{ij}$ for $i,j\in K$. Then for all $j\in K$,
we have $\eta_j'(x_j)=F_K(x_j)/Z_\mathcal{X}'(x_j)$ where the polynomial
$F_K(X)$ is defined as
\[F_K(X)=\binom{k}{2}z_k X^{k-2}+\cdots+\binom{2}{2}z_2=\sum_{j=2}^k z_j\binom{j}{2}X^{j-2} \]
\end{lemma}
\begin{proof}
    For $j\in K$ we have $\tau_j(X)=\frac{Z_\mathcal{X}(X)}{(X-x_j)Z_\mathcal{X}'(x_j)}= \frac{1}{Z_\mathcal{X}'(x_j)}\frac{Z_\mathcal{X}(X)}{X-x_j}$ by definition of Lagrange polynomials.\\
    By doing long division of $Z_\mathcal{X}(X)$
    by $X-x_j$ we have:
    \begin{align*}
        \tau_j(X) &= \frac{1}{Z_\mathcal{X}'(x_j)}\big(z_k X^{k-1} + (x_jz_k + z_{k-1})X^{k-2} + \cdots +
        (x_j^{k-1}z_k + \cdots + z_1)\big) \\
        &= \frac{1}{Z_\mathcal{X}'(x_j)}\sum_{p=0}^{k-1} \left(\sum_{q=p+1}^k z_q x_j^{q-p-1}\right)X^p
    \end{align*}

    Now, we use linearity property of derivative to differentiate:

    $$   \tau_j'(X) = \frac{1}{Z_\mathcal{X}'(x_j)}\sum_{p=0}^{k-1} \left(\sum_{q=p+1}^{k}z_q x_j^{q-p-1}\right)p X^{p-1} $$
    In this step we just differentiated $X^p$ to get $p X^{p-1}$
    $$ \tau_j'(X) = \frac{1}{Z_\mathcal{X}'(x_j)}\sum_{p=1}^{k-1} p  \sum_{q=p+1}^{k}z_q x_j^{q-p-1} X^{p-1} $$
    In this step, we just applied distributive property to the rightmost bracket and brought $p$ outside the sum over $q$\\
    Now we put $X=x_j$:

    $$\tau_j'(x_j) = \frac{1}{Z_\mathcal{X}'(x_j)} \sum_{p=1}^{k-1} p \sum_{q=p+1}^k z_q x_j^{q-2} $$
    $$= \frac{1}{Z_\mathcal{X}'(x_j)}\sum_{q=2}^k z_q x_j^{q-2} \sum_{p=1}^{q-1} p $$
    Here, we reversed the order of the sum. $p$ going from 1 to $k-1$ and $q$ going from $p+1$ to $k$ is same as $q$ going from $2$ to $k$ and $p$ going from 1 to $q-1$
    $$\tau_j'(x_j)= \frac{1}{Z_\mathcal{X}'(x_j)}\sum_{q=2}^k z_q \binom{q}{2} x_j^{q-2} $$
    Here we used that $\sum_{p=1}^{q-1}p=\frac{q(q-1)}{2}=\binom{q}{2}$

    But $q$ is just a dummy variable. We might as well replace $q$ by $j$

    Thus, $$\tau_j'(x_j)= \frac{1}{Z_\mathcal{X}'(x_j)}\sum_{j=2}^k z_j \binom{j}{2} x_j^{j-2} $$
    If we compare this to the definition of $F_K(X)$ given in the lemma statement, we get:
    $$\tau_j'(x_j)= \frac{F_K(x_j)}{Z_\mathcal{X}'(x_j)}$$


\end{proof}

\subsection{Brief Proof Summary of lemma 6.2}
What we want to compute is a sum over the set $K$ for all $i \in I$ or a sum over the set $I$ for all $j \in K$. Let us give a sketch for the first case.

We interpolate a polynomial $p$ in such a way that its evaluations at $\nroots$ correspond to the numerators of the required sum. We take the full $\nroots$ and not subsets of it so that we can apply sumcheck easily later. \\
Next we introduce some rational functions. These will be very useful to reduce the sum we need to compute to an evaluation of the rational functions at some points\\


These evaluations can be reduced to evaluations of polynomials using the simple form of $Z_{\nroots}$. \\

By introducing some more polynomials and using formulas for sumcheck, finally, the entire task can be reduced to calculating $p(0)$ and $p'(X)$ at certain points. \\

The polynomial $p$ turns out to be a product of polynomials: $p(X)=\widehat{Z}_K(X)\cdot q(X)$ and so the next step is to get the evaluations of $\widehat{Z}_K(X)$ and $q(X)$ at those points.
For this, some further lemmas are used to get evaluations of $q(X)$ at degree $q$ many points and thus $q(X)$ is obtained by interpolation. \\
Next $q(0)$ can be computed and thus $p(0)$ is also computed.\\
$Z_K(X)$ and $q'(X)$ can also be computed easily(details in the proof)\\
This leaves computation of $\widehat{Z}_K(X)$ at those points as the main task. For this another lemma is needed related to derivatives of Lagrange polynomials. \\
Eventually the task reduces to needing to compute something which can be computed efficiently using another lemma.\\
Moreover, every single step involves a reduction which can be done in the number of operations allowed in the statement of the lemma. So, we get what we want.\\
For the second case, the proof is very similar, except for some interchanging of $I$ and $K$, and interchanging of $i$ and $j$. \\
This completes a brief summary of the proof.

\subsection{Proof of lemma 6.2}

\begin{proof}

    We give the proof for $e_i$ for all $i \in \setind$ in full detail and briefly mention the modifications needed for $b_j$ for all $j \in K$. \\

    \begin{equation}\label{eq:ei}
    e_i = \sum_{j\in K\setminus \{i\}} \frac{\Delta t_j}{\xi^i - \xi^j}
    \end{equation}
    We need this for all $i \in \setind$. Recall that $\setind \subset K$ in this case.\\
    To compute $e_i$, we will first define a polynomial $p(X)$ of degree at most $N-1$ such that
    $p(\xi^j)=\Delta t_j$ for $j\in K$ and $p(\xi^j)=0$ for $j \in [N]\setminus K$.\\
    Now, because $p(\xi^j)=0$ for $j \in [N]\setminus K$, it is clear that the vanishing polynomial of $H_{[N]\setminus K}$ divides $p(X)$.\\\\

    Thus, there exists a polynomial $q(X)$ of degree at most $|K|-1$ such that:
    \begin{equation}\label{eq:px}
    p(X) = \hat{Z}_K(X)\cdot q(X)
    \end{equation}
    where $\hat{Z}_K(X)=\prod_{i\in [N]\setminus K}(X-\xi^i)$ is the vanishing polynomial of $H_{[N]\setminus K}$\\

    Now introduce the rational functions:
    \begin{gather}\label{eq:fgr}
    f_i(X) = \sum_{j\in [N]\setminus \{i\}}\frac{p(\xi^j)}{X-\xi^j}, i\in \setind \\
    g_i(X) = \sum_{j\in [N]\setminus \{i\}}\frac{p(X)}{X-\xi^j}, i\in \setind \\
    r_i(X) = \sum_{j\in [N]\setminus \{i\}}\frac{p(X) - p(\xi^j)}{X-\xi^j}, i\in \setind
    \end{gather}
    Note that $f_i(\xi^i)=\sum_{j\in [N]\setminus \{i\}}\frac{p(\xi^j)}{\xi^i-\xi^j}$\\\\
    But, by the way $p(X)$ is defined, we have that $f_i(\xi^i)=\sum_{j\in K\setminus \{i\}}\frac{\Delta t_j}{\xi^i-\xi^j}=e_i$\\
    \textbf{Thus, we need $f_i(\xi^i)$ for all $i \in I$.}

    Note that $f_i(X) = g_i(X) - r_i(X)$ for $i\in I$\\
    Thus, $e_i=g_i(\xi^i)-r_i(\xi^i)$\\
    Thus, we need to compute $g_i(\xi^i)$ and $r_i(\xi^i)$ for all $i \in I \subset K$
    \begin{align*}
        g_i(\xi^i) &= p(\xi^i)\sum_{j\in [N]\setminus \{i\}} \frac{1}{\xi^i-\xi^j} \\
        &= \frac{p(\xi^i)Z_{\nroots}''(\xi^i)}{2 Z_{\nroots}'(\xi^i)} \text{From lemma \ref{lem:sumtoder}} \\
        &= \frac{(N-1)\Delta t_i}{2\xi^i}
    \end{align*}

    We used the fact that $Z_H(X)=X^N-1$ and that $p(\xi^i)=\Delta t_i$.\\
    Thus, in $O(|\setind|)$ operations, we get $g_i(\xi^i)$ for all $i$\\
    If we get $r_i(\xi^i)$ for all $i \in \setind$, then getting $f_i(\xi^i)$ will take only $O(|\setind|)$ operations\\\\
    \textbf{So, it suffices to compute $r_i(\xi^i)$ for all $i \in I$}\\

    First of all, we write $r_i(X)$ as:
    \begin{align}\label{eq:rx}
    r_i(X) &= \sum_{j\in [N]}\frac{p(X)-p(\xi^j)}{X-\xi^j} - \frac{p(X)-p(\xi^i)}{X-\xi^i} \nonumber \\
    \intertext{Now, defining the bivariate polynomial $u(X,Y)=(p(X) - p(Y))/(X-Y)$ gives:}
    &r_i(X)= \sum_{j\in [N]}u(X,\xi^j) - u(X,\xi^i) \nonumber
    \end{align}
    Now, define $r(X)=\sum_{j\in [N]}u(X,\xi^j)$\\
    Thus, we get:\\
    $$r_i(X)=r(X)-u(X, \xi^i)$$
    or
    $$r_i(\xi^i)=r(\xi^i) - u(\xi^i,\xi^i)$$
    But, by definition and properties of formal derivative, we have the following polynomial identity
    $$u(X, X)= p'(X)$$
    where $p'(X)$ denotes the derivative of the polynomial $p(X)$ computed in the usual way. Thus, $u(\xi^i, \xi^i)=p'(\xi^i)$\\

    Thus, we have that:
    $$r_i(\xi^i)=r(\xi^i) - p'(\xi^i) $$

    \textbf{Thus, $r_i(\xi^i)$ for all $i \in \setind$ can be computed by evaluating polynomials $r(X)$ and $p'(X)$ at the set $H_{\setind}$}


    Now, from Lemma \ref{lem:sumcheck}, we have that $r(X)=N u(X,0)=N \frac{(p(X) - p(0))}{X}$ and thus $r(\xi^i)$ for $i\in \setind$ is given by:
    $$r(\xi^i)=N \frac{(\Delta t_i - p(0))}{\xi^i}$$
    If we get $p(0)$, then we can get $r(\xi^i)$ for all $i$ in $O(m)$ operations.\\\\
    \textbf{Thus, it suffices to compute $p(0)$ and $p'(X)$ at the set $\{\xi^i:i \in I\}$}\\\\

    Let $p(X)=\widehat{Z}_K(X)\cdot q(X)$
    as in Equation \eqref{eq:px} and let $Z_K(X)$ be the vanishing polynomial for $H_K$\\
    Note that $Z_K(X) \cdot \hat{Z}_K(X)=X^N-1$(by definition)\\

    Notice that $Z_\nroots'(\xi^j)= \frac{N}{\xi^j} $\\
    Moreover, $Z_K(X)$ can be computed by fast multiplication in $O(|K|\log^2 |K|)$ operations and then it can be differentiated monomial by monomial to get $Z_K'(X)$. \\
    Then, we can do fast evaluation to get $Z_K'(X)$ at $\{\xi^j:j\in K\}$ in $O(|K|\log^2 |K|)$ operations\\
    Recall the lemma \ref{lem:zk-hat} which tells us that $\hat{Z}_K(\xi^j)=\frac{Z_{\nroots}'(\xi^j)}{Z_K'(\xi^j)}$.\\
    Thus, we can compute $\hat{Z}_K(\xi^j)=\frac{\frac{N}{\xi^j}}{Z_K'(\xi^j)}$ for all $j \in K$\\\\
    Now, recall from \eqref{eq:px} that $q(\xi^j)=\frac{p(\xi^j)}{\hat{Z}_K(\xi^j)}=\frac{\Delta t_j}{\widehat{Z}_K(\xi^j)}$\\
    So, we can compute $q(\xi^j)$ for all $j \in K$. \\
    As degree of $q(X)$ is strictly less than $|K|$ we can interpolate to get $q(X)$ in $O(|K|\log^2 |K|)$ field operations. \\
    Again by differentiating monomial by monomial, we get $q'(X)$ and by fast evaluation we can get evaluations of $q'(X)$ at $\xi^i$ for all $i \in I$, again in $O(|K|\log^2 |K|)$ field operations.\\\\

    Now we can compute $p(0)$\\
    Consider $p(0)=q(0)\cdot \hat{Z}_K(0)$\\
    But $\hat{Z}_K(0)=\frac{Z_H(0)}{Z_K(0)}=\frac{-1}{Z_K(0)}$\\
    But this can be computed as we know $Z_K(X)$ and $Z_K(0)$ is just its constant term.\\
    Similarly, we know $q(X)$ so can compute $q(0)$ as it is just the constant term. \\
    Thus, we get $p(0)$\\
    \textbf{Now it remains to compute $p'(X)$ at $\{\xi^i:i \in I\}=H_I$\\\\}
    For this, by product rule of derivatives, $p'(X)=q(X) \hat{Z}_K'(X)+q'(X) \hat{Z}_K(X)$\\

    Thus, to get $p'(X)$ at $H_I$, we need $q, q', \hat{Z}_K$ and $\hat{Z}_K'$ at $H_I$. If we get these, then in $O(|I|)$ operations we would get $p'(X)$ at $H_I$.\\
    Whatever we have done so far has given us $q, q', \hat{Z}_K$ at $H_I$\\\\
    \textbf{So, it suffices to compute $\hat{Z}_K'(X)$ at $H_I$\\}

    For this, we recall the second equation of lemma \ref{lem:zk-hat} and replace $X$ by $\xi^i$ to get:\\
    $$  \widehat{Z}_K'(\xi^i) = \sum_{j\in K\setminus \{i\}} \frac{Z_\nroots'(\xi^j)}{Z_K'(\xi^j)}\mu_j'(\xi^i) + \frac{Z_\nroots'(\xi^i)}{Z_K'(\xi^i)}\mu_i'(\xi^i)$$
    This we got by splitting the sum $j \in K$ into 2 sums, one with $j \in K \setminus \{i\}$ and another with $j=i$

    Using lemma \ref{lem:lamda-deriv}, this becomes:
    $$ \widehat{Z}_K'(\xi^i)= N\xi^{-i}\sum_{j\in K\setminus \{i\}}\frac{1}{Z_K'(\xi^j)(\xi^i-\xi^j)} + \frac{N(N-1)}{2\xi^{2i}Z_K'(\xi^i)}$$


    So, if we can compute $\sum_{j\in K\setminus \{i\}}\frac{1}{Z_K'(\xi^j)(\xi^i-\xi^j)}$ for all $i \in I$, then in $O(|I|)$ operations, we will get $\hat{Z}_K'(\xi^i)$ for all $i \in I$ which is what we want.\\

    Let $$\varphi_i=\sum_{j\in K\setminus \{i\}}\frac{1}{Z_K'(\xi^j)(\xi^i-\xi^j)}$$\\\\
    Thus,\textbf{ it suffices to compute $\varphi_i$ for all $i \in I$}\\\\

    Define:
    $$ \Phi_i(X) = \sum_{j\in K\setminus \{i\}} \frac{1}{Z_K'(\xi^j)(X-\xi^j)} $$
    at $X=\xi^i$. Then it is clear that $\varphi_i=\Phi_i(\xi^i)$.\\\\
    Let $\{\eta_i(X)\}_{i\in K}$ be the Lagrange polynomials for the set $\{\xi^i:i\in K\}= H_K$. \\\\
    Then, $\eta_j(X)=\frac{Z_K(X)}{Z_K'(\xi^j)(X-\xi^j)}$.\\
    Thus, $\frac{\eta_j(X)}{Z_K(X)}=\frac{1}{Z_K'(\xi^j)(X-\xi^j)}$\\
    Thus, $\Phi_i(X)$ can be rewritten as:
    $$\Phi_i(X)=\sum_{j\in K\setminus \{i\}} \frac{\eta_j(X)}{Z_K(X)}$$


    or  $$\Phi_i(X)
    = \sum_{j\in K\setminus \{i\}} \frac{\eta_j(X)/(X-\xi^i)}{Z_K(X)/(X-\xi^i)}$$

    We need to compute $\varphi_i=\Phi_i(\xi^i)$ for all $i \in I$


    Thus, putting $X=\xi^i$ in the above, we have:

    $$\varphi_i = \Phi_i(\xi^i) = \left(\sum_{j\in K\setminus \{i\}}\frac{\eta_j(X)/(X-\xi^i)}{Z_K(X)/(X-\xi^i)}\right)(\xi^i)$$
    This can be simplified as:
    $$\varphi_i = \Phi_i(\xi^i) = \left(\sum_{j\in K\setminus \{i\}}\frac{\eta_j(X)/(X-\xi^i)}{\prod_{j \in K \setminus \{i\}}(X-\xi^j)}\right)(\xi^i)$$
    or $$\varphi_i = \Phi_i(\xi^i) = \sum_{j\in K\setminus \{i\}}\left(\frac{\eta_j(X)/(X-\xi^i)}{\prod_{j \in K \setminus \{i\}}(X-\xi^j)}\right)(\xi^i)$$
    or $$\varphi_i = \Phi_i(\xi^i) = \sum_{j\in K\setminus \{i\}}\frac{\left(\eta_j(X)/(X-\xi^i)\right)(\xi^i)}{\left(\prod_{j \in K \setminus \{i\}}(X-\xi^j)\right)(\xi^i)}$$
    or $$\varphi_i =\Phi_i(\xi^i) =  \frac{1}{Z_K'(\xi^i)}\sum_{j\in K\setminus \{i\}}\left(\eta_j(X)/(X-\xi^i)\right)(\xi^i)$$

    Now, $$ \left(\eta_j(X)/(X-\xi^i)\right)(\xi^i)=\left(\frac{Z_K(X)}{(X-\xi^j)(X-\xi^i)Z_K'(\xi^j)}\right)(\xi^i)$$
    or $$\left(\eta_j(X)/(X-\xi^i)\right)(\xi^i)= \frac{Z_K'(\xi^i)}{(\xi^i-\xi^j)Z_K'(\xi^j)}$$

    But if we compute, $\eta_j'(\xi^i)$ by quotient rule, we get:
    $$\eta_j'(\xi^i)=\frac{Z_K'(\xi^i)}{(\xi^i-\xi^j)Z_K'(\xi^j)}$$
    Thus,
    $$\left(\eta_j(X)/(X-\xi^i)\right)(\xi^i)=\eta_j'(\xi^i) $$
    As both are separately equal to $\frac{Z_K'(\xi^i)}{(\xi^i-\xi^j)Z_K'(\xi^j)}$.\\
    Thus, we get:


    $$\varphi_i =\Phi_i(\xi^i) =  \frac{1}{Z_K'(\xi^i)}\sum_{j\in K\setminus \{i\}}\eta_j'(\xi^i)$$
    But, $$\sum_{j\in K\setminus \{i\}}\eta_j'(\xi^i)= \sum_{j\in K}\eta_j'(\xi^i)-\eta_i'(\xi^i)=-\eta_i'(\xi^i)$$
    using that the sum of Lagrange polynomials is $1$ and thus the sum of the derivatives of the Lagrange polynomials is $0$(section A.1)

    Thus,
    $$\varphi_i =\Phi_i(\xi^i) = \frac{-\eta_i'(\xi^i)}{Z_K'(\xi^i)}$$

    If we get $\eta_i'(\xi^i)$ for all $i\in \setind$, then as we already have $Z_K'(\xi^i)$ for all $i \in I$, we get $\varphi_i$ for all $i \in I$ in $O(|I|)$ computations which is what we need.


    \textbf{Thus, it suffices to efficiently compute $\eta_i'(\xi^i)$ for $i\in \setind$}

    For this recall Lemma \ref{lem:tau} and observe that in our setting $\mathcal{X}$ is just $H_K$, $x_j$ are $\xi^j$ and $Z_\mathcal{X}(X)$ which is the vanishing polynomial of $H_K$ is just $Z_K(X)$. The $\eta_i$ in the lemma are the Lagrange polynomials corresponding to $\mathcal{X}=H_K$ so they are same as the $\eta_i$ in our setting\\\\
    Thus, we get:\\
    $$\eta_i'(\xi^i)=\frac{F_K(\xi^i)}{Z_K'(\xi^i)}$$

    $Z_K'(\xi^i)$ we already have for all $i \in I$\\
    If we get $F_K(\xi^i)$ for all $i \in I$, we will be able to get $\eta_i'(\xi^i)$ for all $i \in I$ in $O(|I|)$ operations which is what we want\\\\

    \textbf{Thus, it suffices to compute $F_K(\xi^i)$ for all $i \in I$}\\\\

    But now recall the definition $F_K$. To compute this, we need information about $z_0, z_1, z_2, \cdots, z_k$.\\
    But, we know the polynomial $Z_K$(fast multiplication) hence we know $z_0, z_1, \cdots, z_{k}$\\
    Thus, we get the coefficients of the polynomial $F_K(X)$, hence the polynomial $F_K(X)$\\\\
    Now, by fast evaluation, we can get $F_K(\xi^i)$ for all $i \in I$. This will take $O(|K| \log^2|K|)$ field operations. \\
    \textbf{Thus, we have computed $F_K(\xi^i)$ for all $i \in I$}\\

    This completes the proof for $e_i$ (Note that in the proof we throughout used that $|\setind| \leq |K|$ so that the various $O(|\setind|)$ operations do not contribute to the final complexity) \\\\

    For the case of $b_j$, notice first that it is very similar to the case of $e_i$ except that the roles of $\setind$ and $K$ are reversed and $i$ and $j$ are reversed.

    First of all define $c_i$ for $i$ outside $\setind$ to be 0. For $i \in \setind$, we already know what $c_i$ is(can be computed by fast evaluation).
    Next, define polynomial $p$ of degree atmost $N-1$ such that $p(\xi^i)=c_i$ for all $i \in [N]$.\\

    From here on, we proceed exactly as we did for the $e_i$ case, replacing every $\setind$ with $K$, every $K$ with $\setind$, every $i$ with $j$ and every $j$ with $i$\\

    All the required lemmas are also modified in this way.

    We reach till
    $$\Phi_j(X)=\sum_{i\in I\setminus \{j\}} \frac{\eta_i(X)}{Z_I(X)}$$
    and we need to compute $\varphi_j=\Phi_j(\xi^j)$ for all $j \in K$.\\

    Here, we deviate: For $j \in K\setminus I$, we can very easily compute $\varphi_j=\Phi_j(\xi^j)$ as
    $$\Phi_j(\xi^j)=\sum_{i\in I\setminus \{j\}} \frac{\eta_i(\xi^j)}{Z_I(\xi^j)}$$
    or
    $$\Phi_j(\xi^j)=\frac{1}{Z_I(\xi^j)} \sum_{i\in I}\eta_i(\xi^j)$$
    or
    $$\Phi_j(\xi^j)=\frac{1}{Z_I(\xi^j)}$$

    Using that the sum of all Lagrange polynomials is $1$

    \textbf{So, it suffices to compute $\varphi_j$ for all $j \in \setind$}

    Now, again continue exactly as in the $e_i$ case until we reach:
    $$\varphi_j =\Phi_j(\xi^j) = \frac{-\eta_j'(\xi^j)}{Z_I'(\xi^j)}$$
    for all $j \in \setind$\\

    But $Z_I'(\xi^j)$ we would have already computed for all $j \in \setind$ (fast multiplication and fast evaluation). \\

    \textbf{So, it suffices to compute $-\eta_j'(\xi^j)$ for all $j \in \setind$\\}
    But we computed $\eta_i'(\xi^i)$ for $i\in \setind$ during the computation of $e_i$. Just reuse whatever we got!\\

    This completes the computation for $b_j$ for all $j \in K$.
    Thus, this finishes the proof of lemma \ref{lem:sum computation}
\end{proof}

\section{Reducing indexed lookups to subvector lookups}
Let us say we have a subvector lookup argument $L$ which takes as input two vectors say $a, b$ and determines whether $b$ is a subvector of $a$.\\
\textbf{Assumption-} We assume that $a$ and $b$ are vectors with no repeated elements.\\
If $a$ and $b$ have repetitions, then we remove the repeated elements from each until we get $a'$, $b'$, which have the same elements as $a$, $b$ respectively, except that there are no repetitions.\\
Then we define $L(a,b)$ as $L(a', b')$


Indexed lookup argument takes in 3 vectors $a, b, c$ where $c$ is a vector of indices with $|b|=|c|$ and checks that $b$ is a subvector of $a$ and that $a[c_i]=b_i$ for all $i$. That is it does $L(a, b)$ first and further checks that $a$ restricted to the indices in $c$ gives $b$. Again, we assume that none of the vectors have repetitions.\\\\
So, we want to do $L'(T, v, a)$ using $L(\cdot, \cdot)$.
Let $|T|=N$ and it has no repetitions. Let $|v|=m=|a|$ with no repetitions.
Procedure to calculate $L'(T, v, a)$ is as follows:
\begin{itemize}
    \item First do $L([N], a)$. This checks that $a$ is indeed a vector of indices.
    \item Next do $L(T, v)$. This checks that $v$ is indeed a subvector of $T$.
    \item Choose a $\gamma$ uniformly at random from $\F^*$
    \item Do $L(T+\gamma[N], v+\gamma a)$
    \item If all the 3 callings of $L$ output 'true' then output 'true' for $L'(T, v, a)$
    \item If any of the 3 outputs 'false' then output 'false' for $L'(T, v, a)$
\end{itemize}

\begin{lemma}
    Let the procedure for getting $L'(\cdot, \cdot, \cdot)$ be as above. Then, if $L(\cdot, \cdot)$ is a valid subvector lookup argument then the procedure yields a valid indexed lookup argument $L'(\cdot, \cdot, \cdot)$ with error probability upper bounded by the inverse of the size of the field $\F$
\end{lemma}

\begin{proof}
    To show that the procedure yields a valid indexed lookup argument $L'(T, v, a)$, we need to show the following:
    \begin{enumerate}

        \item If $v$ is indeed a subvector of $T$ with $T$ restricted to $a$ giving $v$, then all the checks mentioned in the procedure hold with high probability
        \item If either $v$ is not a subvector of $T$ or if $v$ is a subvector of $T$ but restriction of $T$ to $a$ is not $v$, then at least one of the checks is false with high probability
    \end{enumerate}

    For 1., suppose $v$ is a subvector of $T$ with $T$ restricted to $a$ giving $v$. Then $L(T, v)$ is true. \\
    Moreover, $a$ is a subvector of $[N]$(else restricting $T$ to $a$ won't even make sense). \\
    So, $L([N], a)$ is true. \\
    Lastly, since $v_i=T[a_i]$ for all $i \in m$, we have that for any $\gamma$,
    $$T[a_i]+\gamma a_i=v_i+\gamma a_i$$ for all $i$ \\
    But $T[a_i]+\gamma a_i$ is of course an element $T+\gamma [N] $ for all $i \in [m]$. So, $\{T[a_i]+\gamma a_i:i \in [m]\}$ is a subvector of $T+\gamma [N] $.\\
    Thus, $\{v_i+\gamma a_i:i \in [m]\}$ is a subvector of $T+\gamma [N] $.\\
    So, $v+\gamma a$ is a subvector of $T+\gamma [N] $.\\
    Thus, $L(T+\gamma[N], v+\gamma a)$ is also true.

    This completes proof of 1.(In this case the high probability is actually 1) \\\\
    For proof of 2., if $v$ is not a subvector of $T$ then $L(T, v)$ is false with probability 1 and we are done \\
    So let us assume that $v$ is a subvector of $T$ and restriction of $T$ to $a$ is not $v$.\\
    If $a$ is not a vector of indices then $L([N], a)$ is false with probability 1 and we are done. So, let us assume that $a$ is a vector of indices( of course by definition of $L'$, we have that $|a|=|v|=m$).\\

    \textbf{It suffices to show that $L(T+\gamma[N], v+\gamma a)$ is false with high probability.}

    Since $v$ is a subvector of $T$ there is a restriction of $T$ which gives $v$. Say the restriction is $b$. Then $|b|=|v|=m$ and $T[b_i]=v_i$ for all $i \in [m]$. \\
    Moreover, by our assumption that restriction of $T$ to $a$ is not $v$, we have that $a \neq b$. \\
    But $a\neq b$ implies there exists a $k \in [m]$ such that $a_k\neq b_k$.\\
    Let $I=\{k \in [m]:a_k\neq b_k\}$\\
    $a \neq b$ implies $I\neq \phi$. Moreover $I \subset \mathbb{N}$
    By well ordering principle, $I$ has a minimum element, say $i$. \\
    Then $a_i \neq b_i$ and $v_i=T[b_i]$. \\
    Let $\gamma$ be picked uniformly at random from $\F^*$\\
    Consider the vector $T+\gamma[N]$. Once $\gamma$ has been chosen, this is a fixed vector. \\
    Also, consider the vector $v+\gamma a$. One of the elements in this vector is $v_i +\gamma a_i$.\\
    We will upper bound the probability that this element is an element of the vector $T+\gamma[N]$. \\\\
    $$\Pr[v_i +\gamma a_i \in T+\gamma[N]]=\Pr[\exists j \in [N] \ni v_i +\gamma a_i=T[j]+\gamma j]$$
    But by union bound:
    $$\Pr[\exists j \in [N] \ni v_i +\gamma a_i=T[j]+\gamma j] \leq \sum_{j=1}^N \Pr[v_i +\gamma a_i=T[j]+\gamma j]$$
    Thus,
    $$\Pr[v_i +\gamma a_i \in T+\gamma[N]] \leq \sum_{j=1}^N \Pr\left[\gamma = \frac{T[j]-v_i}{a_i-j}\right] $$
    But for every $j \in [N]\setminus{a_i, b_i}$, $\frac{T[j]-v_i}{a_i-j}$ is just an element of $\F^{*}$. \\
    Since $\gamma$ is chosen uniformly at random from $\F^{*}$,
    $$\Pr\left[\gamma = \frac{T[j]-v_i}{a_i-j}\right]=\frac{1}{|\F|-1}$$
    And for $j=a_i$ or $j=b_i$, $\frac{T[j]-v_i}{a_i-j}$ is either not defined or is $0$. So probability that $\gamma$ equals to it is just $0$.\\
    Thus, we get that:
    $$\Pr[v_i +\gamma a_i \in T+\gamma[N]] \leq \sum_{j \in [N]\setminus \{a_i, b_i\}} \frac{1}{|\F|-1} =\frac{N-2}{|\F|-1}$$

    All this was done for the smallest element $i$ of the set $I$.\\
    For $v+\gamma a$ to be a subvector of $T+\gamma [N]$, we need that:
    $$v_i +\gamma a_i \in T+\gamma[N]$$ for all $i \in I$\\
    But observe that,
    $$\Pr[v_i +\gamma a_i \in T+\gamma[N] \, \text{for all} \, i \in I] \leq \Pr[v_i +\gamma a_i \in T+\gamma[N]] $$
    So, we have that $$\Pr[v_i +\gamma a_i \in T+\gamma[N] \, \text{for all} \,i \in I] \leq \frac{N-2}{|\F|-1}$$
    So, the probability that $v+\gamma a$ is a subvector of $T+\gamma [N]$ is upper bounded by $\frac{N-2}{|\F|-1}$. \\
    But $N <<|\F|$. So, $\frac{N-2}{|\F|-1}$ is very small. \\
    So, with a large probability of atleast $(1-\frac{N-2}{|\F|-1})$, $v+\gamma a$ is not a subvector of $T+\gamma [N]$. \\
    So, $L(T+\gamma[N], v+\gamma a)$ is false with high probability. \\
    This completes the proof.
\end{proof}