%\usepackage{amssymb}%! Author = nitinsingh
%! Date = 12/04/24
We now detail the steps required to realize batching efficient RAM outlined in the technical overview.

\subsection{Committed Index Lookup}\label{subsec:committed-index-lookup}
In this section, we ``lift'' any committed sub-vector argument to a committed index lookup argument, where
the latter makes a black-box use of the former. We use the trick of random linear combination of vectors to infer
indexed lookup relation among them from sub-vector relation over the aggregated vectors. Similar use of random
 linear combinations has been made in the context of proving permutations in literature (e.g. ~\cite{SCN:GOTV22}).

\begin{lemma}\label{lem:generic-transformation}
Let $\vec{t}\in \F^n$ and let $\vec{a},\vec{v}\in \F^m$ for some positive integers $m,n$. Let $\vec{I}_n$ denote the vector $(1,\ldots,n)$.
Then for $\gamma\gets \F$, $(\vec{v}+\gamma \vec{a})\preceq (\vec{t} + \gamma \vec{I}_n)$ implies
$\vec{v}=\vec{t}[\,\vec{a}\,]$ except with probability $mn/|\F|$.
\end{lemma}
\begin{proof}
We define vectors of linear polynomials $\vec{p}=(p_1,\ldots,p_m)$ and $\vec{q}=(q_1,\ldots,q_n)$ where
$p_i(X) = v_i + a_i X$, $i\in [m]$ and $q_i(X) = t_i + i X$, $i\in [n]$. Now, we see that $\vec{v}=\vec{t}[\,\vec{a}\,]$
if and only if $\vec{p}\preceq \vec{q}$. For $\gamma\in F$, let $\vec{p}_\gamma$ and $\vec{q}_\gamma$ denote the vectors
$(p_1(\gamma),\ldots,p_m(\gamma))$ and $(q_1(\gamma),\ldots,q_n(\gamma))$ respectively. It is obvious that $\vec{p}\preceq \vec{q}$
implies $\vec{p}_\gamma\preceq \vec{q}_\gamma$ for all $\gamma\in \F$. Using Schwartz-Zippel Lemma, it can also be seen that
$\Pr_{{}_{\gamma\gets \F}}[\vec{p}\npreceq \vec{q} \,|\, \vec{p}_\gamma\preceq \vec{q}_\gamma]\leq mn/|\F|$. The bound
follows from the observation that the event occurs only when $\gamma$ is a common root of at least one pair of linear polynomials
$\{(p_i(X),q_j(X)): i\in [m], j\in [n]\}$.
\end{proof}
In Figure ~\ref{fig:committed-index-lookup}, we invoke Lemma ~\ref{lem:generic-transformation} to construct a
committed index lookup argument using a committed sub-vector argument $(\subvecprover,\subvecverifier)$. We formally
state the following lemma, whose proof essentially follows from Lemma ~\ref{lem:generic-transformation}.

\begin{lemma}\label{lem:proto-committed-index-lookup}
Assuming that $(\subvecprover,\subvecverifier)$ is an argument of knowledge for the relation $\RSVEC$ in the AGM, the interactive protocol
in Figure ~\ref{fig:committed-index-lookup} is an argument of knowledge for the relation $\RLOOK$ in the AGM.
\end{lemma}

\begin{figure}[htbp]
\begin{mdframed}
{
%<<<<<<< HEAD
    {\bf Common Input}: $\srs,c_t, c_a, c_v$, $c_I=\gone{I(X)}$ where $I(X)=\enc{I}{\setN}$ encodes the vector $\vec{I}=(1,\ldots,N)\in \F^N$. \\
%=======
%    {\bf Common Input}: $\srs,c_t, c_a, c_v$, $c_I=\gone{I(X)}$ where $I(X)=\enc{\vec{I}_N}{\setN}$
%    with $\vec{I}_N=(1,\ldots,N)$. \\
%>>>>>>> f162aeee62ad0161c590d91a0cd6435404bbcc83
    {\bf Prover's Input}: Vectors $\vec{t}\in \F^N$, $\vec{a},\vec{v}\in\F^m$.
    \begin{enumerate}[leftmargin=1em, label=\arabic*.]
    \item $\verifier$ sends $\gamma\gets \F$.
    \item $\prover$ and $\verifier$ compute: $\tilde{c}_t=\gamma c_I+c_t$, $\tilde{c}_v=\gamma c_a + c_v$.
    \item $\prover$ computes: $\tilde{\vec{t}}=\gamma\vec{I}_N + \vec{t}$, $\tilde{\vec{v}}=\gamma\vec{a} + \vec{v}$.
    \item $\prover$ and $\verifier$ run sub-vector argument $(\subvecprover,\subvecverifier)$ with
    $(\srs,\tilde{c}_t,\tilde{c}_v)$ as the common input and $(\tilde{\vec{t}},\tilde{\vec{v}})$ as $\subvecprover$'s input.
    \item $\verifier$ outputs
    $b\gets \argoutput{\subvecprover}{\subvecverifier}{\tilde{\vec{t}},\tilde{\vec{v}}}{\srs,\tilde{c}_t,\tilde{c}_v}$.
    \end{enumerate}
}
\end{mdframed}
\caption{Committed Index Lookup Argument}
\label{fig:committed-index-lookup}
\end{figure}

\subsection{Almost Identical RAM States}\label{subsec:proximity-ram}
For a vector $\vec{a}\in [N]^m$, let $\uniq{a}=\{a_i: i\in [m]\}$ denote the subset of unique values in $\vec{a}$. We call two
RAM states $\vecT, \vecT'\in \F^N$ to be $\vec{a}$-{\em identical} if $\vecT[i]=\vecT'[i]$ for all $i\not\in\uniq{a}$. As before,
let $T(X),T^*(X)$ and $a(X)$ be polynomials encoding the vectors $\vecT,\vecT'$ (over $\setN$) and $\vec{a}$ (over $\setV$). Let
$c_T, c_T'$ and $c_a$ be the commitments to vectors $\vecT, \vecT'$ and $\vec{a}$ respectively in the group $\Gone$. The polynomial protocol to prove that
$\vecT,\vecT'\in \F^N$ and $\vec{a}\in \F^m$ are $\vec{a}$-identical requires proving the relation
$Z_I(X)(T(X) - T^*(X)) = 0$ over the set $Z_\setN$ where
$I=\uniq{a}$ and $Z_I(X)=\prod_{i\in I}(X-\xi^i)$ is the vanishing polynomial for the set $\setN_I=\{\xi^i: i\in I\}$.
%\moumita{ Prev: To proceed, the honest prover commits to polynomial $Z_I$ and proves (i) $Z_I(T - T') = 0 \text{ mod } Z_\setN$ and (ii) the zeroes
%of $Z_I$ form a subset of $\setN_I$ as defined. }
To proceed, the honest prover commits to polynomial $Z_I(X)$ and proves (i) $Z_I(X)\cdot (T(X) - T^*(X)) = 0 \text{ mod } Z_\setN$ and (ii) the zeroes of $Z_I(X)$ form a subset of zeroes of $\setN_I(X)$ as defined.
Together, the two conditions imply that $T(\xi^i)=T^*(\xi^i)$ for $i\not\in \uniq{a}$.
To prove the first relation, the prover computes the polynomial $D(X)$ as below:
\begin{align}\label{eq:poly-q}
D(X) &= \frac{(T(X)-T^*(X))\cdot Z_I(X)}{Z_\setN(X)} \nonumber \\
&= \sum_{i\in I}\frac{(T(\xi^i) - T^*(\xi^i))\mu_i(X)}{Z_\setN(X)} Z_I(X) \nonumber \\
\intertext{ Substituting, $\Delta_i=T(\xi^i) - T^*(\xi^i)$, $\mu_i(X)=\vpolyN(X)/(\vpolyN'(\xi^i)(X-\xi^i))$ }
&=\sum_{i\in I}\frac{\Delta_i}{Z_\setN'(\xi^i)}\left(\frac{Z_I(X)}{X-\xi^i}\right) = \sum_{i\in I}\frac{\Delta_i Z_I'(\xi^i)}{Z_\setN'(\xi^i)}\kappa_i(X)
\end{align}
In the above, the summation only runs over indices in $I$, as $\Delta i = 0$ for $i\not\in I$. In the final equality, we use
$\kappa_i(X) = Z_I(X)/(Z_I'(\xi^i)(X-\xi^i))$ for $i\in I$ which we recognize as the lagrange basis polynomials for the set
$\{\xi^i: i\in I\}$. Thus, Equation \eqref{eq:poly-q} implies that $D(X)$ is at most degree $|I|-1$ polynomial, with
$D(\xi^i)=\Delta_i Z_I'(\xi^i)/\vpolyN'(\xi^i)$ for $i\in I$.
The prover can therefore interpolate $D(X)$ (in power basis)
in $O(|I|\log^2 |I|)$ $\F$-operations and compute $\gone{D(X)}$ in $O(|I|)$ $\Gone$-operations. The prover sends the
commitment $\gone{D(X)}$ to the verifier. Finally, the verifier can
check the identity $Z_I(X) \cdot  (T(X) - T^*(X)) = D(X) \cdot Z_\setN(X)$ %\moumita{Prev: $Z_I(T - T^*) = D\cdot Z_\setN$ . Alternative: $Z_I(T - T') = D\cdot Z_\setN$} 
by a pairing check. For this, since the tables are committed in $\Gone$, prover will need to send $\elttwo{Z_I(X)}$.

Next, the prover needs to show that zeroes of $Z_I$ are indeed in the set $\setN_I=\{\xi^i: i\in I\}=\{\xi^{a_i}:i\in [m]\}$.
Clearly, it suffices to show that $Z_I(X)$ divides the polynomial $\prod_{i\in [m]}(X - \xi^{a_i})$. To obtain a
polynomial protocol, the prover commits to an auxiliary polynomial $h(X)=\sum_{i=1}^m \xi^{a_i}\tau_i(X)$
which interpolates the vector $\vec{h}=(\xi^{a_1},\ldots,\xi^{a_m})$. The correctness of $h$ polynomial can be
established by showing that the interpolated vector $\vec{h}$ satisfies committed index lookup relation
$\vec{h}=\vec{T}_{\exp}[\,\vec{a}\,]$ where $\vec{T}_{\exp}=(\xi^1,\ldots,\xi^N)$. Moreover, we notice that the polynomial
interpolating the table $\vec{T}_{\exp}$ is particularly simple, i.e, $T_{\exp}(X)=X$, and thus the setup need not be
augmented with table-specific parameters for $\vec{T}_{\exp}$.
Finally, it remains to show that $Z_I(X)$ divides $K(X) = \prod_{i=1}^m (X - h(\nu^i))$. To do so, the prover commits to
$K(X)$ and the quotient polynomial $q(X)=K(X)/Z_I(X)$. The verifier checks the polynomial identities
at $\alpha$, i.e $K(\alpha)=q(\alpha)Z_I(\alpha)$ and $K(\alpha)=\prod_{i=1}^m(\alpha - h(\nu^i))$.
The former is easily accomplished
using evaluation proofs for $K,q$ and $Z_I$ at $\alpha$.
For checking the latter, the prover commits to another polynomial
$u(X)$ satisfying $u(\nu^i)=\prod_{j=1}^{i-1}\big((\alpha - h(\nu^j))/(1 + \beta\tau_1(\nu^j))\big)$ for $i\in [m]$
where $\beta=K(\alpha) - 1$.
The verifier ensures the correctness of $u(X)$ by checking:
\begin{equation}
    \begin{aligned}
        \tau_1(X)(u(X) - 1) &= 0 \text{ mod } Z_{\setV} \\
        u(\nu X)(1+\beta \tau_1(X))-u(X)(\alpha - h(X)) &= 0 \text{ mod } Z_\setV.
    \end{aligned}
    \label{eq:kh-check}
\end{equation}
We prove that the above constraints imply that $K(\alpha)=\prod_{i\in [m]}(\alpha - h(\nu^i))$ in Lemma ~\ref{lem:kh-check}.
Note that in this protocol we require commitment to the polynomial $Z_I$ in both $\Gone$ and $\Gtwo$,
and thus another pairing check is required to show that the $Z_I(X)$ committed in $\Gone$
is the same as the $Z_I(X)$ committed in $\Gtwo$ (used for the real pairing check).
The complete protocol for checking that RAMs $\vecT$ and $\vecT'$ are identical outside indices in $\vec{a}$
is given in Figure ~\ref{fig:a-identical}.

\begin{lemma}\label{lem:kh-check}
There exists a polynomial $u(X)\in \F[X]$ satisfying the identities in Equation ~\eqref{eq:kh-check}
if and only if $K(\alpha)=1+\beta=\prod_{i\in [m]} (\alpha - h(\nu^i))$.
\end{lemma}
\begin{proof}
    Assume that the identitites hold for some polynomial $u(X)$.
    The first identity implies $u(\nu)=1$. From the second identity, we conclude that for all $i\in [m]$, we have
    $u(\nu^{i+1})=u(\nu^i)\cdot ((\alpha - h(\nu^i))/(1+\beta \tau_1(\nu^i)))$, and thus:
    $$1 = u(\nu^{m+1})/u(\nu) = \prod_{i\in [m]}\left(\frac{\alpha - h(\nu^i)}{1+\beta \tau_1(\nu^i)}\right).$$
    We observe that the product of denominators in the above equation is simply $1+\beta$ as $\tau_1(\nu^i)$
    is $0$ for all $i\neq 1$, and thus $1+\beta = \prod_{i=1}^m (\alpha - h(\nu^i))$. In the other direction,
    it is easy to check that $u(X)$ as defined for an honest prover, satisfies the identities in Equation ~\ref{eq:kh-check}.
\end{proof}

%%% complete protocol listing %%%
\begin{figure}[htbp]
    \begin{mdframed}

        {\bf Common Input}: $\srs$, $c_T, c_T', c_a$.

        {\bf Prover's Input}: Vectors $\vecT,\vecT'\in \F^N$, $\vec{a}\in \F^m$. Polynomials $T(X),T^*(X)$ and
        $a(X)$ encoding $\vecT,\vecT'$ and $\vec{a}$ respectively.\\

        {\bf Round 1}: Prover commits to auxiliary polynomials
        \begin{enumerate}[leftmargin=1em, label=\arabic*.]
            \item $\prover$ computes:
                \begin{itemize}[leftmargin=1em, label=-]
                \item $I=\uniq{a}$, $Z_I(X)=\prod_{i\in I}(X-\xi^i)$.
                \item $D(X)=Z_I(X)(T(X) - T^*(X))/\vpolyN(X)$.
                \item $h(X)$ such that $h(\nu^i)=\xi^{a_i}$ for $i\in [m]$.
                \item $K(X)=\prod_{i=1}^m (X - h(\nu^i))$, $q(X)=K(X)/Z_I(X)$.
                \end{itemize}
            \item $\prover$ sends $c_z = \gone{Z_I(X)}$, $c_z'=\gtwo{Z_I(X)}$, $c_d=\gone{D(X)}$, $c_h=\gone{h(X)}$, $c_k=\gone{K(X)}$,
            $c_q=\gone{q(X)}$.
            \item $\verifier$ sends $\alpha\gets \F$.
        \end{enumerate}

        {\bf Round 2}: Prover commits to polynomial $u(X)$.
        \begin{enumerate}[leftmargin=1em, label=\arabic*.]
            \item $\prover$ sets $\beta=K(\alpha)-1$ and interpolates $u(X)$ on $\setV$ such that
                $u(\nu^i)=\prod_{j=1}^{i-1}\big((\alpha - h(\nu^j))/(1 + \beta\tau_1(\nu^j))\big)$ for $i\in [m]$.
            \item $\prover$ sends $c_u=\gone{u(X)}$.
            \item $\verifier$ sends $r\gets \F$.
        \end{enumerate}


        {\bf Round 3}: Prover batches checks in Eq ~\eqref{eq:kh-check}.
        \begin{enumerate}[leftmargin=1em, label=\arabic*.]
            \item $\prover$ computes:
            \begin{align*}
            Q(X)=\big(&u(\nu X)(1+\beta \tau_1(X)) - u(X)(\alpha - h(X)) \\
            &\quad + r\tau_1(X)(u(X)-1)\big)/Z_{\setV}(X)
            \end{align*}
            \item $\prover$ sends $c_Q = \gone{Q(X)}$.
            \item $\verifier$ sends $s\gets \F$.
        \end{enumerate}

        {\bf Round 4}: Prover sends evaluations.
        \begin{enumerate}[leftmargin=1em, label=\arabic*.]
        \item $\prover$ sends $\val{\alpha}{z}=Z_I(\alpha)$, $\val{\alpha}{q}=q(\alpha)$, $\val{\alpha}{K}=K(\alpha)$,
        $\val{s}{Q}=Q(s)$, $\val{s}{u}=u(s)$, $\val{\nu s}{u}=u(\nu s)$, $\val{s}{h}=h(s)$.
        \item $\verifier$ sends $r_\alpha, r_s\gets \F$.
        \end{enumerate}

        {\bf Round 5}: Prover batches evaluation proofs.
        \begin{enumerate}[leftmargin=1em, label=\arabic*.]
        \item Compute:
            \begin{itemize}[leftmargin=1em, label=-]
            \item $p_\alpha(X)=Z_I(X) + r_\alpha q(X) + r_\alpha^2 K(X)$.
            \item $p_s(X) = Q(X) + r_s u(X) + r_s^2 h(X)$.
            \item $\Pi_\alpha = \kzgprove(\srs,p_\alpha,\alpha)$.
            \item $\Pi_s = \kzgprove(\srs,p_s,s)$, $\Pi_{\nu s}$ = $\kzgprove(\srs,u,\nu s)$.
            \end{itemize}
        \item Send $\Pi_\alpha, \Pi_s, \Pi_{\nu s}$.
        \end{enumerate}

        {\bf Round 6}: Verifier checks identities.
        \begin{enumerate}[leftmargin=1em, label=\arabic*.]
        \item $\verifier$ computes $\gone{p_\alpha}=c_z + r_\alpha c_q + r_\alpha^2$, $\gone{p_z}$ = $c_Q + r_s c_u + r_s^2 c_h$.
        \item $\verifier$ checks:
            \begin{itemize}[leftmargin=1em, label=-]
                \item $\val{\alpha}{z}\cdot \val{\alpha}{q}=\val{\alpha}{K}$.
                \item $\val{\nu s}{u}(1+\beta \tau_1(s)) - \val{s}{u}(\alpha - \val{s}{h}) + r\tau_1(s)(\val{s}{u}-1)=\val{s}{Q}Z_\setV(s)$.
                \item $e(c_T - c_T', c_z')=e(c_d, \gtwo{\vpolyN(X)})$.
                \item $e(\gone{1},c_z')=e(c_z, \gtwo{1})$.
                \item $\kzgverify(\srs$ ,
        $\gone{p_\alpha}$, $\val{\alpha}{z}+r_\alpha \val{\alpha}{q} + r_\alpha^2 \val{\alpha}{K}$, $\alpha$, $\Pi_\alpha)$.
                \item $\kzgverify(\srs$,
                $\gone{p_z}$,  $\val{s}{Q}+r_s \val{s}{u} + r_s^2 \val{s}{K}$, $s$, $\Pi_s)$.
                \item $\kzgverify(\srs,c_u, \val{\nu s}{u}, \nu s, \Pi_{\nu s})$.
            \end{itemize}
        \end{enumerate}

        {\bf Round 7}: Check correctness of polynomial $h$.
        \begin{enumerate}[leftmargin=1em, label=\arabic*.]
        \item $\prover$ and $\verifier$ execute committed index lookup argument (Fig ~\ref{fig:committed-index-lookup})
        to check $(\gone{X},c_a,c_h)\in \RLOOK$.
        \item $\verifier$ accepts if the above argument accepts and all the preceding checks succeed.
        \end{enumerate}
    \end{mdframed}
    \caption{Argument for showing RAMs are identical outside small set of indices.}
    \label{fig:a-identical}
\end{figure}

\subsection{Batching-Efficient RAM: Combined Protocol}\label{subsec:all-together}
We put the entire protocol together now. Let $\setind$ denote the set of indices $\{1,\ldots,N\}$, and $\vec{I}_N$
denote the vector $(1,\ldots,N)$. We formally define the committed RAM relation for which we present an argument of
knowledge in this section.
\begin{definition}\label{defn:committed-ram}
We define the {\em committed ram} relation
$\CRAM$ to consist of tuples $((c_T, c_T', c_\op, c_a, c_w),(\vecT, \vecT',\vec{\op},\vec{a},\vec{w}))$
such that:
\begin{itemize}[leftmargin=1em]
    \item $(\vecT,\vec{o},\vecT')\in \LRAM{I}{N}{m}$ for $\vec{o}=(o_1,\ldots,o_m)$ where we have $o_i=(\op_i, a_i, w_i)\in \RAMOp{I}$ for all $i\in [m]$~(here we implicitly view vectors $\vecT$ and $\vecT'$ as RAMs with index column $\setind_N$). 
    
%    for $T=(\setind_N,\vecT)$, $T'=(\setind_N,\vecT')$ and $\vec{o}=(o_1,\ldots,o_m)$
   
    \item $c_T=\KZGcommit(\srs, T(X))$, $c_T'=\KZGcommit(\srs, T^*(X))$, $c_\op = \KZGcommit(\srs,\op(X))$,  $c_a=\KZGcommit(\srs, a(X))$,
    $c_w = \KZGcommit(\srs, w(X))$ where polynomials $T(X), T^*(X)$ encode vectors $\vecT, \vecT'$ over $\setN$, while $\op(X), a(X)$ and
    $w(X)$ encode vectors $\vec{\op}=(\op_1,\ldots,\op_m)$, $\vec{a}$ and $\vec{w}$ over $\setV$.
\end{itemize}
\end{definition}
As outlined in the blueprint, the prover first commits to ``smaller'' RAMs $\Sbf=(\vec{a},\vec{v})$ and $\Sbf'=(\vec{a},\vec{v}')$
where $\vec{v}=\vecT[\vec{a}]$ and $\vec{v}'=\vecT'[\vec{a}]$. The prover commits to $\Sbf$ and $\Sbf'$ by sending commitments
$c_v$ and $c_v'$ to $\vec{v}$ and $\vec{v}'$. Then the prover and verifier execute the committed index lookup protocol to
prove:
\begin{equation}
(c_T, c_a, c_v)\in \RLOOK\, \wedge\, (c_T', c_a, c_v')\in \RLOOK
\end{equation}
The verifier uses a random challenge $\chi\gets \F$ to reduce two instances of $\RLOOK$ to one instance
$(c_T + \chi c_T', c_a, c_v + \chi c_v')\in \RLOOK$. Then, we show that
RAMs $\vecT$ and $\vecT'$ are $\vec{a}$-identical using the protocol in Figure ~\ref{fig:a-identical}, described
in Section ~\ref{subsec:proximity-ram}.
All that remains is to prove is that the operation sequence $\vec{o}$ is consistent with small RAMs $\Sbf$ and $\Sbf'$.
We check this using the argument in Appendix ~\ref{sec:poly-proto-ram-app}, which is obtained by compiling the
polynomial protocol for RAM in Appendix ~\ref{sec:poly-proto-ram} into an argument of knowledge in the AGM.
Specifically, the prover and the verifier set
$c_S = (c_a, c_v)$, $c_S'=(c_a, c_v')$ and $c_o = (c_\op, c_a, c_w)$, and execute the argument of knowledge for
showing $(c_S, c_o, c_S')\in \CLRAM$ (see Definition ~\ref{defn:committed-vram}). We provide the complete protocol
listing in Figure ~\ref{fig:complete-listing}. The protocol in Figure ~\ref{fig:complete-listing} assumes pre-computed parameters
for the tables $\vecT$ and $\vecT'$. The maintenance of these pre-computed parameters in the presence of updates
is detailed in Section ~\ref{sec:update-protocol}.

\begin{theorem}\label{thm:committed-ram}
The protocol in Figure~\ref{fig:complete-listing} is a succinct argument of knowledge for the relation $\CRAM$ in
the AGM, under the $Q$-DLOG assumption for the bilinear group $(\F,\Gone,\Gtwo,\GT,e,g_1,g_2)$.
\end{theorem}

\begin{figure}[t!]
    \begin{mdframed}

        \underline{Setup $(1^\secp,N,m, \vecT, \vecT')$}:
        \begin{itemize}[leftmargin=1em]
            \item $\srs = (\{\gone{\tau^i}\}_{i=0}^N, \{\gtwo{\tau^i}\}_{i=0}^N)$ for $\tau\gets \F$.
            \item $W_2^i=\gtwo{\vpolyN(X)/(X-\xi^i)}$, $i\in [N]$ (needed by prover).
            \item $\gone{\vpolyN(X)}, \gtwo{\vpolyN(X)}$ (known to both $\prover$ and $\verifier$).
        \end{itemize}

        \underline{Precompute $(\vecT, \vecT')$}:
        \begin{itemize}[leftmargin=1em]
            \item $W_1^i=\gtwo{(T(X)-T(\xi^i))/(X-\xi^i)}$, $i\in [N]$,
            \item ${W_1^i}'=\gtwo{(T^*(X) - T^*(\xi^i))/(X-\xi^i)}$, $i\in [N]$.
        \end{itemize}

        {\bf Common Input}: $\srs$, $c_T, c_T', c_\op, c_a, c_w\in \Gone$.\\
        {\bf Prover's Input}: Vectors $\vecT,\vecT',\vec{\op},\vec{a},\vec{w}$ and their encoding polynomials.\\

        {\bf Round 1}: Commit to sub RAMs.
        \begin{enumerate}[leftmargin=1em, label=\arabic*.]
            \item $\prover$ computes $\vec{v}=\vecT[\,\vec{a}\,]$, $\vec{v}'=\vecT'[\,\vec{a}\,]$ and the encoding
            polynomials $v(X)$ and $v^*(X)$.
            \item $\prover$ sends $c_v = \gone{v(X)}$, $c_v'=\gone{v^*(X)}$.
            \item $\verifier$ sends $\chi\gets \F$.
        \end{enumerate}

        {\bf Round 2}: Execute committed index lookup.
        \begin{enumerate}[leftmargin=1em, label=\arabic*.]
            \item $\prover$ and $\verifier$ compute $\hat{c}_T=c_T + \chi c_T'$, $\hat{c}_v=c_v + \chi c_v'$.
            \item $\prover$ computes $\hat{\vec{T}} = \vecT + \chi \vecT'$, $\hat{\vec{v}}= \vec{v} + \chi \vec{v}'$.
            \item $\prover$ and $\verifier$ execute committed index lookup argument in Fig ~\ref{fig:committed-index-lookup},
            with $(\hat{c}_T,c_a,\hat{c}_v)$ as the common input and $(\hat{\vec{T}}, \vec{a},\hat{\vec{v}})$ as prover's input.
        \end{enumerate}

        {\bf Round 3}: Prove RAMs are $\vec{a}$-identical.
        \begin{enumerate}[leftmargin=1em, label=\arabic*.]
            \item $\prover$ and $\verifier$ execute argument in Fig ~\ref{fig:a-identical} with common input
            $(c_T,c_T',c_a)$ and prover's input as $(\vecT,\vecT',\vec{a})$.
        \end{enumerate}

        {\bf Round 4}: Prove sub RAMs are memory-consistent with update.
        \begin{enumerate}[leftmargin=1em, label=\arabic*.]
        \item $\prover$ and $\verifier$ execute argument in Fig ~\ref{fig:covering-protocol}
        to check $(c_S, c_o, c_S')\in \CLRAM$ with $c_S = (c_a, c_v)$, $c_S'=(c_a, c_v')$ and
        $c_o=(c_\op,c_a,c_w)$.
        \item $\verifier$ accepts if all sub-protocols accept.
        \end{enumerate}
    \end{mdframed}
    \caption{Our batching-efficient RAM protocol}
    \label{fig:complete-listing}
\end{figure}

\section{Fast Lookups from Approximate Pre-Processing}\label{sec:update-protocol}


Recall that our lookup protocol in section 5.1 involves certain precomputations by the prover namely $W_1^i, W_2^i, W_3^i$. $W_2^i$ and $W_3^i$ do not depend on the table. However, $W_1^i$ depends on the lookup table and their values will change even if the table changes by a small amount. It is expensive to recompute all the $W_1^i$ for every small change in the table and this will affect the efficiency of our lookup protocol in the long run.\\\\
In this section, we show how to achieve efficient lookups even when the table is changing frequently, as long as the cumulative change in the table is small. \\
In particular, we show how the prover can compute $[Q(X)]_2=\gtwo{\frac{T(X)-T_I(X)}{Z_I(X)}}$ without computing all the $W_1^i$(thus minimizing the overhead).\\
The overhead(as long as the table doesn't change too much) will be much lower than the time needed for the lookup and so is very practical.

\subsection{The idea: Base + Cache approach}

The idea is that we do not compute $W_1^i$ after each change of the table. Instead, this expensive computation will be done periodically for all $i \in [N]$ after say $s \in \mathbb{Z}$ batches.
Let current table $\vecT$ can be represented as $\vecTbase + \vecTcache$ where the vector
$\vecTbase$ denotes the base table (with respect to which $W_1^i$ was last computed for all $i \in [N]$) and the vector $\vecTcache$ corresponds to the changes
that have happened to the base table since the last rebasing (rebasing denotes computation of all $W_1^i$)\\\\
Thus, there is an \textbf{online} phase which happens after every batch (which includes computation of $[Q(X)]_2$ among other things) and an \textbf{offline} phase which consists of the rebasing(this is all prover computations) \\\\




\noindent{\bf Offline Phase}: This computation is executed once after every $s$ rounds. Here, the prover updates the base vector $\vecTbase$ with the changes in the cache vector
$\vecTcache$ by setting $\vecTbase := \vecTbase + \vecTcache$ and simultaneously clears the cache vector by setting
$\vecTcache = 0$.\\
It computes the commitment of $T_b$ as well\\
It also re-computes the $\mathsf{KZG}$ opening proofs $[W_1^i(x)]_2$ for $i\in [N]$ where
$W_1^i(X) = (\Tbasepoly{X} - t_i)/(X-\xi^i)$. Here, $t_i=\Tbasepoly{\xi^i}$ are the coordinates
of the updated base vector $\vecTbase$.\\
As mentioned in section 5.1 this can be done in $O(N\log N)$ group and field operations.\\\\
\noindent{\bf Online Phase}:
The online phase happens for every batch because the purpose of this phase is to ensure that all the things needed for the current execution of the lookup protocol are available. We show how the prover computes the next table $T'$ from the current table $T$ and the new Cache vector from the old cache vector (by an inductive argument this suffices)
\begin{enumerate}[leftmargin=1em]
    \item Prover has the $T$ for the current round and the commitment $[T(x)]_1$ as well(because these are just the $T'$, $[T'(x)]_1$ of the last round)
    \item The $\vecTcache$ and $\vecTcache(X)$ is also updated to the start of the current round (contains information till previous round:$\vecT=\vecTbase+\vecTcache$)
    \item The prover updates the cache using the current batch: $\vec{T'}_{\mathsf{ch}}[i] = \vec{T}_{\mathsf{ch}}[i] + \Delta_i$ for $i\in I$ in $O(m)$ $\F$ operations
    \item Here $\Delta_i$ for all $i \in I$ is the change that will happen to $\vecT$ \textbf{during the current round}
    \item Prover computes the commitment to the new cache polynomial:
    $$[\vec{T'}_{\mathsf{ch}}(x)]_1=[\vec{T}_{\mathsf{ch}}(x)]_1+\sum_{i\in I}\Delta_i[\mu_i(x)]_1$$ in
    $O(m)$ $\Gone$ operations.
    \item Prover also gets ${T}'$ as ${T'}[i]=T_b[i]+ \vec{T'_{\text{ch}}}[i]$ using the $T_b$ and the latest cache
    \item Prover computes the commitment to the new table $\vecT'$: $[T'(x)]_1=[\Tbasepoly{x}]_1+[\vec{T'}_{\mathsf{ch}}(x)]_2$

    \item In addition, the other things (apart from $[Q(x)]_2$) needed for the current round of the lookup protocol are also computed by the prover as described in the lookup protocol in section 5.1 as it is just naive computation



\end{enumerate}

\subsection{Computation of $[Q(X)]_2$}
Clearly, it suffices to efficiently compute $[Q(x)]_2$ where $[Q(x)]_2=\gtwo{\frac{T(x)-T_I(x)}{Z_I(x)}}$. We have the information of $[\Tbasepoly{x}-\Tbasepoly{\xi^i}/(x-\xi^i)]_2$. For this, we have the following lemma:


\begin{lemma}\label{lem:approx-setup}
Let $N,\xi$ be as defined previously. Suppose we are given
$\kzg$ proofs $\{W_i\}_{i=1}^N$ with $W_i=\gtwo{\Tbasepoly{X} - \Tbasepoly{\xi^i}/(X-\xi^i)}$ and where
$\Tbasepoly{X}=\enc{T_{\mathsf{b}}}{\setN}$ encodes a vector $\vecTbase\in \F^N$.
Let $I \subset [N]$.
Then,
there exists an algorithm to compute $\kzg$ multi-opening proof
$[Q(X)]_2=\gtwo{(T(x) - T_I(x)/Z_I(x)}$ for encoding $T(X)=\enc{T}{\setN}$ of vector $\vecT\in \F^N$ using $O((\delta + |I|) \log^2 (\delta + |I|))$ $\F$-operations and $O(\delta + |I|)$ $\Gtwo$-operations.
Here, $\delta$ denotes the hamming distance
between vectors $\vecTbase$ and $\vecT$.
\end{lemma}

\begin{proof}
    We know that $\vecT=\vecTbase+\vecTcache$ and that $T(X)=\Tbasepoly{X}+\Tcachepoly{X}$\\

    Let $K \subset [N]$ be a set which contains $I$ and all those indices $j$ for which $\vecTcache[j] \neq 0$. For these $j$, let $\vecTcache[j]=\Delta t_j$. Then $\Tcachepoly{X}=\sum_{j\in K}\Delta t_j\mu_j(X)$.\\\\
    By definition of $K$, $|K|\leq \delta +|I|$. So, we need to bound $\Gtwo$ operations by $O(|K|)$ and field operations by $O(|K| \log^2|K|)$\\
    First of all note that:
    \begin{align}\label{eq:Q2-upd}
    Q_2(X) &= \sum_{i\in \setind}\frac{1}{z_I'(\xi^i)}\left(\frac{\Tbasepoly{X} - \Tbasepoly{\xi^i}}{X-\xi^i}\right)
    + \sum_{i\in \setind}\frac{1}{z_I'(\xi^i)}\left(\frac{\Tcachepoly{X} - \Tcachepoly{\xi^i}}{X-\xi^i}\right)
    \end{align}

    From the above we can write $\elttwo{Q_2(x)}=\gtwo{\Qbasepoly{x}}+\gtwo{\Qcachepoly{x}}$ where
    \begin{gather*}
        \Qbasepoly{X}=\sum_{i\in \setind}(z_I'(\xi^i))^{-1} (\Tbasepoly{X}-\Tbasepoly{\xi^i})/(X-\xi^i) \\
        \Qcachepoly{X}=\sum_{i\in \setind}(z_I'(\xi^i))^{-1} (\Tcachepoly{X}-\Tcachepoly{\xi^i})/(X-\xi^i)
    \end{gather*}
    We can compute
    $\elttwo{\Qbasepoly{x}}$ from the pre-computed KZG openings of $\Tbasepoly(X)$ at points $\xi^i,i\in I$ using $O(|I|)$ group operations and
    $O(|I|\log^2 |I|)$ field operations(as is done in Caulk/Caulk+).\\\\

    \textbf{Thus, it suffices to describe the computation for $\elttwo{\Qcachepoly{x}}$. }\\
    Using
    $\Tcachepoly{X}=\sum_{j\in K}\Delta t_j\mu_j(X)$ and defining constants $c_i=(1/z_I'(\xi^i))$ for $i\in I$,
    we write $\Qcachepoly{X}$ as:
    \begin{align*}
        \Qcachepoly{X} &= \sum_{i\in \setind} c_i\sum_{j\in K} \Delta t_j\frac{\mu_j(X)-\mu_j(\xi^i)}{X-\xi^i} \\
        &= \sum_{i\in \setind} c_i\Delta t_i\frac{\mu_i(X) - 1}{X-\xi^i} + \sum_{i\in \setind}\sum_{j\in K\setminus\{i\}}c_i\Delta t_j\frac{\mu_j(X)}{X-\xi^i}
    \end{align*}
    Now, we can write $\elttwo{\Qcachepoly{x}}=\elttwo{\Qcachepolyone{x}} + \elttwo{\Qcachepolytwo{x}}$ where:
    \begin{gather*}
        \Qcachepolyone{X}=\sum_{i\in \setind}c_i\Delta t_i\frac{\mu_i(X)-1}{X-\xi^i} \\
        \Qcachepolytwo{X}=\sum_{i\in \setind}\sum_{j\in K\setminus \{i\}} c_i\Delta t_j\frac{\mu_j(X)}{X-\xi^i}
    \end{gather*}

    The term $\elttwo{\Qcachepolyone{x}}$ can be computed using $O(I)$ group operations by pre-computing the
    \textsf{KZG} openings of polynomials $\mu_i(X)$ at $\xi^i$ for $i\in [N]$. That is by precomputing $[\frac{\mu_i(X)-1}{X-\xi^i}]_2$. This requires just $N$ more precomputations and can be done along with the other precomputations which are done in the lookup protocol\\

    \textbf{Thus, it suffices to describe the computation for $\Qcachepolytwo{X}$}:
    \begin{align}\label{eq:Qcachepoly2}
    \Qcachepolytwo{X} &= \sum_{i\in \setind}c_i\sum_{j\in K\setminus \{i\}} \Delta t_j\mu_j(X)/(X-\xi^i) \nonumber \\
    &= \sum_{i\in\setind}c_i\sum_{j\in K\setminus \{i\}}\frac{\Delta t_j}{Z_{\nroots}'(\xi^j)} \frac{Z_{\nroots}(X)}{(X-\xi^i)(X-\xi^j)} \nonumber \\
    \intertext{This used definition of $\mu_j(X)$}
    \intertext{ Now, using $Z_\nroots'(\xi^j)=N\xi^{-j}$ and partial fraction decomposition}
    &\Qcachepolytwo{X}= N^{-1}\sum_{i\in\setind}c_i\sum_{j\in K\setminus \{i\}}\frac{\xi^j\Delta t_j}{\xi^i-\xi^j}
    \left(\frac{Z_\nroots(X)}{X-\xi^i} - \frac{Z_\nroots(X)}{X-\xi^j}\right) \nonumber \\
    &\Qcachepolytwo{X}= N^{-1}\sum_{i\in\setind}\left(c_i\cdot \sum_{j\in K\setminus \{i\}} \frac{\xi^j\Delta t_j}{\xi^i-\xi^j}\right)\frac{Z_\nroots(X)}{X-\xi^i} \nonumber \\
    &\qquad + \sum_{j\in K}\left(\xi^j\Delta t_j\cdot \sum_{i\in \setind\setminus \{j\}}\frac{c_i}{\xi^j-\xi^i}\right)\frac{Z_{\nroots}(X)}{X-\xi^j}
    \end{align}
    In this last equality, the first term is just the first term of the distributive property in finite fields.\\
    The second term is just the second term of the distributive property in finite fields except that the order of the sums is reversed. This follows from the following fact \\

    \begin{fact}
        $\sum_{i \in I} \sum_{j \in K \setminus \{i\}} f(i,j)=\sum_{j \in K} \sum_{i \in I \setminus \{j\}} f(i,j) $
    \end{fact}

    In the above equation \eqref{eq:Qcachepoly2}, let us define:
    \begin{gather*}
        a_i = \sum_{j\in K\setminus \{i\}}\frac{\xi^j\Delta t_j}{\xi^i-\xi^j}, \text{ for } i\in \setind \\
        b_j=  \sum_{i\in \setind\setminus \{j\}}\frac{c_i}{\xi^j - \xi^i}, \text{ for } j\in K \\
        W_3^i(X) = \frac{Z_\nroots(X)}{X-\xi^i}, \text{ for } i\in [N]
    \end{gather*}


    Then, we have:
    \begin{equation}\label{eq:Qcachepoly2commit}
    \elttwo{\Qcachepolytwo{x}} = N^{-1}\left(\sum_{i\in\setind}c_ia_i\elttwo{W_3^i(x)} + \sum_{j\in K}\xi^j \Delta t_j b_j \elttwo{W_3^j(x)}\right)
    \end{equation}

    $c_i$ for all $i \in \setind$ are determined easily by evaluating $Z_{\setind}'(X)$ on $H_{\setind}$ and $[W_3^i(X)]_2$ for all $i \in [N]$ have been computed in the lookup protocol

    So, given $\{a_i\}_{i\in I}, \{b_j\}_{j\in K}$, $\elttwo{\Qcachepolytwo{x}}$ can be computed in $O(|\setind|+|K|)\, \Gtwo$ operations.\\
    But $|\setind|+|K|<2|K|$. So, $\Gtwo$ operations are $O(|K|)$ as required.\\\\

    It remains to bound the number of field operations needed to compute the scalar multipliers $a_i, i \in \setind$ and $b_j, j \in K$. \\

    Let us consider $a_i$ first:\\
    Note that:

    $$ a_i = -\sum_{j\in K\setminus \{i\}}\Delta t_j + \xi^i\sum_{j\in K\setminus \{i\}}\frac{\Delta t_j}{\xi^i-\xi^j} $$
    This is because:
    $$a_i+\sum_{j\in K\setminus \{i\}}\Delta t_j= \sum_{j \in K \setminus \{i\}}\frac{\xi^j\Delta t_j}{\xi^i-\xi^j}+\Delta t_j$$
    $$=\sum_{j \in K \setminus \{i\}}\frac{\xi^i\Delta t_j}{\xi^i-\xi^j} = \xi^i\sum_{j\in K\setminus \{i\}}\frac{\Delta t_j}{\xi^i-\xi^j}$$
    Now, define $\Delta T=\sum_{j\in K}\Delta t_j$\\

    Here computing $\Delta T$ is a one time computation (per batch). It can be computed from the knowledge of $T_{\text{ch}}$ in the online phase.
    We have:
    $$ a_i = -\Delta T + \Delta t_i + \xi^i\sum_{j\in K\setminus\{i\}}\frac{\Delta t_j}{\xi^i-\xi^j} $$
    Suppose we get $\sum_{j\in K\setminus\{i\}}\frac{\Delta t_j}{\xi^i-\xi_j}$ for all $i \in I$ efficiently. Then $a_i$ for all $i \in I$ can be obtained in $O(|I|)$ field operations. \\
    \textbf{Thus, to get $a_i$ for all $i \in I$ it suffices to describe the computation of $e_i=\sum_{j\in K\setminus\{i\}}\frac{\Delta t_j}{\xi^i-\xi^j}$ for all $i \in I$}\\\\
    Our requirement is now to bound the number of field operations for $e_i$ and for $b_j$. For this, we invoke the following lemma with the proof in the appendix.

    \begin{lemma}\label{lem:sum computation}
    Let $e_i$ for all $i \in \setind$ and $b_j$ for all $j \in K$ be as described above.
    Then, $e_i$ for all $i \in I$ and $b_j$ for all $j \in K$ can be computed in $O(|K|\log^2|K|)\, \mathbb{F}$ operations
    \end{lemma}

    From the lemma \ref{lem:sum computation}, we have shown that the field operations needed to get $e_i$ and $b_j$ and thus $\elttwo{\Qcachepolytwo{x}}$ is $O(|K|\log^2|K|)$. \\

    This completes the proof of Lemma \ref{lem:approx-setup}.
\end{proof}

\subsection{Amortized Analysis of the update protocol}
Recall that we were able to get $[Q(X)]_2$ in $O(|K|)$ group operations and $O(|K|\log^2|K|)$ field operations. \\
For concrete analysis, let $s$ be the period after which the rebasing takes place. Also, the lookup happens at maximum of $m$ indices during a single batch. Thus, $|I|\leq m$.\\
This gives an upper bound on $\delta$, that is $ms$ and an upper bound on $K$, that is $ms+m=m(s+1)$.\\

Clearly $O(K)=O(ms)$ and $O(|K|\log^2|K|)=O(ms \log^2(ms))$ so group operations are $O(ms)$ and field operations are $O(ms\log^2(ms))$ \\

Moreover, after every $s$ batches, the rebasing(offline phase) is done which we know takes $O(N\log N)$ group and field operations.\\

So, the amortized number of operations for the offline and online phase in total is:
$O(ms \log^2(ms)+\frac{N\log N}{s})$ $\mathbb{F}$ operations and $O(ms +\frac{N\log N}{s})$ $\Gtwo$ operations\\

The value of $s$ which minimizes the group operations is $\sqrt{\frac{N}{m}}$. For this value of $s$:\\
\textbf{The amortized group operations needed are $\tilde{O}(\sqrt{mN})$}\\
\textbf{The amortized field operations needed are also $\tilde{O}(\sqrt{mN})$}\\
Here $\tilde{O}$ denotes that the polylog factors have been neglected\\\\

