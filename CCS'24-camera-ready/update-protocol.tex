
In this section, we provide details of the algorithm to construct lookup argument for a table $\vecT$,
using pre-computed parameters of a table  which is a small hamming distance away. The dependence on
pre-computed parameters in several recent lookup arguments such as ~\cite{CCS:ZBKMNS22,EPRINT:PosKat22,EPRINT:ZGKMR22,EPRINT:EagFioGab22}
stems from the need to compute an encoded quotient of the form:
\begin{equation}\label{eq:enc-quotient}
\gany{Q} = \sum_{i\in I}c_i\gany{\frac{T(X) - T(\xi^i)}{X-\xi^i}}
\end{equation}
for some $O(m)$ sized set $I$. The quotient in Equation ~\eqref{eq:enc-quotient} can be computed in $O(m)$ cost
when the quotients $\gany{(T(X)-T(\xi^i))/(X-\xi^i)}$ are available for all $i\in [N]$. In this section
we exhibit an algorithm which computes the above with $O((m+\delta)\log^2(m+\delta))$ cost, given access to
similar quotients for a table at hamming distance $\delta$ from $\vecT$. We now describe our approach.
%Recall that our lookup protocol in section 5.1 involves certain precomputations by the prover namely $W_1^i, W_2^i, W_3^i$. $W_2^i$ and $W_3^i$ do not depend on the table. However, $W_1^i$ depends on the lookup table and their values will change even if the table changes by a small amount. It is expensive to recompute all the $W_1^i$ for every small change in the table and this will affect the efficiency of our lookup protocol in the long run.\\\\
%In this section, we show how to achieve efficient lookups even when the table is changing frequently, as long as the cumulative change in the table is small. \\
%In particular, we show how the prover can compute $[Q(X)]_2=\gtwo{\frac{T(X)-T_I(X)}{Z_I(X)}}$ without computing all the $W_1^i$(thus minimizing the overhead).\\
%The overhead(as long as the table doesn't change too much) will be much lower than the time needed for the lookup and so is very practical.

\subsection{Base + Cache approach}\label{subsec:base-cache}
The key idea we employ is to express the current table $\vecT\in \F^N$ as $\vecTbase + \vecTcache$, where $\vecTbase$ is the table for which we assume that
the encoded quotients are available (via the $O(N\log N)$ computation), and $\vecTcache$ captures the changes to the table since. We will periodically update (say
after $s$ batch updates) $\vecTbase$ to current table state, and re-compute all the quotients (we call it the {\em offline} phase).
We will revisit the question on choosing $s$ optimally later. Let $I\subseteq [N]$ denote the set of indices in the current batch of $m$ lookups. The {\em online}
phase of our proof generation involves computing the sum in Equation \eqref{eq:enc-quotient} for the table $\vecT$.
%that we do not compute $W_1^i$ after each change of the table. Instead, this expensive computation will be done periodically for all $i \in [N]$ after say $s \in \mathbb{Z}$ batches.
%Let current table $\vecT$ can be represented as $\vecTbase + \vecTcache$ where the vector
%$\vecTbase$ denotes the base table (with respect to which $W_1^i$ was last computed for all $i \in [N]$) and the vector $\vecTcache$ corresponds to the changes
%that have happened to the base table since the last rebasing (rebasing denotes computation of all $W_1^i$)\\\\
%Thus, there is an \textbf{online} phase which happens after every batch (which includes computation of $[Q(X)]_2$ among other things) and an \textbf{offline} phase which consists of the rebasing(this is all prover computations) \\\\
\begin{comment}
    \noindent{\bf Offline Phase}: This computation is executed once after every $s$ rounds. Here, the prover updates the base vector $\vecTbase$ with the changes in the cache vector
    $\vecTcache$ by setting $\vecTbase := \vecTbase + \vecTcache$ and simultaneously clears the cache vector by setting
    $\vecTcache = 0$.\\
    It computes the commitment of $T_b$ as well\\
    It also re-computes the $\mathsf{KZG}$ opening proofs $[W_1^i(X)]_2$ for $i\in [N]$ where
    $W_1^i(X) = (\Tbasepoly{X} - t_i)/(X-\xi^i)$. Here, $t_i=\Tbasepoly{\xi^i}$ are the coordinates
    of the updated base vector $\vecTbase$.\\
    As mentioned in section 5.1 this can be done in $O(N\log N)$ group and field operations.\\\\
    \noindent{\bf Online Phase}:
    The online phase happens for every batch because the purpose of this phase is to ensure that all the things needed for the current execution of the lookup protocol are available. We show how the prover computes the next table $T'$ from the current table $T$ and the new Cache vector from the old cache vector (by an inductive argument this suffices)
    \begin{enumerate}[leftmargin=1em]
        \item Prover has the $T$ for the current round and the commitment $[T(X)]_1$ as well(because these are just the $T'$, $[T'(x)]_1$ of the last round)
        \item The $\vecTcache$ and $\vecTcache(X)$ is also updated to the start of the current round (contains information till previous round:$\vecT=\vecTbase+\vecTcache$)
        \item The prover updates the cache using the current batch: $\vec{T'}_{\mathsf{ch}}[i] = \vec{T}_{\mathsf{ch}}[i] + \Delta_i$ for $i\in I$ in $O(m)$ $\F$ operations
        \item Here $\Delta_i$ for all $i \in I$ is the change that will happen to $\vecT$ \textbf{during the current round}
        \item Prover computes the commitment to the new cache polynomial:
        $$[\vec{T'}_{\mathsf{ch}}(X)]_1=[\vec{T}_{\mathsf{ch}}(X)]_1+\sum_{i\in I}\Delta_i[\mu_i(X)]_1$$ in
        $O(m)$ $\Gone$ operations.
        \item Prover also gets ${T}'$ as ${T'}[i]=T_b[i]+ \vec{T'_{\text{ch}}}[i]$ using the $T_b$ and the latest cache
        \item Prover computes the commitment to the new table $\vecT'$: $[T'(X)]_1=[\Tbasepoly{X}]_1+[\vec{T'}_{\mathsf{ch}}(X)]_1$

        \item In addition, the other things (apart from $[Q(X)]_2$) needed for the current round of the lookup protocol are also computed by the prover as described in the lookup protocol in section 5.1 as it is just naive computation

    \end{enumerate}
    \subsection{Computation of $[Q(X)]_2$}
    Clearly, it suffices to efficiently compute $[Q(X)]_2$ where $[Q(X)]_2=\gtwo{\frac{T(X)-T_I(X)}{Z_I(X)}}$. We have the information of $[\Tbasepoly{X}-\Tbasepoly{\xi^i}/(X-\xi^i)]_2$. For this, we have the following lemma:
\end{comment}
The following Theorem determines the efficiency of the online phase of our prover.
\begin{theorem}\label{thm:approx-setup}
Let $N,\xi$ be as defined previously. Given
$\kzg$ proofs $\{W_i\}_{i=1}^N$ with $W_i=\gany{\Tbasepoly{X} - \Tbasepoly{\xi^i}/(X-\xi^i)}$, where
$\Tbasepoly{X}=\enc{T_{\mathsf{b}}}{\setN}$ encodes a vector $\vecTbase\in \F^N$, for any $I\subseteq [N]$, there exists an algorithm to compute $\gany{Q}$ as given in Equation ~\eqref{eq:enc-quotient}
for polynomial $T(X)=\enc{T}{\setN}$ encoding the vector $\vecT\in \F^N$ using $O((\delta + |I|) \log^2 (\delta + |I|))$ $\F$-operations
and $O(\delta + |I|)$ $\mathbb{G}$-operations. Here, $\delta$ denotes the hamming distance
between vectors $\vecTbase$ and $\vecT$.
\end{theorem}
\begin{proof}
    Let $\vecT=\vecTbase+\vecTcache$ and thus $T(X)=\Tbasepoly{X}+\Tcachepoly{X}$.
    Define $K=I\cup \{j\in [N]: \vecTcache[\,j\,]\neq 0\}$ as a set which captures the indices where the current table $\vecT$ differs from the base $\vecTbase$,
    where we explicitly also include the lookup indices $I$ in $K$. For $j\in K$, let $\vecTcache[j]=\Delta t_j$. Then $\Tcachepoly{X}=\sum_{j\in K}\Delta t_j\mu_j(X)$.
    %By definition of $K$, $|K|\leq \delta +|I|$. So, we need to bound $\Gtwo$ operations by $O(|K|)$ and field operations by $O(|K| \log^2|K|)$\\
    %First of all note that:
    We write the quotient $Q(X)$ as:
        {
    \begin{equation*}
        \begin{aligned}
            Q(X) = \sum_{i\in \setind}c_i\left(\frac{\Tbasepoly{X} - \Tbasepoly{\xi^i}}{X-\xi^i}\right)
            + \sum_{i\in \setind}c_i\left(\frac{\Tcachepoly{X} - \Tcachepoly{\xi^i}}{X-\xi^i}\right)
        \end{aligned}
        %\label{eq:Q2-upd}
    \end{equation*}
    }
    From above, we have $\gany{Q(x)}=\gany{\Qbasepoly{x}}+\gany{\Qcachepoly{x}}$ where
    \begin{gather*}
        \Qbasepoly{X}=\sum_{i\in \setind}c_i (\Tbasepoly{X}-\Tbasepoly{\xi^i})/(X-\xi^i) \\
        \Qcachepoly{X}=\sum_{i\in \setind}c_i(\Tcachepoly{X}-\Tcachepoly{\xi^i})/(X-\xi^i)
    \end{gather*}
    We can compute
    $\gany{\Qbasepoly{X}}$ from the pre-computed KZG openings of $\Tbasepoly{X}$ at points $\xi^i,i\in I$ using $O(|I|)$ group operations and
    $O(|I|\log^2 |I|)$ field operations. Therefore, it suffices to compute $\gany{\Qcachepoly{X}}$ efficiently.
    %\textbf{Thus, it suffices to describe the computation for $\elttwo{\Qcachepoly{X}}$. }\\
    Using $\Tcachepoly{X}=\sum_{j\in K}\Delta t_j\mu_j(X)$
    we write $\Qcachepoly{X}$ as linear combination of table-independent polynomials:
    \begin{align*}
        \Qcachepoly{X} &= \sum_{i\in \setind} c_i\sum_{j\in K} \Delta t_j\frac{\mu_j(X)-\mu_j(\xi^i)}{X-\xi^i} \\
        &= \sum_{i\in \setind} c_i\Delta t_i\frac{\mu_i(X) - 1}{X-\xi^i} + \sum_{i\in \setind}\sum_{j\in K\setminus\{i\}}c_i\Delta t_j\frac{\mu_j(X)}{X-\xi^i}
    \end{align*}
    Now, we can write $\gany{\Qcachepoly{X}}=\elany{\Qcachepolyone{X}} + \elany{\Qcachepolytwo{X}}$ where:
        {
    \begin{gather*}
        \Qcachepolyone{X}=\sum_{i\in \setind}c_i\Delta t_i\frac{\mu_i(X)-1}{X-\xi^i},\,
        \Qcachepolytwo{X}=\sum_{i\in \setind}\sum_{j\in K\setminus \{i\}} c_i\Delta t_j\frac{\mu_j(X)}{X-\xi^i}
    \end{gather*}
    }
    The term $\gany{\Qcachepolyone{X}}$ can be computed using $O(|I|)$ group operations by augmenting the setup with pre-computed
    $\kzg$ opening proofs of polynomials $\mu_i(X)$ at $\xi^i$ for $i\in [N]$. This adds $O(N)$ to the setup parameters, while the computation
    can be done in $O(N\log N)$ time with methods similar to existing pre-computed parameters. This eventually leaves us with $\elany{\Qcachepolytwo{X}}$.
    %That is by precomputing $[\frac{\mu_i(X)-1}{X-\xi^i}]_2$. This requires just $N$ more precomputations and can be done along with the other precomputations which are done in the lookup protocol\\
    Next, we synthesize the polynomial $\Qcachepolytwo{X}$ in a form that reduces group operations required to compute its encoding.
    %\textbf{Thus, it suffices to describe the computation for $\elttwo{\Qcachepolytwo{X}}$}:
    \begin{align}\label{eq:Qcachepoly2}
    &\Qcachepolytwo{X} = \sum_{i\in \setind}c_i\sum_{j\in K\setminus \{i\}} \Delta t_j\mu_j(X)/(X-\xi^i) \nonumber \\
    &\quad = \sum_{i\in\setind}c_i\sum_{j\in K\setminus \{i\}}\frac{\Delta t_j}{Z_{\nroots}'(\xi^j)} \frac{Z_{\nroots}(X)}{(X-\xi^i)(X-\xi^j)} \nonumber \\
    %\intertext{Above, we expanded $\mu_j(X)$. Now using $Z_\nroots'(\xi^j)=N\xi^{-j}$ and using partial fractions}
    &\quad = N^{-1}\sum_{i\in\setind}c_i\sum_{j\in K\setminus \{i\}}\frac{\xi^j\Delta t_j}{\xi^i-\xi^j}
    \left(\frac{Z_\nroots(X)}{X-\xi^i} - \frac{Z_\nroots(X)}{X-\xi^j}\right) \nonumber \\
    &\quad = N^{-1}\sum_{i\in\setind}\left(c_i\cdot \sum_{j\in K\setminus \{i\}} \frac{\xi^j\Delta t_j}{\xi^i-\xi^j}\right)\frac{Z_\nroots(X)}{X-\xi^i} \nonumber \\
    &\qquad + N^{-1} \sum_{j\in K}\left(\xi^j\Delta t_j\cdot \sum_{i\in \setind\setminus \{j\}}\frac{c_i}{\xi^j-\xi^i}\right)\frac{Z_{\nroots}(X)}{X-\xi^j}
    \end{align}
    In the first step, we substituted $\mu_j(X)$, while in the final step we re-arranged the summation to accumulate the scalar factor for
    each distinct polynomial of the form $\vpolyN(X)/(X-\xi^i)$. Define scalars $a_i$, $i\in I$ and $b_j$, $j\in K$ as below:
    %this last equality, the first term is just the first term of the distributive property in finite fields.\\
    %The second term is just the second term of the distributive property in finite fields except that the order of the sums is reversed. This follows from the following fact \\
    %\begin{fact}
    %    $\sum_{i \in I} \sum_{j \in K \setminus \{i\}} f(i,j)=\sum_{j \in K} \sum_{i \in I \setminus \{j\}} f(i,j) $
    %\end{fact}
    %In the above equation \eqref{eq:Qcachepoly2}, let us define:
    \begin{gather}\label{eq:scalars1}
    a_i = \sum_{j\in K\setminus \{i\}}\frac{\xi^j\Delta t_j}{\xi^i-\xi^j}, i\in \setind\quad
    b_j=  \sum_{i\in \setind\setminus \{j\}}\frac{c_i}{\xi^j - \xi^i}, j\in K
        %W_3^i(X) = \frac{Z_\nroots(X)}{X-\xi^i}, \text{ for } i\in [N]
    \end{gather}
    Now, define
    $W_3^j:=\gany{\vpolyN(X)/(X-\xi^j)}$. We see that $W_3^j$ is just the KZG opening proof of the polynomial $\vpolyN(X)$ evaluated at $\xi^j$ for $j \in [N]$. These can be precomputed one time and it adds $O(N)$ to the setup parameters and the computation can be done in $O(N\log N)$ time.\\ Now, we see that   $\elany{\Qcachepolytwo{X}}$ can be written as linear combination of $O(|K|+|I|)$ group elements.
    \begin{equation}\label{eq:Qcachepoly2commit}
    \gany{\Qcachepolytwo{X}} = N^{-1}\left(\sum_{i\in\setind}(c_ia_i)\cdot W_3^i + \sum_{j\in K}(\xi^j \Delta t_j b_j)\cdot W_3^j\right)
    \end{equation}
    Now, $c_i$ are known constants depending on the specific lookup scheme. So, given $\{a_i\}_{i\in I}, \{b_j\}_{j\in K}$, $\gany{\Qcachepolytwo{X}}$ can be computed in $O(|\setind|+|K|)$ group operations.
    While we have diligently reduced the group operations, we still seem to need $O(|I||K|)=O(m\delta)$ field operations. We clearly need better than
    naive way of computing the scalars in \eqref{eq:scalars1} to obtain additive overhead in $\delta$. This is what we consider next.
    %Routine calculation shows that we can write $a_i$ as:
    %\begin{equation*}
    % a_i = -\Delta T + \Delta t_i + \xi^i\sum_{j\in K\setminus\{i\}}\frac{\Delta t_j}{\xi^i-\xi^j}, i\in I
    %  \end{equation*}
    % where in the above, we have $\Delta T=\sum_{j\in K}\Delta t_j$.
    %a_i = -\sum_{j\in K\setminus \{i\}}\Delta t_j + \xi^i\sum_{j\in K\setminus \{i\}}\frac{\Delta t_j}{\xi^i-\xi^j} $$
    %This is because:
    %$$a_i+\sum_{j\in K\setminus \{i\}}\Delta t_j= \sum_{j \in K \setminus \{i\}}\frac{\xi^j\Delta t_j}{\xi^i-\xi^j}+\Delta t_j$$
    %$$=\sum_{j \in K \setminus \{i\}}\frac{\xi^i\Delta t_j}{\xi^i-\xi^j} = \xi^i\sum_{j\in K\setminus \{i\}}\frac{\Delta t_j}{\xi^i-\xi^j}$$
    %Now, define $\Delta T=\sum_{j\in K}\Delta t_j$\\
	 %Here computing $\Delta T$ is a one time computation (per batch). It can be computed from the knowledge of $T_{\text{ch}}$ in the online phase.
    %We have:
    %$$  $$
    %Suppose we get $\sum_{j\in K\setminus\{i\}}\frac{\Delta t_j}{\xi^i-\xi_j}$ for all $i \in I$ efficiently. Then $a_i$ for all $i \in I$ can be obtained in $O(|I|)$ field operations. \\
    %\textbf{Thus, to get $a_i$ for all $i \in I$ it suffices to describe the computation of $e_i=\sum_{j\in K\setminus\{i\}}\frac{\Delta t_j}{\xi^i-\xi^j}$ for all $i \in I$}\\\\
    Let $d_j:=\xi^j \Delta t_j$. Then we have from Eq ~\eqref{eq:scalars1}:\\
    \begin{gather}\label{eq:scalars}
    a_i = \sum_{j\in K\setminus \{i\}}\frac{d_j}{\xi^i-\xi^j}, i\in \setind\quad
    b_j=  \sum_{i\in \setind\setminus \{j\}}\frac{c_i}{\xi^j - \xi^i}, j\in K
        %W_3^i(X) = \frac{Z_\nroots(X)}{X-\xi^i}, \text{ for } i\in [N]
    \end{gather}
    So, to compute $a_i$ and $b_j$, it suffices to compute {\em reciprocal sums} efficiently. Our next lemma claims that
    such reciprocal sums can be computed efficiently.
    
    %Our requirement is now to bound the number of field operations for $e_i$ and for $b_j$. For this, we invoke the following lemma with the proof in the appendix.
    \begin{lemma}\label{lem:sum-computation}
    Let $I\subset K\subset [N]$ and let $a_i$ for all $i \in \setind$ and $b_j$ for all $j \in K$ be as described above.
    Then, $a_i$ for all $i \in I$ and $b_j$ for all $j \in K$ can be computed in $O(|K|\log^2|K|)\, \mathbb{F}$ operations.
    \end{lemma}
    \begin{proof}[Proof-Sketch]
        We provide a very high-level overview of the proof idea for $a_i$.
        First, we mention that the special case of the lemma when $d_j=1$ for all $j\in K$ admits an efficient computation due to the following identity
        proved in Lemma D.1 of the full version of the paper~\cite{full-ver}. For $Z_K(X)=\prod_{i\in K}(X-\xi^i)$, we have
        \begin{equation*}
            \frac{Z_K''(\xi^i)}{Z_K'(\xi^i)} = 2\sum_{j\in K\setminus \{i\}}\frac{1}{\xi^i-\xi^j}.
        \end{equation*}
        The polynomial $Z_K$ can be computed in $O(|K|\log^2|K|)$ and its first two
        derivatives can also be evaluated on the set $\{\xi^i: i\in I\}$ with the same complexity. However, to deal with arbitrary values of $d_j$ we
        need more ingenuity. %We will {\em imagine} $d_j$ to be $p(\xi^j)$ for some polynomial $p(X)$, such that $p(\xi^j)=0$ for $j\not\in K$.
        %We will not compute such a polynomial $p$, as it has degree $O(N)$, but view it as an ``oracle'' which we can hopefully query at the points we need. 
        We defer the detailed proof to the full version of the paper~\cite{full-ver}. 
%        Then it can be seen that $a_i=g_i(\xi^i) - r_i(\xi^i)$ for rational functions $g_i(X)$ and $r_i(X)$ defined by:
%        \begin{align}\label{eq:rat-fun-f}
%        g_i(X) =\sum_{j\in [N]\setminus i}\frac{p(X)}{X-\xi^j},\quad
%        r_i(X) =\sum_{j\in [N]\setminus i} \frac{p(X)-p(\xi^j)}{X-\xi^j}
%        \end{align}
%        Now, $g_i(\xi^i)$ for $i\in I$ turns out to be (using the special case above):
%        $$p(\xi^i)\sum_{j\in K\setminus \{i\}} 1/(\xi^i-\xi^j)=d_i(Z_K''(\xi^i)/Z_K'(\xi^i))/2$$
%        Defining $u(X,Y)=(p(X)-p(Y))/(X-Y)$, we can write $r_i(\xi^i)$ as:
%        \begin{equation}
%            r_i(X) = \sum_{j\in [N]}u(X,\xi^j) - u(X,\xi^i)
%        \end{equation}
%        Observe that $u(X,X)=p'(X)$ and so $u(X,X)$ gives the formal derivative of polynomial $p(X)$. We get
%        $r_i(\xi^i)=r(\xi^i)-p'(\xi^i)$ for all $i\in I$, where $r(X)=\sum_{j\in [N]}u(X,\xi^j)$. Fortunately,
%        $r(X)$ is simply $Nu(X,0)=N(p(X) - p(0))/X$, a fact that follows from uni-variate sum-check. The problem
%        thus reduces to being able to compute derivatives $p'(\xi^i)$ for $i\in I$ and the value $p(0)$. Before
%        concluding the proof-sketch, we briefly highlight the structure of the polynomial $p(X)$. Since $p(X)$
%        vanishes for $p(\xi^i)$ for $i\not\in K$, it can we written as the product $\widehat{Z}_K(X)q(X)$ where
%        $\widehat{Z}_K$ is the vanishing polynomial of ``complementary'' roots of unity $\{\xi^i: i\not\in K\}$
%        and $q$ is a low-degree ($<K$) polynomial. Assuming we can interpolate $q(X)$, we can write:
%        \[ p'(\xi^i) = \widehat{Z}_K(\xi^i)q'(\xi^i) + \widehat{Z}_K'(\xi^i)q(\xi^i) \]
%        In the above expression, we require evaluations of high-degree polynomials $\widehat{Z}_K(X)$ and $\widehat{Z}_K'(X)$
%        at $\xi^i$, $i\in I$. This is discussed in Lemma D.3 and other related lemmas in Appendix D.2 of full version of the paper~\cite{full-ver},
%        and motivates the at times tedious algebra there. We conclude the proof-sketch, deferring the missing details to the full
%        proof in Appendix D.2 of full version of the paper~\cite{full-ver}.
	\end{proof}
    From Lemma ~\ref{lem:sum-computation}, we conclude that the scalars $a_i,i\in I$ and $b_j, j\in K$ can be computed in
    time $O(|K|\log^2 |K|)$, which proves the bound in Theorem ~\ref{thm:approx-setup}.
    %have shown that the field operations needed to get $e_i$ and $b_j$ and thus $\elttwo{\Qcachepolytwo{X}}$ is $O(|K|\log^2|K|)$. \\

    %This completes the proof of Lemma \ref{lem:approx-setup}.
\end{proof}

\subsection{Amortized Sublinear Batching}\label{sec:amortization}
We now return to the question of how frequently should we run the offline phase to compute full parameters.
For concrete analysis, let $s$ be the period after which the rebasing takes place; i.e., after $s$ batches of $m$ operations
each, we set the base table $\vecTbase$ to the current table, setting $\vecTcache=\vec{0}$. At this point we also compute all
encoded quotients for $\vecTbase$ using the $O(N\log N)$ algorithm of ~\cite{EPRINT:FeiKho23}. Consider $\delta\leq ms$ as
the upper-bound on $\delta$, and distributing the cost of re-basing, the amortized overhead for the batch of $m$ operations is:
$O(ms \log^2(ms)+\frac{N\log N}{s})$ $\mathbb{F}$-operations and $O(ms +\frac{N\log N}{s})$ $\mathbb{G}$-operations. Ignoring
the logarithmic factors, the cost is minimized by setting $s\approx \sqrt{N/m}$, resulting in amortized prover overhead of
$\wt{O}(\sqrt{mN})$. We note that the above analysis considers the worst case scenario, where each update affects a distinct
position in the table. In settings, where frequency of updates is non-uniform accross positions in the table (e.g, in the
blockchain example, if bulk of transactions come from small number of clients), we may be able to defer the offline phase even longer.
Same is also true for settings where updates to the table are infrequent.