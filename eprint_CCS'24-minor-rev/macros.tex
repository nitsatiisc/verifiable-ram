%!TEX root = ../main.tex

%------------------------------
%  Do not add notation macros to this file 
%  use notation.tex instead
%-------------------------------

%----- Full version switch --------------------------------------------------------
\newboolean{submission} 
\newboolean{finalversion} 
\newboolean{fullversion} 

\newcommand{\issubmission}{%
	\setboolean{submission}{true}
	\setboolean{finalversion}{false} 
	\setboolean{fullversion}{false} 
}

\newcommand{\isfinalversion}{%
	\setboolean{submission}{false}
	\setboolean{finalversion}{true} 
	\setboolean{fullversion}{false} 
}

\newcommand{\isfullversion}{%
	\setboolean{submission}{false}
	\setboolean{finalversion}{false} 
	\setboolean{fullversion}{true} 
}

\newcommand{\submission}[1]{\ifthenelse{\boolean{submission}}{#1}{}}
\newcommand{\finalversion}[1]{\ifthenelse{\boolean{finalversion}}{#1}{}}
\newcommand{\fullversion}[1]{\ifthenelse{\boolean{fullversion}}{#1}{}}
% #1 for submission #2 for finalversion #3 for fullversion
\newcommand{\whichversion}[3]{\ifthenelse{\boolean{submission}}{#1}{\ifthenelse{\boolean{finalversion}}{#2}{#3}}}





%----- Theorems ----------------------------------------------------------------


%Non-LNCS theorem environments
%\newtheorem{theorem}{Theorem}
%\newtheorem{lemma}{Lemma}
%\newtheorem{corollary}{Corollary}
%\newtheorem{definition}{Definition}
%\newtheorem{example}{Example}
%\newtheorem{remark}{Remarks}
%\newtheorem{proof}{Proof}

%This might work with amsthm
%\makeatletter
%\newtheorem{repeatthm@}{Theorem}{\bfseries}{\itshape}
%\newenvironment{repeatthm}[1]{%
	%    \def\therepeatthm@{\ref{#1}}
	%    \repeatthm@
	%}
%{\endrepeatthm@}
%\makeatother


%LNCS related theorem environments%
%\spnewtheorem*{theorem*}{Theorem}{\bfseries}{\itshape}
%\spnewtheorem*{theoremnn}{Theorem}{\bfseries}{\itshape}
%\spnewtheorem*{lemma*}{Lemma}{\bfseries}{\itshape}
%\newenvironment{claimproof}[1]{\par\noindent\underline{Proof}:\space#1}{\hfill $\blacksquare$}
%\spnewtheorem*{definition*}{Definition}{\bfseries}{\itshape}
%\spnewtheorem{fancyclaim}{Claim}{\bfseries}{\itshape}

%This only works with LNCS!
% \makeatletter
% \spnewtheorem{repeatthm@}{Theorem}{\bfseries}{\itshape}
% \newenvironment{repeatthm}[1]{%
	%     \def\therepeatthm@{\ref{#1}}
	%     \repeatthm@
	% }
% {\endrepeatthm@}
% \spnewtheorem{repeatlem@}{Lemma}{\bfseries}{\itshape}
% \newenvironment{repeatlem}[1]{%
	%     \def\therepeatlem@{\ref{#1}}
	%     \repeatlem@
	% }
% {\endrepeatlem@}
% \spnewtheorem{repeatcor@}{Corollary}{\bfseries}{\itshape}
% \newenvironment{repeatcor}[1]{%
	%     \def\therepeatcor@{\ref{#1}}
	%     \repeatcor@
	% }
% {\endrepeatcor@}
% \makeatother

%-------------------------------------------------------------------------------
%  Magic Stuff below
%-------------------------------------------------------------------------------

%------ Quote ------------------------------------------------------------------
\renewcommand{\quote}{\list{}{\rightmargin=\leftmargin\topsep=0pt}\item\relax}

%------ Subsection and Paragraph -----------------------------------------------

% Saving space in case of deadlines

%\makeatletter
%\renewcommand{\section}{\abovedisplayskip 3\p@ \@plus3\p@ \@minus1\p@%
	%                      \belowdisplayskip 5\p@ \@plus3\p@ \@minus1\p@%
	%                      \abovedisplayshortskip 0pt \@plus2\p@%
	%                      \belowdisplayshortskip 0pt \@plus2\p@ \@minus0\p@%
	%                      \@startsection{section}{1}{\z@}%
	%                       {-10\p@ \@plus -4\p@ \@minus -4\p@}%
	%                       {6\p@ \@plus 4\p@ \@minus 4\p@}%
	%                       {\normalfont\large\bfseries\boldmath
		%                        \rightskip=\z@ \@plus 8em\pretolerance=10000 }}
%\renewcommand{\subsection}{\@startsection{subsection}{2}{\z@}%
	%                      {-6\p@ \@plus -4\p@ \@minus -4\p@}%
	%                      {2\p@ \@plus 2\p@ \@minus 2\p@}%
	%                      {\normalfont\normalsize\bfseries\boldmath
		%                       \rightskip=\z@ \@plus 8em\pretolerance=10000 }}
%\renewcommand{\subsubsection}{\@startsection{paragraph}{4}{\z@}%
	%                      {-8\p@ \@plus -4\p@ \@minus -4\p@}%
	%                      {-5\p@ \@plus -0.22em \@minus -0.1em}%
	%                      {\normalfont\normalsize\bfseries\boldmath
		%                      }}
\renewcommand{\paragraph}[1]{\medskip\noindent{\bf #1}}
%\renewcommand{\paragraph}{\@startsection{paragraph}{4}{\z@}%
	%                      {-8\p@ \@plus -4\p@ \@minus -4\p@}%
	%                      {-5\p@ \@plus -0.22em \@minus -0.1em}%
	%                      {\normalfont\normalsize\bf
		%                     }}
%\makeatother

%----- Algorithm Environment ---------------------------------------------------
%Header for Algorithms/Functionalities
\newcommand{\algoHead}[1]{\vspace{0.2em} \underline{\textbf{#1}} \vspace{0.3em}}
\newcommand{\algoHeadExt}[2]{\vspace{0.2em} \underline{\textbf{#1} #2} \vspace{0.3em}}

%Multiline Algo-States
\makeatletter
\algnewcommand{\ExtendedState}[1]{\State
	\parbox[t]{\dimexpr\linewidth-\ALG@thistlm}{\hangindent=\algorithmicindent\strut\hangafter=3#1\strut}}
\makeatother

%Algorithms States
\algnewcommand\algorithmicinput{\textbf{Input:}}
\algnewcommand\Input{\item[\algorithmicinput]}
\renewcommand{\algorithmicensure}{\textbf{Output:}}

%Algo Comments
\algrenewcommand{\algorithmiccomment}[1]{{\color{gray}// #1}}

% \newtcolorbox{titlebox}[5]{enhanced,center,colframe=black,colback=white,boxrule={#3},arc={#2},auto outer arc,%
	%  breakable,pad at break*=5pt,vfill before first,before={\par\medskip\noindent},after={\par\medskip},top=12pt,left=4pt,%
	%  enlarge top by=7pt,%enlarge bottom by=7pt,%
	%  title={\rule[-.3\baselineskip]{0pt}{\baselineskip}\normalsize\sffamily\bfseries #1}, varwidth boxed title*=-30pt,
	%  attach boxed title to top left={yshift=-10pt,xshift=10pt}, coltitle=black,
	%  boxed title style={colback=white,boxrule={#5},arc={#4},auto outer arc}
	%  }

\newtcolorbox{titlebox}[5]{enhanced,center,colframe=black,colback=white,boxrule={#3},arc={#2},auto outer arc,%
	breakable,pad at break*=5pt,vfill before first,before={\par\medskip\noindent},after={\par\medskip},top=12pt,left=4pt,%
	enlarge top by=7pt,%enlarge bottom by=7pt,%
	title={\rule[-.3\baselineskip]{0pt}{\baselineskip}\normalsize\sffamily\bfseries #1}, varwidth boxed title*=-30pt,
	attach boxed title to top left={yshift=-10pt,xshift=10pt}, coltitle=black,
	boxed title style={colback=white,boxrule={#5},arc={#4},auto outer arc}
}


\newenvironment{systembox}[1]
{\vspace{\baselineskip}\begin{titlebox}{Functionality \normalfont #1}{2.5pt}{1pt}{3.5pt}{1pt}}
	{\end{titlebox}}

\newenvironment{protocolbox}[1]
{\begin{titlebox}{Protocol \normalfont #1}{0.5pt}{0.5pt}{1pt}{0.75pt}}
	{\end{titlebox}}



\newenvironment{processbox}[1]
{\begin{titlebox}{Process \normalfont #1}{0.5pt}{0.5pt}{1pt}{0.75pt}}
	{\end{titlebox}}

\newenvironment{algobox}[1]
{\begin{titlebox}{Algorithm \normalfont #1}{0.5pt}{0.5pt}{1pt}{0.75pt}}
	{\end{titlebox}}
%{\begin{titlebox}{Algorithm \normalfont #1}{commonbox}{normal}}
	%{\end{titlebox}}

\newenvironment{simubox}[1]
{\begin{titlebox}{Simulator \normalfont #1}{0.5pt}{0.5pt}{1pt}{0.75pt}}
	{\end{titlebox}}


%----- Reference magic ---------------------------------------------------------
%Enable reference of descriptions
\makeatletter
\let\orgdescriptionlabel\descriptionlabel
\renewcommand*{\descriptionlabel}[1]{%
	\let\orglabel\label
	\let\label\@gobble
	\phantomsection
	\edef\@currentlabel{#1}%
	%\edef\@currentlabelname{#1}%
	\let\label\orglabel
	\orgdescriptionlabel{#1}%
}
\makeatother



%NDSS
%\newtheorem{theorem}{Theorem}[section]
%\newtheorem{lemma}[theorem]{Lemma}
%\newtheorem{definition}{Theorem}[section]
%\newtheorem{remark}[theorem]{Lemma}
%\newtheorem{corollary}[theorem]{Lemma}

%\theoremstyle{definition}
%\newtheorem{definition}{Definition}
%
%\theoremstyle{theorem}
%\newtheorem{theorem}{Theorem}%[section]
%
%\theoremstyle{remark}
%\newtheorem{lemma}{Lemma}
%\newtheorem{remark}{Remark}
%\newtheorem{corollary}{Corollary}