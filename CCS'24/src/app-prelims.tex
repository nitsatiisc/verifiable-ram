\section{More Preliminaries}

\subsection{Polynomial Commitment Scheme}
\label{sec:pcs_def}
The notion of a polynomial commitment scheme that allows the prover to open evaluations of the committed polynomial succinctly was introduced in~\cite{AC:KatZavGol10} who gave a construction under the trusted setup assumption.
A polynomial commitment scheme over $\F$ is a tuple $\pc = 
(\pcsetup,\pccommit,\pcopen,\pceval)$ where:

\begin{itemize}
    \item $\pcsetup(1^\secp,D) \rightarrow \pp$. On input security parameter $\secp$, and an upper bound $D \in \mathbb{N}$ on the degree, $\pcsetup$ generates public parameters $\pp$.
    
    \item $\pccommit(\pp, f(X),d) \rightarrow (C,\mathbf{\tilde{c}})$. On input the public parameters $\pp$, and a univariate polynomial $f(X) \in \F[X]$ with degree at most $d \leq D$, $\pccommit$ outputs a commitment to the polynomial $C$, and additionally an opening hint $\mathbf{\tilde{c}}$.
	
	\item $\pcopen(\pp,f(X),d,C,\mathbf{\tilde{c}}) \rightarrow b$. On input the public parameters $\pp$, the commitment $C$ and the opening hint $\mathbf{\tilde{c}}$, a polynomial $f(X)$ of degree $d \leq D$, $\pcopen$ outputs a bit indicating accept or reject. 
    
	    \item $\pceval(\pp,C,d,x,v;f(X)) \rightarrow b $. A public coin interactive protocol 
	    $\langle P_{\mathsf{eval}}(f(X)), V_{\mathsf{eval}} \rangle(\pp,C,d,z,v)$ between a PPT prover and a PPT verifier. The parties have as common input public parameters $\pp$, commitment $C$, degree $d$, evaluation point $x$, and claimed evaluation $v$. The prover has, in addition, the opening $f(X)$ of $C$, with $\deg(f) \leq d$. At the end of the protocol, the verifier outputs $1$ indicating accepting the proof that $f(x)=v$, or outputs $0$ indicating rejecting the proof.
		
		\end{itemize}
		
A polynomial commitment scheme must satisfy completeness, binding and extractability.

\begin{definition}[Completeness]
\label{def:pcs-comp}
For all polynomials $f(X) \in \F[X]$ of degree $d \leq D$, for all $x \in \F$,
\[
\Pr \left( 
\begin{matrix}
 %\pccheck(\pp,C,d,z,v,\pi) = 1 
 b=1
\end{matrix}
 \,:\,
 \begin{matrix}
\pp \leftarrow \pcsetup(1^\secp,D) \\
 (C,\mathbf{\tilde{c}}) \leftarrow \pccommit(\pp, f(X),d) \\
 v \leftarrow f(x) \\
 b \leftarrow \pceval(\pp,C,d,x,v;f(X))
 %\pi \leftarrow \pceval(\pp,f(X),d,z) \\
\end{matrix}
 \right) = 1.
 \]
\end{definition}

\begin{definition}[Binding] 
\label{def:pcs-binding-app}
A polynomial commitment scheme $\pc$ is binding if for all PPT $\Adv$, the following probability is negligible in $\secp$:
\[
\Pr \left( 
\begin{matrix}
 \pcopen(\pp,f_0,d,C,\mathbf{\tilde{c}_0}) =1 \wedge \\
 \pcopen(\pp,f_1,d,C,\mathbf{\tilde{c}_1}) = 1 \wedge \\
 f_0 \neq f_1
\end{matrix}
 \,:\,
 \begin{matrix}
\pp \leftarrow \pcsetup(1^\secp,D) \\
 (C,f_0,f_1,\mathbf{\tilde{c}_0}, \mathbf{\tilde{c}_1},d) \leftarrow \Adv(\pp) 
\end{matrix}
 \right).
 \]
\end{definition}

\begin{definition}[Extractability]
\label{def:pcs-ext-app}
For any PPT adversary $\Adv = (\Adv_{1},\Adv_{2})$, there exists a PPT algorithm $\ext$ such that the following probability is negligible in $\secp$:
	 \[
    \Pr\left(
      \begin{matrix}
      %\pccheck(\pp,C,d,z,v,\pi) = 1 \wedge
      b = 1 \wedge
      \R_{\pceval}(\pp,C,x,v; \tilde{f},\mathbf{\tilde{c}}) = 0
      \end{matrix}
      \,:\,
      \begin{matrix}
         \pp \leftarrow \pcsetup(1^\secp,D) \\
          (C,d,x,v,\mathsf{st}) \leftarrow \Adv_{1}(\pp) \\
         (\tilde{f},\mathbf{\tilde{c}}) \leftarrow \ext^{\Adv_{2}}(\pp)\\
         b \leftarrow \langle \Adv_{2}(\mathsf{st}), V_{\mathsf{eval}} \rangle(\pp,C,d,x,v)
      \end{matrix}
    \right).
  \]
  where the relation $\R_{\pceval}$ is defined as follows:
\begin{align*}
        \R_{\pceval} &= \{\left((\pp,C \in \mathbb{G},\; x \in \F, \; v \in \F);\; (f(X), \mathbf{\tilde{c}}) \right) : \\
        &\qquad (
        \pcopen(\pp,f,d,C,\mathbf{\tilde{c}}) = 1 )
         \land v = f(x)  \}
    \end{align*} 
    
\end{definition}

We denote by $\mathsf{Prove}, \mathsf{Verify}$, the non-interactive prover and verifier algorithms obtained by applying FS to the 
$\pceval$ public-coin interactive protocol.

\begin{definition}[Succinctness]
\label{def:pcs-succinct}
We require the commitments and the evaluation proofs to be of size independent of the degree of the polynomial, that is the scheme is \emph{proof succinct} if
$|C|$ is $\mathsf{poly}(\secp)$, $|\pi|$ is $\mathsf{poly}(\secp)$ where $\pi$ is the transcript obtained by applying FS to $\pceval$. Additionally, the scheme is \emph{verifier succinct} if $\pceval$ runs in time
$\mathsf{poly}(\secp) \cdot \allowbreak \mathsf{log} (d)$ for the verifier.
\end{definition}


\subsection{Succinct Argument of Knowledge}
\label{sec:aok}