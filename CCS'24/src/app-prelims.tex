\section{More Preliminaries}

\subsection{Polynomial Commitment Scheme}
\label{sec:pcs_def}
The notion of a polynomial commitment scheme (PCS) that allows the prover to open evaluations of the committed polynomial succinctly was introduced in~\cite{AC:KatZavGol10} who gave a construction under the trusted setup assumption.
A polynomial commitment scheme over $\F$ is a tuple $\pc = 
(\pcsetup,\pccommit,\pcopen,\pceval)$ where:

\begin{itemize}
    \item $\pp \leftarrow \pcsetup(1^\secp,D) $. On input security parameter $\secp$, and an upper bound $D \in \mathbb{N}$ on the degree, $\pcsetup$ generates public parameters $\pp$.
    
    \item $(C,\mathbf{\tilde{c}}) \leftarrow \pccommit(\pp, f(X),d) $. On input the public parameters $\pp$, and a univariate polynomial $f(X) \in \F[X]$ with degree at most $d \leq D$, $\pccommit$ outputs a commitment to the polynomial $C$, and additionally an opening hint $\mathbf{\tilde{c}}$.
	
	\item $b \leftarrow \pcopen(\pp,f(X),d,C,\mathbf{\tilde{c}})$. On input the public parameters $\pp$, the commitment $C$ and the opening hint $\mathbf{\tilde{c}}$, a polynomial $f(X)$ of degree $d \leq D$, $\pcopen$ outputs a bit indicating accept or reject. 
    
	    \item $b \leftarrow \pceval(\pp,C,d,x,v;f(X)) $. A public coin interactive protocol 
	    $\langle P_{\mathsf{eval}}(f(X)), V_{\mathsf{eval}} \rangle(\pp,C,d,z,v)$ between a PPT prover and a PPT verifier. The parties have as common input public parameters $\pp$, commitment $C$, degree $d$, evaluation point $x$, and claimed evaluation $v$. The prover has, in addition, the opening $f(X)$ of $C$, with $\deg(f) \leq d$. At the end of the protocol, the verifier outputs $1$ indicating accepting the proof that $f(x)=v$, or outputs $0$ indicating rejecting the proof.
		
		\end{itemize}
		
A polynomial commitment scheme must satisfy completeness, binding and extractability.

\begin{definition}[Completeness]
\label{def:pcs-comp}
For all polynomials $f(X) \in \F[X]$ of degree $d \leq D$, for all $x \in \F$,
\[
\Pr \left( 
\begin{matrix}
 %\pccheck(\pp,C,d,z,v,\pi) = 1 
 b=1
\end{matrix}
 \,:\,
 \begin{matrix}
\pp \leftarrow \pcsetup(1^\secp,D) \\
 (C,\mathbf{\tilde{c}}) \leftarrow \pccommit(\pp, f(X),d) \\
 v \leftarrow f(x) \\
 b \leftarrow \pceval(\pp,C,d,x,v;f(X))
 %\pi \leftarrow \pceval(\pp,f(X),d,z) \\
\end{matrix}
 \right) = 1.
 \]
\end{definition}

\begin{definition}[Binding] 
\label{def:pcs-binding-app}
A polynomial commitment scheme $\pc$ is binding if for all PPT $\Adv$, the following probability is negligible in $\secp$:
\[
\Pr \left( 
\begin{matrix}
 \pcopen(\pp,f_0,d,C,\mathbf{\tilde{c}_0}) =1 \wedge \\
 \pcopen(\pp,f_1,d,C,\mathbf{\tilde{c}_1}) = 1 \wedge \\
 f_0 \neq f_1
\end{matrix}
 \,:\,
 \begin{matrix}
\pp \leftarrow \pcsetup(1^\secp,D) \\
 (C,f_0,f_1,\mathbf{\tilde{c}_0}, \mathbf{\tilde{c}_1},d) \leftarrow \Adv(\pp) 
\end{matrix}
 \right).
 \]
\end{definition}

\begin{definition}[Knowledge Soundness]
\label{def:pcs-ext-app}
For any PPT adversary $\Adv = (\Adv_{1},\Adv_{2})$, there exists a PPT algorithm $\ext$ such that the following probability is negligible in $\secp$:
	 \[
    \Pr\left(
      \begin{matrix}
      %\pccheck(\pp,C,d,z,v,\pi) = 1 \wedge
      b = 1 \wedge \\
      \R_{\pceval}(\pp,C,x,v; \tilde{f},\mathbf{\tilde{c}}) = 0
      \end{matrix}
      \,:\,
      \begin{matrix}
         \pp \leftarrow \pcsetup(1^\secp,D) \\
          (C,d,x,v,\mathsf{st}) \leftarrow \Adv_{1}(\pp) \\
         (\tilde{f},\mathbf{\tilde{c}}) \leftarrow \ext^{\Adv_{2}}(\pp)\\
         b \leftarrow \langle \Adv_{2}(\mathsf{st}), V_{\mathsf{eval}} \rangle(\pp,C,d,x,v)
      \end{matrix}
    \right).
  \]
  where the relation $\R_{\pceval}$ is defined as follows:
\begin{align*}
        \R_{\pceval} &= \{\left((\pp,C \in \mathbb{G},\; x \in \F, \; v \in \F);\; (f(X), \mathbf{\tilde{c}}) \right) : \\
        &\qquad (
        \pcopen(\pp,f,d,C,\mathbf{\tilde{c}}) = 1 )
         \land v = f(x)  \}
    \end{align*} 
    
\end{definition}

We denote by $\mathsf{Prove}, \mathsf{Verify}$, the non-interactive prover and verifier algorithms obtained by applying FS to the 
$\pceval$ public-coin interactive protocol, giving a non-interactive PCS scheme $(\pcsetup,\pccommit,\allowbreak \mathsf{Prove},\mathsf{Verify})$.

\begin{definition}[Succinctness]
\label{def:pcs-succinct}
We require the commitments and the evaluation proofs to be of size independent of the degree of the polynomial, that is the scheme is \emph{proof succinct} if
$|C|$ is $\mathsf{poly}(\secp)$, $|\pi|$ is $\mathsf{poly}(\secp)$ where $\pi$ is the transcript obtained by applying FS to $\pceval$. Additionally, the scheme is \emph{verifier succinct} if $\pceval$ runs in time
$\mathsf{poly}(\secp) \cdot \allowbreak \mathsf{log} (d)$ for the verifier.
\end{definition}


\subsection{Succinct Argument of Knowledge}
\label{sec:aok}

Let $\R$ be a NP-relation and $\L$ be the corresponding NP-language, where $\L=\{x : \exists$ $w$ such that $(x,w) \in \R\}$. Here, a prover $\prover$ aims to convince a verifier $\verifier$ that $x\in \L$ by proving that it knows a witness $w$ for a public statement $x$ such that $(x,w)\in \R$. An interactive argument of knowledge for a relation $\R$ consists of a PPT algorithm $\setup$, that takes an input the security parameter $\secp$, and outputs the public parameter $\pp$, and a pair of interactive PPT algorithms $\langle\prover,\verifier\rangle$, where $\prover$ takes as input $(\pp,x,w)$ and $\verifier$ takes as input $(\pp,x)$. An interactive argument of knowledge  $\langle\prover,\verifier\rangle$, must satisfy completeness and knowledge soundness. % and a succinct argument of knowledge must additionally satisfy the property of succinctness.

\begin{definition}[Completeness]
	\label{def:aok-comp}
	For all security parameter $\secp \in \mathbb{N}$ and statement $x$ and witness $w$ such that $(x,w)\in \R$, we have
	\[
	\Pr \left( 
	\begin{matrix}
		%\pccheck(\pp,C,d,z,v,\pi) = 1 
		b=1
	\end{matrix}
	\,:\,
	\begin{matrix}
		\pp \leftarrow \setup(1^\secp) \\
		b \leftarrow \langle\prover (w) ,\verifier\rangle (\pp,x)
	\end{matrix}
	\right) = 1.
	\]
\end{definition}

\begin{definition}[Knowledge Soundness]
	\label{def:aok-ks}
	%An interactive argument of knowledge  $\langle\prover,\verifier\rangle$ satisfies knowledge soundness with error $\kappa$, if for
	For any PPT malicious prover $\prover^* = (\prover_1^*,\prover_2*)$, there exists a PPT algorithm $\ext$ such that the following probability is negligible:
	\[
		\Pr \left( 
		\begin{matrix}
			b=1 \wedge \\
			(x,w)\not\in \R
		\end{matrix}
		\,:\,
		\begin{matrix}
			\pp \leftarrow \setup(1^\secp) \\
			(x,\mathsf{st}) \leftarrow \prover_1^*(1^\secp,\pp) \\
			b \leftarrow \langle\prover_2^* (\mathsf{st}) ,\verifier\rangle (\pp,x) \\
			w \leftarrow \ext^{\prover_2^*} (\pp,x)
		\end{matrix}
		\right) .
		\]	
\end{definition}

A \emph{succinct} argument of knowledge $\langle\prover,\verifier\rangle$ for a relation $\R$, must satisfy completeness and knowledge soundness and additionally ensure that the communication complexity between prover and verifier, as well as the verification complexity is bounded by $\mathsf{poly}(\secp,\log |w|)$, where $w$ is the witness for the relation.


