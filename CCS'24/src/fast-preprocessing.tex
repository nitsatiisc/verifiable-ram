%! Author = nitin
%! Date = 10/12/23

% Preamble
\documentclass[11pt]{article}

% Packages
\usepackage{llncsconf}
\usepackage{times}
\usepackage{amsmath,amsthm,amsfonts,enumitem,tikz, subcaption}
\usepackage{amssymb}
\usetikzlibrary{arrows,arrows.meta,shapes.arrows, calc, positioning,matrix}

\newtheorem{theorem}{Theorem}[section]
\newtheorem{lemma}{Lemma}[section]
\newtheorem{definition}{Definition}[section]


%! Author = nitin
%! Date = 03/11/23

\newcommand{\npol}{\mathsf{NP}}
\newcommand{\enc}[2]{\mathsf{Enc}_{{}_{#2}}(\vec{#1})}
\newcommand{\vecA}{\vec{a}}
\newcommand{\vecB}{\vec{b}}
\newcommand{\vecC}{\vec{c}}
\newcommand{\vecD}{\vec{d}}
\newcommand{\vecP}{\vec{p}}
\newcommand{\vecQ}{\vec{q}}
\newcommand{\vecS}{\vec{s}}
\newcommand{\vecT}{\vec{T}}
\newcommand{\vecX}{\vec{x}}
\newcommand{\vecY}{\vec{y}}
\newcommand{\vecZ}{\vec{z}}
\newcommand{\vecW}{\vec{w}}
\newcommand{\vecI}{\vec{I}}
\newcommand{\vecind}{\overrightarrow{\ind}}
\newcommand{\vecop}{\overrightarrow{\op}}
\newcommand{\vecval}{\overrightarrow{\val}}
\newcommand{\setind}{I}
\newcommand{\ins}{\mathsf{ins}}
\newcommand{\ind}{\mathsf{ind}}
\newcommand{\op}{\mathsf{op}}
\newcommand{\vecins}{\overrightarrow{\ins}}

%%% Base table and cache table
\newcommand{\vecTbase}{\vec{T}_{\mathsf{b}}}
\newcommand{\vecTcache}{\vec{T}_{\mathsf{ch}}}
\newcommand{\Tbasepoly}[1]{T_{\mathsf{b}}({#1})}
\newcommand{\Tcachepoly}[1]{T_{\mathsf{ch}}({#1})}
\newcommand{\Qbasepoly}[1]{Q_{\mathsf{b}}({#1})}
\newcommand{\Qcachepoly}[1]{Q_{\mathsf{ch}}({#1})}
\newcommand{\Qcachepolyone}[1]{Q'_{\mathsf{ch}}({#1})}
\newcommand{\Qcachepolytwo}[1]{Q''_{\mathsf{ch}}({#1})}
\newcommand{\Zinv}[1]{(\mathbb{Z}_#1)^{\times}}
\newcommand{\Zn}{\mathbb{Z}_N}
\newcommand{\setH}{\mathbb{K}}
\newcommand{\setV}{\mathbb{V}}
\newcommand{\setN}{\mathbb{H}}
\newcommand{\vpolyN}{Z_\mathbb{H}}
\newcommand{\mroots}{\mathbb{V}}
\newcommand{\nroots}{\mathbb{H}}
\newcommand{\iroots}{\mathbb{H}_I}
\newcommand{\BG}{\mathsf{BG}}
\newcommand{\F}{\mathbb{F}}
\newcommand{\Fp}{{\mathbb{F}_p}}
\newcommand{\Zp}{{\mathbb{Z}_p}}
\newcommand{\Zq}{{\mathbb{Z}_q}}
\newcommand{\G}{\mathbb{G}}
\newcommand{\N}{\mathbb{N}}
\newcommand{\V}{\mathsf{V}}
\newcommand{\Z}{\mathsf{Z}}

\newcommand{\Gone}{\mathbb{G}_1}
\newcommand{\Gtwo}{\mathbb{G}_2}
\newcommand{\GT}{\mathbb{G}_T}
\newcommand{\srs}{\mathsf{srs}}
\newcommand{\cm}{\mathsf{cm}}
\newcommand{\wt}[1]{\widetilde{#1}}
\newcommand{\outp}{\rightarrow}
\newcommand{\pp}{\mathsf{pp}}
\newcommand{\pk}{\ensuremath{\mathsf{pk}}}
\newcommand{\bal}{\ensuremath{\mathsf{bal}}}
\newcommand{\txno}{\ensuremath{\mathsf{txno}}}
\newcommand{\load}{\ensuremath{\mathsf{LD}}}
\newcommand{\store}{\ensuremath{\mathsf{STR}}}
\newcommand{\RAM}[2]{\ensuremath{\mathsf{RAM}}_{\mathcal{#1}, {#2}}}
\newcommand{\RAMOp}[1]{\ensuremath{\mathcal{O}_{{}_\mathcal{#1}}}}
\newcommand{\Rupd}[1]{\ensuremath{\mathsf{Upd}_{\mathcal{#1}}}}
\newcommand{\LRAM}[3]{\ensuremath{\mathsf{LRAM}_{\mathcal{#1},#2,#3}}}
\newcommand{\LTr}{\ensuremath{\mathcal{L}_{\mathsf{Tr}}}}
\newcommand{\tr}{\ensuremath{\mathsf{tr}}}
\newcommand{\TOT}{\ensuremath{\mathsf{TimeTr}}}
\newcommand{\AOT}{\ensuremath{\mathsf{AddrTr}}}
\newcommand{\evalH}[1]{\ensuremath{\vec{#1}_{|_\setH}}}
\newcommand{\evalV}[1]{\ensuremath{\vec{#1}_{|_\setV}}}
\newcommand{\LLconcat}{\ensuremath{\tilde{\mathcal{L}}_{\mathsf{concat}}}}
\newcommand{\Lperm}{\ensuremath{\mathcal{L}_{\mathsf{perm}}}}
\newcommand{\RRAM}[1]{\ensuremath{R}_{\mathsf{ram},#1}}
\newcommand{\RLOOK}{\ensuremath{\mathcal{R}^{\mathsf{lookup}}_{\srs,N,m}}}
\newcommand{\CRAM}{\ensuremath{\mathcal{R}^{\mathsf{ram}}_{\srs,N,m}}}
\newcommand{\CLRAM}{\ensuremath{\mathcal{R}^{\mathsf{LRAM}}_{\srs,m}}}
\newcommand{\val}[2]{\ensuremath{\mathsf{V}_{{#2},{#1}}}}
\newcommand{\uniq}[1]{\ensuremath{\mathsf{uniq}(\vec{{#1}})}}
% group encodings
\newcommand{\gone}[1]{\ensuremath{\left[{#1}\right]_1}}
\newcommand{\gtwo}[1]{\ensuremath{\left[{#1}\right]_2}}
\newcommand{\gany}[1]{\ensuremath{\left[{#1}\right]_g}}

% provers, verifiers, adversaries
\newcommand{\prover}{\ensuremath{\mathcal{P}}}
\newcommand{\verifier}{\ensuremath{\mathcal{V}}}
\newcommand{\Adv}{\ensuremath{\mathcal{A}}}
\newcommand{\kzg}{\ensuremath{\mathsf{KZG}}}
\newcommand{\kzgprove}{\ensuremath{\mathsf{KZG.Prove}}}
\newcommand{\kzgverify}{\ensuremath{\mathsf{KZG.Verify}}}
\newcommand{\kzgcommit}{\ensuremath{\mathsf{KZG.Commit}}}

%%%Author Comments--------------------------------------------
\newcommand{\moumita}[1]{\textcolor{orange}{M: #1}}
\newcommand{\chaya}[1]{\textcolor{cyan}{Chaya: #1}}
\newcommand{\sikhar}[1]{\textcolor{red}{Sikhar: #1}}
\newcommand{\nitin}[1]{\textcolor{blue}{Nitin: #1}}
\newcommand{\shubh}[1]{\textcolor{green}{Shubh: #1}}

%group elements and commitments
\newcommand{\eltone}[1]{[#1]_1}
\newcommand{\elttwo}[1]{[#1]_2}
\newcommand{\elany}[1]{[#1]_g}
\newcommand{\elt}[1]{[\,{#1}\,]}
\newcommand{\Comone}[1]{\mathsf{Com}_1(#1)}
\newcommand{\Comtwo}[1]{\mathsf{Com}_2(#1)}
\newcommand{\ComT}[1]{\mathsf{Com}_T(#1)}

\newtheorem{fact}{Fact}[section]

%%%KZG 
\newcommand{\Com}[1]{\mathsf{Com}(#1)}
\newcommand{\KZGsetup}{\mathsf{KZG.Setup}}
\newcommand{\KZGcommit}{\mathsf{KZG.Commit}}
\newcommand{\KZGopen}{\mathsf{KZG.Prove}}
\newcommand{\KZGverify}{\mathsf{KZG.Verify}}
\newcommand{\out}{\mathsf{out}}
\newcommand{\ppt}{\mathrm{PPT}}
% \newcommand{\in}{\mathsf{in}}
\newcommand{\rangeproof}{\mathsf{Range}}

\newcommand{\setup}{\mathsf{Setup}}
\newcommand{\commit}{\mathsf{Commit}}
\newcommand{\open}{\mathsf{Open}}
\newcommand{\verify}{\mathsf{Verify}}

%%%Elements
\newcommand{\abf}{\textbf{a}}
\newcommand{\bbf}{\textbf{b}}
\newcommand{\cbf}{\textbf{c}}
\newcommand{\ibf}{\textbf{i}}
\newcommand{\jbf}{\textbf{j}}
\newcommand{\kbf}{\textbf{k}}
\newcommand{\pbf}{\textbf{p}}
\newcommand{\qbf}{\textbf{q}}
\newcommand{\rbf}{\textbf{r}}
\newcommand{\sbf}{\textbf{s}}
\newcommand{\vbf}{\textbf{v}}
\newcommand{\xbf}{\textbf{x}}
\newcommand{\ybf}{\textbf{y}}
\newcommand{\zbf}{\textbf{z}}
\newcommand{\Tbf}{\textbf{T}}
\newcommand{\Sbf}{\textbf{S}}
\newcommand{\Mbf}{\textbf{M}}

%%%Sets
\newcommand{\doubleH}{\mathbb{H}}
\newcommand{\Hi}{{\mathbb{H}_I}}
\newcommand{\doubleV}{\mathbb{V}}
\newcommand{\Vm}{{\mathbb{V}_m}}
\newcommand{\zeroset}{\{0\}}


%% Bilinear group generator
\newcommand{\bg}{\ensuremath{\mathsf{BG}}}
%% Security parameter
\newcommand{\secp}{\ensuremath{\lambda}}
\newcommand{\negl}{\ensuremath{\mathsf{negl}}}
\title{Superfast Amortized KZG proofs from Coset sums}

% Document
\begin{document}
    \maketitle
    \begin{abstract}
    In this note, we present a linear time algorithm that computes KZG proofs of opening over
        a smooth subgroup of size $n$ of a polynomial of degree $d$ using $n+d$ group and
        field operations. This improves the existing algorithm of Fiest and Khovratovich
        \cite{EPRINT:FeiKho23}, by shaving off the $\log(n+d)$ factor and concretely resulting
        in about $50\times$ improvement in proving time. Our improvement directly impacts the
        practical viability of recent lookup arguments such as Caulk ~\cite{CCS:ZBKMNS22},
        Caulk+ ~\cite{EPRINT:PosKat22}, CQ ~\cite{EPRINT:EagFioGab22} which crucially
        rely on such proofs in the pre-processing phase.
    \end{abstract}

    \section{Introduction}\label{sec:introduction}
    Recently, there has been tremendous interest in designing ``lookup'' primitive from pre-computed tables,
    which is seen as a promising way to optimise proof generation using zkSNARKs. Several efficient lookup
    arguments have been proposed recently, both for protocols based on uni-variate polynomials ~\cite{CCS:ZBKMNS22,
    EPRINT:PosKat22, EPRINT:EagFioGab22,EPRINT:ZGKMR22} and those based on multi-variate polynomials (Lasso).
    The mentioned approaches based on uni-variate polynomials essentially encode a vector (a table) as a
    polynomial whose degree matches the size of the vector. The lookup protocols involve pre-processing this
    vector (say of size $n$), to compute $O(n)$-size parameters, such that subsequently showing $m$ lookups
    from the table incurs proving cost independent of $n$. Specifically, the pre-processing parameters consist
    of KZG proofs of opening of the polynomial encoding the table, at all $n$ roots of unity (over which the polynomial
    interpolates the vector). The existing approach to accomplish this using the techniques of Fiest and Khovratovich
    ~\cite{EPRINT:FeiKho23}, which incurs $O((n+d)\log (n+d))$ group and field operations. While clearly superior
    to the naive approaches involving $O(n^2)$ operations, the approach becomes prohibitive due to several FFTs
    over groups that need to be performed. The algorithm also involves several algorithms from computation algebra,
    ported to work over groups (instead of finite fields) which makes the implementation challenging. In the next
    section, we present an algorithm which only incurs $n+d$ group and field operations and is conceptually simpler
    with easier implementation. It relies on elegant {\em uni-variate sum-check} based approach which essentially
    simplifies sums of polynomials over structured sets such as subgroups or cosets.

    \section{Improved Amortized KZG Proofs}\label{sec:kzgproofs}
    Let $\F$ be a finite field of large prime modulus $p$ and suppose that $\F$ contains a subgroup
    $\nroots=\{1,\omega,\ldots,\omega^{n-1}\}$ for some integer $n\in \N$. Let $\BG=(\Gone,\Gtwo,\GT,g_1,g_2)$
    be a bilinear group and $e:\Gone\times \Gtwo\rightarrow \GT$ be a non-degenerate efficiently computable pairing map.
    We also assume that all parties have access to the structured reference string $\srs = (\gone{1},\gone{s},\ldots,\gone{s^n},
    \gtwo{1},\ldots,\gtwo{s^n})$. Let $T(X)\in \F[X]$ be a polynomial of degree $d<n$. A KZG proof $\pi_i$ for opening
    of $T(X)$ at $\omega^i$, consists of $\Gtwo$-encoding of the polynomial $(TX) - T(\omega^i))/(X-\omega^i)$, i.e,
    \begin{equation*}
        \pi_i = \left[\frac{T(X)-T(\omega^i)}{X-\omega^i}\right]_2
    \end{equation*}


    \bibliographystyle{plain}
    \bibliography{crypto_crossref, main}




\end{document}