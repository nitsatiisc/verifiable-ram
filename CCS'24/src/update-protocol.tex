
The recent progress and interest in lookup arguments has been stellar, as exemplified by the series of works
both in uni-variate setting ~\cite{CCS:ZBKMNS22,EPRINT:PosKat22,EPRINT:ZGKMR22,EPRINT:EagFioGab22} and multi-variate
setting ~\cite{lasso}. However, the excellent online efficiency of current constructions relies on expensive
$\wt{O}(|T|)$ table-specific  pre-computation for a table $T$, or on tables exhibiting tensor structure as in ~\cite{lasso}.
This limits their application in settings where such assumptions are not viable, for example when tables model account balances in
a layer 2 (L2) blockchain network. We make the first attempt in this direction. Our key idea is to extend the utiltiy of pre-computed
parameters for a table $\vecT$, to proving lookups from tables $\vecT'\neq \vecT$. Essentially, we show that for $\delta=\Delta(\vecT, \vecT')$,
an argument for $m$ lookups from $\vecT'$ incurs an additional prover overhead of $(m+\delta)\log^2(m+\delta)$. We note that overhead is additive
in $\delta$ and that too only {\em quasi} linear. Our competitive overhead rests on several innovative applications of algebraic
algorithms, which are summarised in Section ~\ref{subsec:comp-algebra-app}.

\noindent{\bf Naive approaches are inadequate}: We note that the aforementioned constructions of lookup arguments require encoded quotients
of the form $\gany{(T(X)-T(\xi^i))/(X-\xi^i)}$ for upto $m$ values of $i$ during the proof generation. While constructions ~\cite{CCS:ZBKMNS22,EPRINT:PosKat22}
consider quotients encoded in the group $\Gtwo$, the protocol in ~\cite{EPRINT:EagFioGab22} encodes them in $\Gone$. We use a generic $[\,\cdot\,]_g$ to
account for protocol-specific choices. We also see that even a small change to the table requires one to update all the quotients (the polynomial $T(X)$ is
common to all quotients). Updating the $|\vecT|$ quotients for each batch is clearly infeasible. One could consider delaying the updation of the quotients, till
the time they are actually required in a proof, which happens when the corresponding index in the table is involved in lookup. However, each of the $m$ quotients
is now potentially ``lagging'' by $\delta$ updates, so we would need $\Omega(m\delta)$ group operations to refresh all of them. This gives us multiplicative degradation
with $\delta$, and is clearly unsustainable for reasonable values of $\delta$. We abandon the idea of computing individual encoded quotients, and instead attempt
to directly compute the aggregate encoding $\gany{(T(X)-T_I(X))/Z_I(X)}$, which as seen earlier is given by the summation below:
\begin{equation}\label{eq:encoded-quotient}
\gany{\frac{T(X)-T_I(X)}{Z_I(X)}} = \sum_{i\in I}\frac{1}{Z_I'(\xi^i)}\gany{\frac{T(X)-T(\xi^i)}{X-\xi^i}}
\end{equation}
We now describe our approach.
%Recall that our lookup protocol in section 5.1 involves certain precomputations by the prover namely $W_1^i, W_2^i, W_3^i$. $W_2^i$ and $W_3^i$ do not depend on the table. However, $W_1^i$ depends on the lookup table and their values will change even if the table changes by a small amount. It is expensive to recompute all the $W_1^i$ for every small change in the table and this will affect the efficiency of our lookup protocol in the long run.\\\\
%In this section, we show how to achieve efficient lookups even when the table is changing frequently, as long as the cumulative change in the table is small. \\
%In particular, we show how the prover can compute $[Q(X)]_2=\gtwo{\frac{T(X)-T_I(X)}{Z_I(X)}}$ without computing all the $W_1^i$(thus minimizing the overhead).\\
%The overhead(as long as the table doesn't change too much) will be much lower than the time needed for the lookup and so is very practical.

\subsection{Base + Cache approach}\label{subsec:base-cache}
The key idea we employ is to express the current table $\vecT\in \F^N$ as $\vecTbase + \vecTcache$, where $\vecTbase$ is the table for which we assume that
the encoded quotients are available (via the $O(N\log N)$ computation), and $\vecTcache$ captures the changes to the table since. We will periodically update (say
after $s$ batch updates) $\vecTbase$ to current table state, and re-compute all the quotients (we call it the {\em offline} phase).
We will revisit the question on choosing $s$ optimally later. Let $I\subseteq [N]$ denote the set of indices in the current batch of $m$ lookups. The {\em online}
phase of our proof generation involves computing the sum in Equation \eqref{eq:encoded-quotient} for the table $\vecT$.
%that we do not compute $W_1^i$ after each change of the table. Instead, this expensive computation will be done periodically for all $i \in [N]$ after say $s \in \mathbb{Z}$ batches.
%Let current table $\vecT$ can be represented as $\vecTbase + \vecTcache$ where the vector
%$\vecTbase$ denotes the base table (with respect to which $W_1^i$ was last computed for all $i \in [N]$) and the vector $\vecTcache$ corresponds to the changes
%that have happened to the base table since the last rebasing (rebasing denotes computation of all $W_1^i$)\\\\
%Thus, there is an \textbf{online} phase which happens after every batch (which includes computation of $[Q(X)]_2$ among other things) and an \textbf{offline} phase which consists of the rebasing(this is all prover computations) \\\\
\begin{comment}
    \noindent{\bf Offline Phase}: This computation is executed once after every $s$ rounds. Here, the prover updates the base vector $\vecTbase$ with the changes in the cache vector
    $\vecTcache$ by setting $\vecTbase := \vecTbase + \vecTcache$ and simultaneously clears the cache vector by setting
    $\vecTcache = 0$.\\
    It computes the commitment of $T_b$ as well\\
    It also re-computes the $\mathsf{KZG}$ opening proofs $[W_1^i(X)]_2$ for $i\in [N]$ where
    $W_1^i(X) = (\Tbasepoly{X} - t_i)/(X-\xi^i)$. Here, $t_i=\Tbasepoly{\xi^i}$ are the coordinates
    of the updated base vector $\vecTbase$.\\
    As mentioned in section 5.1 this can be done in $O(N\log N)$ group and field operations.\\\\
    \noindent{\bf Online Phase}:
    The online phase happens for every batch because the purpose of this phase is to ensure that all the things needed for the current execution of the lookup protocol are available. We show how the prover computes the next table $T'$ from the current table $T$ and the new Cache vector from the old cache vector (by an inductive argument this suffices)
    \begin{enumerate}[leftmargin=1em]
        \item Prover has the $T$ for the current round and the commitment $[T(X)]_1$ as well(because these are just the $T'$, $[T'(x)]_1$ of the last round)
        \item The $\vecTcache$ and $\vecTcache(X)$ is also updated to the start of the current round (contains information till previous round:$\vecT=\vecTbase+\vecTcache$)
        \item The prover updates the cache using the current batch: $\vec{T'}_{\mathsf{ch}}[i] = \vec{T}_{\mathsf{ch}}[i] + \Delta_i$ for $i\in I$ in $O(m)$ $\F$ operations
        \item Here $\Delta_i$ for all $i \in I$ is the change that will happen to $\vecT$ \textbf{during the current round}
        \item Prover computes the commitment to the new cache polynomial:
        $$[\vec{T'}_{\mathsf{ch}}(X)]_1=[\vec{T}_{\mathsf{ch}}(X)]_1+\sum_{i\in I}\Delta_i[\mu_i(X)]_1$$ in
        $O(m)$ $\Gone$ operations.
        \item Prover also gets ${T}'$ as ${T'}[i]=T_b[i]+ \vec{T'_{\text{ch}}}[i]$ using the $T_b$ and the latest cache
        \item Prover computes the commitment to the new table $\vecT'$: $[T'(X)]_1=[\Tbasepoly{X}]_1+[\vec{T'}_{\mathsf{ch}}(X)]_1$

        \item In addition, the other things (apart from $[Q(X)]_2$) needed for the current round of the lookup protocol are also computed by the prover as described in the lookup protocol in section 5.1 as it is just naive computation

    \end{enumerate}
    \subsection{Computation of $[Q(X)]_2$}
    Clearly, it suffices to efficiently compute $[Q(X)]_2$ where $[Q(X)]_2=\gtwo{\frac{T(X)-T_I(X)}{Z_I(X)}}$. We have the information of $[\Tbasepoly{X}-\Tbasepoly{\xi^i}/(X-\xi^i)]_2$. For this, we have the following lemma:
\end{comment}
The following Theorem determines the efficiency of the online phase of our prover.
\begin{theorem}\label{thm:approx-setup}
Let $N,\xi$ be as defined previously. Suppose we are given
$\kzg$ proofs $\{W_i\}_{i=1}^N$ with $W_i=\gany{\Tbasepoly{X} - \Tbasepoly{\xi^i}/(X-\xi^i)}$, where
$\Tbasepoly{X}=\enc{T_{\mathsf{b}}}{\setN}$ encodes a vector $\vecTbase\in \F^N$.
Let $I \subset [N]$, $\setN_I=\{\xi^i:i\in I\}$, $Z_I(X)$ denote the vanishing polynomial of $\setN_I$ and
$T_I(X)$ be the restriction of polynomial $T(X)$ on $\setN_I$.
Then, there exists an algorithm to compute $\kzg$ multi-opening proof
$\gany{Q(X)}=\gany{(T(X) - T_I(X)/Z_I(X)}$ for encoding $T(X)=\enc{T}{\setN}$ of vector $\vecT\in \F^N$ using $O((\delta + |I|) \log^2 (\delta + |I|))$ $\F$-operations
and $O(\delta + |I|)$ $\mathbb{G}$-operations. Here, $\delta$ denotes the hamming distance
between vectors $\vecTbase$ and $\vecT$.
\end{theorem}
\begin{proof}
    Let $\vecT=\vecTbase+\vecTcache$ and thus $T(X)=\Tbasepoly{X}+\Tcachepoly{X}$.
    Define $K=I\cup \{j\in [N]: \vecTcache[\,j\,]\neq 0\}$ as a set which captures the indices where the current table $\vecT$ differs from the base $\vecTbase$,
    where we explicitly also include the lookup indices $I$ in $K$. For $j\in K$, let $\vecTcache[j]=\Delta t_j$. Then $\Tcachepoly{X}=\sum_{j\in K}\Delta t_j\mu_j(X)$.
    %By definition of $K$, $|K|\leq \delta +|I|$. So, we need to bound $\Gtwo$ operations by $O(|K|)$ and field operations by $O(|K| \log^2|K|)$\\
    %First of all note that:
    We write the quotient $Q(X)$ as:
        {\small
    \begin{equation*}
        \begin{aligned}
            Q(X) = \sum_{i\in \setind}\frac{1}{z_I'(\xi^i)}\left(\frac{\Tbasepoly{X} - \Tbasepoly{\xi^i}}{X-\xi^i}\right)
            + \sum_{i\in \setind}\frac{1}{z_I'(\xi^i)}\left(\frac{\Tcachepoly{X} - \Tcachepoly{\xi^i}}{X-\xi^i}\right)
        \end{aligned}
        %\label{eq:Q2-upd}
    \end{equation*}
    }

    From above, we have $\gany{Q(x)}=\gany{\Qbasepoly{x}}+\gany{\Qcachepoly{x}}$ where
    \begin{gather*}
        \Qbasepoly{X}=\sum_{i\in \setind}(Z_I'(\xi^i))^{-1} (\Tbasepoly{X}-\Tbasepoly{\xi^i})/(X-\xi^i) \\
        \Qcachepoly{X}=\sum_{i\in \setind}(Z_I'(\xi^i))^{-1} (\Tcachepoly{X}-\Tcachepoly{\xi^i})/(X-\xi^i)
    \end{gather*}
    We can compute
    $\gany{\Qbasepoly{X}}$ from the pre-computed KZG openings of $\Tbasepoly{X}$ at points $\xi^i,i\in I$ using $O(|I|)$ group operations and
    $O(|I|\log^2 |I|)$ field operations. Therefore, it suffices to compute $\gany{\Qcachepoly{X}}$ efficiently.
    %\textbf{Thus, it suffices to describe the computation for $\elttwo{\Qcachepoly{X}}$. }\\
    Using $\Tcachepoly{X}=\sum_{j\in K}\Delta t_j\mu_j(X)$ and setting $c_i=(1/z_I'(\xi^i))$ for $i\in I$,
    we write $\Qcachepoly{X}$ as linear combination of table-independent polynomials:
    \begin{align*}
        \Qcachepoly{X} &= \sum_{i\in \setind} c_i\sum_{j\in K} \Delta t_j\frac{\mu_j(X)-\mu_j(\xi^i)}{X-\xi^i} \\
        &= \sum_{i\in \setind} c_i\Delta t_i\frac{\mu_i(X) - 1}{X-\xi^i} + \sum_{i\in \setind}\sum_{j\in K\setminus\{i\}}c_i\Delta t_j\frac{\mu_j(X)}{X-\xi^i}
    \end{align*}
    Now, we can write $\gany{\Qcachepoly{X}}=\elany{\Qcachepolyone{X}} + \elany{\Qcachepolytwo{X}}$ where:
        {\small
    \begin{gather*}
        \Qcachepolyone{X}=\sum_{i\in \setind}c_i\Delta t_i\frac{\mu_i(X)-1}{X-\xi^i},\,
        \Qcachepolytwo{X}=\sum_{i\in \setind}\sum_{j\in K\setminus \{i\}} c_i\Delta t_j\frac{\mu_j(X)}{X-\xi^i}
    \end{gather*}
    }
    The term $\gany{\Qcachepolyone{X}}$ can be computed using $O(|I|)$ group operations by augmenting the setup with pre-computed
    $\kzg$ opening proofs of polynomials $\mu_i(X)$ at $\xi^i$ for $i\in [N]$. This adds $O(N)$ to the setup parameters, while the computation
    can be done in $O(N\log N)$ time with methods similar to existing pre-computed parameters. This eventually leaves us with $\elany{\Qcachepolytwo{X}}$.
    %That is by precomputing $[\frac{\mu_i(X)-1}{X-\xi^i}]_2$. This requires just $N$ more precomputations and can be done along with the other precomputations which are done in the lookup protocol\\
    Next, we synthesize the polynomial $\Qcachepolytwo{X}$ in a form that reduces group operations required to compute its encoding.
    %\textbf{Thus, it suffices to describe the computation for $\elttwo{\Qcachepolytwo{X}}$}:
    \begin{align}\label{eq:Qcachepoly2}
    &\Qcachepolytwo{X} = \sum_{i\in \setind}c_i\sum_{j\in K\setminus \{i\}} \Delta t_j\mu_j(X)/(X-\xi^i) \nonumber \\
    &\quad = \sum_{i\in\setind}c_i\sum_{j\in K\setminus \{i\}}\frac{\Delta t_j}{Z_{\nroots}'(\xi^j)} \frac{Z_{\nroots}(X)}{(X-\xi^i)(X-\xi^j)} \nonumber \\
    %\intertext{Above, we expanded $\mu_j(X)$. Now using $Z_\nroots'(\xi^j)=N\xi^{-j}$ and using partial fractions}
    &\quad = N^{-1}\sum_{i\in\setind}c_i\sum_{j\in K\setminus \{i\}}\frac{\xi^j\Delta t_j}{\xi^i-\xi^j}
    \left(\frac{Z_\nroots(X)}{X-\xi^i} - \frac{Z_\nroots(X)}{X-\xi^j}\right) \nonumber \\
    &\quad = N^{-1}\sum_{i\in\setind}\left(c_i\cdot \sum_{j\in K\setminus \{i\}} \frac{\xi^j\Delta t_j}{\xi^i-\xi^j}\right)\frac{Z_\nroots(X)}{X-\xi^i} \nonumber \\
    &\qquad + \sum_{j\in K}\left(\xi^j\Delta t_j\cdot \sum_{i\in \setind\setminus \{j\}}\frac{c_i}{\xi^j-\xi^i}\right)\frac{Z_{\nroots}(X)}{X-\xi^j}
    \end{align}
    In the first step, we substituted $\mu_j(X)$, while in the final step we re-arranged the summation to accumulate the scalar factor for
    each distinct polynomial of the form $\vpolyN(X)/(X-\xi^i)$. Define scalars $a_i$, $i\in I$ and $b_j$, $j\in K$ as below:
    %this last equality, the first term is just the first term of the distributive property in finite fields.\\
    %The second term is just the second term of the distributive property in finite fields except that the order of the sums is reversed. This follows from the following fact \\
    %\begin{fact}
    %    $\sum_{i \in I} \sum_{j \in K \setminus \{i\}} f(i,j)=\sum_{j \in K} \sum_{i \in I \setminus \{j\}} f(i,j) $
    %\end{fact}
    %In the above equation \eqref{eq:Qcachepoly2}, let us define:
    \begin{gather}\label{eq:scalars}
    a_i = \sum_{j\in K\setminus \{i\}}\frac{\xi^j\Delta t_j}{\xi^i-\xi^j}, i\in \setind\quad
    b_j=  \sum_{i\in \setind\setminus \{j\}}\frac{c_i}{\xi^j - \xi^i}, j\in K
        %W_3^i(X) = \frac{Z_\nroots(X)}{X-\xi^i}, \text{ for } i\in [N]
    \end{gather}
    Now, recalling that
    $W_3^j=\gany{\vpolyN(X)/(X-\xi^i)}$, we see that $\elany{\Qcachepolytwo{X}}$ can be written as linear combination of $O(|K|+|I|)$ group elements.
    \begin{equation}\label{eq:Qcachepoly2commit}
    \gany{\Qcachepolytwo{X}} = N^{-1}\left(\sum_{i\in\setind}(c_ia_i)\cdot W_3^i + \sum_{j\in K}(\xi^j \Delta t_j b_j)\cdot W_3^j\right)
    \end{equation}
    Now $c_i=(z_I(\xi^i))^{-1}$ for all $i \in \setind$ can be determined in $O(|I|\log^2 |I|)$ field operations by evaluating $Z_{\setind}'(X)$ on
    $H_{\setind}$. So, given $\{a_i\}_{i\in I}, \{b_j\}_{j\in K}$, $\gany{\Qcachepolytwo{X}}$ can be computed in $O(|\setind|+|K|)$ group operations.
    While we have diligently reduced the group operations, we still seem to need $O(|I||K|)=O(m\delta)$ field operations. We clearly need better than
    naive way of computing the scalars in \eqref{eq:scalars} to obtain additive overhead in $\delta$. This is what we consider next.
    Routine calculation shows that we can write $a_i$ as:
    \begin{equation*}
        a_i = -\Delta T + \Delta t_i + \xi^i\sum_{j\in K\setminus\{i\}}\frac{\Delta t_j}{\xi^i-\xi^j}, i\in I
    \end{equation*}
    where in the above, we have $\Delta T=\sum_{j\in K}\Delta t_j$.
    %a_i = -\sum_{j\in K\setminus \{i\}}\Delta t_j + \xi^i\sum_{j\in K\setminus \{i\}}\frac{\Delta t_j}{\xi^i-\xi^j} $$
    %This is because:
    %$$a_i+\sum_{j\in K\setminus \{i\}}\Delta t_j= \sum_{j \in K \setminus \{i\}}\frac{\xi^j\Delta t_j}{\xi^i-\xi^j}+\Delta t_j$$
    %$$=\sum_{j \in K \setminus \{i\}}\frac{\xi^i\Delta t_j}{\xi^i-\xi^j} = \xi^i\sum_{j\in K\setminus \{i\}}\frac{\Delta t_j}{\xi^i-\xi^j}$$
    %Now, define $\Delta T=\sum_{j\in K}\Delta t_j$\\

    %Here computing $\Delta T$ is a one time computation (per batch). It can be computed from the knowledge of $T_{\text{ch}}$ in the online phase.
    %We have:
    %$$  $$
    %Suppose we get $\sum_{j\in K\setminus\{i\}}\frac{\Delta t_j}{\xi^i-\xi_j}$ for all $i \in I$ efficiently. Then $a_i$ for all $i \in I$ can be obtained in $O(|I|)$ field operations. \\
    %\textbf{Thus, to get $a_i$ for all $i \in I$ it suffices to describe the computation of $e_i=\sum_{j\in K\setminus\{i\}}\frac{\Delta t_j}{\xi^i-\xi^j}$ for all $i \in I$}\\\\
    Above implies that to compute $a_i, i\in I$ efficiently, it is sufficient to efficiently
    compute $e_i=\sum_{j\in K\setminus\{i\}}\frac{\Delta t_j}{\xi^i-\xi^j}$ for all $i \in I$. Our next lemma claims that
    such {\em reciprocal sums} can be computed efficiently. The computation of $b_j,j\in K$ can also be reduced a similar computation.
    We defer this reduction and the full proof of Lemma ~\ref{lem:sum-computation} to the Appendix, but illustrate the key ideas in the proof.
    %Our requirement is now to bound the number of field operations for $e_i$ and for $b_j$. For this, we invoke the following lemma with the proof in the appendix.
    \begin{lemma}\label{lem:sum-computation}
    Let $I\subset K\subset [N]$ and let $e_i$ for all $i \in \setind$ and $b_j$ for all $j \in K$ be as described above.
    Then, $e_i$ for all $i \in I$ and $b_j$ for all $j \in K$ can be computed in $O(|K|\log^2|K|)\, \mathbb{F}$ operations
    \end{lemma}
    \begin{proof}[Proof-Sketch]
        First, we mention that the special case of the lemma when $\Delta t_j=1$ for all $j\in K$ admits an efficient computation due to the following identity
        proved in Lemma ~\ref{lem:sumtoder}.
        \begin{equation*}
            \frac{Z_K''(\xi^i)}{Z_K'(\xi^i)} = 2\sum_{j\in K\setminus \{i\}}\frac{1}{\xi^i-\xi^j}
        \end{equation*}
        for $Z_K(X)=\prod_{i\in K}(X-\xi^i)$. The polynomial $Z_K$ can be computed in $O(|K|\log^2|K|)$ and subsequent evaluations of its first two
        derivatives can also be evaluated on the set $\{\xi^i: i\in I\}$ with the same complexity. However, to deal with arbitrary values of $\Delta t_j$ we
        need more ingenuity. We will {\em imagine} $\Delta t_j$ to be $p(\xi^j)$ for some polynomial $p(X)$. Moreover, we demand that $p(\xi^j)=0$ for $j\not\in K$.
        We will not compute such a polynomial $p$, as it has degree $O(N)$, but view it as an ``oracle'' which we can hopefully query at the points we need.
        Then it can be seen that $e_i=g_i(\xi^i) - r_i(\xi^i)$ for rational functions $g_i(X)$ and $r_i(X)$ defined by:
        \begin{align}\label{eq:rat-fun-f}
        g_i(X) =\sum_{j\in [N]\setminus i}\frac{p(X)}{X-\xi^j},\quad
        r_i(X) =\sum_{j\in [N]\setminus i} \frac{p(X)-p(\xi^j)}{X-\xi^j}
        \end{align}
        Now, $g_i(\xi^i)$ for $i\in I$ turns out to be (using the special case above):
        $$p(\xi^i)\sum_{j\in K\setminus \{i\}} 1/(\xi^i-\xi^j)=\Delta t_i (Z_K''(\xi^i)/Z_K'(\xi^i))/2$$
        Defining $u(X,Y)=(p(X)-p(Y))/(X-Y)$, we can write $r_i(\xi^i)$ as:
        \begin{equation}
            r_i(X) = \sum_{j\in [N]}u(X,\xi^j) - u(X,\xi^i)
        \end{equation}
        Using the fact that $u(X,X)=p'(X)$ gives the formal derivative of polynomial $p(X)$, we get
        $r_i(\xi^i)=r(\xi^i)-p'(\xi^i)$ for all $i\in I$, where $r(X)=\sum_{j\in [N]}u(X,\xi^j)$. Fortunately,
        $r(X)$ is simply $Nu(X,0)=N(p(X) - p(0))/X$, a fact that follows from univariate sum-check. The problem
        thus reduces to being able to compute derivates $p'(\xi^i)$ for $i\in I$ and the value $p(0)$. In the
        complete proof, we show how we leverage special structure of the high degree polynomial $p(X)$ to compute
        the above efficiently, without computing $p(X)$ itself. This concludes the proof-sketch of the lemma.
    \end{proof}

    From the lemma ~\ref{lem:sum-computation}, we conclude that the scalars $a_i,i\in I$ and $b_j, j\in K$ can be computed in
    time $O(|K|\log^2 |K|)$, which proves the bound in Theorem ~\ref{thm:approx-setup}.
    %have shown that the field operations needed to get $e_i$ and $b_j$ and thus $\elttwo{\Qcachepolytwo{X}}$ is $O(|K|\log^2|K|)$. \\

    %This completes the proof of Lemma \ref{lem:approx-setup}.
\end{proof}

\subsection{Amortized Sublinear Batching}\label{sec:amortization}
We now return to the question of how frequently should we run the offline phase to compute full parameters.
For concrete analysis, let $s$ be the period after which the rebasing takes place; i.e., after $s$ batches of $m$ operations
each, we set the base table $\vecTbase$ to the current table, setting $\vecTcache=\vec{0}$. At this point we also compute all
encoded quotients for $\vecTbase$ using the $O(N\log N)$ algorithm of ~\cite{EPRINT:FeiKho23}. Consider $\delta\leq ms$ as
the upper-bound on $\delta$, and distributing the cost of re-basing, the amortized overhead for the batch of $m$ operations is:
$O(ms \log^2(ms)+\frac{N\log N}{s})$ $\mathbb{F}$-operations and $O(ms +\frac{N\log N}{s})$ $\mathbb{G}$-operations. Ignoring
the logarithmic factors, the cost is minimized by setting $s\approx \sqrt{N/m}$, resulting in amortized prover overhead of
$\wt{O}(\sqrt{mN})$. We note that the above analysis considers the worst case scenario, where each update affects a distinct
position in the table. In settings, where frequency of updates is non-uniform accross positions in the table (e.g, in the
blockchain example, if bulk of transactions come from small number of clients), we may be able to defer the offline phase even longer.
Same is also true for settings where updates to the table are infrequent.