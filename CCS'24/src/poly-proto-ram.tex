In this section, we formalize the notions of RAM, RAM operations and
execution transcripts etc., and subsequently introduce relevant relations over them.
We have earlier used a vector $\vecT\in \F^n$ to model a RAM of size $n$, where the $i^{th}$ entry $\vecT[\,i\,]$
implicitly corresponds to index (address) $i$. Here, we will consider a generalisation that will be useful later.
We will allow a RAM to explicitly associate a value $v\in \F$ to an index $a$ from an {\em index space}
$\mathcal{I}\subseteq \F$, such that the association is unambiguous.
\begin{definition}[RAM]\label{defn:RAM}
    Given $n\in \N$, finite field $\F$ and a set $\mathcal{I}\subseteq \F$, a RAM of size $n$ over indices $\mathcal{I}$
    is a tuple $T=(\vec{a},\vec{v})\in \mathcal{I}^n\times \F^n$ such that $\forall\, i,j\in [n]$  $v_i=v_j$ whenever $a_i=a_j$.
    We think of $T$ as a table with vectors $\vec{a}$ and $\vec{v}$ denoting its columns. The set of all such
    tables will be denoted by $\RAM{I}{n}$.
    %\begin{align*}
    %        T[\,a\,] = \begin{cases}
    %                   v \text{ if } \, \exists i\in [n] \text{ s.t. } (a,v) = (a_i,v_i), \\
     %                  \bot \text{ otherwise }
    %                \end{cases}
    %\end{align*}
\end{definition}

For a table $T=(\vec{a},\vec{v})\in \RAM{I}{n}$, we refer to tuples $(a_i,v_i)$, $i\in [n]$ as records of the table $T$.
We use the access notation $v=T[a]$ to mean that $(a,v)$ is a record of $T$.
Note that our definition allows identical records to be repeated. When we refer to a vector $\vecT\in \F^n$ as a RAM, we implicitly
assume $T=(\setind_n,\vecT)$ where $\setind_n=(1,2,\ldots,n)$ and we have $\vecT[i]=t_i$.
For a RAM $T\in \RAM{I}{n}$, a RAM operation is a three tuple $(\op,a,v)$ with $\op\in \{0,1\}$,
$a\in \setind$ and $v\in \F$. An operation with $\op=0$ is called a {\em load} operation which denotes reading a value $v$
mapped to index $a$ in the RAM. Similarly, an operation with $\op=1$ is called a {\em store} operation,
which denotes associating the value $v$ with index $a$ in the RAM.
We use $\RAMOp{I}$ to denote the set of all RAM operations with index set $\setind$.
%We say than an operation
%$(\op,a,v)\in \{0,1\}\times \setind\times \F$ is {\em admissible} with respect to RAM $T\in \RAM{I}{n}$ if
%$T[\,a\,]=v$ whenever $\op=0$ (i.e, the operation loads a value which agrees with the value in the RAM at the specified index).

\subsection{Correctness of RAM Update}\label{subsec:ram-update}
The versatility of the RAM primitive stems from its updatability. While a load operation leaves the RAM unchanged, the store operation
updates the value in the RAM associated with the referenced index. We model the update via the function
$\Rupd{I}$ which takes RAM $T\in \RAM{I}{n}$, operation
$o=(\op,a,v)\in \RAMOp{I}$ as inputs and returns an updated RAM $T'\in \RAM{I}{n}$.
The updated RAM $T'=\Rupd{I}(T,o)$ satisfies
$T'=T$ if $\op=0$ while for $\op=1$ it satisfies $T'[\,a\,]=v$  and $T'[\,x\,]=T[\,x\,]$ for $x\neq a$. The central problem
in verifiable RAM protocols is to establish that a sequence of operations $\vec{o}=(o_1,\ldots,o_m)$ are correct with
respect to the initial RAM state $T$ and the final RAM state $T'$. This involves ensuring
that all load operations read the value which is consistent with updates to the RAM as a result of preceding
store operations, and that $T'$ is the final state. We say that an operation $o=(\op,a,v)$ is {\em load-consistent}
with respect to RAM $T$ if $v=T[a]$ whenever $o$ is a load operation (store operations are vacuously defined to be load-consistent).
We formally define the notion of consistency below:

\begin{definition}[Consistent Operations]\label{defn:consistent-operations}
    Let $n\in \N$ and $T,T'\in \RAM{I}{n}$ for some index set $\setind$. We say that a sequence of operations
    $\vec{o}=(o_1,\ldots,o_k)\in \RAMOp{I}^k$ over $\setind$ is {\em consistent} with RAM states
    $T,T'$ if for all $i\in [k]$, $T_{i}=\Rupd{I}(T_{i-1},o_i)$ and operation $o_i$ is load-consistent with respect to $T_{i-1}$. Here
    we assume $T_0=T$ and $T_k=T'$.
\end{definition}

For $m,n\in \N$, let $\LRAM{I}{m}{n}$ denote the language consisting of tuples $(T,\vec{o},T')$ with $T,T'\in \RAM{I}{n}$ and $\vec{o}\in (\RAMOp{I})^m$
such that $\vec{o}$ is consistent with $T,T'$.
Next, we formalize the folklore technique of checking correctness of RAM operations
using {\em address-ordered transcripts}.

\subsection{Consistency Check via Transcripts}\label{subsec:transcripts}
A {\em transcript} is time-stamped sequence of operations executed on a RAM.
More formally, given a RAM $T=(\vec{a},\vec{v})\in \RAM{I}{n}$,
operation sequence $\vec{o}=(o_1,\ldots,o_m)$ with $o_i=(\bar{\op_i},\bar{a}_i,\bar{v}_i)\in \RAMOp{I}$ and RAM $T'=(\vec{a}', \vec{v}')\in \RAM{I}{n}$,
the {\em time ordered transcript} for the tuple $(T,\vec{o},T')$ is given by the table $\tr$ with $k=2n+m$ rows and four columns
$\tr = (\vec{t},\vec{\op},\vec{A},\vec{V})$  defined as follows: (i) $\vec{t}=\setind_{k}=(1,\ldots,k)$,
(ii) $\vec{\op}=0^n||(\bar{\op}_1,\ldots,\bar{\op}_m)||0^n$,
(iii) $\vec{A}=\vec{a}||(\bar{a}_1,\ldots,\bar{a}_m)||\vec{a}'$ and
(iv) $\vec{V}=\vec{v}||(\bar{v}_1,\ldots,\bar{v}_m)||\vec{v}'$. The $i^{th}$ row of the table $\tr$ is
$(t_i,\op_i,A_i,V_i)$ for $i\in [k]$. The first $n$ records in $\tr$ correspond to the contents of $T$,
the next $m$ records correspond to the operations in $\vec{o}$ and final $n$
records correspond to contents of $T'$. The timestamp column $\vec{t}$ is added to order operations with the same index.
Notationally, we write $\tr=\TOT(T_0,\vec{\op},T_1)$.

We call a transcript $\tr=(\vec{t},\vec{\op},\vec{A},\vec{V})$ to be {\em address ordered} if $A_i\leq A_{i+1}$ for $i\in [k-1]$ and
$t_i < t_{i+1}$ whenever $A_i=A_{i+1}$. For a transcript $\tr=(\vec{t},\vec{\op},\vec{A},\vec{V})$ with $k$ records and a
permutation $\sigma:[k]\rightarrow [k]$, we use $\sigma(\tr)$ to denote the transcript
$(\sigma(\vec{t}),\sigma(\vec{\op}),\sigma(\vec{A}),\sigma(\vec{V}))$
obtained by permuting the records of $\tr$ according to the permutation $\sigma$.
An address ordered transcript for tuple $(T,\vec{o},T')$ is defined as $\tr^\ast=\sigma(\tr)$ where $\tr=\TOT(T,\vec{o},T')$ and $\sigma$ is
a permutation such that $\tr^\ast$ is an address ordered. We denote it by $\tr^\ast=\AOT(T,\vec{o},T')$.
We say that an address ordered transcript $\tr=(\vec{t},\vec{\op},\vec{A},\vec{V})$ satisfies {\em load-store correctness}
if for all pairs of consecutive records $(t_i,\op_i,A_i,V_i)$ and $(t_{i+1},\op_{i+1},A_{i+1},V_{i+1})$ we have $V_{i+1}=V_i$
whenever $\op_{i+1}=0$ (load operation) and
$A_i=A_{i+1}$, i.e, a load operation does not change the value at an index.
We formally state the folklore technique for enforcing memory consistency in our setting.
\begin{lemma}\label{lem:consistency-check}
Let $\F$ be a finite field, $m,n\in \N$ be positive integers and $\setind\subseteq \F$. Then $(T,\vec{o},T')\in \LRAM{I}{n}{m}$ if and only if
    the address ordered transcript $\tr^\ast=\AOT(T,\vec{o},T')$ satisfies load-store correctness.
\end{lemma}

\subsection{Polynomial Encoding}\label{subsec:poly-encoding}
The aim of this section is to encode artefacts such as RAM state, operations and transcripts as polynomials, and
translate checking memory consistency (equivalently, checking membership in $\LRAM{I}{n}{m}$) to checking polynomial identities
over the encoded polynomials. For
simplicity and its usefulness later, we consider the case $m=n$, and accordingly check the membership in the language $\LRAM{I}{m}{m}$.
Let $k=3m$ and let $\omega$ be the $k^{th}$ root of unity in $\F$.
Let $\nu=\omega^3$, and thus $\nu$ is an $m^{th}$ root of unity in $\F$ (We assume, these roots exist in $\F$).
We define sets $\setH$ and $\setV$ below consisting of $k^{th}$ and $m^{th}$ roots of unity respectively.
\begin{gather}\label{eq:interpolation-sets}
    \setH = \{\omega,\ldots,\omega^k\},\quad \setV = \{\nu,\ldots,\nu^m\}
\end{gather}
Let $\{\lambda_i(X)\}_{i=1}^k$ be lagrange basis polynomials
for the set $\setH$ and $\{\tau_i(X)\}_{i=1}^m$ be the lagrange polynomials for the set $\setV$ satisfying
$\lambda_i(\omega^j)=\delta_{ij}$ for $i,j\in [k]$ and $\tau_i(\nu^j)=\delta_{ij}$ for $i,j\in [m]$.
We use $\setH$ to encode a vector $\vec{f}=(f_1,\ldots,f_k)$ of size $k$ as the polynomial $f(X)\in \F_{<k}[X]$ such that $f(\omega^i)=f_i$ for $i\in [k]$.
Similarly, we use $\setV$ to encode a vector
$\vec{g}=(g_1,\ldots,g_m)$ of size $m$ as the polynomial $g(X)\in F_{<m}[X]$ such that $g(\nu^i)=g_i$ for $i\in [m]$. We use
the notation $\enc{f}{\setH}$ and $\enc{g}{\setV}$ to denote polynomial encodings of vectors $\vec{f}\in \F^k$ and $\vec{g}\in \F^m$
respectively.
In the other direction, for a polynomial $f(X)\in \F[X]$,
we use $\vec{f}_{|_\setH}$ and $\vec{f}_{|_\setV}$ to denote the vectors $(f(\omega^1),\ldots,f(\omega^k))$ and $(f(\nu^1),\ldots,f(\nu^m))$ respectively.
The encoding of vectors as polynomials can be succinctly described using lagrange basis polynomials.
Then we have $\enc{f}{\setH}=\sum_{i=1}^k f_i\lambda_i(X)$ and $\enc{g}{\setV}=\sum_{i=1}^m g_i\tau_i(X)$.

We extend the encoding of vectors to encode RAM, operations and transcripts.
For a RAM $T=(\vec{a},\vec{v})\in \RAM{I}{m}$, we define its encoding
$\wt{T}=(a(X),v(X))$ where $a(X),v(X)\in \F_{<m}[X]$ encode vectors $\vec{a}, \vec{v}$ respectively.
Given an operation sequence
$\vec{o}=(o_1,\ldots,o_m)$ with $o_i=(\bar{\op}_i,\bar{a}_i,\bar{v}_i)$ we encode $\vec{o}$ as $\wt{O}=(\bar{\op}(X),\bar{a}(X),\bar{v}(X))$
where $\bar{\op}(X)$ encodes the
vector $\vec{\op}=(\bar{\op}_1,\ldots,\bar{\op}_m)$, $\bar{a}(X)$ encodes the vector $(\bar{a}_1,\ldots,\bar{a}_m)$ and $\bar{v}(X)$
encodes the vector $(\bar{v}_1,\ldots,\bar{v}_m)$.
Finally, a transcript $\tr=(\vec{t},\vec{\op},\vec{A},\vec{V})$ for tuples $(T,\vec{o},T')$ where $T,T'$ are RAMs of size $m$, and $\vec{o}$ is an
operation sequence of size $m$ is encoded as $\wt{\tr}$ $=$ $(t(X)$, $\op(X)$, $A(X)$, $V(X))$ where the polynomials
encode the respective vectors in $\F^k$.

\subsection{Relations over Polynomial Encodings}\label{subsec:encoded-relations}
In this section, we describe polynomial checks for two important relations we need in subsequent sections, viz,
(i) checking concatenation of vectors and (ii) checking monotonicity and load-store consistency of a transcript. Let
$k,m$ be integers with $k=3m$ as in previous section. The lemma below specifies the polynomial checks for verifying
that vector $\vec{v}\in \F^k$ is concatenation of vectors $\vec{a},\vec{b},\vec{c}$ in $\F^m$.

\begin{lemma}\label{lem:vec-concatenation}
Let $\vec{a},\vec{b},\vec{c}\in \F^m$  and $v\in \F^k$ be vectors encoded by polynomials
$a(X),b(X),c(X)$ and $v(X)$ respectively. Then,
{\small
\begin{align}
    a(X^3) - v(X)  &= 0  \quad \text{ mod $Z(X)$ } \tag{A1}\label{eq:A1}\\
    b(X^3) - v(\omega^m X)  &= 0   \quad \text{ mod $Z(X)$ } \tag{A2}\label{eq:A2}\\
    c(X^3) - v(\omega^{2m} X) &= 0 \quad \text{ mod $Z(X)$ } \tag{A3}\label{eq:A3}
\end{align}
}
for $Z(X)=\prod_{i=1}^m (X-\omega^i)$ if and only if $\vec{v}=\vec{a}||\vec{b}||\vec{c}$.
\end{lemma}
\begin{proof}
    Assume that the polynomial identities hold. Substituting $X=\omega^i$ for $i\in [m]$ in above equations implies
    for $i\in [m]$: $a_i=v_i$ (Eq \eqref{eq:A1}), $b_i=v_{m+i}$ (Eq \eqref{eq:A2}) and $c_i=v_{2m+i}$ (Eq \eqref{eq:A3}),
    which together imply $\vec{v}=\vec{a}||\vec{b}||\vec{c}$. Converse follows by observing that $\vec{v}=\vec{a}||\vec{b}||\vec{c}$
    implies that $v(X) = a(X^3)$, $v(\omega^m X)=b(X^3)$ and $v(\omega^{2m} X)=c(X^3)$ holds for all $X=\omega^i, i\in [m]$.
    Thus, the equalities hold modulo the polynomial $Z(X)$ as defined above.
\end{proof}

Next, we specify polynomial checks on the encoding of a transcript to ensure it satisfies address-ordering and load-store consistency.
Let $\tr=(\vec{t},\vec{\op},\vec{A},\vec{V})$ be a transcript encoded as
$\wt{\tr}=(t(X),\op(X),A(X),V(X))$. Recall that we need to check two conditions on $\tr$, viz (i) {\em monotonicty}:
the transcript is sorted by address and timestamp respectively, i.e, $A_i\leq A_{i+1}$ for all $i < k$ and
$t_i < t_{i+1}$ when $A_i=A_{i+1}$, (ii) {\em load-store consistency}: whenever $\op_{i+1}=0$ and $A_i=A_{i+1}$,
we have $V_i=V_{i+1}$. To do so, we exhibit disjoint sets $I_1,I_2$ with $I_1\uplus I_2=[k]$ such that: (i) for all
$i\in I_1$, $A_i < A_{i+1}$, (ii) for all $i\in I_2$, $(A_i = A_{i+1})\wedge (t_i < t_{i+1})$ and (iii) for all $i\in I_2$,
$(\op_i=1)\vee (V_i = V_{i+1})$. We note that the conditions on the sets $I_1$ and $I_2$ ensures monotonicity.
Moreover it can be seen that load-store consistency requirements are satisfied for all $i\in I_1$ (as $A_i\neq A_{i+1}$).
Similarly,load-store consistency also holds for all $i\in I_2$.
It remains to exhibit the sets and show that they satisfy the above invariants using polynomials, as in the following
lemma:
\begin{lemma}\label{lem:addr-ordered-transcript}
Let $\wt{\tr}$ be a polynomial encoding of transcript $\tr$ of size $k$, given by polynomials $t(X),\op(X),A(X)$ and $V(X)$.
Then for $kN<|\F|$, $\tr$ is address ordered and satisfies load-store consistency if and only if there exist polynomials
 $Z_1,Z_2,\delta_T,\delta_A$
such that the following hold:
{\small
\begin{align}%\label{eq:loadstore-consitency-constraints}
& Z_2(X)(A(\omega X) - A(X) - \delta_A(X)) = 0 \text{ mod } \mathbb{Z}_\setH(X) \tag{C1} \label{eq:C1} \\
& A(\omega X) - A(X) = 0  \text{ mod } Z_2(X) \tag{C2} \label{eq:C2} \\
& t(\omega X) - t(X) - \delta_T(X) = 0  \text{ mod } Z_2(X) \tag{C3} \label{eq:C3} \\
& (\op(X) - 1)(V(\omega X) - V(X)) = 0  \text{ mod } Z_2(X) \tag{C4} \label{eq:C4} \\
& Z_1(X)\cdot Z_2(X) = \mathbb{Z}_\setH(X) \quad  \tag{C5} \label{eq:C5} \\
& 1\leq A(\omega) < N  \quad \tag{C6} \label{eq:C6} \\
& 1\leq t(\omega^i) < N, 1\leq \delta_A(\omega^i)<N,\ 1\leq \delta_T(\omega^i)<N \text{ for } i\in [k] \tag{C7} \label{eq:C7}
\end{align}
}
\end{lemma}

\section{Succinct Argument for Verifiable RAM}\label{sec:poly-proto-ram}
The polynomial encodings in the previous section can be used to polynomial protocol for
checking the membership in the language $\LRAM{I}{m}{m}$ for $m\in \N$. The polynomial protocol can be subsequently
be compiled into a succinct argument using an extractable polynomial commitment scheme.
In this section, we use $\kzg$ polynomial commitment scheme to obtain a succinct argument for checking membership in $\LRAM{I}{m}{m}$
in the Algebraic Group Model (AGM).
At a high level, to prove $(\vec{T},\vec{o},\vec{T'})\in \LRAM{I}{m}{m}$, the prover
constructs time ordered transcript $\tr$ and then permutes it to obtain the address sorted transcript $\tr^\ast$.
It then sends the polynomial encodings of $\vec{T},\vec{o},\vec{T'},\tr$ and $\tr^\ast$ to the verifier, who verifies that:
\begin{enumerate}[leftmargin=1em]
\item The time ordered transcript is correctly constructed, i.e, $\tr=\TOT(\vec{T},\vec{o},\vec{T'})$.
\item The transcript $\tr^\ast$ is a permutation of the transcript $\tr$, i.e, $\tr^\ast=\sigma(\tr)$ for some permutation $\sigma$ of $[k]$.
\item The transcript $\tr^\ast$ is address ordered and satisfies load-store consistency.
\end{enumerate}
In fact, we consider memebership in the above language under commitments. Let $\srs$ denote a $\kzg$ setup over a bilinear group, with
prime order groups $\Gone, \Gtwo$ and $\GT$. We canonically commit to RAM, operation sequences and transcripts by committing to their
polynomial encodings. Commitment of an encoding represented as tuple of polynomials is simply the tuple consisting of commitments of the component
polynomials. We now define
the relation $\CLRAM$ below, and present a succinct argument for the same.
\begin{definition}\label{defn:committed-vram}
Let $\CLRAM$ consist of tuples $((c_T, c_o, c_T'), (T, \vec{o},T'))$ where $c_T$ $=$ $\kzgcommit(\srs, \wt{T})$,
$c_T'$ $=$ $\kzgcommit(\srs, \wt{T'})$,
$c_o$ $=$ $\kzgcommit(\srs,\wt{O})$ commit to $T$, $T'$ and $\vec{o}$ with $(T,\vec{o},T')\in \LRAM{I}{m}{m}$.
\end{definition}
\noindent In the above definition we have $c_T=(c_a,c_v)$ where $c_a$ and $c_v$ are $\kzg$ commitments to polynomials $a(X)$ and $v(X)$ in
the encoding $\wt{T}=(a(X), v(X))$. Similarly we parse $c_T'=(c_a', c_v')$ and $c_o=(\bar{c}_\op, \bar{c}_a, \bar{c}_v)$ (see Section
\ref{subsec:poly-encoding} for polynomial encodings).
For proving relation \ref{defn:committed-vram}, prover's input consists of initial RAM state $T=(\vec{a},\vec{v})$,
final RAM state $T'=(\vec{a'},\vec{v'})$, operation sequence $\vec{o}=(o_1,\ldots,o_m)$ with $o_i=(\bar{\op}_i,\bar{a}_i,\bar{v}_i)$,
time-ordered transcript $\tr=(\vec{t},\vec{\op},\vec{A},\vec{V})$ and address-ordered transcript $\tr^\ast=(\vec{t^\ast},\vec{\op^\ast},
\vec{A^\ast},\vec{V^\ast})$ obtained from $\tr$ using a permutation $\sigma:[k]\rightarrow [k]$. Verifier's input consists of the
commitments $c_T, c_o$ and $c_T'$ as described above.

The prover starts the protocol by sending commitments $c_\tr$ and $c^\ast_\tr$ to the transcripts $\tr$ and $\tr^\ast$ respectively.
To show that $\tr$ is correctly formed, the prover needs to prove the concatenations:
(i) $\vec{\op}=0^m||(\bar{\op}_1,\ldots,\bar{\op}_m)||0^m$, (ii) $\vec{A}=\vec{a}||(\bar{a}_1,\ldots,\bar{a}_m)||\vec{a'}$
and (iii) $\vec{V}=\vec{v}||(\bar{v}_1,\ldots,\bar{v}_m)||\vec{v'}$. Note that the time-stamp column $\vec{t}$ is implicitly assumed
to be $(1,\ldots,k)$.
The verifier checks the concatenations using Lemma \ref{lem:vec-concatenation}.
It uses a random challenge $\beta$ to reduce the three concatenations to one concatenation, and uses another challenge $\gamma$
to reduce the three polynomial checks in Lemma \ref{lem:vec-concatenation} to a single check.
The complete polynomial protocol is detailed in Figure \ref{fig:time-ordered-transcript}.

%\begin{tcolorbox}
\begin{figure}[htbp]
\centering
\begin{mdframed}
{\footnotesize
\noindent{\bf Common Input}: Commitments $c_T=(c_a,c_v)$, $c_o=(\bar{c}_\op, \bar{c}_a, \bar{c}_v)$, $c_T'=(c_a', c_v')$
and $c_\tr=(c_t, c_\op, c_A, c_V)$ to $T,o,T'$ and $\tr$ respectively. Commitment $\gone{Z}$ to the polynomial
$Z(X)=\prod_{i=1}^m (X-\omega^i)$.
\begin{enumerate}[leftmargin=2em]
\item $\verifier\rightarrow\prover$: Verifier sends $\beta,\gamma\gets \F$.
\item $\prover$ computes:
\begin{align}\label{eq:poly-constraints}
    & G_1(X) = a(X) + \beta v(X),\, G_2(X) = \bar{a}(X) + \beta \bar{v}(X) + \beta^2 \bar{\op}(X) \tag{A1} \\
    & G_3(X) = a'(X) + \beta v'(X),\, G(X) = A(X) + \beta V(X) + \beta^2 \op(X) \tag{A2} \\
    & H(X) = G_1(X) + \gamma G_2(X) + \gamma^2 G_3(X), \tag{A3} \\
    & Q(X) = (H(X^3) - G(X) - \gamma G(\omega^m X) - \gamma^2 G(\omega^{2m} X))/Z(X) \tag{A4}
\end{align}
\item $\prover\rightarrow\verifier$: The prover sends commitment $\gone{Q}$ to $Q(X)$.
\item $\verifier\rightarrow\prover$: Sends $s\gets\F$.
\item $\prover\rightarrow\verifier$: Sends evaluations $\val{s}{G}=G(s)$, $\val{\omega^m s}{G}=G(\omega^m s)$,
$\val{\omega^{2m}s}{G}=G(\omega^{2m} s)$, $\val{s^3}{H}=H(s^3)$, $\val{s}{Q}=Q(s)$ and $\val{s}{Z}=Z(s)$.
\item $\verifier\rightarrow\prover$: Sends $r\gets\F$.
\item $\prover\rightarrow\verifier$: Sends $\kzg$ proofs:
\begin{itemize}[leftmargin=2em]
\item $\Pi_G=\kzgprove(\srs,G,(s,\omega^m s, \omega^{2m}s))$.
\item $\Pi_H=\kzgprove(\srs,H,s^3)$.
\item $\Pi_F=\kzgprove(\srs,F,s)$ where $F(X)=Z(X) + rQ(X)$.
\end{itemize}
\item $\verifier$ computes: Compute commitments $\gone{G},\gone{H}$ and $\gone{F}$ using homomorphism. It then checks:
\begin{itemize}[leftmargin=2em]
\item $\kzgverify(\srs,\gone{G},(s,\omega^m s, \omega^{2m}s), (\val{s}{G}, \val{\omega^m s}{G}, \val{\omega^{2m} s}{G}),\Pi_G)$.
\item $\kzgverify(\srs,\gone{H}, s^3, \val{s^3}{H},\Pi_H)$.
\item $\kzgverify(\srs,\gone{F}, s, \val{s}{Z} + r\val{s}{Q}, \Pi_F)$.
\item $\val{s}{Q}\cdot \val{s}{Z} =? \val{s^3}{H}-\val{s}{G}-\gamma \val{\omega^m s}{G}-\gamma^2\val{\omega^{2m}s}{G}$.
\end{itemize}
\item $\verifier$ outputs: Accept if all the above checks succeeds, otherwise it rejects.
\end{enumerate}
}
\end{mdframed}
\vspace*{-5mm}
\caption{Protocol: Check correctness of time-ordered transcript}
\label{fig:time-ordered-transcript}
\end{figure}
%\end{tcolorbox}

Next, we show a polynomial protocol for proving that the transcript $\tr^\ast$ is a permutation of the transcript $\tr$.
We first recall the permutation argument for vectors from ~\cite{EPRINT:GabWilCio19}.
\begin{lemma}[Permutation Check \cite{EPRINT:GabWilCio19}]\label{lem:perm-argument}
Let $f(X), g(X)$ be polynomials in $\F[\,X\,]$. Then, the vectors $\vec{f}, \vec{g}\in \F^k$ encoded by the polynomials
are permutations of each other if and only if with overwhelming probability over the choice of $\alpha\gets \F$,
there exists a polynomial $z(X)$ satisfying the polynomial constraints:
{\small
\begin{align}
\lambda_1(X)(z(X) -1) &= 0 \text{ mod } Z_\setH(X) \tag{B1} \label{eq:B1} \\
(\alpha - g(X))z(\omega X) &= (\alpha - f(X))z(X) \text{ mod } Z_\setH(X) \tag{B2} \label{eq:B2}
\end{align}
}
\end{lemma}
The polynomial protocol in Figure \ref{fig:permutated-transcripts} essentially invokes the above argument on
the random linear combination of the columns of the respective transcripts.
%\begin{tcolorbox}
\begin{figure}[htbp]
\centering
\begin{mdframed}
    {\footnotesize
    \noindent{\bf Common Input}: Commitments $c_\tr=(c_t,c_\op,c_A, c_V)$ and $c_\tr^\ast=(c_t^\ast,c_\op^\ast, c_A^\ast, c_V^\ast)$
    of transcripts $\tr$ and $\tr^\ast$ respectively.
    \begin{enumerate}[leftmargin=2em]
        \item $\verifier\rightarrow\prover$: Sends $\alpha,\beta, \chi\gets \F$
        \item $\prover$ computes $f(X)=t(X) + \beta \op(X) + \beta^2 A(X) + \beta^3 V(X)$, $g(X) = t^\ast(X) + \beta \op^\ast(X)$
        $+ \beta^2 A^\ast(X) + \beta^3 V^\ast(X)$. It then computes polynomials $z(X),q(X)$ as:
        \begin{itemize}[leftmargin=1em]
        \item Interpolate polynomial $z(X)$ of degree $k-1$ such that $z(\omega)=1$ and
         $z(\omega^{i+1})=\prod_{j=1}^i (\alpha - f(\omega^j))/(\alpha - g(\omega^j))$ for $1\leq i\leq k$.
        \item Compute $q(X) = ((\alpha - g(X))z(\omega X) - (\alpha - f(X))z(X) + \chi\lambda_1(X)(z(X) - 1))/\mathbb{Z}_{\setH}(X)$.
        \end{itemize}
        \item $\prover\rightarrow\verifier$: Sends commitments $\gone{z}$ and $\gone{q}$ to polynomials $z(X)$ and $q(X)$ respectively.
        \item $\verifier$ computes: The verifier checks that $q(X)Z_\setH(X)$ = $(\alpha - g(X))z(\omega X)-(\alpha - f(X))z(X) + \chi\lambda_1(X)(z(X) - 1)$
        by requesting evaluations and $\kzg$ proofs of polynomials $f,g$ and $q$ at a random point. It accepts if the check succeeds.
    \end{enumerate}
}
\end{mdframed}
\vspace*{-5mm}
\caption{Protocol: Check permutation of transcripts}
\label{fig:permutated-transcripts}
\end{figure}
%\end{tcolorbox}
Finally, we see that Lemma \ref{lem:addr-ordered-transcript} implies a polynomial protocol to check that the transcript
$\tr^\ast$ is address ordered and satisfies load-store consistency. We skip the formal protocol details, which
essentially involves the prover identifying sets
$I_1, I_2$ as described in Section \ref{subsec:encoded-relations} and sending auxiliary polynomials $Z_1(X),Z_2(X),\delta^\ast_A(X)$ and $\delta^\ast_T(X)$ to the verifier.
The verifier then checks the identities (C1)-(C6) in Lemma \ref{lem:addr-ordered-transcript}.
The range checks in (C7) can be checked using polynomial protocols in sub-vector lookup arguments such as Caulk+ ~\cite{EPRINT:PosKat22}, CQ
~\cite{EPRINT:EagFioGab22}. For completeness, we also include a protocol for enforcing range checks using committed index lookup
described in Section \ref{subsec:committed-index-lookup}.

\begin{comment}
\begin{enumerate}[leftmargin=2em]
\item $\prover\rightarrow\verifier$: For the transcript $\tr^\ast=(\vec{t^\ast},\vec{\op^\ast},\vec{A^\ast},\vec{V^\ast})$,
the prover computes encoding polynomials $\wt{\tr^\ast}=({t^\ast}(X),\op^\ast(X), A^\ast(X),V^\ast(X))$. Next, the prover
computes sets $I_1$ and $I_2$ as in Section \ref{subsec:encoded-relations}, and then computes polynomials
$Z_b(X)=\prod_{i\in I_b}(X-\omega^i)$ for $b=1,2$.
The prover also interpolates polynomials $\delta^\ast_A(X)$ and $\delta^\ast_T(X)$ satisfying
$\delta^\ast_A(\omega^i) = A^\ast_{i+1} - A^\ast_i$ for $i\in I_1$ and $\delta^\ast_T(\omega^i)=t^\ast_{i+1}-t^\ast_i$ for $i\not\in I_1$.
The prover sends all the above polynomials to the verifier.
\item $\verifier$ computes: The verifier checks relations (C1)-(C6) in Lemma \ref{lem:addr-ordered-transcript}. The range constraints
in (C7) can be verified using polynomial protocols in sub-vector lookup arguments such as Caulk+ ~\cite{EPRINT:PosKat22}, CQ
~\cite{EPRINT:EagFioGab22}.
\end{enumerate}


\subsection{Succinct Argument for Verifiable RAM}\label{subsec:succ-args}
Polynomial protocols can be compiled into succinct arguments of knowledge using an extractable polynomial commitment scheme. At a
high level the compilation involves the prover sending ``commitments'' to the message polynomials, whereas the verifier checks
the polynomial identities probabilistically by requesting evaluation proofs of the polynomials at random points.
We complile the polynomial protocol for RAM to a succinct argument of knowledge in the Algebraic Group Model (AGM)
 using $\kzg$ as the extractable polynomial commitment scheme. We also leverage homomorphism of $\kzg$ scheme
to avoid sending commitments and evaluations for polynomials which are linear combinations of previously committed polynomials.
We now present an argument for verifiable RAM, which combines the polynomial protocols in this section, and instantiates
them using $\kzg$ commitments. We make standard optimisations of batching several identity checks into one where applicable.
Let $\srs=\{\gone{\tau^i},\gtwo{\tau^i}\}_{i=0}^d$
denote the $\kzg$ setup parameters for degree $d\geq k$ for the bilinear group $\mathsf{BG}=(\Gone,\Gtwo,\gone{1},\gtwo{1},e)$. Let
$\F=\F_p$ be the finite field where $p$ is the order of the groups.


\begin{figure}[t]
\begin{subfigure}{\linewidth}
\centering
{\footnotesize
\begin{enumerate}[leftmargin=1em]
    \item $\prover\rightarrow\verifier$: Send polynomial commitments $\gone{a(X)}$, $\gone{v(X)}$, $\gone{a'(X)}$, $\gone{v'(X)}$,
    $\gone{\bar{\op}(X)}$, $\gone{\bar{a}(X)}, \gone{\bar{v}(X)}$, $\gone{\op(X)}$, $\gone{A(X)}$, $\gone{V(X)}$, $\gone{t^\ast(X)}$,
    $\gone{\op^\ast(X)}$, $\gone{A^\ast(X)}$, $\gone{V^\ast(X)}$, $\gone{\delta_A^\ast(X)}$, $\gone{\delta_T^\ast(X)}$.
    \item $\verifier\rightarrow\prover$: Send $\alpha,\beta,\gamma\gets \F$.
    \item $\prover$ computes: Compute sets $I_1=\{i\in [k]: A^\ast_i\neq A^\ast_{i+1}\}$, and $I_2=[k]\setminus I_1$. Next, compute
    polynomials:
    \begin{align*}
        &Z_1(X)=\prod_{i\in I_1}(X-\omega^i),\quad Z_2(X)=\prod_{i\in I_2}(X-\omega^i), \\
        &H(X) = \begin{bmatrix} 1 & \gamma & \gamma^2 \end{bmatrix}
        \begin{bmatrix}
            a(X) & v(X) & 0 \\
            a'(X) & v'(X) & 0 \\
            \bar{a}(X) & \bar{v}(X) & \bar{\op}(X)
        \end{bmatrix}
        \begin{bmatrix}
        1 \\
        \beta \\
        \beta^2
        \end{bmatrix}, \\
        &G(X) = A(X) + \beta V(X) + \beta^2 \op(X), \\
        &Q(X) = \big(H(X^3) - G(X) - \gamma G(\omega^m X) - \gamma^2 G(\omega^{2m} X)\big)/Z(X), \\
        &f(X) = G(X) + \beta^3 t(X),\, g(X) = A^{\ast}(X) + \beta V^{\ast}(X) + \beta^2 \op^{\ast}(X) + \beta^3 t^{\ast}(X), \\
        &z(X) = \sum_{i=1}^k \lambda_i(X)\prod_{j=1}^{i-1} (\alpha - g(\omega^j))/(\alpha - f(\omega^j)), \\
        &Q_1(X) = \frac{1}{\mathbb{Z}_\setH(X)}\Big( (\alpha - g(X))z(\omega X) - (\alpha - f(X))z(X) \\
        &\qquad\qquad + \beta Z_2(X)(A^\ast(\omega X) - A^\ast(X) - \delta_A^\ast(X))\Big), \\
        &Q_2(X) = \frac{1}{Z_2(X)}\Big((A^\ast(\omega X) - A^\ast(X))
         + \beta(t^\ast(\omega X) - t^\ast(X)-\delta_T^\ast(X)) \\
        &\qquad\qquad + \beta^2 (\op^\ast(X) - 1)(V^\ast(\omega X) - V^\ast(X))\Big)
    \end{align*}
    \item $\prover\rightarrow \verifier$: Send $\gone{Z_1(X)}$, $\gone{Z_2(X)}$, $\gone{Q(X)}$, $\gone{z(X)}$,
    $\gone{Q(X)}$, $\gone{Q_1(X)}$, $\gone{Q_2(X)}$.
    \item $\verifier\rightarrow\prover$: $s\gets \F$.
    \item $\prover\rightarrow\verifier$: Send evaluations $\val{Z_1}{s}=Z_1(s)$, $\val{Z_2}{s}=Z_2(s)$,
    $\val{G}{s}=G(s)$, $\val{Q}{s}=Q(s)$, $\val{Z}{s}=Z(s)$, $\val{A^\ast}{s}=A^\ast(s)$, $\val{V^\ast}{s}=V^\ast(s)$,
    $\val{\op^\ast}{s}=\op^\ast(s)$, $\val{t}{s}=t(s)$, $\val{t^\ast}{s}=t^\ast(s)$, $\val{z}{s}=z(s)$, $\val{Q_1}{s}=Q_1(s)$,
    $\val{Q_2}{s}=Q_2(s)$, $\val{H}{s^3}=H(s^3)$, $\val{G}{\omega^m s}=G(\omega^m s)$, $\val{\delta^\ast_A}{s}=\delta_A^\ast(s)$,
    $\val{\delta_T^\ast}{s}=\delta_T^\ast(s)$
    $\val{G}{\omega^{2m} s}=G(\omega^{2m} s)$,
    $\val{A^\ast}{\omega s}=A^\ast(\omega s)$, $\val{V^\ast}{\omega s}=V^\ast(s)$, $\val{t^\ast}{\omega s}=t^\ast(\omega s)$,
    $\val{z}{\omega s}=z(\omega s)$, $\val{A^\ast}{\omega}=A^\ast(\omega)$.

    \item $\verifier\rightarrow\prover$: Send $r \gets \F$.
    \item $\prover$ computes: Compute batched $\kzg$ proofs:
    \begin{alignat*}{3}
        &P_1 && &\quad=\quad  && &Z_1+r Z_2 + r^2 G + r^3 Q + r^4 Z + r^5 A^{\ast} + r^6 V^{\ast} + r^7 \op^{\ast} \\
        & && & && &\quad  + r^8 t + r^9 t^{\ast} + r^{10} z + r^{11} Q_1 + r^{12} Q_2 + r^{13} \delta_A^{\ast} + r^{14} \delta_T^{\ast} \\
        &P_2 && &\quad=\quad && &A^{\ast} + r V^{\ast} + r^2 z \\
        &\Pi_s && &\quad=\quad && &\kzgprove(P_1, s) \\
        &\Pi_{\omega s} && &\quad=\quad && &\kzgprove(P_2, \omega s) \\
        &\Pi_{s^3} && &\quad=\quad && &\kzgprove(H, s^3) \\
        &\Pi_{\{\omega^m s, \omega^{2m} s\}} && &\quad=\quad && &\kzgprove(G, (\omega^m s, \omega^{2m} s))
    \end{alignat*}
    \item $\prover\rightarrow \verifier$: Send $\Pi_s,\Pi_{\omega s}, \Pi_{s^3},\Pi_{\{\omega^m s, \omega^{2m} s\}}$ and
    $\val{H}{s^3},\val{G}{\omega^m s}, \val{G}{\omega^{2m} s}$.

    \item $\verifier$ computes: Compute commitments and evaluations for linear combinations.
    \begin{align*}
        \gone{G(X)} &= \gone{A(X)}+\beta\gone{V(X)}+\beta^2\gone{\op(X)}, \\
        \gone{H(X)} &= \begin{bmatrix} 1 & \gamma & \gamma^2 \end{bmatrix}
        \begin{bmatrix}
            \gone{a(X)} & \gone{v(X)} & \gone{0} \\
            \gone{a'(X)} & \gone{v'(X)} & \gone{0} \\
            \gone{\bar{a}(X)} & \gone{\bar{v}(X)} & \gone{\bar{\op}(X)}
        \end{bmatrix}
        \begin{bmatrix}
            1 \\
            \beta \\
            \beta^2
        \end{bmatrix}, \\
        \gone{P_1(X)} &= \gone{Z_1(X)}+r \gone{Z_2(X)} + r^2 \gone{G(X)} + r^3 \gone{Q(X)} + r^4 \gone{Z(X)} + r^5 \gone{A^{\ast}(X)} \\
         &\qquad + r^6 \gone{V^{\ast}(X)} + r^7 \gone{\op^{\ast}(X)} + r^8 \gone{t(X)} + r^9 \gone{t^{\ast}(X)} + r^{10} \gone{z(X)} \\
         &\qquad + r^{11} \gone{Q_1(X)} + r^{12} \gone{Q_2(X)} + r^{13} \gone{\delta_A^{\ast}(X)} + r^{14} \gone{\delta_T^{\ast}(X)}, \\
        \gone{P_2(X)} &= \gone{A^\ast(X)} + r\gone{V^\ast(X)} + r^2\gone{z(X)},\\
        \val{P_1}{s} &= \val{Z_1}{s}+r \val{Z_2}{s} + r^2 \val{G}{s} + r^3 \val{Q}{s} + r^4 \val{Z}{s} + r^5 \val{A^{\ast}}{s} \\
         &\qquad + r^6 \val{V^{\ast}}{s} + r^7 \val{\op^{\ast}}{s} + r^8 \val{t}{s} + r^9 \val{t^{\ast}}{s} + r^{10} \val{z}{s} \\
         &\qquad + r^{11} \val{Q_1}{s} + r^{12} \val{Q_2}{s} + r^{13} \val{\delta_A^{\ast}}{s} + r^{14} \val{\delta_T^{\ast}}{s}, \\
        \val{P_2}{\omega s} &= \val{A^\ast}{s} + r\val{V^\ast}{s} + r^2\val{z}{s}. \\
        \val{g}{s} &= \val{A^\ast}{s} + \beta \val{V^\ast}{s} + \beta^2 \val{\op^\ast}{s} + \beta^3 \val{t^\ast}{s}, \\
        \val{f}{s} &= \val{G}{s} + \beta^3 \val{t}{s}.
    \end{align*}

    \item $\verifier$ checks polynomial identities at $s$:
    \begin{alignat*}{3}
        &\val{Q}{s}\val{Z}{s} && &\quad \stackrel{?}{=}\quad  && &\val{H}{s^3} - \val{G}{s} - \gamma \val{G}{\omega^m s} - \gamma^2 \val{G}{\omega^{2m} s} \\
        &\val{Q_1}{s}(s^k-1) && &\quad \stackrel{?}{=}\quad && &(\alpha - \val{g}{s})\val{z}{\omega s} - (\alpha - \val{f}{s})\val{z}{s}
        +\beta \val{Z_2}{s}(\val{A^\ast}{\omega s}-\val{A^\ast}{s}-\val{\delta_A^\ast}{s}). \\
        &\val{Q_2}{s}\val{Z_2}{s} && &\quad \stackrel{?}{=}\quad && &(\val{A^\ast}{\omega s} - \val{A^\ast}{s}) + \beta (\val{t^\ast}{s} - \val{t}{s} - \val{\delta}{s}) \\
        & && & && &\qquad + \beta^2 (\val{\op^\ast}{s} - 1)(\val{V^\ast}{\omega s}-\val{V^\ast}{s}) \\
        &\val{Z_1}{s}\val{Z_2}{s} && &\quad \stackrel{?}{=}\quad && &s^k - 1 \\
    \end{alignat*}
    \item $\verifier$ checks evaluation proofs:
    \begin{alignat*}{3}
        &\kzgverify(\gone{P_1(X)},\val{P_1}{s},s, \Pi_s) && &\quad\stackrel{?}{=}\quad && &1 \\
        &\kzgverify(\gone{P_2(X)},\val{P_2}{\omega s},\Pi_{\omega s}) && &\quad\stackrel{?}{=}\quad && &1 \\
        &\kzgverify(\gone{H(X)},\val{H}{s^3},\Pi_{s^3}) && &\quad\stackrel{?}{=}\quad && &1 \\
        &\kzgverify(\gone{G(X)},(\val{G}{\omega^m s},\val{G}{\omega^{2m} s}),
        (\omega^m s, \omega^{2m} s), \Pi_{\{\omega^m s, \omega^{2m} s\}}) && &\quad\stackrel{?}{=}\quad && &1 \\
        &\kzgverify(\gone{A^\ast(X)}, \val{A^\ast}{\omega}, \omega, \Pi_\omega) && &\quad\stackrel{?}{=}\quad && &1
    \end{alignat*}
    \item $\verifier$ enforces range checks: Invoke the lookup argument $\varPi_{\mathsf{lookup}}$ to check that
    polynomials $\delta_A^\ast(X)$, $\delta_T^\ast(X)$ and $t^\ast(X)$ encode values in $[0,N]$.
    \item $\verifier$ outputs: The verifier outputs accept (1) if all the preceeding checks pass, else it outputs reject (0).
\end{enumerate}
}
\end{subfigure}
\caption{Argument of Knowledge for Verifiable RAM}
\label{fig:aok-vram}
\end{figure}
\end{comment}

\begin{comment}
Let $\LLconcat$ denote the language consisting of polynomial tuples $(a(X)$, $b(X)$, $c(X)$, $v(X))$ satisfying the identities
in Lemma ~\ref{lem:vec-concatenation}. To check membership of $(T,\vec{o},T',\tr)$ in the language $\LTr$, let $\wt{T}=(a(X),v(X))$, $\wt{T'}=(a'(X),v'(X))$,
$\wt{O}=(\bar{\op}(X),\bar{a}(X),\bar{v}(X))$ and $\wt{\tr}=(t(X),o(X),A(X),V(X))$ be the polynomial
encodings of $T,T',\vec{o}$ and $\tr$ respectively.
From the construction of time ordered transcript $\tr$ outlined
in Section \ref{subsec:transcripts}, we see that the equivalent constraints on the encodings are given by:
\begin{align*}\label{eq:tot-poly-constraints}
t(X) &= \sum_{i=1}^k i\lambda_i(X) \quad (= \mathsf{Enc}(\setind_k)) \\
(0, \bar{\op}(X), 0, o(X)) &\in \LLconcat \\
(a(X), \bar{a}(X), a'(X), A(X)) &\in \LLconcat \\
(v(X), \bar{v}(X), v'(X), V(X)) &\in \LLconcat
\end{align*}

We can probabilistically combine the different polynomial identities into one and obtain the following:
\begin{lemma}
    Let the polynomial $Z(X)=\prod_{i\in [m]}(X-\omega^i)$ and $\rho=\omega^m$ be as before.
    Then, we have $(T,\vec{o},T',\tr)\in \LTr$ if and only if the encoding polynomials as defined above satisfy
    the identity $G_\gamma(X) = 0  \text{ mod } Z(X)$ and $\evalH{t}=(1,\ldots,k)$ with overwhelming
    probability over the choice of $\gamma\gets \F$. Here the polynomial $G_\gamma(X)$ is defined as:
    \begin{multline*}
        G_\gamma(X) = o(X) + \gamma(\bar{\op}(X^3) - o(\rho X)) + \gamma^2 o(\rho^2 X) \\
        + \gamma^3(a(X^3) - A(X)) + \gamma^4(\bar{a}(X^3) - A(\rho X)) + \gamma^5(a'(X^3) - A(\rho^2 X)) \\
        + \gamma^6(v(X^3) - V(X)) + \gamma^7(\bar{v}(X^3) - V(\rho X)) + \gamma^8(v'(X^3) - V(\rho^2 X))
    \end{multline*}
\end{lemma}

Next, we consider the language $\Lperm$ consisting of pairs $(\tr, \tr^\ast)$ of $k$-length transcripts such
that $\tr^\ast=\sigma(\tr)$ for some permutation $\sigma:[k]\rightarrow [k]$. We now describe constraints
to check the same using polynomial encodings of the transcripts. Let encodings $\wt{\tr},\wt{\tr}^\ast$ be
given by polynomials as below:
\begin{align*}\label{eq:tr-encodings}
\wt{\tr} &= (t(X),\op(X),A(X),V(X)) \\
\wt{\tr}^\ast &= (t^\ast(X), \op^\ast(X), A^\ast(X), V^\ast(X))
\end{align*}
We need to show that for some $\sigma:[k]\rightarrow [k]$, $\evalH{p^\ast}=\sigma(\evalH{p})$ for $p\in \{t,\op,A,V\}$.
Again, for $\gamma\gets \F$, with overwhelming probability it is equivalent to establishing that polynomial
$f^\ast(X)=t^\ast(X)+\gamma \op^\ast(X) + \gamma^2 A^\ast(X) + \gamma^3 V^\ast(X)$ encodes a permutation of
vector encoded by $f(X)=t(X)+\gamma \op(X) + \gamma^2 A(X) + \gamma^3 V(X)$. We recall the following variation
of the grand product argument to show that two polynomials encode vectors which are permutations of each other.
\end{comment}







