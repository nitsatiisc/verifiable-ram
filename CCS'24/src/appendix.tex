\appendix
\section{More Preliminaries}

\subsection{Polynomial Commitment Scheme}
\label{sec:pcs_def}
The notion of a polynomial commitment scheme (PCS) that allows the prover to open evaluations of the committed polynomial succinctly was introduced in~\cite{AC:KatZavGol10} who gave a construction under the trusted setup assumption.
A polynomial commitment scheme over $\F$ is a tuple $\pc = 
(\pcsetup,\pccommit,\pcopen,\pceval)$ where:

\begin{itemize}
    \item $\pp \leftarrow \pcsetup(1^\secp,D) $. On input security parameter $\secp$, and an upper bound $D \in \mathbb{N}$ on the degree, $\pcsetup$ generates public parameters $\pp$.
    
    \item $(C,\mathbf{\tilde{c}}) \leftarrow \pccommit(\pp, f(X),d) $. On input the public parameters $\pp$, and a univariate polynomial $f(X) \in \F[X]$ with degree at most $d \leq D$, $\pccommit$ outputs a commitment to the polynomial $C$, and additionally an opening hint $\mathbf{\tilde{c}}$.
	
	\item $b \leftarrow \pcopen(\pp,f(X),d,C,\mathbf{\tilde{c}})$. On input the public parameters $\pp$, the commitment $C$ and the opening hint $\mathbf{\tilde{c}}$, a polynomial $f(X)$ of degree $d \leq D$, $\pcopen$ outputs a bit indicating accept or reject. 
    
	    \item $b \leftarrow \pceval(\pp,C,d,x,v;f(X)) $. A public coin interactive protocol 
	    $\langle P_{\mathsf{eval}}(f(X)), V_{\mathsf{eval}} \rangle(\pp,C,d,z,v)$ between a PPT prover and a PPT verifier. The parties have as common input public parameters $\pp$, commitment $C$, degree $d$, evaluation point $x$, and claimed evaluation $v$. The prover has, in addition, the opening $f(X)$ of $C$, with $\deg(f) \leq d$. At the end of the protocol, the verifier outputs $1$ indicating accepting the proof that $f(x)=v$, or outputs $0$ indicating rejecting the proof.
		
		\end{itemize}
		
A polynomial commitment scheme must satisfy completeness, binding and extractability.

\begin{definition}[Completeness]
\label{def:pcs-comp}
For all polynomials $f(X) \in \F[X]$ of degree $d \leq D$, for all $x \in \F$,
\[
\Pr \left( 
\begin{matrix}
 %\pccheck(\pp,C,d,z,v,\pi) = 1 
 b=1
\end{matrix}
 \,:\,
 \begin{matrix}
\pp \leftarrow \pcsetup(1^\secp,D) \\
 (C,\mathbf{\tilde{c}}) \leftarrow \pccommit(\pp, f(X),d) \\
 v \leftarrow f(x) \\
 b \leftarrow \pceval(\pp,C,d,x,v;f(X))
 %\pi \leftarrow \pceval(\pp,f(X),d,z) \\
\end{matrix}
 \right) = 1.
 \]
\end{definition}

\begin{definition}[Binding] 
\label{def:pcs-binding-app}
A polynomial commitment scheme $\pc$ is binding if for all PPT $\Adv$, the following probability is negligible in $\secp$:
\[
\Pr \left( 
\begin{matrix}
 \pcopen(\pp,f_0,d,C,\mathbf{\tilde{c}_0}) =1 \wedge \\
 \pcopen(\pp,f_1,d,C,\mathbf{\tilde{c}_1}) = 1 \wedge \\
 f_0 \neq f_1
\end{matrix}
 \,:\,
 \begin{matrix}
\pp \leftarrow \pcsetup(1^\secp,D) \\
 (C,f_0,f_1,\mathbf{\tilde{c}_0}, \mathbf{\tilde{c}_1},d) \leftarrow \Adv(\pp) 
\end{matrix}
 \right).
 \]
\end{definition}

\begin{definition}[Extractability]
\label{def:pcs-ext-app}
For any PPT adversary $\Adv = (\Adv_{1},\Adv_{2})$, there exists a PPT algorithm $\ext$ such that the following probability is negligible in $\secp$:
	 \[
    \Pr\left(
      \begin{matrix}
      %\pccheck(\pp,C,d,z,v,\pi) = 1 \wedge
      b = 1 \wedge \\
      \R_{\pceval}(\pp,C,x,v; \tilde{f},\mathbf{\tilde{c}}) = 0
      \end{matrix}
      \,:\,
      \begin{matrix}
         \pp \leftarrow \pcsetup(1^\secp,D) \\
          (C,d,x,v,\mathsf{st}) \leftarrow \Adv_{1}(\pp) \\
         (\tilde{f},\mathbf{\tilde{c}}) \leftarrow \ext^{\Adv_{2}}(\pp)\\
         b \leftarrow \langle \Adv_{2}(\mathsf{st}), V_{\mathsf{eval}} \rangle(\pp,C,d,x,v)
      \end{matrix}
    \right).
  \]
  where the relation $\R_{\pceval}$ is defined as follows:
\begin{align*}
        \R_{\pceval} &= \{\left((\pp,C \in \mathbb{G},\; x \in \F, \; v \in \F);\; (f(X), \mathbf{\tilde{c}}) \right) : \\
        &\qquad (
        \pcopen(\pp,f,d,C,\mathbf{\tilde{c}}) = 1 )
         \land v = f(x)  \}
    \end{align*} 
    
\end{definition}

We denote by $\mathsf{Prove}, \mathsf{Verify}$, the non-interactive prover and verifier algorithms obtained by applying FS to the 
$\pceval$ public-coin interactive protocol, giving a non-interactive PCS scheme $(\pcsetup,\pccommit,\allowbreak \mathsf{Prove},\mathsf{Verify})$.

\begin{definition}[Succinctness]
\label{def:pcs-succinct}
We require the commitments and the evaluation proofs to be of size independent of the degree of the polynomial, that is the scheme is \emph{proof succinct} if
$|C|$ is $\mathsf{poly}(\secp)$, $|\pi|$ is $\mathsf{poly}(\secp)$ where $\pi$ is the transcript obtained by applying FS to $\pceval$. Additionally, the scheme is \emph{verifier succinct} if $\pceval$ runs in time
$\mathsf{poly}(\secp) \cdot \allowbreak \mathsf{log} (d)$ for the verifier.
\end{definition}


\subsection{Succinct Argument of Knowledge}
\label{sec:aok}

Let $\R$ be a NP-relation and $\L$ be the corresponding NP-language, where $\L=\{x : \exists$ $w$ such that $(x,w) \in \R\}$. Here, a prover $\prover$ aims to convince a verifier $\verifier$ that $x\in \L$ by proving that it knows a witness $w$ for a public statement $x$ such that $(x,w)\in \R$. An interactive argument of knowledge for a relation $\R$ consists of a PPT algorithm $\setup$, that takes an input the security parameter $\secp$, and outputs the public parameter $\pp$, and a pair of interactive PPT algorithms $\langle\prover,\verifier\rangle$, where $\prover$ takes as input $(\pp,x,w)$ and $\verifier$ takes as input $(\pp,x)$. An interactive argument of knowledge  $\langle\prover,\verifier\rangle$, must satisfy completeness and knowledge soundness. % and a succinct argument of knowledge must additionally satisfy the property of succinctness.

\begin{definition}[Completeness]
	\label{def:aok-comp}
	For all security parameter $\secp \in \mathbb{N}$ and statement $x$ and witness $w$ such that $(x,w)\in \R$, we have
	\[
	\Pr \left( 
	\begin{matrix}
		%\pccheck(\pp,C,d,z,v,\pi) = 1 
		b=1
	\end{matrix}
	\,:\,
	\begin{matrix}
		\pp \leftarrow \setup(1^\secp) \\
		b \leftarrow \langle\prover (w) ,\verifier\rangle (\pp,x)
	\end{matrix}
	\right) = 1.
	\]
\end{definition}

\begin{definition}[Knowledge Soundness]
	\label{def:aok-ks}
	%An interactive argument of knowledge  $\langle\prover,\verifier\rangle$ satisfies knowledge soundness with error $\kappa$, if for
	For any PPT malicious prover $\prover^*$, there exists a PPT algorithm $\ext$ with an oracle access to the prover $\prover^*$ such that the following probability is negligible
	\[
		\Pr \left( 
		\begin{matrix}
			%\pccheck(\pp,C,d,z,v,\pi) = 1 
			b=1 \wedge \\
			(x,w)\in \R
		\end{matrix}
		\,:\,
		\begin{matrix}
			\pp \leftarrow \setup(1^\secp) \\
			b \leftarrow \langle\prover (w) ,\verifier\rangle (\pp,x) \\
			w \leftarrow \ext^{\prover^*} (\pp,x)
		\end{matrix}
		\right) = 1.
		\]	
\end{definition}

A succinct argument of knowledge $\langle\prover,\verifier\rangle$ for a relation $\R$, must satisfy completeness and knowledge soundness and additionally ensure that the communication complexity between prover and verifier, as well as the verification complexity is bounded by $O(\log |w|)$, where $w$ is the witness for the relation.



\section{Argument for RAM From Polynomial Protocols}\label{sec:poly-proto-ram-app}
In this section, we give a self-contained argument of knowledge for membership in the language
$\LRAM{I}{m}{m}$ introduced in Section \ref{sec:model-for-ram}. We first consider the polynomial encoding
of different RAM artefacts.
\subsection{Polynomial Encoding}\label{subsec:poly-encoding}
%The aim of this section is to encode artefacts such as RAM state, operations and transcripts as polynomials, and
%translate checking memory consistency (equivalently, checking membership in $\LRAM{I}{n}{m}$) to checking polynomial identities
%over the encoded polynomials. For
%simplicity and its usefulness later, we consider the case $m=n$, and accordingly check the membership in the language $\LRAM{I}{m}{m}$.
Let $k=3m$ and let $\omega$ be a primitive $k^{th}$ root of unity in $\F$.
Let $\nu=\omega^3$, and thus $\nu$ is a primitive $m^{th}$ root of unity in $\F$ (We assume, these roots exist in $\F$).
We recall $\setV$ as the subgroup consisting of $m^{th}$ roots of unity with associated Lagrange basis polynomials
$\{\tau_i(X)\}_{i\in [m]}$, while we additionally introduce the set $\setH$ of $k^{th}$ roots of unity with
$\{\lambda_i(X)\}_{i\in [k]}$ as the associated Lagrange polynomials.
\begin{gather}\label{eq:interpolation-sets}
\setH = \{\omega,\ldots,\omega^k\},\quad \setV = \{\nu,\ldots,\nu^m\}
\end{gather}
As before, we define the encoding of vectors in $\vec{f}\in \F^k$ as $\enc{f}{\setH}=\sum_{i\in [k]}f_i\lambda_i(X)$.
%Let $\{\lambda_i(X)\}_{i=1}^k$ be lagrange basis polynomials
%for the set $\setH$ and $\{\tau_i(X)\}_{i=1}^m$ be the lagrange polynomials for the set $\setV$ satisfying
%$\lambda_i(\omega^j)=\delta_{ij}$ for $i,j\in [k]$ and $\tau_i(\nu^j)=\delta_{ij}$ for $i,j\in [m]$.
%We use $\setH$ to encode a vector $\vec{f}=(f_1,\ldots,f_k)$ of size $k$ as the polynomial $f(X)\in \F_{<k}[X]$ such that $f(\omega^i)=f_i$ for $i\in [k]$.
%Similarly, we use $\setV$ to encode a vector
%$\vec{g}=(g_1,\ldots,g_m)$ of size $m$ as the polynomial $g(X)\in F_{<m}[X]$ such that $g(\nu^i)=g_i$ for $i\in [m]$. We use
%the notation $\enc{f}{\setH}$ and $\enc{g}{\setV}$ to denote polynomial encodings of vectors $\vec{f}\in \F^k$ and $\vec{g}\in \F^m$
%respectively.
%In the other direction, for a polynomial $f(X)\in \F[X]$,
%we use $\vec{f}_{|_\setH}$ and $\vec{f}_{|_\setV}$ to denote the vectors $(f(\omega^1),\ldots,f(\omega^k))$ and $(f(\nu^1),\ldots,f(\nu^m))$ respectively.
%The encoding of vectors as polynomials can be succinctly described using lagrange basis polynomials.
%Then we have $\enc{f}{\setH}=\sum_{i=1}^k f_i\lambda_i(X)$ and $\enc{g}{\setV}=\sum_{i=1}^m g_i\tau_i(X)$.
We canonically extend the encoding of vectors to encode RAM, operations and transcripts by encoding their component vectors.
Thus, for a RAM $T=(\vec{a},\vec{v})\in \RAM{I}{m}$, we define its encoding
$\wt{T}=(a(X),v(X))$ where $a(X),v(X)\in \F_{<m}[X]$ encode vectors $\vec{a}, \vec{v}$ respectively.
Given an operation sequence
$\vec{o}=(o_1,\ldots,o_m)$ with $o_i=(\bar{\op}_i,\bar{a}_i,\bar{v}_i)$ we encode $\vec{o}$ as
$\wt{O}$ $=$ $(\bar{\op}(X)$ ,$\bar{a}(X)$ ,$\bar{v}(X))$
where $\bar{\op}(X)$ encodes the
vector $\vec{\op}=(\bar{\op}_1,\ldots,\bar{\op}_m)$, $\bar{a}(X)$ encodes the vector $(\bar{a}_1,\ldots,\bar{a}_m)$ and
$\bar{v}(X)$encodes the vector $(\bar{v}_1,\ldots,\bar{v}_m)$.
Finally, a transcript $\tr=(\vec{t},\vec{\op},\vec{A},\vec{V})$ for tuples $(T,\vec{o},T')$ where $T,T'$ are RAMs of size $m$,
and $\vec{o}$ is an operation sequence of size $m$ is encoded as $\wt{\tr}$ $=$ $(t(X)$, $\op(X)$, $A(X)$, $V(X))$
where the polynomials $t(X),\op(X),V(X)$ and $A(X)$ encode the respective vectors in $\F^k$ (See Section \ref{sec:model-for-ram}).

\subsection{Relations over Polynomial Encodings}\label{subsec:encoded-relations}
In this section, we describe polynomial checks for two important relations we need in subsequent sections, viz,
(i) checking concatenation of vectors and (ii) checking monotonicity and load-store consistency of a transcript.
The lemma below specifies the polynomial identities for verifying
that vector $\vec{v}\in \F^k$ is concatenation of vectors $\vec{a},\vec{b},\vec{c}$ in $\F^m$.

\begin{lemma}\label{lem:vec-concatenation}
Let $\vec{a},\vec{b},\vec{c}\in \F^m$  and $v\in \F^k$ be vectors encoded by polynomials
$a(X),b(X),c(X)$ and $v(X)$ respectively. Then,
\begin{align}
    a(X^3) - v(X)  &= 0  \quad \text{ mod $Z(X)$ } \tag{A1}\label{eq:A1}\\
    b(X^3) - v(\omega^m X)  &= 0   \quad \text{ mod $Z(X)$ } \tag{A2}\label{eq:A2}\\
    c(X^3) - v(\omega^{2m} X) &= 0 \quad \text{ mod $Z(X)$ } \tag{A3}\label{eq:A3}
\end{align}
for $Z(X)=\prod_{i=1}^m (X-\omega^i)$ if and only if $\vec{v}=\vec{a}||\vec{b}||\vec{c}$.
\end{lemma}
\begin{proof}
    Assume that the polynomial identities hold. Substituting $X=\omega^i$ for $i\in [m]$ in above equations implies
    for $i\in [m]$: $a_i=v_i$ (Eq \eqref{eq:A1}), $b_i=v_{m+i}$ (Eq \eqref{eq:A2}) and $c_i=v_{2m+i}$ (Eq \eqref{eq:A3}),
    which together imply $\vec{v}=\vec{a}||\vec{b}||\vec{c}$. Converse follows by observing that $\vec{v}=\vec{a}||\vec{b}||\vec{c}$
    implies that $v(X) = a(X^3)$, $v(\omega^m X)=b(X^3)$ and $v(\omega^{2m} X)=c(X^3)$ holds for all $X=\omega^i, i\in [m]$.
    Thus, the equalities hold modulo the polynomial $Z(X)$ as defined above.
\end{proof}


Next, we specify polynomial checks on the encoding of a transcript to ensure it satisfies address-ordering and load-store consistency.
Let $N$ be an upper bound on the values of $\vec{A}$, i.e, the index set $\setind\subseteq [N]$.
Let $\tr=(\vec{t},\vec{\op},\vec{A},\vec{V})$ be a transcript encoded as
$\wt{\tr}=(t(X),\op(X),A(X),V(X))$. Recall that we need to check two conditions on $\tr$, viz, (i) {\em monotonicty}:
the transcript is sorted by address and timestamp respectively, i.e, $A_i\leq A_{i+1}$ for all $i < k$ and
$t_i < t_{i+1}$ whenever $A_i=A_{i+1}$, (ii) {\em load-store consistency}: whenever $\op_{i+1}=0$ and $A_i=A_{i+1}$,
we have $V_i=V_{i+1}$.
To do so, we exhibit disjoint sets $I_1,I_2$ with $I_1\uplus I_2=[k]$ such that: (i) for all
$i\in I_1$, $A_i < A_{i+1}$, (ii) for all $i\in I_2$, $(A_i = A_{i+1})\wedge (t_i < t_{i+1})$ and (iii) for all $i\in I_2$,
$(\op_i=1)\vee (V_i = V_{i+1})$. We vacuously allow $k$ to be part of either of the sets.
Note that the conditions on the sets $I_1$ and $I_2$ ensures monotonicity.
Moreover, it can be seen that load-store consistency requirements are satisfied for all $i\in I_1$ (as $A_i\neq A_{i+1}$).
Similarly,load-store consistency also holds for all $i\in I_2$.
It remains to exhibit the sets and show that they satisfy the above invariants using polynomials, as in the following
lemma:
\begin{lemma}\label{lem:addr-ordered-transcript}
Let $\wt{\tr}$ be a polynomial encoding of transcript $\tr$ of size $k$, given by polynomials $t(X),\op(X),A(X)$ and $V(X)$,
with index set $[N]$. Then assuming $kN<|\F|$, $\tr$ is address ordered and satisfies load-store consistency if and only if there exist polynomials
$Z_1,Z_2,\delta_T,\delta_A$
such that the following hold:
\begin{align}%\label{eq:loadstore-consitency-constraints}
    & A(\omega X) - A(X) - \delta_A(X)= 0 \text{ mod }\, \mathbb{Z}_1(X) \tag{C1} \label{eq:C1} \\
    & A(\omega X) - A(X) = 0  \text{ mod } Z_2(X) \tag{C2} \label{eq:C2} \\
    & t(\omega X) - t(X) - \delta_T(X) = 0  \text{ mod } Z_2(X) \tag{C3} \label{eq:C3} \\
    & (\op(X) - 1)(V(\omega X) - V(X)) = 0  \text{ mod } Z_2(X) \tag{C4} \label{eq:C4} \\
    & Z_1(X)\cdot Z_2(X) = \mathbb{Z}_\setH(X) \quad  \tag{C5} \label{eq:C5} \\
    & 1\leq A(\omega^i) \leq N  \quad \tag{C6} \label{eq:C6} \\
    & 1\leq t(\omega^i) \leq N, 1\leq \delta_A(\omega^i)\leq N,\ 1\leq \delta_T(\omega^i)\leq N \text{ for } i\in [k] \tag{C7} \label{eq:C7}
\end{align}
\end{lemma}
\begin{proof}
    Suppose there exist polynomials $Z_1(X),Z_2(X),\delta_T(X)$ and $\delta_A(X)$ satisfying above identities. From Equation
    ~\eqref{eq:C5}, we conclude that their exist sets $I_1,I_2$ with $I_1\uplus I_2=[k]$ such that $Z_b(X)$, $b\in \{0,1\}$ is the
    vanishing polynomial of the set $\{\omega^i: i\in I_b\}$. We now note that the following are true for $i\in I_1$:
    \begin{itemize}[leftmargin=1em]
        \item $A(\omega^{i+1})-A(\omega^i)=\delta_A(\omega^i)$. Since $1\leq \delta_A(\omega^i)\leq N$, this ensures $A_i < A_{i+1}$ for the vector $\vec{A}$ encoded
        by $A(X)$. We note that $kN < |\F|$ implies there is no overflow modulo the field characteristic.
    \end{itemize}
    Similarly, it can be seen that for $i\in I_2$, we must have (i) $A_i=A_{i+1}\wedge t_i < t_{i+1}$
    and (ii) $\op_i=1\,\vee\,V_i=V_{i+1}$. Together these imply that the encoded transcript is address-ordered.
\end{proof}

Protocols facilitating the checks mentioned in Lemma \ref{lem:vec-concatenation} and Lemma \ref{lem:addr-ordered-transcript} are presented in Figure \ref{fig:concatenation} and \ref{fig:encoded-relations} respectively.

\begin{figure}[htbp]
	\begin{mdframed}
		{
			{\bf Common Input}: Commitments $c_a$, $c_b$, $c_c$, $c_v$, and $\gone{Z}$ (to the polynomial
			$Z(X)=\prod_{i=1}^m (X-\omega^i)$). \\
			{\bf Prover's Input}: Vectors $\vec{a},\vec{b},\vec{c}\in\F^m$ and $\vec{v}\in \F^k$.
			\begin{enumerate}[leftmargin=1em, label=\arabic*.]
				\item $\verifier$ sends $\gamma\gets \F$.
				\item $\prover$ computes:
				\begin{align}
					& h(X) = a(X) + \gamma b(X) + \gamma^2 c(X),\\
					& Q(X) = (h(X^3) - v(X) - \gamma v(\omega^m X) - \gamma^2 v(\omega^{2m} X))/Z(X)
				\end{align}
				\item $\prover$ sends commitment $\gone{Q}$ = $\gone{Q(X)}$.
				\item $\verifier$ sends $s\gets\F$.
				\item $\prover$ sends evaluations $\val{s}{v}=v(s)$, $\val{\omega^m s}{v}=v(\omega^m s)$,
				$\val{\omega^{2m}s}{v}=v(\omega^{2m} s)$, $\val{s^3}{h}=h(s^3)$, $\val{s}{Q}=Q(s)$ and $\val{s}{Z}=Z(s)$.
				\item $\verifier$ sends $r\gets\F$.
				\item $\prover$ computes $\kzg$ proofs:
				\begin{itemize}[leftmargin=1em]
					\item $\Pi_v=\kzgprove(\srs,v,(s,\omega^m s, \omega^{2m}s))$.
					\item $\Pi_h=\kzgprove(\srs,h,s^3)$.
					\item $\Pi_f=\kzgprove(\srs,f,s)$ where $f(X)=Z(X) + rQ(X)$.
				\end{itemize}
				\item $\prover$ sends $\Pi_v$, $\Pi_h$ and $\Pi_f$.
				\item $\verifier$ Computes commitments $\gone{h}$ and $\gone{f}$ using homomorphism.
				\item $\verifier$ checks:
				\begin{itemize}[leftmargin=1em]
					\item $\kzgverify(\srs,\gone{v}, \vec{e}_v, \vec{p}_v, \Pi_v)$ where
					$\vec{p}_v$ = $(s,\omega^m s, \omega^{2m}s)$ and $\vec{e}_v$ = $(\val{s}{v}, \val{\omega^m s}{v}, \val{\omega^{2m} s}{v})$.
					\item $\kzgverify(\srs,\gone{h},  \val{s^3}{h}, s^3,\Pi_h)$.
					\item $\kzgverify(\srs,\gone{f},  \val{s}{Z} + r\val{s}{Q}, s,\Pi_f)$.
					\item $\val{s}{Q}\cdot \val{s}{Z} = \val{s^3}{h}-\val{s}{v}-\gamma \val{\omega^m s}{v}-\gamma^2\val{\omega^{2m}s}{v}$.
				\end{itemize}
				\item $\verifier$ outputs accept if all the above checks succeed, else it rejects.
			\end{enumerate}
		}
	\end{mdframed}
	%\vspace*{-5mm}
	\caption{Protocol: Check concatenation via commitments}
	\label{fig:concatenation}
\end{figure}

\begin{figure}[htbp]

    \begin{mdframed}
    {
            {\bf Common Input}: Commitments $c_t$, $c_{op}$, $c_A$ and $c_V$ to $t,op,A$ and $V$ respectively where $\tr=(t, op, A, v)$ is supposed to be address ordered\\
    {\bf Prover's Input}: $\tr=(t, op, A, v)$ and its polynomial encoding $\wt{\tr}$
        \begin{enumerate}[leftmargin=1em, label=\arabic*]
            \item Prover determines sets $I_1, I_2$ as described in section A.2
            \item Prover computes polynomials $Z_1(X), Z_2(X), \delta_T(X), \delta_A(X)$
            \item $\prover$ sends $\gone{Z_1(X)}$, $\gone{Z_2(X)}$, $\gone{\delta_T(X)}$, $\gone{\delta_A(X)}$
            \item $\verifier$ sends $\gamma\gets \F$.
            \item $\prover$ computes:

            $$Q_1(X) =  (A(\omega X) - A(X) - \delta_A(X)) / \Z_1(X)$$
            $$Q_2(X) =  [(A(\omega X) - A(X))+\gamma(t(\omega X) - t(X) - \delta_T(X))+$$
            $$ \gamma^2 (\op(X) - 1)(V(\omega X) - V(X))] / Z_2(X) $$


            \item $\prover$ sends commitments $\gone{Q_1(X)}$, $\gone{Q_2(X)}$
            \item $\verifier$ sends $s\gets\F$
            \item $\prover$ sends evaluations $\val{s}{A}=A(s)$, $\val{\omega s}{A}=A(\omega s)$, $\val{s}{\delta_A} = \delta_A(s)$, $\val{s}{t}=t(s)$, $\val{\omega s}{t}=t(\omega s)$, $\val{s}{\delta_T} = \delta_T(s)$, $\val{s}{\op}=\op(s)$, $\val{s}{V}=V(s)$, $\val{\omega s}{V}=V(\omega s)$, $\val{s}{Q_1}=Q_1(s)$, $\val{s}{Q_2}=Q_2(s)$, $\val{s}{Z_1}=Z_1(s)$, $\val{s}{Z_2}=Z_2(s)$
            \item $\verifier$ checks
            \begin{itemize}
                \item $\val{s}{Q_1}\cdot \val{s}{Z_1} = (\val{\omega s}{A} - \val{s}{A} - \val{s}{\delta_A})$
                \item  $\val{s}{Q_2}\cdot \val{s}{Z_2} = (\val{\omega s}{A} - \val{s}{A})+\gamma (\val{\omega s}{t} - \val{s}{t} - \val{s}{\delta_T})+\gamma^2 (\val{s}{\op}-1)(\val{\omega s}{V} - \val{s}{V}) $
                \item $\val{s}{Z_1} \cdot \val{s}{Z_2}=s^{3m}-1$

            \end{itemize}
            \item $\verifier$ sends $r_1, r_2\gets\F$.
            \item $\prover$ computes:
            \begin{itemize}
                \item $\Phi_{ws}(X)= A(X)+r_1 t(X) +r_1^2 V(X)$
                \item $\Phi_s(X)=A(X)+r_2 \delta_A(X)+r_2^2 t(X)+ r_2^3 \delta_T(X)+r_2^4 \op(X) +r_2^5 V(X)+r_2^6 Q_1(X)+ r_2^7 Q_2(x)+r_2^8 Z_1(X)+r_2^9 Z_2(X)$
                \item $\prover$ computes $\Pi_{\omega s} = \KZGopen(\srs, \Phi_{\omega s}(X), \omega s)$
                \item $\prover$ computes $\Pi_{s} = \KZGopen(\srs, \Phi_{s}(X),s)$

            \end{itemize}
            \item $\prover$ sends $\Pi_{\omega s}, \Pi_{s}$
            \item $\verifier$ computes:
            \begin{itemize}
                \item $[\Phi_{\omega s}(X)]_1= c_A +r_1 c_t + r_1^2 c_V$
                \item $[\Phi_{s}(X)]_1= c_A +r_2 [\delta_A(X)]_1 + r_2^2 c_t +r_2^3 [\delta_T(X)]_1 +r_2^4 c_{\op} +r_2^5 c_V +r_2^6 [Q_1(X)]_1 +r_2^7 [Q_2(X)]_1 +r_2^8 [Z_1(X)]_1 + r_2^9 [Z_2(X)]_1 $
                \item $V_{ws}= \val{\omega s}{A}+r_1 \val{\omega s}{t}+ r_1^2 \val{\omega s}{V}$
                \item $V_s=\val{s}{A}+r_2 \val{s}{\delta_A}+r_2^2 \val{s}{t}+r_2^3 \val{s}{\delta_T}+r_2^4 \val{s}{\op} +r_2^5 \val{s}{V} +r_2^6 \val{s}{Q_1} +r_2^7 \val{s}{Q_2}+r_2^8 \val{s}{Z_1} +r_2^9 \val{s}{Z_2}$
            \end{itemize}
            \item $\verifier$ checks:
            \begin{itemize}
                \item $\KZGverify(\srs, \gone{\Phi_{ws}}, V_{ws}, \omega s, \Pi_{\omega s})$
                \item $\KZGverify(\srs, \gone{\Phi_{s}}, V_{s}, s, \Pi_{s})$
            \end{itemize}

            \item $\left<\prover, \verifier\right>$ run 4 instances of subvector lookup protocol $L(\cdot, \cdot)$, namely $L([N], A), L([N], t), L([N], \delta_A), L([N], \delta_t)$ \\
            (eg. using \cite{EPRINT:EagFioGab22})
            \item $\verifier$ accepts if all checks succeed and the lookups succeed, rejects otherwise
        \end{enumerate}
    }
    \end{mdframed}

    \caption{Protocol: Check transcript is address ordered}
    \label{fig:encoded-relations}
\end{figure}

\section{Succinct Argument for Verifiable RAM}\label{sec:poly-proto-ram}
The polynomial encodings in the previous section can be used to polynomial protocol for
checking the membership in the language $\LRAM{I}{m}{m}$ for $m\in \N$. The polynomial protocol can be subsequently
be compiled into a succinct argument using an extractable polynomial commitment scheme.
In this section, we use $\kzg$ polynomial commitment scheme to obtain a succinct argument for checking membership in $\LRAM{I}{m}{m}$
in the Algebraic Group Model (AGM).
At a high level, to prove $(\vec{T},\vec{o},\vec{T'})\in \LRAM{I}{m}{m}$, the prover
constructs time ordered transcript $\tr$ and then permutes it to obtain the address sorted transcript $\tr^\ast$.
It then sends the polynomial encodings of $\vec{T},\vec{o},\vec{T'},\tr$ and $\tr^\ast$ to the verifier, who verifies that:
\begin{enumerate}[leftmargin=1em]
    \item The time ordered transcript is correctly constructed, i.e, $\tr=\TOT(\vec{T},\vec{o},\vec{T'})$.
    \item The transcript $\tr^\ast$ is a permutation of the transcript $\tr$, i.e, $\tr^\ast=\sigma(\tr)$ for some permutation $\sigma$ of $[k]$.
    \item The transcript $\tr^\ast$ is address ordered and satisfies load-store consistency.
\end{enumerate}
In fact, we consider memebership in the above language under commitments. Let $\srs$ denote a $\kzg$ setup over a bilinear group, with
prime order groups $\Gone, \Gtwo$ and $\GT$. We canonically commit to RAM, operation sequences and transcripts by committing to their
polynomial encodings. Commitment of an encoding represented as tuple of polynomials is simply the tuple consisting of commitments of the component
polynomials. We now define
the relation $\CLRAM$ below, and present a succinct argument for the same.
\begin{definition}\label{defn:committed-vram}
Let $\CLRAM$ consist of tuples $((c_T, c_o, c_T'), (T, \vec{o},T'))$ where $c_T$ $=$ $\kzgcommit(\srs, \wt{T})$,
$c_T'$ $=$ $\kzgcommit(\srs, \wt{T'})$,
$c_o$ $=$ $\kzgcommit(\srs,\wt{O})$ commit to $T$, $T'$ and $\vec{o}$ with $(T,\vec{o},T')\in \LRAM{I}{m}{m}$.
\end{definition}
\noindent In the above definition we have $c_T=(c_a,c_v)$ where $c_a$ and $c_v$ are $\kzg$ commitments to polynomials $a(X)$ and $v(X)$ in
the encoding $\wt{T}=(a(X), v(X))$. Similarly we parse $c_T'=(c_a', c_v')$ and $c_o=(\bar{c}_\op, \bar{c}_a, \bar{c}_v)$ (see Section
\ref{subsec:poly-encoding} for polynomial encodings).
For proving relation \ref{defn:committed-vram}, prover's input consists of initial RAM state $T=(\vec{a},\vec{v})$,
final RAM state $T'=(\vec{a'},\vec{v'})$, operation sequence $\vec{o}=(o_1,\ldots,o_m)$ with $o_i=(\bar{\op}_i,\bar{a}_i,\bar{v}_i)$,
time-ordered transcript $\tr=(\vec{t},\vec{\op},\vec{A},\vec{V})$ and address-ordered transcript $\tr^\ast=(\vec{t^\ast},\vec{\op^\ast},
\vec{A^\ast},\vec{V^\ast})$ obtained from $\tr$ using a permutation $\sigma:[k]\rightarrow [k]$. Verifier's input consists of the
commitments $c_T, c_o$ and $c_T'$ as described above.

The prover starts the protocol by sending commitments $c_\tr$ and $c^\ast_\tr$ to the transcripts $\tr$ and $\tr^\ast$ respectively.
To show that $\tr$ is correctly formed, the prover needs to prove the concatenations:
(i) $\vec{\op}=0^m||(\bar{\op}_1,\ldots,\bar{\op}_m)||0^m$, (ii) $\vec{A}=\vec{a}||(\bar{a}_1,\ldots,\bar{a}_m)||\vec{a'}$
and (iii) $\vec{V}=\vec{v}||(\bar{v}_1,\ldots,\bar{v}_m)||\vec{v'}$. Note that the time-stamp column $\vec{t}$ is implicitly assumed
to be $(1,\ldots,k)$.
The verifier checks the concatenations using Lemma \ref{lem:vec-concatenation}.
It uses a random challenge $\beta$ to reduce the three concatenations to one concatenation, and uses another challenge $\gamma$
to reduce the three polynomial checks in Lemma \ref{lem:vec-concatenation} to a single check.
The complete polynomial protocol is detailed in Figure \ref{fig:time-ordered-transcript}.

%\begin{tcolorbox}
\begin{figure}[htbp]

    \begin{mdframed}
    {
            {\bf Common Input}: Commitments $c_T=(c_a,c_v)$, $c_o=(\bar{c}_\op, \bar{c}_a, \bar{c}_v)$, $c_T'=(c_a', c_v')$
        and $c_\tr=(c_t, c_\op, c_A, c_V)$ to $T,o,T'$ and $\tr$(which is supposed to be the time ordered transcript) respectively. Commitment $\gone{Z(X)}$ to the polynomial
        $Z(X)=\prod_{i=1}^m (X-\omega^i)$\\
    {\bf Prover's Input}: $\tr, T, T', \vec{o}$ and their polynomial encodings, $Z(X)$
        \begin{enumerate}[leftmargin=1em, label=\arabic*.]
            \item $\verifier$ sends $\beta,\gamma\gets \F$.
            \item $\prover$ computes:
            \begin{align}\label{eq:poly-constraints}
            & G_1(X) = a(X) + \beta v(X),\, G_2(X) = \bar{a}(X) + \beta \bar{v}(X) + \beta^2 \bar{\op}(X) \tag{A1} \\
            & G_3(X) = a'(X) + \beta v'(X),\, G(X) = A(X) + \beta V(X) + \beta^2 \op(X) \tag{A2} \\
            & H(X) = G_1(X) + \gamma G_2(X) + \gamma^2 G_3(X), \tag{A3} \\
            & Q(X) = [(H(X^3) - G(X) - \gamma G(\omega^m X) - \gamma^2 G(\omega^{2m} X))]/Z(X) \tag{A4}
            \end{align}
            \item $\prover$ sends commitment $\gone{Q}$ to $Q(X)$.
            \item $\verifier$ sends $s\gets\F$.
            \item $\prover$ sends evaluations $\val{s}{G}=G(s)$, $\val{\omega^m s}{G}=G(\omega^m s)$,
            $\val{\omega^{2m}s}{G}=G(\omega^{2m} s)$, $\val{s^3}{H}=H(s^3)$, $\val{s}{Q}=Q(s)$ and $\val{s}{Z}=Z(s)$.
            \item $\verifier$ sends $r\gets\F$.
            \item $\prover$ sends $\kzg$ proofs:
            \begin{itemize}[leftmargin=1em]
                \item $\Pi_G=\kzgprove(\srs,G(X),(s,\omega^m s, \omega^{2m}s))$.
                \item $\Pi_H=\kzgprove(\srs,H(X),s^3)$.
                \item $\Pi_F=\kzgprove(\srs,F(X),s)$ where $F(X)=Z(X) + rQ(X)$.
            \end{itemize}
            \item $\verifier$ computes $\gone{G(X)},\gone{H(X)}$ and $\gone{F(X)}$ using homomorphism.
            \item $\verifier$ checks:
            \begin{itemize}[leftmargin=1em]
                \item $\kzgverify(\srs,\gone{G}, (\val{s}{G}, \val{\omega^m s}{G}, \val{\omega^{2m} s}{G}),\\(s,\omega^m s, \omega^{2m}s),\Pi_G)$.
                \item $\kzgverify(\srs,\gone{H}, \val{s^3}{H}, s^3, \Pi_H)$.
                \item $\kzgverify(\srs,\gone{F}, \val{s}{Z} + r\val{s}{Q}, s, \Pi_F)$.
                \item $\val{s}{Q}\cdot \val{s}{Z} = \val{s^3}{H}-\val{s}{G}-\gamma \val{\omega^m s}{G}-\gamma^2\val{\omega^{2m}s}{G}$.
            \end{itemize}
            \item $\verifier$ accepts if all the above checks succeeds, otherwise it rejects.
        \end{enumerate}
    }
    \end{mdframed}
%    \vspace*{-5mm}
    \caption{Protocol: Check correctness of time-ordered transcript}
    \label{fig:time-ordered-transcript}
\end{figure}
%\end{tcolorbox}

Next, we show a polynomial protocol for proving that the transcript $\tr^\ast$ is a permutation of the transcript $\tr$.
We first recall the permutation argument for vectors from ~\cite{EPRINT:GabWilCio19}.
\begin{lemma}[Permutation Check \cite{EPRINT:GabWilCio19}]\label{lem:perm-argument}
Let $f(X), g(X)$ be polynomials in $\F[\,X\,]$. Then, the vectors $\vec{f}, \vec{g}\in \F^k$ encoded by the polynomials
are permutations of each other if and only if with overwhelming probability over the choice of $\alpha\gets \F$,
there exists a polynomial $z(X)$ satisfying the polynomial constraints:
    {\small
\begin{align}
    \lambda_1(X)(z(X) -1) &= 0 \text{ mod } Z_\setH(X) \tag{B1} \label{eq:B1} \\
    (\alpha - g(X))z(\omega X) &= (\alpha - f(X))z(X) \text{ mod } Z_\setH(X) \tag{B2} \label{eq:B2}
\end{align}
}
\end{lemma}
The polynomial protocol in Figure \ref{fig:permutated-transcripts} essentially invokes the above argument on
the random linear combination of the columns of the respective transcripts.
%\begin{tcolorbox}
\begin{figure}[htbp]

    \begin{mdframed}
    {
            {\bf Common Input}: Commitments $c_\tr=(c_t,c_\op,c_A, c_V)$ and $c_\tr^\ast=(c_t^\ast,c_\op^\ast, c_A^\ast, c_V^\ast)$
        of transcripts $\tr$ and $\tr^\ast$ respectively.\\
    {\bf Prover's Input}: Transcripts $\tr, \tr^{*}$ and their polynomial encodings, permutation $\sigma$ such that $\tr^{*}=\sigma(\tr)$
        \begin{enumerate}[leftmargin=1em, label=\arabic*]
            \item $\verifier$ sends $\alpha,\beta, \chi\gets \F$
            \item $\prover$ computes $f(X)=t(X) + \beta \op(X) + \beta^2 A(X) + \beta^3 V(X)$, $g(X) = t^\ast(X) + \beta \op^\ast(X)$
            $+ \beta^2 A^\ast(X) + \beta^3 V^\ast(X)$. It then computes polynomials $z(X),q(X)$ as:
            \begin{itemize}[leftmargin=1em]
                \item Interpolate polynomial $z(X)$ of degree $k-1$ such that $z(\omega)=1$ and
                $z(\omega^{i+1})=\prod_{j=1}^i (\alpha - f(\omega^j))/(\alpha - g(\omega^j))$ for $1\leq i\leq k-1$.
                \item Compute $q(X) = ((\alpha - g(X))z(\omega X) - (\alpha - f(X))z(X) + \chi\lambda_1(X)(z(X) - 1))/\mathbb{Z}_{\setH}(X)$.
            \end{itemize}
            \item $\prover$ sends commitments $\gone{z(X)}$ and $\gone{q(X)}$ to polynomials $z(X)$ and $q(X)$ respectively.
            \item $\verifier$ computes commitments $[f]_1, [g]_1$ by homomorphism
            \item $\verifier$ checks that $q(X)Z_\setH(X)$ = $(\alpha - g(X))z(\omega X)-(\alpha - f(X))z(X) + \chi\lambda_1(X)(z(X) - 1)$
            by requesting evaluations and $\kzg$ proofs of polynomials $f,g, q, z$ at a random point, say $s$ and evaluation and $\kzg$ proof of $z$ at $\omega s$.
            \item $\verifier$ accepts if all the checks succeed, else it rejects
        \end{enumerate}
    }
    \end{mdframed}
%    \vspace*{-5mm}
    \caption{Protocol: Check permutation of transcripts}
    \label{fig:permutated-transcripts}
\end{figure}


Finally, we see that Lemma \ref{lem:addr-ordered-transcript} implies a polynomial protocol to check that the transcript
$\tr^\ast$ is address ordered and satisfies load-store consistency, which
essentially involves the prover identifying sets
$I_1, I_2$ as described in Section \ref{subsec:encoded-relations} and sending auxiliary polynomials $Z_1(X),Z_2(X),\delta^\ast_A(X)$ and $\delta^\ast_T(X)$ to the verifier.
The verifier then checks the identities (C1)-(C6) in Lemma \ref{lem:addr-ordered-transcript}.
The range checks in (C7) can be checked using polynomial protocols in sub-vector lookup arguments such as Caulk+ ~\cite{EPRINT:PosKat22}, CQ
~\cite{EPRINT:EagFioGab22}. The protocol(compiled using KZG commitments in AGM) can be found in figure \ref{fig:encoded-relations}. The overall covering protocol for $\CLRAM$ which combines figures \ref{fig:time-ordered-transcript}, \ref{fig:permutated-transcripts}, \ref{fig:encoded-relations} in this order can be found in figure \ref{fig:covering-protocol}.
\begin{figure}[htbp]

    \begin{mdframed}
    {
            {\bf Common Input}: Commitments $c_T=(c_a,c_v)$, $c_o=(\bar{c}_\op, \bar{c}_a, \bar{c}_v)$, $c_T'=(c_a', c_v')$

            {\bf Prover's Input}: $ T, T', \vec{o}$ and their polynomial encodings
        \begin{enumerate}[leftmargin=1em, label=\arabic*.]
            \item $\prover$ computes:
            \begin{itemize}
                \item $\tr$, which is the time ordered transcript corresponding to $T, \vec{o}, T'$, and its polynomial encoding
                \item $Z(X)=\prod_{i=1}^m (X-\omega^i)$
                \item $c_\tr=(c_t, c_\op, c_A, c_V)$ which is the commitment of $\tr$
                \item $[Z(X)]_1$
            \end{itemize}

            \item $\prover$ sends  $c_\tr=(c_t, c_\op, c_A, c_V)$ and $[Z(X)]_1$
            \item $\left<\prover, \verifier \right>$ run the protocol for checking correctness of time ordered transcript as in figure \ref{fig:time-ordered-transcript}
            \item $\prover$ computes the address ordered transcript $\tr^{*}$ from the time ordered transcript $\tr$ by permuting suitably. It also computes its polynomial encoding
            \item $\prover$ computes the permutation $\sigma$ such that $\tr^{*}= \sigma(\tr)$ (this is computed along with the $\tr^{*}$)
            \item $\prover$ computes the commitment $c_\tr^\ast=(c_t^\ast,c_\op^\ast, c_A^\ast, c_V^\ast)$ of $\tr^{*}$
            \item $\prover$ sends  $c_\tr^\ast$
            \item $\left<\prover, \verifier \right>$ run the protocol for checking that the two transcripts are permuatations of each other as in figure \ref{fig:permutated-transcripts}
            \item $\left<\prover, \verifier \right>$ run the protocol for checking the constraints given in lemma \ref{lem:addr-ordered-transcript} as in figure \ref{fig:encoded-relations}
            \item $\verifier$ accepts if all the three sub protocols lead to accept. $\verifier$ rejects otherwise
        \end{enumerate}
    }
    \end{mdframed}
%    \vspace*{-5mm}
    \caption{Protocol:Overall protocol for the relation $\CLRAM$}
    \label{fig:covering-protocol}
\end{figure}

\section{Proof of Lemma ~\ref{lem:sum-computation}}
%Recall the statement of lemma \ref{lem:sum computation}. We want to compute $e_i$ for all $i \in \setind$ and $b_j$ for all $j \in K$ in $O(|K|\log^2|K|)$ $\mathbb{F}$ operations.\\\\
Before starting the proof, we collect some preliminaries which will be useful in the proof.

\subsection{Computational Algebra Preliminaries}\label{subsec:comp-algebra-app}
Let $\F$ be a finite field of prime order $p$ and $\G$ be a cyclic additive group of order $p$ with generator $g$. For $s \in \F$, we use
the notation $\elt{s}$ to denote the group element $s\cdot g$. Assume that $\F$ contains $n^{th}$ root of unity $\xi$
satisfying $\xi^n=1$ for large $n$ (All polynomial degrees are assumed less than $n$).

\begin{fact}[\textsf{Fast Evaluation}]\label{fc:fft}
Let $f\in \F[X]$ be a polynomial of degree $<d$ and $(\xi_1,\ldots,\xi_r)\in \F^r$ be distinct points in $\F$.
Then the vector $(f(\xi_1),\ldots,f(\xi_r))$ can be computed in $O((d+r)\log (d+r))$ $\F$ operations if $\xi_1,\ldots,\xi_r$ form roots
of unity, and in $O((d+r)\log^2(d+r))$ $\F$ operations otherwise.
\end{fact}

\begin{fact}[\textsf{Fast Interpolation}]\label{fc:ifft}
Let $\xi_1,\ldots,\xi_d$ be distinct points in $\F$ and $(v_1,\ldots,v_d)\in \F^d$. Then $(f_0,\ldots,f_{d-1})\in \F^d$
can be computed in $O(d\log^2 d)$ operations in $\F$ such that $f(\xi_i)=v_i$ for all $i\in [d]$ where
$f(X)=\sum_{i=0}^{d-1}f_iX^i$.
\end{fact}

\begin{fact}[\textsf{Fast Multiplication}]\label{fc:mult}
Let $\xi_1,\ldots,\xi_r$ be distinct points in $\F$. Then coefficients of $f(X)=\prod_{i=1}^r (X-\xi_i)$
can be computed in $O(r\log^2 r)$ operations in $\F$.
\end{fact}

\begin{fact}[\textsf{Multi KZG proofs} ~\cite{EPRINT:FeiKho23}]\label{fc:multkzg}
Let $\{\elt{x^i}\}_{i=1}^d$ be given for some $x\in \F$. Then for set of $r$ distinct points $\xi_1,\ldots,\xi_r$,
and a polynomial $f(X)\in \F[X]$ of degree $<d$,the vector $(\elt{h_1(x)},\ldots,\elt{h_r(x)})$,
where $h_i(X) = (f(X) - f(\xi_i))/(X - \xi_i)$ can be computed in
$O((r+d)\log(r+d))$ group and field operations when $\xi_1,\ldots,\xi_r$ are roots of unity, and in
$O(rlog^2 r + d\log d)$ group and field operations otherwise.
\end{fact}

\begin{fact}[\textsf{Lagrange Polynomials}]\label{fc:lagrange}
Let $\mathbb{S}=\{\xi_1,\ldots,\xi_r\}$ be a set of $r$ distinct points and let $\tau_1(X),\ldots,\tau_r(X)$ be
the corresponding lagrange polynomials of degree $r-1$ each. Let $Z_{\mathbb{S}}(X)=\prod_{i=1}^r (X-\xi_r)$ denote the vanishing polynomial
for $\mathbb{S}$. Then we have:
\begin{gather*}
    \sum_{i=1}^r \tau_i(X) = 1 \\
    \tau_i(X) = \frac{Z_{\mathbb{S}}(X)}{Z_{\mathbb{S}}'(\xi_i)(X-\xi_i)} \text{ for all } i\in [r]
\end{gather*}
\end{fact}

\noindent{\bf Formal Derivative}: For a polynomial $p(X) \in \F[X]$, we define the formal derivative of $p(X)$ as the polynomial
$u(X,X)$ where $u(X,Y)=\frac{p(X)-p(Y)}{X-Y}$. It can be seen that $u(X,X)$ equals the polynomial $p'(X)$ obtained by differentiating
$p(X)$ according to regular rules of calculus.
%Let us denote the formal derivative of $p(X)$ as $p^f(X)$.\\
%Then, $p^f(X)= u(X,X)$ where $u(X,Y)=\frac{p(X)-p(Y)}{X-Y}$\\\\
%It is well known that the formal derivative of a polynomial satisfies all the properties that we want the derivative of a polynomial to satisfy, like linearity, product rule, quotient rule etc.\\\\
%As a result, the formal derivative $p^f(X)$ defined as above can be computed from $p(X)$ in the \textit{usual} way, that is $\frac{d}{dx}X^k=k X^{k-1}$ and extend using linearity of derivative.\\
%Henceforth, we will represent the formal derivative of $p(X)$ as $p'(X)$ and because of the above statements, we will compute it in the usual way and freely use all the properties of derivatives on it.\\

\subsection{Some Useful Results}\label{subsec:sub-results}
We state and prove some facts which are used later througout the proof.

\begin{lemma}\label{lem:sumtoder}
For $K\subset [N]$, define $H_K$ to be $\{\xi^i:i \in K\}$. Let $p(X)$ be the vanishing polynomial of $H_{K}$.
Let $p'(X)$ and $p''(X)$ denote the formal first derivative and second derivative of $p(X)$ respectively.
Then, $p''(\xi^i)/p'(\xi^i)=2 \cdot \sum_{j\in K\setminus \{i\}} 1/(\xi^i-\xi^j)$ for all $i \in K$
\end{lemma}

\begin{proof}
    Observe that $p'(X)=\sum_{i\in K}\prod_{j\in K\setminus \{i\}}(X-\xi^j)$ and
    $p''(X)=\sum_{i\in K}\sum_{j\in K\setminus \{i\}}\prod_{k\in K\setminus \{i,j\}}(X-\xi^k)$. Thus for $r\in K$,
    we have:
    \begin{align*}
        p'(\xi^r) &= \prod_{j \in K \setminus \{r\}}(\xi^r-\xi^j), \\
        p''(\xi^r)&=\sum_{j\in K\setminus \{r\}}\prod_{k\in K\setminus \{r,j\}} (\xi^r-\xi^k)
        +\sum_{i\in K\setminus \{r\}}\prod_{k\in K\setminus \{r,i\}} (\xi^r-\xi^k)
    \end{align*}
    Note that only non-zero products in the expansion of $p''(\xi^r)$ occur when $i=r$ or $j=r$, resulting in the two summands for the same in the above equation.
    Moreover, we notice that both summands are the same, giving us $p''(\xi^r)=2\sum_{i\in K\setminus \{r\}}\prod_{k\in \setminus \{r,i\}}(\xi^r-\xi^k)$. One may
    now verify that $p'(\xi^r)/p''(\xi^r)$ gives the desired result claimed in the lemma.
\end{proof}
%To compute $p''(\xi^r)$, note that $p''(X)$ is a (double) sum of many products. When one substitutes $\xi^r$ in a product, the product is non zero if and only if $r=i\, \text{or}\, r=j$\\\\

%    Putting $\xi^r$ gives:
%    $$ p''(\xi^r)=\sum_{j\in K\setminus \{r\}}\prod_{k\in K\setminus \{r,j\}} (\xi^r-\xi^k)+$$
%    $$ \sum_{i\in K\setminus \{r\}}\prod_{k\in K\setminus \{r,i\}} (\xi^r-\xi^k) $$
%    But this is just,
%    $$2\cdot \sum_{j\in K\setminus \{r\}}\prod_{k\in K\setminus \{r,j\}} (\xi^r-\xi^k) $$

%    Now, we are left to show that $\frac{\frac{p''(\xi^r)}{2}}{p'(\xi^r})=\sum_{j\in K\setminus \{r\}} 1/(\xi^r-\xi^j)$, or
%    $$\sum_{j\in K\setminus \{r\}}\prod_{k\in K\setminus \{r,j\}} (\xi^r-\xi^k)=\prod_{j \in K \setminus \{r\}}(\xi^r-\xi^j) \sum_{j\in K\setminus \{r\}} 1/(\xi^r-\xi^j)$$

%    But we observe that:
%    $$\sum_{j\in K\setminus \{r\}}\prod_{k\in K\setminus \{r,j\}} (\xi^r-\xi^k)=\sum_{j\in K\setminus \{r\}}\frac{\prod_{k\in K\setminus \{r\}} (\xi^r-\xi^k)}{(\xi^r-\xi^j)}$$

%    But $\prod_{k\in K\setminus \{r\}} (\xi^r-\xi^k)$ is independent of $j$, so it can be pulled outside the sum:
%    $$\sum_{j\in K\setminus \{r\}}\prod_{k\in K\setminus \{r,j\}} (\xi^r-\xi^k)=\prod_{k\in K\setminus \{r\}} (\xi^r-\xi^k) \sum_{j\in K\setminus \{r\}} 1/(\xi^r-\xi^j) $$
%    But $k$ is just a dummy variable. We can replace $k$ by $j$ on the RHS:
%    $$\sum_{j\in K\setminus \{r\}}\prod_{k\in K\setminus \{r,j\}} (\xi^r-\xi^k)=\prod_{j\in K\setminus \{r\}} (\xi^r-\xi^j) \sum_{j\in K\setminus \{r\}} 1/(\xi^r-\xi^j) $$
%    This completes the proof.
%\end{proof}

\begin{comment}
    ************ ALTERNATIVE PROOF **************
    \textbf{An alternative proof of \ref{lem:sumtoder}}:\\
    An alternative proof is possible by inducting on the cardinality of $K\subset N$.\\
    The base case where $K$ has one element, say $i$ is trivial. This is because in that case $p$ is a linear polynomial $X-\xi^i$ and LHS=$\frac{p''}{p'}=\frac{0}{1}=0$ and on RHS we are summing over the empty set so the sum is $0$.\\\\
    Now let $|K|=k$ and the elements of $K$ be $a_1, a_2, \cdots, a_k$.
    Consequently $p(X)=\prod_{i \in K}(X-\xi^i)$\\
    We assume that the result holds for this case. Thus,
    $$p''(\xi^i)/p'(\xi^i)=2 \cdot \sum_{j\in K\setminus \{i\}} 1/(\xi^i-\xi^j)$$ for all $i \in K$.\\
    Suppose we introduce another element in the set $K$, call it $\alpha$
    The vanishing polynomial for the new $K$(call it $K')$ is $q(X)=p(X)(X-\xi^{\alpha})$\\
    Then by the product rule for derivatives, we have:
    $$q'(X)=p'(X)(X-\xi^{\alpha})+p(X)$$ and
    $$q''(X)=2p'(X)+p''(X)(X-\xi^{\alpha})$$

    It suffices to show that $$\frac{q''(\xi^i)}{q'(\xi^i)}=2\sum_{j\in K'\setminus \{i\}}\frac{1}{\xi^i-\xi^j}$$
    for all $i \in K'$\\
    For this we make 2 cases:\\
    \textbf{Case 1-}$i \in K$
    Then
    $$q'(\xi^i)=p'(\xi^i)(\xi^i-\xi^{\alpha})$$ and
    $$q''(\xi^i)=2p'(\xi^i)+p''(\xi^i)(\xi^i-\xi^{\alpha})$$
    This used that $p(\xi^i)=0$\\
    Thus, $$\frac{q''(\xi^i)}{q'(\xi^i)}=\frac{2p'(\xi^i)+p''(\xi^i)(\xi^i-\xi^{\alpha})}{p'(\xi^i)(\xi^i-\xi^{\alpha})}$$
    Dividing numerator and denominator by $p'(\xi^i)(\xi^i-\xi^{\alpha})$ gives:
    $$\frac{q''(\xi^i)}{q'(\xi^i)}=\frac{2}{\xi^i-\xi^\alpha}+\frac{p''(\xi^i)}{p'(\xi^i)}$$
    By induction hypothesis this becomes:
    $$\frac{q''(\xi^i)}{q'(\xi^i)}=\frac{2}{\xi^i-\xi^\alpha}+2 \cdot \sum_{j\in K\setminus \{i\}} 1/(\xi^i-\xi^j)$$
    Note that $$\frac{1}{\xi^i-\xi^\alpha}+ \sum_{j\in K\setminus \{i\}} 1/(\xi^i-\xi^j)=\sum_{j\in K'\setminus \{i\}}\frac{1}{\xi^i-\xi^j}$$
    Using that $K'=K \cup \{\alpha\}$\\
    Thus, $$\frac{q''(\xi^i)}{q'(\xi^i)}=2\sum_{j\in K'\setminus \{i\}}\frac{1}{\xi^i-\xi^j}$$ which completes case 1. \\\\
    \textbf{Case 2-}$i=\alpha$. Then
    $$q'(\xi^{\alpha})=p(\xi^{\alpha})$$ and
    $$q''(\xi^{\alpha})=2p'(\xi^{\alpha})$$
    These we get by replacing $X$ by $\xi^{\alpha}$ in the expressions of $q', q''$
    So, it is sufficient to prove that:
    $$\frac{2p'(\xi^{\alpha})}{p(\xi^{\alpha})}=2\sum_{j\in K'\setminus \{\alpha\}}\frac{1}{\xi^{\alpha}-\xi^j}$$

    But $$p(\xi^{\alpha})=\prod_{i \in K}(\xi^{\alpha}-\xi^i)=\prod_{i \in K'\setminus \{\alpha\}}(\xi^{\alpha}-\xi^i)$$ and $$p'(\xi^{\alpha})=\sum_{j \in K} \prod_{i \in K \setminus \{j\}}(\xi^{\alpha}-\xi^i)=\sum_{j \in K'\setminus \{\alpha\}} \prod_{i \in K' \setminus \{j, \alpha\}}(\xi^{\alpha}-\xi^i)$$
    But now, note that $$\sum_{j \in K'\setminus \{\alpha\}} \prod_{i \in K' \setminus \{j, \alpha\}}(\xi^{\alpha}-\xi^i)=\prod_{i \in K' \setminus \{ \alpha\}}(\xi^{\alpha}-\xi^i)\sum_{j \in K'\setminus \{\alpha\}}\frac{1}{\xi^{\alpha}-\xi^j}$$
    So, $$\frac{2p'(\xi^{\alpha})}{p(\xi^{\alpha})}=\frac{2\cdot \prod_{i \in K' \setminus \{ \alpha\}}(\xi^{\alpha}-\xi^i)\sum_{j \in K'\setminus \{\alpha\}}\frac{1}{\xi^{\alpha}-\xi^j}}{\prod_{i \in K'\setminus \{\alpha\}}(\xi^{\alpha}-\xi^i)}=$$\\
    $$=2\cdot\sum_{j \in K'\setminus \{\alpha\}}\frac{1}{\xi^{\alpha}-\xi^j}$$

    This completes the proof of case 2 and thus the proof of the lemma.
\end{comment}

\begin{lemma}[Sumcheck]\label{lem:sumcheck}
Let $u(X,Y)$ be a bi-variate polynomial over a finite field $\F$ with degree less than $N$ in each of the variables and
$\nroots$ be defined as the group of $N^{\text{th}}$ roots of unity $(N<<|\F|)$
with generator $\xi \in \F$. Then $\sum_{i\in [N]}u(X,\xi^i)= Nu(X,0)$
\end{lemma}

\begin{proof}
    For some $d<N$, we write $u(X,Y)=a_0+a_1 Y+ a_2 Y^2+\cdots+ a_d Y^d$
    where each $a_i$ is a polynomial in $X$ of degree less than $N$.
    Now we write the sum:
    \begin{align*}
        \sum_{i\in [N]}u(X,\xi^i) &= N a_0+a_1(\xi^+\xi^2+\cdots+\xi^N) \\
        &\, +a_2(\xi^2+\xi^4+\cdots+\xi^{2N})+ \cdots + a_d(\xi^d+\cdots +\xi^{Nd})
    \end{align*}
    But for any $\alpha=\xi^k$ for $k<N$, $\alpha+\alpha^2+\cdots \alpha^N=0$. Thus, all terms vanish except the first term
    and hence $\sum_{i\in [N]}u(X,\xi^i)=N a_0$.
    The lemma follows by observing that $a_0=u(X, 0)$.
\end{proof}

\begin{lemma}\label{lem:zk-hat}
Let $Z_\nroots(X)$ be the vanishing polynomial for $\nroots$, let $\widehat{Z}_K(X)$ and  $Z_K(X)$ be the vanishing polynomials for $H_{[N]\setminus K}$ and $H_K$ respectively.
Let $\mu_1(X),\ldots,\mu_N(X)$ be Lagrange polynomials for the set $\nroots=\{\xi,\ldots,\xi^N\}$. Then:
\begin{gather}
    \widehat{Z}_K(X) = \sum_{j\in K}\frac{Z_\nroots'(\xi^j)}{Z_K'(\xi^j)}\mu_j(X), \\
    \widehat{Z}_K'(X)= \sum_{j\in K}\frac{Z_\nroots'(\xi^j)}{Z_K'(\xi^j)}\mu_j'(X)
\end{gather}
\end{lemma}
We use the following standard observation:
\begin{fact}
    If polynomials $f,g$ of degree $<N$ agree on $N$ points, then they are equal as polynomials, that is, $f(X)=g(X)$
\end{fact}

\begin{proof}
    Note that the second equation follows from the first by linearity of derivatives, so it suffices to prove the first equation.
    Both the sides of the identity are polynomials of degree $<N$, so it suffices to show their evaluations are identical over $N$ distcint points.
    In particular we show their evaluations are identical over $\nroots$.
    Consider evaluating LHS and RHS at $\xi^i$ for $i \in [N]\setminus K$.
    The left side is $0$ by definition of $\hat{Z}_K(X)$, while the right hand side is zero by the properties of Lagrange polynomials.
    Now consider evaluations LHS and RHS at $\xi^i$ for $i \in K$.
    The RHS is $\frac{Z_\nroots'(\xi^i)}{Z_K'(\xi^i)}$ by properties of Lagrange polynomials, while the
    LHS is $\prod_{j \in [N]\setminus K} (\xi^i-\xi^j)$\\
    Multiplying dividing by $\prod_{ j \in K \setminus \{i\}}(\xi^i-\xi^j)$ gives:
    $$LHS = \frac{\prod_{j \in [N] \setminus \{i\}} (\xi^i-\xi^j)}{\prod_{j \in K \setminus \{i\}}(\xi^i-\xi^j)}$$
    Which is $\frac{Z_\nroots'(\xi^i)}{Z_K'(\xi^i)}$, the same as the right hand side.
    This proves the claim.
\end{proof}

\begin{lemma}\label{lem:lamda-deriv}
Let $\mu_1,\ldots,\mu_N$ be the lagrange polynomials for the set $\nroots=\{\xi^i:i\in [N]\}$
of the $N^{th}$ roots of unity. Then we have:
\begin{equation*}
    \mu_i'(\xi^j) = \begin{cases}
                        \frac{(N-1)}{2\xi^{i}}  \text{ if } j=i\\
                        \frac{\xi^i}{\xi^j(\xi^j-\xi^i)} \text{ otherwise }
    \end{cases}
\end{equation*}
\end{lemma}

\begin{proof}
    Suppose first that $i \neq j$.
    We know that $\mu_i(X)= \frac{Z_H(X)}{Z_H'(\xi^i)(X-\xi^i)}$.
    Thus, by applying quotient rule(we can apply quotient rule because $\mu_i$ is defined at $\xi^j$ as $j\neq i$):
    $$\mu_i'(X) \cdot Z_H'(\xi^i)= \frac{(X-\xi^i)(N\cdot X^{N-1})-(X^N-1)}{(X-\xi^i)^2}$$
    Substituting $X$ by $\xi^j$, we get:
    $$\mu_i'(\xi^j) \cdot \frac{N}{\xi^i}= \frac{N(\xi^j-\xi^i)}{\xi^j (\xi^j-\xi^i)^2}$$
    Thus, we get:
    $$\mu_i'(\xi^j)=\frac{\xi^i}{\xi^j(\xi^j-\xi^i)}$$
    Now consider $i=j$:
    Then, $\mu_i(X)=\frac{\prod_{j \in [N] \setminus \{i\}}(X-\xi^j)}{Z_H'(\xi^i)}$\\
    Thus,  $\mu_i(X)\cdot Z_H'(\xi^i)=\prod_{j \in [N] \setminus \{i\}}(X-\xi^j)$\\
    Thus, $\mu_i'(X) \cdot \frac{N}{\xi^i}=\sum_{j \in [N] \setminus \{i\}}\prod_{k \in [N]\setminus\{i,j\}}(X-\xi^k)$\\
    Now, substituting $X$ by $\xi^i$ gives:
    \begin{align*}
        \mu_i'(\xi^i) \cdot \frac{N}{\xi^i} &= \sum_{j \in [N] \setminus \{i\}}\prod_{k \in [N]\setminus\{i,j\}}(\xi^i-\xi^k) \\
        &=\sum_{j \in [N] \setminus \{i\}}\frac{\prod_{k \in [N]\setminus\{i\}}(\xi^i-\xi^k)}{\xi^i-\xi^j} \\
        &= \prod_{k \in [N]\setminus\{i\}}(\xi^i-\xi^k)\sum_{j\in [N]\setminus \{i\}}\frac{1}{\xi^i-\xi^j} \\
        &=Z_H'(\xi^i)\sum_{j\in [N]\setminus \{i\}}\frac{1}{\xi^i-\xi^j}
    \end{align*}
    Cancelling off $Z_H'(\xi^i)=N/\xi^i$ in the above, and using Lemma ~\ref{lem:sumtoder}, we get:
    \begin{align*}
        \mu_i'(\xi^i) &= \sum_{j\in [N]\setminus \{i\}}\frac{1}{\xi^i-\xi^j} =\frac{Z_H''(\xi^i)}{2 Z_H'(\xi^i)}
        =\frac{N-1}{2\xi^i}
    \end{align*}

\end{proof}

\begin{lemma}\label{lem:tau}
Let $K\subseteq \F$ be a set of cardinality $k$ and $\mathcal{X}=\{x_j:j\in K\}$ be a set
where $x_j$ for $j\in K$ are distinct elements of $\F$. Let $Z_\mathcal{X}(X)=z_k X^k+\cdots+z_0$ denote the vanishing polynomial of $\mathcal{X}$
and $\{\eta_j(X)\}_{j\in K}$ denote the lagrange polynomials such that $\eta_i(x_j)=\delta_{ij}$ for $i,j\in K$. Then for all $j\in K$,
we have $\eta_j'(x_j)=F_K(x_j)/Z_\mathcal{X}'(x_j)$ where the polynomial
$F_K(X)$ is defined as
\[F_K(X)=\binom{k}{2}z_k X^{k-2}+\cdots+\binom{2}{2}z_2=\sum_{j=2}^k z_j\binom{j}{2}X^{j-2} \]
\end{lemma}
\begin{proof}
    For $j\in K$ we have $\tau_j(X)=\frac{Z_\mathcal{X}(X)}{(X-x_j)Z_\mathcal{X}'(x_j)}= \frac{1}{Z_\mathcal{X}'(x_j)}\frac{Z_\mathcal{X}(X)}{X-x_j}$ by definition of Lagrange polynomials.\\
    By doing long division of $Z_\mathcal{X}(X)$
    by $X-x_j$ we have:
    \begin{align*}
        \tau_j(X) &= \frac{1}{Z_\mathcal{X}'(x_j)}\big(z_k X^{k-1} + (x_jz_k + z_{k-1})X^{k-2} + \cdots\\ +
        (x_j^{k-1}z_k + \cdots + z_1)\big)
        &= \frac{1}{Z_\mathcal{X}'(x_j)}\sum_{p=0}^{k-1} \left(\sum_{q=p+1}^k z_q x_j^{q-p-1}\right)X^p
    \end{align*}

    Now, we use linearity property of derivative to differentiate:

    $$   \tau_j'(X) = \frac{1}{Z_\mathcal{X}'(x_j)}\sum_{p=0}^{k-1} \left(\sum_{q=p+1}^{k}z_q x_j^{q-p-1}\right)p X^{p-1} $$
    In this step we just differentiated $X^p$ to get $p X^{p-1}$
    $$ \tau_j'(X) = \frac{1}{Z_\mathcal{X}'(x_j)}\sum_{p=1}^{k-1} p  \sum_{q=p+1}^{k}z_q x_j^{q-p-1} X^{p-1} $$
    In this step, we just applied distributive property to the rightmost bracket and brought $p$ outside the sum over $q$\\
    Now we put $X=x_j$:

    $$\tau_j'(x_j) = \frac{1}{Z_\mathcal{X}'(x_j)} \sum_{p=1}^{k-1} p \sum_{q=p+1}^k z_q x_j^{q-2} $$
    $$= \frac{1}{Z_\mathcal{X}'(x_j)}\sum_{q=2}^k z_q x_j^{q-2} \sum_{p=1}^{q-1} p $$
    Here, we reversed the order of the sum. $p$ going from 1 to $k-1$ and $q$ going from $p+1$ to $k$ is same as $q$ going from $2$ to $k$ and $p$ going from 1 to $q-1$
    $$\tau_j'(x_j)= \frac{1}{Z_\mathcal{X}'(x_j)}\sum_{q=2}^k z_q \binom{q}{2} x_j^{q-2} $$
    Here we used that $\sum_{p=1}^{q-1}p=\frac{q(q-1)}{2}=\binom{q}{2}$

    But $q$ is just a dummy variable. We might as well replace $q$ by $j$

    Thus, $$\tau_j'(x_j)= \frac{1}{Z_\mathcal{X}'(x_j)}\sum_{j=2}^k z_j \binom{j}{2} x_j^{j-2} $$
    If we compare this to the definition of $F_K(X)$ given in the lemma statement, we get:
    $$\tau_j'(x_j)= \frac{F_K(x_j)}{Z_\mathcal{X}'(x_j)}$$


\end{proof}

\begin{comment}
\subsection{Brief Proof Summary of lemma \ref{lem:sum-computation}}\label{subsec:sum-computational}
What we want to compute is a sum over the set $K$ for all $i \in I$ or a sum over the set $I$ for all $j \in K$. Let us give a sketch for the first case.

We interpolate a polynomial $p$ in such a way that its evaluations at $\nroots$ correspond to the numerators of the required sum. We take the full $\nroots$ and not subsets of it so that we can apply sumcheck easily later.
Next we introduce some rational functions. These will be very useful to reduce the sum we need to compute to an evaluation of the rational functions at some points.
These evaluations can be reduced to evaluations of polynomials using the simple form of $Z_{\nroots}$. By introducing some more polynomials and using formulas for sumcheck, finally, the entire task can be reduced to calculating $p(0)$ and $p'(X)$ at certain points.

The polynomial $p$ turns out to be a product of polynomials: $p(X)=\widehat{Z}_K(X)\cdot q(X)$ and so the next step is to get the evaluations of $\widehat{Z}_K(X)$ and $q(X)$ at those points.
For this, some further lemmas are used to get evaluations of $q(X)$ at degree $q$ many points and thus $q(X)$ is obtained by interpolation.
Next $q(0)$ can be computed and thus $p(0)$ is also computed.
$Z_K(X)$ and $q'(X)$ can also be computed easily(details in the proof)\\
This leaves computation of $\widehat{Z}_K(X)$ at those points as the main task. For this another lemma is needed related to derivatives of Lagrange polynomials.
Eventually the task reduces to needing to compute something which can be computed efficiently using another lemma.\\
Finally, every single step involves a reduction which can be done in the number of operations allowed in the statement of the lemma. So, we get what we want.\\\\
For the second case, the proof is very similar, except for some interchanging of $I$ and $K$, and interchanging of $i$ and $j$. \\
This completes a brief summary of the proof.
\end{comment}

\subsection{Proof of lemma ~\ref{lem:sum-computation}}\label{subsec:sum-computation}

\begin{proof}

    We give the proof for $e_i$ for all $i \in \setind$ in full detail and briefly mention the modifications needed to compute $b_j$ for all $j \in K$. \\

    \begin{equation}\label{eq:ei}
    e_i = \sum_{j\in K\setminus \{i\}} \frac{\Delta t_j}{\xi^i - \xi^j}
    \end{equation}
    We need the above for all $i \in \setind$. Recall that we assume $\setind \subset K$ in this case.
    To compute $e_i$, we first define a polynomial $p(X)$ of degree at most $N-1$ such that
    $p(\xi^j)=\Delta t_j$ for $j\in K$ and $p(\xi^j)=0$ for $j \in [N]\setminus K$.
    Therefore, the vanishing polynomial of $H_{[N]\setminus K}$ divides $p(X)$ and
    there exists a polynomial $q(X)$ of degree at most $|K|-1$ such that:
    \begin{equation}\label{eq:px}
    p(X) = \hat{Z}_K(X)\cdot q(X)
    \end{equation}
    where $\hat{Z}_K(X)=\prod_{i\in [N]\setminus K}(X-\xi^i)$ is the vanishing polynomial of $H_{[N]\setminus K}$\\

    Now introduce the rational functions:
    \begin{gather}\label{eq:fgr}
    f_i(X) = \sum_{j\in [N]\setminus \{i\}}\frac{p(\xi^j)}{X-\xi^j}, i\in \setind \\
    g_i(X) = \sum_{j\in [N]\setminus \{i\}}\frac{p(X)}{X-\xi^j}, i\in \setind \\
    r_i(X) = \sum_{j\in [N]\setminus \{i\}}\frac{p(X) - p(\xi^j)}{X-\xi^j}, i\in \setind
    \end{gather}
    Note that $f_i(\xi^i)=\sum_{j\in [N]\setminus \{i\}}\frac{p(\xi^j)}{\xi^i-\xi^j}$.
    But, by the way $p(X)$ is defined, we have that $f_i(\xi^i)=\sum_{j\in K\setminus \{i\}}\frac{\Delta t_j}{\xi^i-\xi^j}=e_i$\\
    \textbf{Thus, we need $f_i(\xi^i)$ for all $i \in I$.}

    Note that $f_i(X) = g_i(X) - r_i(X)$ for $i\in I$.
    Thus, $e_i=g_i(\xi^i)-r_i(\xi^i)$.
    Thus, we need to compute $g_i(\xi^i)$ and $r_i(\xi^i)$ for all $i \in I \subset K$.
    \begin{align*}
        g_i(\xi^i) &= p(\xi^i)\sum_{j\in [N]\setminus \{i\}} \frac{1}{\xi^i-\xi^j} \\
        &= \frac{p(\xi^i)Z_{\nroots}''(\xi^i)}{2 Z_{\nroots}'(\xi^i)} \text{From lemma \ref{lem:sumtoder}} \\
        &= \frac{(N-1)\Delta t_i}{2\xi^i}
    \end{align*}

    We used the fact that $Z_H(X)=X^N-1$ and that $p(\xi^i)=\Delta t_i$.
    Thus, in $O(|\setind|)$ operations, we get $g_i(\xi^i)$ for all $i$.
    If we compute $r_i(\xi^i)$ for all $i \in \setind$, then computing $f_i(\xi^i)$ takes only
    $O(|\setind|)$ operations. Therefore, it suffices to compute $r_i(\xi^i)$ for all $i \in I$.
    We write $r_i(X)$ as:
    \begin{align}\label{eq:rx}
    r_i(X) &= \sum_{j\in [N]}\frac{p(X)-p(\xi^j)}{X-\xi^j} - \frac{p(X)-p(\xi^i)}{X-\xi^i} \nonumber \\
    \intertext{Now, defining the bivariate polynomial $u(X,Y)=(p(X) - p(Y))/(X-Y)$ gives:}
    &r_i(X)= \sum_{j\in [N]}u(X,\xi^j) - u(X,\xi^i) \nonumber
    \end{align}
    Next, we define $r(X)=\sum_{j\in [N]}u(X,\xi^j)$.
    Thus, we have:
    $$r_i(X)=r(X)-u(X, \xi^i)$$
    or
    $$r_i(\xi^i)=r(\xi^i) - u(\xi^i,\xi^i)$$
    But, by definition and properties of formal derivative, we have the following polynomial identity
    $$u(X, X)= p'(X)$$
    where $p'(X)$ denotes the derivative of the polynomial $p(X)$ computed in the usual way.
    Thus, $u(\xi^i, \xi^i)=p'(\xi^i)$ from which we conclude:
    $$r_i(\xi^i)=r(\xi^i) - p'(\xi^i) $$

    \textbf{Thus, $r_i(\xi^i)$ for all $i \in \setind$ can be computed by evaluating polynomials $r(X)$ and $p'(X)$ at the set $H_{\setind}$.}
    Now, from Lemma \ref{lem:sumcheck}, we have that $r(X)=N u(X,0)=N \frac{(p(X) - p(0))}{X}$ and thus $r(\xi^i)$ for $i\in \setind$ is given by:
    $$r(\xi^i)=N \frac{(\Delta t_i - p(0))}{\xi^i}$$
    If we get $p(0)$, then we can get $r(\xi^i)$ for all $i$ in $O(m)$ operations.
    \textbf{Thus, it suffices to compute $p(0)$ and $p'(X)$ at the set $\{\xi^i:i \in I\}$}\\\\
    Let $p(X)=\widehat{Z}_K(X)\cdot q(X)$
    as in Equation \eqref{eq:px} and let $Z_K(X)$ be the vanishing polynomial for $H_K$.
    Note that $Z_K(X) \cdot \hat{Z}_K(X)=X^N-1$(by definition).
    Notice that $Z_\nroots'(\xi^j)= \frac{N}{\xi^j} $.
    Moreover, $Z_K(X)$ can be computed by fast multiplication in $O(|K|\log^2 |K|)$ operations and then it can be differentiated monomial by monomial to get $Z_K'(X)$.\\
    Then, we can do fast evaluation to get $Z_K'(X)$ at $\{\xi^j:j\in K\}$ in $O(|K|\log^2 |K|)$ operations\\
    Recall that Lemma \ref{lem:zk-hat} tells us that $\hat{Z}_K(\xi^j)=\frac{Z_{\nroots}'(\xi^j)}{Z_K'(\xi^j)}$ and
     thus we can compute $\hat{Z}_K(\xi^j)=\frac{\frac{N}{\xi^j}}{Z_K'(\xi^j)}$ for all $j \in K$.
     Moreover, recall from \eqref{eq:px} that $q(\xi^j)=\frac{p(\xi^j)}{\hat{Z}_K(\xi^j)}=\frac{\Delta t_j}{\widehat{Z}_K(\xi^j)}$.
    So, we can compute $q(\xi^j)$ for all $j \in K$.
    As degree of $q(X)$ is strictly less than $|K|$ we can interpolate to get $q(X)$ in $O(|K|\log^2 |K|)$ field operations.
    Again by differentiating monomial by monomial, we get $q'(X)$ and by fast evaluation we can get evaluations of $q'(X)$ at $\xi^i$ for all $i \in I$, again in $O(|K|\log^2 |K|)$ field operations.

    Now we can compute $p(0)$:
    Since $p(0)=q(0)\cdot \hat{Z}_K(0)$, and
    $\hat{Z}_K(0)=\frac{Z_H(0)}{Z_K(0)}=\frac{-1}{Z_K(0)}$, this enables us to compute $p(0)$,
    since we know $Z_K(X)$ and $Z_K(0)$ is just its constant term.
    Similarly, we know $q(X)$ so can compute $q(0)$ as it is just the constant term.
    Thus, we can compute $p(0)$.

    \textbf{Now it remains to compute $p'(X)$ at $\{\xi^i:i \in I\}=H_I$.}
    Using the  product rule for derivatives we have: $p'(X)=q(X) \hat{Z}_K'(X)+q'(X) \hat{Z}_K(X)$.

    Thus, to get $p'(X)$ at $H_I$, we need $q, q', \hat{Z}_K$ and $\hat{Z}_K'$ at $H_I$.
    If we get these, then in $O(|I|)$ operations we would get $p'(X)$ at $H_I$.
    So far we have computed $q, q', \hat{Z}_K$ at $H_I$. There's only one last remaining item which
    is $\hat{Z}_K'(X)$ at the set $H_I$.

    \textbf{So, it suffices to compute $\hat{Z}_K'(X)$ at $H_I$\\}

    For this, we recall the second equation of lemma \ref{lem:zk-hat} and replace $X$ by $\xi^i$ to get:
    $$  \widehat{Z}_K'(\xi^i) = \sum_{j\in K\setminus \{i\}} \frac{Z_\nroots'(\xi^j)}{Z_K'(\xi^j)}\mu_j'(\xi^i) + \frac{Z_\nroots'(\xi^i)}{Z_K'(\xi^i)}\mu_i'(\xi^i)$$
    We obtained the above by splitting the sum $j \in K$ into two sums, one with $j \in K \setminus \{i\}$ and other with $j=i$.
    Using lemma \ref{lem:lamda-deriv}, this becomes:
    $$ \widehat{Z}_K'(\xi^i)= N\xi^{-i}\sum_{j\in K\setminus \{i\}}\frac{1}{Z_K'(\xi^j)(\xi^i-\xi^j)} + \frac{N(N-1)}{2\xi^{2i}Z_K'(\xi^i)}.$$
    So, if we can compute $\sum_{j\in K\setminus \{i\}}\frac{1}{Z_K'(\xi^j)(\xi^i-\xi^j)}$ for all $i \in I$, then in $O(|I|)$ operations, we will get $\hat{Z}_K'(\xi^i)$ for all $i \in I$ which is what we want.

    Let $$\varphi_i=\sum_{j\in K\setminus \{i\}}\frac{1}{Z_K'(\xi^j)(\xi^i-\xi^j)}$$
    Thus,\textbf{ it suffices to compute $\varphi_i$ for all $i \in I$}\\

    Define:
    $$ \Phi_i(X) = \sum_{j\in K\setminus \{i\}} \frac{1}{Z_K'(\xi^j)(X-\xi^j)} $$
    at $X=\xi^i$. Then it is clear that $\varphi_i=\Phi_i(\xi^i)$.
    Let $\{\eta_i(X)\}_{i\in K}$ be the Lagrange polynomials for the set $\{\xi^i:i\in K\}= H_K$.
    Then, $\eta_j(X)=\frac{Z_K(X)}{Z_K'(\xi^j)(X-\xi^j)}$.
    Thus, $\frac{\eta_j(X)}{Z_K(X)}=\frac{1}{Z_K'(\xi^j)(X-\xi^j)}$.\\
    Thus, $\Phi_i(X)$ can be rewritten as:
    \begin{align*}
        \Phi_i(X)&=\sum_{j\in K\setminus \{i\}} \frac{\eta_j(X)}{Z_K(X)}\\
        &= \sum_{j\in K\setminus \{i\}} \frac{\eta_j(X)/(X-\xi^i)}{Z_K(X)/(X-\xi^i)}
    \end{align*}

    We need to compute $\varphi_i=\Phi_i(\xi^i)$ for all $i \in I$. Thus, putting $X=\xi^i$ in the above, we have:

    $$\varphi_i = \Phi_i(\xi^i) = \left(\sum_{j\in K\setminus \{i\}}\frac{\eta_j(X)/(X-\xi^i)}{Z_K(X)/(X-\xi^i)}\right)(\xi^i)$$
    This can be simplified as:
    \begin{align*}\varphi_i = \Phi_i(\xi^i) &= \left(\sum_{j\in K\setminus \{i\}}\frac{\eta_j(X)/(X-\xi^i)}{\prod_{k \in K \setminus \{i\}}(X-\xi^k)}\right)(\xi^i)\\
    &= \sum_{j\in K\setminus \{i\}}\left(\frac{\eta_j(X)/(X-\xi^i)}{\prod_{k \in K \setminus \{i\}}(X-\xi^k)}\right)(\xi^i)\\
    &= \sum_{j\in K\setminus \{i\}}\frac{\left(\eta_j(X)/(X-\xi^i)\right)(\xi^i)}{\left(\prod_{k \in K \setminus \{i\}}(X-\xi^k)\right)(\xi^i)}\\
    &= \frac{1}{Z_K'(\xi^i)}\sum_{j\in K\setminus \{i\}}\left(\eta_j(X)/(X-\xi^i)\right)(\xi^i)
    \end{align*}

    Now, \begin{align*}
             \left(\eta_j(X)/(X-\xi^i)\right)(\xi^i)&=\left(\frac{Z_K(X)}{(X-\xi^j)(X-\xi^i)Z_K'(\xi^j)}\right)(\xi^i)\\
             &= \frac{Z_K'(\xi^i)}{(\xi^i-\xi^j)Z_K'(\xi^j)}
    \end{align*}

    But if we compute, $\eta_j'(\xi^i)$ by quotient rule, we get:
    $$\eta_j'(\xi^i)=\frac{Z_K'(\xi^i)}{(\xi^i-\xi^j)Z_K'(\xi^j)}$$
    Thus,
    $$\left(\eta_j(X)/(X-\xi^i)\right)(\xi^i)=\eta_j'(\xi^i) $$
    As both are separately equal to $\frac{Z_K'(\xi^i)}{(\xi^i-\xi^j)Z_K'(\xi^j)}$.\\
    Thus, we get:


    $$\varphi_i =\Phi_i(\xi^i) =  \frac{1}{Z_K'(\xi^i)}\sum_{j\in K\setminus \{i\}}\eta_j'(\xi^i)$$
    But, $$\sum_{j\in K\setminus \{i\}}\eta_j'(\xi^i)= \sum_{j\in K}\eta_j'(\xi^i)-\eta_i'(\xi^i)=-\eta_i'(\xi^i)$$
    using that the sum of Lagrange polynomials is $1$ and thus the sum of the derivatives of the Lagrange polynomials is $0$

    Thus,
    $$\varphi_i =\Phi_i(\xi^i) = \frac{-\eta_i'(\xi^i)}{Z_K'(\xi^i)}$$

    If we get $\eta_i'(\xi^i)$ for all $i\in \setind$, then as we already have $Z_K'(\xi^i)$ for all $i \in I$, we get $\varphi_i$ for all $i \in I$ in $O(|I|)$ computations which is what we need. \textbf{Thus, it suffices to efficiently compute $\eta_i'(\xi^i)$ for $i\in \setind$}

    For this recall Lemma \ref{lem:tau} and observe that in our setting $\mathcal{X}$ is just $H_K$, $x_j$ are $\xi^j$ and $Z_\mathcal{X}(X)$ which is the vanishing polynomial of $H_K$ is just $Z_K(X)$. The $\eta_i$ in the lemma are the Lagrange polynomials corresponding to $\mathcal{X}=H_K$ so they are same as the $\eta_i$ in our setting\\\\
    Thus, we get:
    $$\eta_i'(\xi^i)=\frac{F_K(\xi^i)}{Z_K'(\xi^i)}$$

    Since we already have $Z_K'(\xi^i)$ for all $i \in I$,
    it suffices to compute $F_K(\xi^i)$ for all $i \in I$ to compute $\eta_i'(\xi^i)$ , $i \in I$.
    % in $O(|I|)$ operations which is what we want.
    \textbf{Thus, it suffices to compute $F_K(\xi^i)$ for all $i \in I$}\\

    From the definition $F_K$ (Lemma ~\ref{lem:tau}), it can be computed once we know
    coefficients  $z_0, z_1, z_2, \cdots, z_k$ of the polynomial $Z_K(X)$.
    But, we know the polynomial $Z_K$ (via fast multiplication) hence we know $z_0, z_1, \cdots, z_{k}$.
    Thus, we get the coefficients of the polynomial $F_K(X)$, hence the polynomial $F_K(X)$.
    Now, by fast evaluation, we can get $F_K(\xi^i)$ for all $i \in I$. This will take $O(|K| \log^2|K|)$ field operations.
    \textbf{Thus, we have computed $F_K(\xi^i)$ for all $i \in I$}\\

    This completes the proof for $e_i$ (Note that in the proof we throughout used that $|\setind| \leq |K|$ so that the various $O(|\setind|)$ operations do not contribute to the final complexity).\\
    For the case of $b_j$, notice first that it is very similar to the case of $e_i$ except that the roles of $\setind$ and $K$ are reversed and $i$ and $j$ are reversed.

    First of all define $c_i$ for $i$ outside $\setind$ to be 0. For $i \in \setind$, we already know what $c_i$ is(can be computed by fast evaluation).
    Next, define polynomial $p$ of degree atmost $N-1$ such that $p(\xi^i)=c_i$ for all $i \in [N]$.
    From here on, we proceed exactly as we did for the $e_i$ case, replacing every $\setind$ with $K$, every $K$ with $\setind$, every $i$ with $j$ and every $j$ with $i$.
    All the required lemmas are also modified in this way.

    We reach till
    $$\Phi_j(X)=\sum_{i\in I\setminus \{j\}} \frac{\eta_i(X)}{Z_I(X)}$$
    and we need to compute $\varphi_j=\Phi_j(\xi^j)$ for all $j \in K$.\\

    Here, we deviate: For $j \in K\setminus I$, we can very easily compute $\varphi_j=\Phi_j(\xi^j)$ as
    \begin{align*}
        \Phi_j(\xi^j)&=\sum_{i\in I\setminus \{j\}} \frac{\eta_i(\xi^j)}{Z_I(\xi^j)}\\
        &=\frac{1}{Z_I(\xi^j)} \sum_{i\in I}\eta_i(\xi^j)\\
        &=\frac{1}{Z_I(\xi^j)}
    \end{align*}

    Using that the sum of all Lagrange polynomials is $1$. \textbf{So, it suffices to compute $\varphi_j$ for all $j \in \setind$}

    Now, again continue exactly as in the $e_i$ case until we reach:
    $$\varphi_j =\Phi_j(\xi^j) = \frac{-\eta_j'(\xi^j)}{Z_I'(\xi^j)}$$
    for all $j \in \setind$. But $Z_I'(\xi^j)$ we would have already computed for all $j \in \setind$ (fast multiplication and fast evaluation). \\
    \textbf{So, it suffices to compute $-\eta_j'(\xi^j)$ for all $j \in \setind$\\}
    But we computed $\eta_i'(\xi^i)$ for $i\in \setind$ during the computation of $e_i$. Just reuse whatever we got!
    This completes the computation for $b_j$ for all $j \in K$.
    Thus, this finishes the proof of lemma \ref{lem:sum-computation}
\end{proof}


\section{Reducing indexed lookups to subvector lookups}\label{sec:generic-transformation-app}
We give the proof of Lemma ~\ref{lem:generic-transformation} here.
\begin{proof}[Proof of Lemma ~\ref{lem:generic-transformation}]
Assume that there exists $i\in [m]$ such that $v_i\neq \vec{t}[\,a_i\,]$.
 We bound the probability
that the check $\vec{v} + \gamma \vec{a}\leq \vec{t}+\gamma \vec{I}_n$ succeeds for
$\gamma\gets \F$ chosen uniformly. The probability is clearly upper bound by the probability
that $v_i + \gamma a_i\in \vec{t}+\gamma \vec{I}_n$, or $v_i+\gamma a_i=t_j + \gamma j$ for
some $j\in [n]$, or $v_i + \gamma (a_i - j)=t_j$. We note that for $a_i\neq j$ (by assumption),
the left hand side is uniform in $\F$, and hence occurs in $\vec{t}$ with probability at most
$n/|\F|$. This concludes the proof of the lemma.
\end{proof}

\begin{comment}
Let us say we have a subvector lookup argument $L$ which takes as input two vectors say $a, b$ and determines whether $b$ is a subvector of $a$. It outputs $1$ if $b$ is a subvector of $a$ and $0$ if $b$ is not a subvector of $a$.
Recall that $b$ is said to be a subvector of $a$ if every element of $b$ is in $a$\\
Indexed lookup argument $L'$ takes in 3 vectors $a, b, c$ where $c$ is a vector of indices with $|b|=|c|$ and checks that $b$ is a subvector of $a$ and that $a[c_i]=b_i$ for all $i$. That is, it does $L(a, b)$ first and further checks that $a$ restricted to the indices in $c$(in a particular order) gives $b$(in that particular order).
If $b$ is a subvector of $a$ and $a[c_i]=b_i$ for all $i$ then it outputs $1$ else it outputs $0$.\\

\textbf{Note:} Given $a,b$ with $b$ subvector of $a$, there will always exist a $c$ with $|c|=|b|$ such that $a[c_i]=b_i$ for all $i$. The $c$ may not be unique though. \\\\
So, we are given $T, v, a$ with $|a|=|v|$ and we want to compute $L'(T, v, a)$ using $L(\cdot, \cdot)$.
Let $|T|=N$. Let $|v|=m=|a|$.
Procedure to compute $L'(T, v, a)$ is as follows:
\begin{itemize}
    \item First compute $L([N], a)$. This checks that $a$ is indeed a vector of indices. Call the output of this as $b_1$
    \item Next compute $L(T, v)$. This checks that $v$ is indeed a subvector of $T$. Call the output of this as $b_2$
    \item Choose a $\gamma$ uniformly at random from $\F^*$
    \item Compute $L(T+\gamma[N], v+\gamma a)$ which checks that $v+\gamma a$ is a subvector of $T+\gamma [N]$. Call the output $b_3$.
    \item $L'(T, v, a)$ is then computed as $b_1 \wedge b_2 \wedge b_3$
\end{itemize}
This procedure works for any $T,v,a$ with $|v|=|a|$. So, this gives a procedure for calculating $L'(\cdot, \cdot, \cdot)$.

\begin{lemma}\label{lem:reduction}
Let the procedure for getting $L'(\cdot, \cdot, \cdot)$ be as above. Then, if $L(\cdot, \cdot)$ is a valid subvector lookup argument then the procedure yields a valid indexed lookup argument $L'(\cdot, \cdot, \cdot)$ with an error probability upper bounded by $\frac{O(1)}{|\F|}$.
\end{lemma}

\begin{proof}
    To show that the procedure yields a valid indexed lookup argument $L'(T, v, a)$, we need to show the following:
    \begin{enumerate}

        \item If $v$ is indeed a subvector of $T$ with $T$ restricted to $a$ giving $v$, then $b_1=b_2=b_3=1$ with probability lower bounded by $1-\frac{O(1)}{|\F|}$
        \item If either $v$ is not a subvector of $T$ or if $v$ is a subvector of $T$ but restriction of $T$ to $a$ is not $v$, then at least one of $b_1, b_2, b_3$ is $0$ with probability lower bounded by $1-\frac{O(1)}{|\F|}$
    \end{enumerate}

    For (1), suppose $v$ is a subvector of $T$ with $T$ restricted to $a$ giving $v$. Then $L(T, v)=b_2=1$.
    Moreover, $a$ is a subvector of $[N]$(else restricting $T$ to $a$ won't even make sense).
    So, $L([N], a)=b_1=1$.
    Lastly, since $v_i=T[a_i]$ for all $i \in m$, we have that for any $\gamma \in \F^{*}$,
    $$T[a_i]+\gamma a_i=v_i+\gamma a_i$$ for all $i$ \\
    But $T[a_i]+\gamma a_i$ is of course an element of $T+\gamma [N] $ for all $i \in [m]$.(Every $a_i \in [N]$).
    So, $\{T[a_i]+\gamma a_i:i \in [m]\}$ is a subvector of $T+\gamma [N] $.
    Thus, $\{v_i+\gamma a_i:i \in [m]\}$ is a subvector of $T+\gamma [N] $.
    So, $v+\gamma a$ is a subvector of $T+\gamma [N] $.
    Thus, $L(T+\gamma[N], v+\gamma a)=b_3=1$

    This completes proof of (1).(In this case the probability is actually 1) \\\\
    For proof of (2). if $v$ is not a subvector of $T$ then $L(T, v)=b_2=0$ with probability 1 and we are done.
    So let us assume that $v$ is a subvector of $T$ and restriction of $T$ to $a$ is not $v$.
    If $a$ is not a vector of indices (that is, some $a_i \notin [N]$) then $L([N], a)=b_1=0$ with probability 1 and we are done.
    So, let us assume that $a$ is a vector of indices (of course by definition of $L'$, we have that $|a|=|v|=m$).\\
    \textbf{It suffices to show that $L(T+\gamma[N], v+\gamma a)=b_3$ is 0 with probability lower bounded by $1-\frac{O(1)}{|\F|}$.}

    Since $v$ is a subvector of $T$ there is a restriction of $T$ which gives $v$. Say the restriction is $b$. Then $|b|=|v|=m$ and $T[b_i]=v_i$ for all $i \in [m]$.
    Moreover, by our assumption that restriction of $T$ to $a$ is not $v$, we have that $a \neq b$.
    But $a\neq b$ implies there exists a $k \in [m]$ such that $a_k\neq b_k$.
    As restriction of $T$ to $a$ is not $v$, there exists a $k \in [m]$ such that $T[a_k]\neq v_k$\\
    Let $I=\{k \in [m]: T[a_k]\neq v_k\}$.
    By the fact that restriction of $T$ to $a$ is not $v$ , $I \neq \phi$.
    Also, $I \subset [N]$.
    By well ordering principle, $I$ has a smallest element, say $i$.
    Then for this $i$, $T[a_i]\neq v_i$. Moreover, as $v_i=T[b_i]$; $a_i \neq b_i$ \\

    Let $\gamma$ be picked uniformly at random from $\F^*$.
    Consider the vector $T+\gamma[N]$. Once $\gamma$ has been chosen, this is a fixed vector..
    Also, consider the vector $v+\gamma a$. One of the elements in this vector is $v_i +\gamma a_i$.
    We will upper bound the probability that this element is an element of the vector $T+\gamma[N]$. \\
    $$\Pr[v_i +\gamma a_i \in T+\gamma[N]]=\Pr[\exists j \in [N] \ni v_i +\gamma a_i=T[j]+\gamma j]$$
    But by union bound:
    $$\Pr[\exists j \in [N] \ni v_i +\gamma a_i=T[j]+\gamma j] \leq \sum_{j=1}^N \Pr[v_i +\gamma a_i=T[j]+\gamma j]$$
    Thus,
    $$\Pr[v_i +\gamma a_i \in T+\gamma[N]] \leq \sum_{j=1}^N \Pr\left[\gamma = \frac{T[j]-v_i}{a_i-j}\right] $$

    Let $M'$ be defined as $M'=\{j\in [N] \ni T[j]=v_i \}$.
    Let $M$ be defined as $M' \cup \{a_i\}$. Clearly $M \subset [N]$.\\
    But now, consider $\frac{T[j]-v_i}{a_i-j}$. For $j=a_i$ this is not defined. And for $j$ such that $T[j]=v_i$, this is 0. So, for such values of $j$, probability that $\gamma$ equals to it is $0$.
    For other values of $j$, $\frac{T[j]-v_i}{a_i-j}$ is an element of $\F^{*}$ and since $\gamma$ is a random element of $F^{*}$, the probability that $\gamma$ equals to it is $\frac{1}{|\F|-1}$\\
    In other words,
    \begin{gather}
        \Pr\left[\gamma = \frac{T[j]-v_i}{a_i-j}\right]=\frac{1}{|\F|-1}\, \forall j \in [N]\setminus M\\
        \Pr\left[\gamma = \frac{T[j]-v_i}{a_i-j}\right]=0 \, \forall j \in M
    \end{gather}

    Thus, we get that:
    $$\Pr[v_i +\gamma a_i \in T+\gamma[N]] \leq
    \sum_{j \in [N]\setminus M} \Pr\left[\gamma = \frac{T[j]-v_i}{a_i-j}\right]+$$
    $$\sum_{j \in M} \Pr\left[\gamma = \frac{T[j]-v_i}{a_i-j}\right]=\sum_{j \in [N]\setminus M} \Pr\left[\gamma = \frac{T[j]-v_i}{a_i-j}\right]$$
    $$=\sum_{j \in [N]\setminus M} \frac{1}{|\F|-1}=\frac{|[N]\setminus M|}{|\F|-1} \leq \frac{N-2}{|\F|-1}$$
    We used first that $[N]\setminus M$ and $M$ are disjoint and union is $[N]$, then we used the above equalities for $\Pr\left[\gamma = \frac{T[j]-v_i}{a_i-j}\right]$ and finally used that $|M| \geq 2$ as $M$ atleast contains $a_i$ and $b_i$.\\
    So we have obtained:
    $$\Pr[v_i +\gamma a_i \in T+\gamma[N]] \leq \frac{N-2}{|\F|-1}$$
    All this was done for the smallest element $i$ of the set $I$.
    For $v+\gamma a$ to be a subvector of $T+\gamma [N]$, we need that:
    $$v_i +\gamma a_i \in T+\gamma[N]$$ for all $i \in I$.
    But observe that,
    $$\Pr[v_i +\gamma a_i \in T+\gamma[N] \, \text{for all} \, i \in I] \leq \Pr[v_i +\gamma a_i \in T+\gamma[N]] \leq \frac{N-2}{|\F|-1}$$
    So, we have that $$\Pr[v_i +\gamma a_i \in T+\gamma[N] \, \text{for all} \,i \in I] \leq \frac{N-2}{|\F|-1}$$
    So, the probability that $v+\gamma a$ is a subvector of $T+\gamma [N]$ is upper bounded by $\frac{N-2}{|\F|-1}$.
    But $N <<|\F|$. So, $\frac{N-2}{|\F|-1}$ is $\frac{O(1)}{|\F|}$. \textbf{Thus, the error probability is upper bounded by $\frac{O(1)}{|\F|}$}.\\
    So, with a probability of atleast $(1-\frac{N-2}{|\F|-1})$, $v+\gamma a$ is not a subvector of $T+\gamma [N]$.
    So, $L(T+\gamma[N], v+\gamma a)=b_3$ is 0 with probability lower bounded by $1-\frac{O(1)}{|\F|}$. \\
    This completes the proof of lemma \ref{lem:reduction}
\end{proof}
\end{comment}

\begin{figure}[t!]
    \begin{mdframed}
        \begin{center}
            Continued from Figure \ref{fig:complete-listing}
        \end{center}
        {\bf Round 4}: Prover sends evaluations.
        \begin{enumerate}[leftmargin=1em, label=\arabic*.]
            \item $\prover$ computes $\val{\alpha}{v}=v(\alpha)$, $\val{\alpha}{a}=a(\alpha)$, $\val{\alpha}{h}=h(\alpha)$, $\val{\alpha}{K}=K(\alpha)$,
            $\val{h(\alpha)}{g}=g(h(\alpha))$, $\val{\alpha}{Q}=Q(\alpha)$, $\val{\alpha}{q}=q(\alpha)$, $\val{\alpha}{Z}=Z_I(\alpha)$.
            \item $\prover$ sends $\val{\alpha}{v}$, $\val{\alpha}{a}$, $\val{\alpha}{h}$, $\val{\alpha}{K}$, $\val{h(\alpha)}{g}$, $\val{\alpha}{Q}$,
            $\val{\alpha}{q}$, $\val{\alpha}{Z}$
            \item $\prover$ computes polynomial $u(X)$ as in Section ~\ref{subsec:proximity-ram}
            and sends $c_u=\gone{u(X)}$.
            \item $\verifier$ checks $\val{\alpha}{Q}(\alpha^m-1)=\val{h(\alpha)}{g}-\val{\alpha}{v}-\gamma \val{\alpha}{a}$.
            \item $\verifier$ checks $\val{\alpha}{Z}\val{\alpha}{q}=\val{\alpha}{K}$.
            \item $\verifier$ sets $\beta=\val{\alpha}{K}-1$ and sends $\theta\gets\F$.
        \end{enumerate}

        {\bf Round 5}: Check correctness of $K$.
        \begin{enumerate}[leftmargin=1em, label=\arabic*.]
            \item $\prover$ computes:
            \begin{align*}
                Q'(X) &= \big(u(\nu X)(1+\beta \tau_1(X))-u(X)(\alpha - h(X)) \\
                &\quad + \theta \tau_1(X)(u(X)-1)\big)/Z_\setV(X).
            \end{align*}
            \item $\prover$ sends $c_Q'=\gone{Q'(X)}$.
            \item $\verifier$ sends $x\gets \F$.
        \end{enumerate}

        {\bf Round 6}: Prover sends more evaluations.
        \begin{enumerate}[leftmargin=1em, label=\arabic*.]
            \item $\prover$ computes $\val{x}{u}=u(x)$, $\val{\nu x}{u}=u(\nu x)$, $\val{x}{h}=h(x)$, $\val{x}{Q'}=Q'(x)$
            \item $\prover$ sends $\val{x}{u}$, $\val{\nu x}{u}$, $\val{x}{h}$, $\val{x}{Q'}$.
            \item $\verifier$ checks $\val{x}{Q'}(x^m-1)=\val{\nu x}{u}(1+\beta \tau_1(x))-\val{x}{u}(\alpha - \val{x}{h})$.
            \item $\verifier$ sends $r_a, r_h, r_q, r_v, r_K, r_Q, r_Z\gets \F$ and $r_h', r_u', r_Q'\gets \F$.
        \end{enumerate}

        \begin{center}
            Continue in Figure ~\ref{fig:complete-listing-3}.
        \end{center}


    \end{mdframed}
    \caption{Batching-Efficient RAM Protocol: Continued}
    \label{fig:complete-listing-2}
\end{figure}

\begin{figure}[t!]
    \begin{mdframed}
        \begin{center}
            Continued from Figure \ref{fig:complete-listing-2}
        \end{center}
        \item {\bf Round 7}: Check aggregated evaluation.
        \begin{enumerate}[leftmargin=1em, label=\arabic*.]
            \item $\prover$ computes:
            \begin{align*}
                \Phi_\alpha(X) &= r_a a(X)+ r_h h(X) + r_q q(X) + r_v v(X) \\
                &\quad + r_K K(X) + r_Q Q(X) + r_Z Z_I(X) \\
                \Phi_x(X) &= r_h' h(X) + r_u' u(X)+r_Q'Q'(X)
            \end{align*}
            \item $\prover$ computes $\Pi_\alpha = \KZGopen(\srs, \Phi_\alpha(X), \alpha)$.
            \item $\prover$ computes $\Pi_x = \KZGopen(\srs, \Phi_x(X), x)$.
            \item $\prover$ computes $\Pi_g = \KZGopen(\srs, g(X), \val{\alpha}{h})$.
            \item $\prover$ computes $\Pi_u = \KZGopen(\srs, u(X), \nu x)$.
            \item $\prover$ sends $\Pi_\alpha$, $\Pi_x$, $\Pi_g$ and $\Pi_u$.
            \item $\verifier$ computes:
            \begin{align*}
                \gone{\Phi_\alpha} &= r_a c_a + r_h c_h + r_q c_q + r_v c_v + r_z c_z + r_K c_K + r_Q c_Q. \\
                \gone{\Phi_x} &= r_h' c_h + r_u' c_u + r_Q' c_Q' \\
                V_{\alpha} &= r_a \val{\alpha}{a} + r_h \val{\alpha}{h} + r_q \val{\alpha}{q} + r_v \val{\alpha}{v} \\
                &\quad + r_z\val{\alpha}{Z} r_K \val{\alpha}{K} + r_Q \val{\alpha}{Q}. \\
                V_x &= r_h' \val{x}{h} + r_u' \val{x}{u}+r_Q' \val{x}{Q'}
            \end{align*}
            \item $\verifier$ checks:
            \begin{itemize}[leftmargin=1em]
                \item $\KZGverify(\srs, \gone{\Phi_\alpha}, V_\alpha, \alpha, \Pi_\alpha)$.
                \item $\KZGverify(\srs, \gone{\Phi_x}, V_x, x, \Pi_x)$.
                \item $\KZGverify(\srs, c_g, \val{h(\alpha)}{g}, \val{\alpha}{h}, \Pi_g)$.
                \item $\KZGverify(\srs, c_u,\val{\nu x}{u}, \nu x, \Pi_u)$.
            \end{itemize}
            \item $\prover$ and $\verifier$ set $c_S=(c_a, c_v)$, $c_S'=(c_a, c_v')$, $c_o=(c_\op, c_a, c_w)$.
            \item $\prover$ and $\verifier$ execute argument for $(c_S, c_o, c_S')\in \CLRAM$ (Section ~\ref{sec:poly-proto-ram}).
        \end{enumerate}
    \end{mdframed}
    \caption{Batching-Efficient RAM Protocol-Continued}
    \label{fig:complete-listing-3}
\end{figure}
