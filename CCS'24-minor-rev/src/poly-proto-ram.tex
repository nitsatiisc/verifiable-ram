In this section, we briefly review and formalize existing memory-checking techniques to ensure
correctness of RAM operations. The formal definitions for various relations involved in memory checking
will be used to describe polynomial protocol for RAM in Appendix~\ref{sec:poly-proto-ram-app}.

\subsection{Correctness of RAM Update}\label{subsec:ram-update}
The versatility of the RAM primitive stems from its updatability. While a load operation leaves the RAM unchanged, the store operation
updates the value in the RAM associated with the referenced index. We model the update via the function
$\Rupd{I}$ which takes RAM $T\in \RAM{I}{n}$, operation
$o=(\op,a,v)\in \RAMOp{I}$ as inputs and returns an updated RAM $T'\in \RAM{I}{n}$.
The updated RAM $T'=\Rupd{I}(T,o)$ satisfies
$T'=T$ if $\op=0$ while for $\op=1$ it satisfies $T'[\,a\,]=v$  and $T'[\,x\,]=T[\,x\,]$ for $x\neq a$. The central problem
in verifiable RAM protocols is to establish that a sequence of operations $\vec{o}=(o_1,\ldots,o_m)$ are correct with
respect to the initial RAM state $T$ and the final RAM state $T'$. This involves ensuring
that all load operations read the value which is consistent with updates to the RAM as a result of preceding
store operations, and that $T'$ is the final state. We say that an operation $o=(\op,a,v)$ is {\em load-consistent}
with respect to RAM $T$ if $v=T[a]$ whenever $o$ is a load operation (store operations are vacuously defined to be load-consistent).
We formally define the notion of consistency below:

\begin{definition}[Consistent Operations]\label{defn:consistent-operations}
    Let $n\in \N$ and $T,T'\in \RAM{I}{n}$ for some index set $\setind$. We say that a sequence of operations
    $\vec{o}=(o_1,\ldots,o_k)\in \RAMOp{I}^k$ over $\setind$ is {\em consistent} with RAM states
    $T,T'$ if for all $i\in [k]$, $T_{i}=\Rupd{I}(T_{i-1},o_i)$ and operation $o_i$ is load-consistent with respect to $T_{i-1}$. Here
    we assume $T_0=T$ and $T_k=T'$.
\end{definition}

For $m,n\in \N$, let $\LRAM{I}{m}{n}$ denote the language consisting of tuples $(T,\vec{o},T')$ with $T,T'\in \RAM{I}{n}$ and $\vec{o}\in (\RAMOp{I})^m$
such that $\vec{o}$ is consistent with $T,T'$.
Next, we formalize the folklore technique of checking correctness of RAM operations
using {\em address-ordered transcripts}.

\subsection{Consistency Check via Transcripts}\label{subsec:transcripts}
A {\em transcript} is time-stamped sequence of operations executed on a RAM.
More formally, given a RAM $T=(\vec{a},\vec{v})\in \RAM{I}{n}$,
operation sequence $\vec{o}=(o_1,\ldots,o_m)$ with $o_i=(\bar{\op_i},\bar{a}_i,\bar{v}_i)\in \RAMOp{I}$ and RAM $T'=(\vec{a}', \vec{v}')\in \RAM{I}{n}$,
the {\em time ordered transcript} for the tuple $(T,\vec{o},T')$ is given by the table $\tr$ with $k=2n+m$ rows and four columns
$\tr = (\vec{t},\vec{\op},\vec{A},\vec{V})$  defined as follows: (i) $\vec{t}=\setind_{k}=(1,\ldots,k)$,
(ii) $\vec{\op}=0^n||(\bar{\op}_1,\ldots,\bar{\op}_m)||0^n$,
(iii) $\vec{A}=\vec{a}||(\bar{a}_1,\ldots,\bar{a}_m)||\vec{a}'$ and
(iv) $\vec{V}=\vec{v}||(\bar{v}_1,\ldots,\bar{v}_m)||\vec{v}'$. The $i^{th}$ row of the table $\tr$ is
$(t_i,\op_i,A_i,V_i)$ for $i\in [k]$. The first $n$ records in $\tr$ correspond to the contents of $T$,
the next $m$ records correspond to the operations in $\vec{o}$ and final $n$
records correspond to contents of $T'$. The timestamp column $\vec{t}$ is added to order operations with the same index.
Notationally, we write $\tr=\TOT(T,\vec{o},T')$.

We call a transcript $\tr=(\vec{t},\vec{\op},\vec{A},\vec{V})$ to be {\em address ordered} if $A_i\leq A_{i+1}$ for $i\in [k-1]$ and
$t_i < t_{i+1}$ whenever $A_i=A_{i+1}$. For a transcript $\tr=(\vec{t},\vec{\op},\vec{A},\vec{V})$ with $k$ records and a
permutation $\sigma:[k]\rightarrow [k]$, we use $\sigma(\tr)$ to denote the transcript
$(\sigma(\vec{t}),\sigma(\vec{\op}),\sigma(\vec{A}),\sigma(\vec{V}))$
obtained by permuting the records of $\tr$ according to the permutation $\sigma$.
An address ordered transcript for tuple $(T,\vec{o},T')$ is defined as $\tr^\ast=\sigma(\tr)$ where $\tr=\TOT(T,\vec{o},T')$ and $\sigma$ is
a permutation such that $\tr^\ast$ is address ordered. We denote it by $\tr^\ast=\AOT(T,\vec{o},T')$.
We say that an address ordered transcript $\tr=(\vec{t},\vec{\op},\vec{A},\vec{V})$ satisfies {\em load-store correctness}
if for all pairs of consecutive records $(t_i,\op_i,A_i,V_i)$ and $(t_{i+1},\op_{i+1},A_{i+1},V_{i+1})$ we have $V_{i+1}=V_i$
whenever $\op_{i+1}=0$ (load operation) and
$A_i=A_{i+1}$, i.e, a load operation does not change the value at an index.
We formally state the folklore technique for enforcing memory consistency in our setting.
\begin{lemma}\label{lem:consistency-check}
Let $\F$ be a finite field, $m,n\in \N$ be positive integers and $\setind\subseteq \F$. Then $(T,\vec{o},T')\in \LRAM{I}{n}{m}$ if and only if
    the address ordered transcript $\tr^\ast=\AOT(T,\vec{o},T')$ satisfies load-store correctness.
\end{lemma}

The consistency check in Lemma \ref{lem:consistency-check} can be encoded as an arithmetic circuit of size $\wt{O}(m+n)$, thus
yielding an argument of knowledge for the language $\LRAM{I}{n}{m}$ with prover complexity quasi-linear in
$m+n$. For completeness, we present a self-contained argument of knowledge for
$\LRAM{I}{m}{m}$ ($m=n$) based on the ``polynomial protocol'' framework defined in ~\cite{Gabizon2019PLONKPO}.



\begin{comment}
\begin{enumerate}[leftmargin=2em]
\item $\prover\rightarrow\verifier$: For the transcript $\tr^\ast=(\vec{t^\ast},\vec{\op^\ast},\vec{A^\ast},\vec{V^\ast})$,
the prover computes encoding polynomials $\wt{\tr^\ast}=({t^\ast}(X),\op^\ast(X), A^\ast(X),V^\ast(X))$. Next, the prover
computes sets $I_1$ and $I_2$ as in Section \ref{subsec:encoded-relations}, and then computes polynomials
$Z_b(X)=\prod_{i\in I_b}(X-\omega^i)$ for $b=1,2$.
The prover also interpolates polynomials $\delta^\ast_A(X)$ and $\delta^\ast_T(X)$ satisfying
$\delta^\ast_A(\omega^i) = A^\ast_{i+1} - A^\ast_i$ for $i\in I_1$ and $\delta^\ast_T(\omega^i)=t^\ast_{i+1}-t^\ast_i$ for $i\not\in I_1$.
The prover sends all the above polynomials to the verifier.
\item $\verifier$ computes: The verifier checks relations (C1)-(C6) in Lemma \ref{lem:addr-ordered-transcript}. The range constraints
in (C7) can be verified using polynomial protocols in sub-vector lookup arguments such as Caulk+ ~\cite{EPRINT:PosKat22}, CQ
~\cite{EPRINT:EagFioGab22}.
\end{enumerate}


\subsection{Succinct Argument for Verifiable RAM}\label{subsec:succ-args}
Polynomial protocols can be compiled into succinct arguments of knowledge using an extractable polynomial commitment scheme. At a
high level the compilation involves the prover sending ``commitments'' to the message polynomials, whereas the verifier checks
the polynomial identities probabilistically by requesting evaluation proofs of the polynomials at random points.
We complile the polynomial protocol for RAM to a succinct argument of knowledge in the Algebraic Group Model (AGM)
 using $\kzg$ as the extractable polynomial commitment scheme. We also leverage homomorphism of $\kzg$ scheme
to avoid sending commitments and evaluations for polynomials which are linear combinations of previously committed polynomials.
We now present an argument for verifiable RAM, which combines the polynomial protocols in this section, and instantiates
them using $\kzg$ commitments. We make standard optimisations of batching several identity checks into one where applicable.
Let $\srs=\{\gone{\tau^i},\gtwo{\tau^i}\}_{i=0}^d$
denote the $\kzg$ setup parameters for degree $d\geq k$ for the bilinear group $\mathsf{BG}=(\Gone,\Gtwo,\gone{1},\gtwo{1},e)$. Let
$\F=\F_p$ be the finite field where $p$ is the order of the groups.


\begin{figure}[t]
\begin{subfigure}{\linewidth}
\centering
{\footnotesize
\begin{enumerate}[leftmargin=1em]
    \item $\prover\rightarrow\verifier$: Send polynomial commitments $\gone{a(X)}$, $\gone{v(X)}$, $\gone{a'(X)}$, $\gone{v'(X)}$,
    $\gone{\bar{\op}(X)}$, $\gone{\bar{a}(X)}, \gone{\bar{v}(X)}$, $\gone{\op(X)}$, $\gone{A(X)}$, $\gone{V(X)}$, $\gone{t^\ast(X)}$,
    $\gone{\op^\ast(X)}$, $\gone{A^\ast(X)}$, $\gone{V^\ast(X)}$, $\gone{\delta_A^\ast(X)}$, $\gone{\delta_T^\ast(X)}$.
    \item $\verifier\rightarrow\prover$: Send $\alpha,\beta,\gamma\gets \F$.
    \item $\prover$ computes: Compute sets $I_1=\{i\in [k]: A^\ast_i\neq A^\ast_{i+1}\}$, and $I_2=[k]\setminus I_1$. Next, compute
    polynomials:
    \begin{align*}
        &Z_1(X)=\prod_{i\in I_1}(X-\omega^i),\quad Z_2(X)=\prod_{i\in I_2}(X-\omega^i), \\
        &H(X) = \begin{bmatrix} 1 & \gamma & \gamma^2 \end{bmatrix}
        \begin{bmatrix}
            a(X) & v(X) & 0 \\
            a'(X) & v'(X) & 0 \\
            \bar{a}(X) & \bar{v}(X) & \bar{\op}(X)
        \end{bmatrix}
        \begin{bmatrix}
        1 \\
        \beta \\
        \beta^2
        \end{bmatrix}, \\
        &G(X) = A(X) + \beta V(X) + \beta^2 \op(X), \\
        &Q(X) = \big(H(X^3) - G(X) - \gamma G(\omega^m X) - \gamma^2 G(\omega^{2m} X)\big)/Z(X), \\
        &f(X) = G(X) + \beta^3 t(X),\, g(X) = A^{\ast}(X) + \beta V^{\ast}(X) + \beta^2 \op^{\ast}(X) + \beta^3 t^{\ast}(X), \\
        &z(X) = \sum_{i=1}^k \lambda_i(X)\prod_{j=1}^{i-1} (\alpha - g(\omega^j))/(\alpha - f(\omega^j)), \\
        &Q_1(X) = \frac{1}{\mathbb{Z}_\setH(X)}\Big( (\alpha - g(X))z(\omega X) - (\alpha - f(X))z(X) \\
        &\qquad\qquad + \beta Z_2(X)(A^\ast(\omega X) - A^\ast(X) - \delta_A^\ast(X))\Big), \\
        &Q_2(X) = \frac{1}{Z_2(X)}\Big((A^\ast(\omega X) - A^\ast(X))
         + \beta(t^\ast(\omega X) - t^\ast(X)-\delta_T^\ast(X)) \\
        &\qquad\qquad + \beta^2 (\op^\ast(X) - 1)(V^\ast(\omega X) - V^\ast(X))\Big)
    \end{align*}
    \item $\prover\rightarrow \verifier$: Send $\gone{Z_1(X)}$, $\gone{Z_2(X)}$, $\gone{Q(X)}$, $\gone{z(X)}$,
    $\gone{Q(X)}$, $\gone{Q_1(X)}$, $\gone{Q_2(X)}$.
    \item $\verifier\rightarrow\prover$: $s\gets \F$.
    \item $\prover\rightarrow\verifier$: Send evaluations $\val{Z_1}{s}=Z_1(s)$, $\val{Z_2}{s}=Z_2(s)$,
    $\val{G}{s}=G(s)$, $\val{Q}{s}=Q(s)$, $\val{Z}{s}=Z(s)$, $\val{A^\ast}{s}=A^\ast(s)$, $\val{V^\ast}{s}=V^\ast(s)$,
    $\val{\op^\ast}{s}=\op^\ast(s)$, $\val{t}{s}=t(s)$, $\val{t^\ast}{s}=t^\ast(s)$, $\val{z}{s}=z(s)$, $\val{Q_1}{s}=Q_1(s)$,
    $\val{Q_2}{s}=Q_2(s)$, $\val{H}{s^3}=H(s^3)$, $\val{G}{\omega^m s}=G(\omega^m s)$, $\val{\delta^\ast_A}{s}=\delta_A^\ast(s)$,
    $\val{\delta_T^\ast}{s}=\delta_T^\ast(s)$
    $\val{G}{\omega^{2m} s}=G(\omega^{2m} s)$,
    $\val{A^\ast}{\omega s}=A^\ast(\omega s)$, $\val{V^\ast}{\omega s}=V^\ast(s)$, $\val{t^\ast}{\omega s}=t^\ast(\omega s)$,
    $\val{z}{\omega s}=z(\omega s)$, $\val{A^\ast}{\omega}=A^\ast(\omega)$.

    \item $\verifier\rightarrow\prover$: Send $r \gets \F$.
    \item $\prover$ computes: Compute batched $\kzg$ proofs:
    \begin{alignat*}{3}
        &P_1 && &\quad=\quad  && &Z_1+r Z_2 + r^2 G + r^3 Q + r^4 Z + r^5 A^{\ast} + r^6 V^{\ast} + r^7 \op^{\ast} \\
        & && & && &\quad  + r^8 t + r^9 t^{\ast} + r^{10} z + r^{11} Q_1 + r^{12} Q_2 + r^{13} \delta_A^{\ast} + r^{14} \delta_T^{\ast} \\
        &P_2 && &\quad=\quad && &A^{\ast} + r V^{\ast} + r^2 z \\
        &\Pi_s && &\quad=\quad && &\kzgprove(P_1, s) \\
        &\Pi_{\omega s} && &\quad=\quad && &\kzgprove(P_2, \omega s) \\
        &\Pi_{s^3} && &\quad=\quad && &\kzgprove(H, s^3) \\
        &\Pi_{\{\omega^m s, \omega^{2m} s\}} && &\quad=\quad && &\kzgprove(G, (\omega^m s, \omega^{2m} s))
    \end{alignat*}
    \item $\prover\rightarrow \verifier$: Send $\Pi_s,\Pi_{\omega s}, \Pi_{s^3},\Pi_{\{\omega^m s, \omega^{2m} s\}}$ and
    $\val{H}{s^3},\val{G}{\omega^m s}, \val{G}{\omega^{2m} s}$.

    \item $\verifier$ computes: Compute commitments and evaluations for linear combinations.
    \begin{align*}
        \gone{G(X)} &= \gone{A(X)}+\beta\gone{V(X)}+\beta^2\gone{\op(X)}, \\
        \gone{H(X)} &= \begin{bmatrix} 1 & \gamma & \gamma^2 \end{bmatrix}
        \begin{bmatrix}
            \gone{a(X)} & \gone{v(X)} & \gone{0} \\
            \gone{a'(X)} & \gone{v'(X)} & \gone{0} \\
            \gone{\bar{a}(X)} & \gone{\bar{v}(X)} & \gone{\bar{\op}(X)}
        \end{bmatrix}
        \begin{bmatrix}
            1 \\
            \beta \\
            \beta^2
        \end{bmatrix}, \\
        \gone{P_1(X)} &= \gone{Z_1(X)}+r \gone{Z_2(X)} + r^2 \gone{G(X)} + r^3 \gone{Q(X)} + r^4 \gone{Z(X)} + r^5 \gone{A^{\ast}(X)} \\
         &\qquad + r^6 \gone{V^{\ast}(X)} + r^7 \gone{\op^{\ast}(X)} + r^8 \gone{t(X)} + r^9 \gone{t^{\ast}(X)} + r^{10} \gone{z(X)} \\
         &\qquad + r^{11} \gone{Q_1(X)} + r^{12} \gone{Q_2(X)} + r^{13} \gone{\delta_A^{\ast}(X)} + r^{14} \gone{\delta_T^{\ast}(X)}, \\
        \gone{P_2(X)} &= \gone{A^\ast(X)} + r\gone{V^\ast(X)} + r^2\gone{z(X)},\\
        \val{P_1}{s} &= \val{Z_1}{s}+r \val{Z_2}{s} + r^2 \val{G}{s} + r^3 \val{Q}{s} + r^4 \val{Z}{s} + r^5 \val{A^{\ast}}{s} \\
         &\qquad + r^6 \val{V^{\ast}}{s} + r^7 \val{\op^{\ast}}{s} + r^8 \val{t}{s} + r^9 \val{t^{\ast}}{s} + r^{10} \val{z}{s} \\
         &\qquad + r^{11} \val{Q_1}{s} + r^{12} \val{Q_2}{s} + r^{13} \val{\delta_A^{\ast}}{s} + r^{14} \val{\delta_T^{\ast}}{s}, \\
        \val{P_2}{\omega s} &= \val{A^\ast}{s} + r\val{V^\ast}{s} + r^2\val{z}{s}. \\
        \val{g}{s} &= \val{A^\ast}{s} + \beta \val{V^\ast}{s} + \beta^2 \val{\op^\ast}{s} + \beta^3 \val{t^\ast}{s}, \\
        \val{f}{s} &= \val{G}{s} + \beta^3 \val{t}{s}.
    \end{align*}

    \item $\verifier$ checks polynomial identities at $s$:
    \begin{alignat*}{3}
        &\val{Q}{s}\val{Z}{s} && &\quad \stackrel{?}{=}\quad  && &\val{H}{s^3} - \val{G}{s} - \gamma \val{G}{\omega^m s} - \gamma^2 \val{G}{\omega^{2m} s} \\
        &\val{Q_1}{s}(s^k-1) && &\quad \stackrel{?}{=}\quad && &(\alpha - \val{g}{s})\val{z}{\omega s} - (\alpha - \val{f}{s})\val{z}{s}
        +\beta \val{Z_2}{s}(\val{A^\ast}{\omega s}-\val{A^\ast}{s}-\val{\delta_A^\ast}{s}). \\
        &\val{Q_2}{s}\val{Z_2}{s} && &\quad \stackrel{?}{=}\quad && &(\val{A^\ast}{\omega s} - \val{A^\ast}{s}) + \beta (\val{t^\ast}{s} - \val{t}{s} - \val{\delta}{s}) \\
        & && & && &\qquad + \beta^2 (\val{\op^\ast}{s} - 1)(\val{V^\ast}{\omega s}-\val{V^\ast}{s}) \\
        &\val{Z_1}{s}\val{Z_2}{s} && &\quad \stackrel{?}{=}\quad && &s^k - 1 \\
    \end{alignat*}
    \item $\verifier$ checks evaluation proofs:
    \begin{alignat*}{3}
        &\kzgverify(\gone{P_1(X)},\val{P_1}{s},s, \Pi_s) && &\quad\stackrel{?}{=}\quad && &1 \\
        &\kzgverify(\gone{P_2(X)},\val{P_2}{\omega s},\Pi_{\omega s}) && &\quad\stackrel{?}{=}\quad && &1 \\
        &\kzgverify(\gone{H(X)},\val{H}{s^3},\Pi_{s^3}) && &\quad\stackrel{?}{=}\quad && &1 \\
        &\kzgverify(\gone{G(X)},(\val{G}{\omega^m s},\val{G}{\omega^{2m} s}),
        (\omega^m s, \omega^{2m} s), \Pi_{\{\omega^m s, \omega^{2m} s\}}) && &\quad\stackrel{?}{=}\quad && &1 \\
        &\kzgverify(\gone{A^\ast(X)}, \val{A^\ast}{\omega}, \omega, \Pi_\omega) && &\quad\stackrel{?}{=}\quad && &1
    \end{alignat*}
    \item $\verifier$ enforces range checks: Invoke the lookup argument $\varPi_{\mathsf{lookup}}$ to check that
    polynomials $\delta_A^\ast(X)$, $\delta_T^\ast(X)$ and $t^\ast(X)$ encode values in $[0,N]$.
    \item $\verifier$ outputs: The verifier outputs accept (1) if all the preceeding checks pass, else it outputs reject (0).
\end{enumerate}
}
\end{subfigure}
\caption{Argument of Knowledge for Verifiable RAM}
\label{fig:aok-vram}
\end{figure}
\end{comment}

\begin{comment}
Let $\LLconcat$ denote the language consisting of polynomial tuples $(a(X)$, $b(X)$, $c(X)$, $v(X))$ satisfying the identities
in Lemma ~\ref{lem:vec-concatenation}. To check membership of $(T,\vec{o},T',\tr)$ in the language $\LTr$, let $\wt{T}=(a(X),v(X))$, $\wt{T'}=(a'(X),v'(X))$,
$\wt{O}=(\bar{\op}(X),\bar{a}(X),\bar{v}(X))$ and $\wt{\tr}=(t(X),o(X),A(X),V(X))$ be the polynomial
encodings of $T,T',\vec{o}$ and $\tr$ respectively.
From the construction of time ordered transcript $\tr$ outlined
in Section \ref{subsec:transcripts}, we see that the equivalent constraints on the encodings are given by:
\begin{align*}\label{eq:tot-poly-constraints}
t(X) &= \sum_{i=1}^k i\lambda_i(X) \quad (= \mathsf{Enc}(\setind_k)) \\
(0, \bar{\op}(X), 0, o(X)) &\in \LLconcat \\
(a(X), \bar{a}(X), a'(X), A(X)) &\in \LLconcat \\
(v(X), \bar{v}(X), v'(X), V(X)) &\in \LLconcat
\end{align*}

We can probabilistically combine the different polynomial identities into one and obtain the following:
\begin{lemma}
    Let the polynomial $Z(X)=\prod_{i\in [m]}(X-\omega^i)$ and $\rho=\omega^m$ be as before.
    Then, we have $(T,\vec{o},T',\tr)\in \LTr$ if and only if the encoding polynomials as defined above satisfy
    the identity $G_\gamma(X) = 0  \text{ mod } Z(X)$ and $\evalH{t}=(1,\ldots,k)$ with overwhelming
    probability over the choice of $\gamma\gets \F$. Here the polynomial $G_\gamma(X)$ is defined as:
    \begin{multline*}
        G_\gamma(X) = o(X) + \gamma(\bar{\op}(X^3) - o(\rho X)) + \gamma^2 o(\rho^2 X) \\
        + \gamma^3(a(X^3) - A(X)) + \gamma^4(\bar{a}(X^3) - A(\rho X)) + \gamma^5(a'(X^3) - A(\rho^2 X)) \\
        + \gamma^6(v(X^3) - V(X)) + \gamma^7(\bar{v}(X^3) - V(\rho X)) + \gamma^8(v'(X^3) - V(\rho^2 X))
    \end{multline*}
\end{lemma}

Next, we consider the language $\Lperm$ consisting of pairs $(\tr, \tr^\ast)$ of $k$-length transcripts such
that $\tr^\ast=\sigma(\tr)$ for some permutation $\sigma:[k]\rightarrow [k]$. We now describe constraints
to check the same using polynomial encodings of the transcripts. Let encodings $\wt{\tr},\wt{\tr}^\ast$ be
given by polynomials as below:
\begin{align*}\label{eq:tr-encodings}
\wt{\tr} &= (t(X),\op(X),A(X),V(X)) \\
\wt{\tr}^\ast &= (t^\ast(X), \op^\ast(X), A^\ast(X), V^\ast(X))
\end{align*}
We need to show that for some $\sigma:[k]\rightarrow [k]$, $\evalH{p^\ast}=\sigma(\evalH{p})$ for $p\in \{t,\op,A,V\}$.
Again, for $\gamma\gets \F$, with overwhelming probability it is equivalent to establishing that polynomial
$f^\ast(X)=t^\ast(X)+\gamma \op^\ast(X) + \gamma^2 A^\ast(X) + \gamma^3 V^\ast(X)$ encodes a permutation of
vector encoded by $f(X)=t(X)+\gamma \op(X) + \gamma^2 A(X) + \gamma^3 V(X)$. We recall the following variation
of the grand product argument to show that two polynomials encode vectors which are permutations of each other.
\end{comment}







