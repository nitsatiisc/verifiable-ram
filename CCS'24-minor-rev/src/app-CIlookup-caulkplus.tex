\section{Committed Index Lookup from Caulk+}\label{app:committed-index-lookup}

In this section, we present an explicit (non-black-box) adaptation of~\cite{EPRINT:PosKat22} to obtain a committed index lookup, which again incurs costs comparable to a single instance of the underlying sub-vector protocol. Let $m,N\in \N$ be fixed parameters with $m < N$ and let $\srs$ denote a $\kzg$ setup of degree $d\geq N$ over bi-linear group $(\F$, $\Gone$, $\Gtwo$, $\GT$, $e$, $\gone{1}$, $\gtwo{1}$, $[1]_t)$. Recall that the committed index
lookup relation in Definition ~\ref{defn:comm-index-lookup} involves the prover showing knowledge of vectors $\vecT\in \F^N$,
$\vec{a}\in \F^m$ and $\vec{v}\in \F^m$ corresponding to public commitments $c_T, c_a$ and $c_v$ such that they
satisfy $v_i = \vecT[\,a_i\,] = T_{a_i}$.
We present a polynomial protocol for the same, which is an adaptation of the lookup protocol from Caulk+ ~\cite{EPRINT:PosKat22}.
However, here we do not aim for zero-knowledge. Let $T(X)=\enc{t}{\setN}$, $a(X)=\enc{a}{\setV}$ and
$v(X)=\enc{v}{\setV}$ denote the polynomials encoding the vectors $\vec{t},\vec{a}$ and $\vec{v}$ respectively.
The verifier knows commitments to these polynomials at the start of the protocol.
Now $v_i = \vec{t}[a_i]$ for $i\in [m]$ is equivalent to $v(\nu^i) = T(\xi^{a(\nu^i)})$ for $i\in [m]$. To
obtain a polynomial protocol, the prover interpolates a polynomial $h(X)=\sum_{i=1}^m \xi^{a_i}\tau_i(X)$, which satisfies
$h(\nu^i)=\xi^{a(\nu^i)}$. To show that polynomial $h$ correctly ``exponentiates'' evaluations of $a(X)$, we consider the
inverting polynomial $\ell(X)=\sum_{i=1}^N i\mu_i(X)$ which behaves like ``log'' over $\setN$ by evaluating to $i$ on $\xi^i$. Now, we see
that all constraints are encoded as polynomial identities below:
\begin{equation}
	\begin{aligned}
		\ell(h(X)) &= a(X) \quad \text{mod } Z_{\setV}\\  % & \quad\text{ encodes } & \quad \forall i\in [m]:& h(\nu^i) = \xi^{a(\nu^i)}  \\
		T(h(X)) &= v(X) \quad \text{mod } Z_\setV \\ % \quad\text{ encodes } & \quad \forall i\in [m]:& v_i = \vec{t}[a_i] \\
		Z_{\setN}(h(X)) &= 0 \qquad \text{mod } Z_\setV  %&\quad\text{ encodes } & \quad \forall i \in [m]:& h(\nu^i)\in \setN
	\end{aligned}
	\label{eq:comm-index-lookup}
\end{equation}
The last polynomial identity ensures that evaluations of $h$ on $\setV$ lie in $\setN$ (the set of roots of $\vpolyN$). Since the polynomial $\ell$ is one-one
over $\setN$, the first equation implies $h(\nu^i)=\xi^{a_i}$ for all $i\in [m]$. The desired relation $v_i=T_{a_i}$ now follows from the second identity.
The above formulation involves composition with polynomials $\ell,T$ and $\vpolyN$ of degree $O(N)$, which is inefficient. We use the trick from
\cite{EPRINT:PosKat22}, where we work with low-degree restrictions of $O(N)$-degree polynomials such as $T, \ell$ over the set
$\setN_I=\{{h(\nu^i)}: i\in [m]\}=\{\xi^{a_i}:i\in I\}\subseteq \setN$, where $I=\{a_i: i\in [m]\}$. The prover
commits to the polynomials $Z_I(X)=\prod_{i\in I}(X-\xi^i)$, $h(X)$ and low degree ($<m$) restrictions $T_I, \ell_I$ of $T$ and $\ell$
on the $\setN_I$ respectively. The polynomial protocol then checks the following:
\begin{equation}
	\begin{alignedat}{3}
		T(X) - T_I(X) &= 0 \quad \text{ mod } Z_I ,&\quad& T_I(h(X)) &= v(X) \quad \text{ mod } Z_{\setV} \\
		\ell(X) - \ell_I(X) &= 0 \quad \text{ mod } Z_I ,&\quad& \ell_I(h(X)) &= a(X) \quad \text{ mod } Z_{\setV} \\
		Z_{\setN}(X) &= 0 \quad \text{ mod } Z_I ,&\quad& Z_I(h(X)) &= 0 \quad \text{ mod } Z_{\setV}
	\end{alignedat}
	\label{eq:poly-comm-index}
\end{equation}
It must be noted that the above identities imply the earlier polynomial identities in \eqref{eq:comm-index-lookup}. This is so because evaluations
of $h$ on $\setV$ are roots of $Z_I$, which implies $T_I(h(\nu^i))=T(h(\nu^i))$, $\ell_I(h(\nu^i))=\ell(h(\nu^i))$ and $\vpolyN(h(\nu^i))=0$ over $\setV$.
While the identities on the left still involve a degree $N$ polynomial, we can use the $\srs$ to check the polynomial
identity at the point $\tau$ encoded in the $\srs$. For example, we can evaluate the encoded quotient $\gtwo{Q(X)} =$
$\gtwo{\frac{(T(X) - T_I(X)}{Z_I(X)}}$ using the relation:
\begin{equation*}
	\gtwo{\frac{T(X)-T_I(X)}{Z_I(X)}} = \sum_{i\in I}\frac{1}{Z_I'(\xi^i)}\gtwo{\frac{T(X)-t_i}{X-\xi^i}}
\end{equation*}
By pre-computing the $\kzg$ proofs $W_1^i=\gtwo{\frac{T(X)-t_i}{X-\xi^i}}$ for all $i\in [N]$, the encoded quotient can be
evaluated using $O(m)$ $\Gtwo$-operations and $O(m\log^2 m)$ $\F$-operations.
The identity is then checked using a real pairing check
$$e(\gone{T(X)}-\gone{T_I(X)},\gtwo{1})=e(\gone{Z_I(X)},\gtwo{Q(X)}).$$
Similarly, we also pre-compute the encoded
quotients $W_2^i=\gtwo{\frac{\ell(X) - i}{X-\xi^i}}$ and $W_3^i=\gtwo{\frac{\vpolyN(X)}{X-\xi^i}}$ for all $i\in [N]$.
The quotients can be computed in time $O(N\log N)$ using the techniques in ~\cite{EPRINT:FeiKho23}. Using $\kzg$ commitment
scheme the polynomial relations over $Z_\setV$ can be checked in a standard manner
by having the prover send evaluation proofs for the committed polynomials at a random point chosen by the verifier.
The total prover effort incurred is $O(m^2)$ group and field operations.
Thus, we have:
\begin{lemma}\label{lem:comm-index-lookup}
	Assuming $\kzg$ is extractable polynomial commitment scheme, there exists a succinct argument of knowledge for
	the relation $\RLOOK$ with prover complexity of $O(m^2)$, given access to pre-computed parameters of size $O(N)$.
\end{lemma}
